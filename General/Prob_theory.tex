\documentclass[11pt]{article}
\usepackage{spikey}
\usepackage{amsmath}
\usepackage{amssymb}
\usepackage{soul}
\usepackage{float}
\usepackage{graphicx}
\usepackage{hyperref}
\usepackage{xcolor}
\usepackage{chngcntr}
\usepackage{centernot}
\usepackage[shortlabels]{enumitem}

\usepackage[margin=1truein]{geometry}

\title{Notes on Probability Theory}
\date{\today}
\author{Tianyu Du}

\counterwithin{equation}{section}
\counterwithin{theorem}{section}
\counterwithin{lemma}{section}
\counterwithin{corollary}{section}
\counterwithin{proposition}{section}
\counterwithin{remark}{section}
\counterwithin{example}{section}

\begin{document}
	\maketitle
	\tableofcontents
	
	\section{Preliminaries}
		\begin{definition}
			A \textbf{probability space} is a triple $(\Omega, \mc{F}, P)$ where $\Omega$ is the \textbf{sample space}, $\mc{F}$ is a $\sigma$-algebra of $\Omega$ (\textbf{events}) and $P: \mc{F} \to [0,1]$ is the \textbf{probability function}.
		\end{definition}

		\begin{remark}
			$(\Omega, \mc{F})$ is a \textbf{measurable space} or \textbf{Borel space}.
		\end{remark}
		
		\begin{definition}
			A \textbf{algebra}, $\mc{A}$, of set $X$ is a collection of subsets of $X$ closed under complementation and \emph{finite} union.
		\end{definition}
		
		\begin{definition}
			A \textbf{$\sigma$-algebra} of set $X$ is a collection of subsets of $X$ closed under complementation and \emph{countable} union.
		\end{definition}
		
		\begin{definition}
			A \textbf{semi-algebra} $\mc{S}$ is a collection of sets closed under intersection such that $S \in \mc{S}$ implies that $S^c$ is a \emph{finite disjoint} union of sets in $\mc{S}$.
		\end{definition}
		
		\begin{lemma}
			If $\mc{S}$ is a semi-algebra, then the set $S^c$ of \emph{finite disjoint} unions of sets in $\mc{S}$ is an algebra, called the \textbf{algebra generated by $\mc{S}$}.
		\end{lemma}
		
		\begin{definition}
			A \textbf{measure} on algebra is a function $\mu: \mc{A} \to \R$ such that
			\begin{enumerate}[(i)]
				\item $\mu(A) \geq \mu(\emptyset) = 0\ \forall A \in \mc{A}$,
				\item and countably additive for \emph{disjoint} set $\{A_i\}_i$
				\begin{equation}
					\mu(\cup_i A_i) = \sum_i \mu(A_i)
				\end{equation}
			\end{enumerate}
		\end{definition}
		
%		\begin{definition}
%			A \textbf{measure} on $\mc{F}$ is a function $\mu:\mc{F} \to \R$ satisfying $\mu(A) \geq \mu(\emptyset) = 0$ for all $A \in \mc{F}$. And $\mu$ is \emph{countably} additive for \emph{disjoint} $\{A_i\}_i$. That's
%			\begin{equation}
%				\mu(\cup_i A_i) = \sum_i \mu(A_i)
%			\end{equation}
%		\end{definition}
		
		\begin{definition}
			A measure $\mu$ on $\mc{F}$ is a \textbf{probability measure} if $\mu(\Omega) = 1$.
		\end{definition}
		
		\begin{definition}
			The \textbf{Borel $\sigma$-algebra} $\mc{B}$ on a topological space is the smallest $\sigma$-algebra \emph{containing all open sets}.
		\end{definition}
		
		\begin{theorem}
			For each \emph{right continuous, non-decreasing} function $F$ such that $\lim_{x \to -\infty} F = 0$ and $\lim_{x \to \infty} F = 1$, there is an \emph{unique} measure defined on the Borel sets of $\R$ with 
			\begin{equation}
				P((a,b]) \equiv F(b) - F(a)
			\end{equation}
		\end{theorem}
		
		\begin{definition}
			A collection $\mc{P}$ of sets is a \textbf{$\pi$-system} is it's closed under intersection.
		\end{definition}
		
		\begin{definition}
			A collection of sets $\mc{L}$ is a \textbf{$\lambda$-system} if
		\end{definition}
\end{document}























