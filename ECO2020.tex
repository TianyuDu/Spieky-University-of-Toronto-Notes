\documentclass{article}
\usepackage{spikey}
\usepackage{amsmath}
\usepackage{mathrsfs}
\usepackage{amssymb}
\usepackage{soul}
\usepackage{float}
\usepackage{graphicx}
\usepackage{hyperref}
\usepackage{xcolor}
\usepackage{chngcntr}
\usepackage{centernot}
\usepackage[shortlabels]{enumitem}
\usepackage[margin=1truein]{geometry}
\usepackage{tkz-graph}
\usepackage{dsfont}

\counterwithin{equation}{section}
\counterwithin{figure}{section}

\usepackage[
    type={CC},
    modifier={by-nc},
    version={4.0},
]{doclicense}

\title{ECO2020 Microeconomic Theory I (PhD) \\ \small Individual Decision Making, Market Equilibrium, Market Failure, and Other Topics.}
\date{\today}
\author{Tianyu Du}
\begin{document}
	\maketitle
	\doclicenseThis
	\begin{itemize}
		\item GitHub: \url{https://github.com/TianyuDu/Spikey_UofT_Notes}
		\item Website: \url{TianyuDu.com/notes}
	\end{itemize}
	\tableofcontents
	\newpage
	
	\section{Chapter 1. Preference and Choice}
	\subsection{Preference Relations}
	
		\begin{definition} \quad
			\begin{enumerate}[(i)]
				\item The \textbf{strict preference} relation, $\succ$, is defined by
					\begin{equation}
						x \succ y \iff x \succsim y \land \neg (y \succsim x)
					\end{equation}
				\item The \textbf{indifference} relation, $\sim$, is defined by
					\begin{equation}
						x \sim y \iff x \succsim y \land y \succsim x
					\end{equation}
			\end{enumerate}
		\end{definition}
	
		\begin{definition}[1.B.1]
			The preference relation $\succsim$ is \textbf{rational} if it possesses the following two properties
			\begin{enumerate}[(i)]
				\item \emph{Completeness} 
					\begin{equation}
						\forall x, y \in X,\ x \succsim y \lor y \succsim x
					\end{equation}
				\item \emph{Transitivity}
					\begin{equation}
						\forall x, y, z \in X,\ x \succsim y \land y \succsim z \implies x \succsim z
					\end{equation}
			\end{enumerate}
		\end{definition}
	
		\begin{proposition}[1.B.1]
			If $\succsim$ is rational, then
			\begin{enumerate}[(i)]
				\item $\succ$ is both \textbf{reflexive} ($\neg\ x \succ x$) and \textbf{transitive} ($x \succ y \land y \succ z \implies x \succ z$);
				\item $\sim$ is both \textbf{reflexive} and \textbf{transitive};
				\item $x \succ y \succsim z \implies x \succ z$.
			\end{enumerate}
		\end{proposition}
		
		\begin{example}
			Typical scenarios when transitivity of preference is violated:
			\begin{enumerate}[(i)]
				\item \emph{Just perceptible differences};
				\item \emph{Framing problem};
				\item \emph{Observed preference might from the result of the interaction of several more primitive rational preferences (Condorcet paradox)};
				\item \emph{Change of tastes}.
			\end{enumerate}
		\end{example}
		
		\begin{definition}[1.B.2]
			A function $u: X \to \R$ is a \textbf{utility function representing preference relation} $\succsim$ if
			\begin{equation}
				\forall x, y \in X,\ x \succsim y \iff u(x) \geq u(y)
			\end{equation}
		\end{definition}
		
		\begin{proposition}[1.B.2]
			If a preference relation $\succsim$ can be represented by a utility function, then $\succsim$ is rational.
		\end{proposition}
	
	\subsection{Choice Rules}
		\begin{definition}
			A \textbf{choice structure}, $(\mathscr{B}, C(\cdot))$, is a tuple consists of
			\begin{enumerate}[(i)]
				\item The collection of \textbf{budget sets} $\mathscr{B}$, which is a set of nonempty subsets of $X$.
				\item The \textbf{choice rule}, $C(B) \subset B$, is a \emph{correspondence} for every $B \subset \mathscr{B}$ denotes the individual's choice from among the alternatives in $B$. If $C(B)$ is not a singleton, it can be interpreted as the \emph{acceptable alternatives} in $B$, which the individual would actually chosen if the decision-making process is run repeatedly. 
			\end{enumerate}
		\end{definition}
		
		\begin{definition}[1.C.1]
			The choice structure $(\mathscr{B}, C(\cdot))$ satisfies the \textbf{weak axiom of revealed preference} if
			\begin{equation}
				\Big(\underbrace{
					\exists B \in \mathscr{B}\ s.t.\ x, y \in B \land x \in C(B)
					}_{\tx{$x \succsim y$ revealed.}}
				\Big)
				\implies 
				\Big(
					\forall B' \in \mathscr{B}\ s.t.\ x, y \in B',\ y \in C(B') \implies x \in C(B')
				\Big)
			\end{equation}
		\end{definition}
		
		\begin{definition}
			Given a choice structure $(\mathscr{B}, C(\cdot))$, the \textbf{revealed preference relation} $\succsim^*$ is defined as
			\begin{equation}
				x \succsim^* y \iff \exists B \in \mathscr{B}\ s.t.\ x, y \in B \land x \in C(B)
			\end{equation}
		\end{definition}
		
		\begin{remark}[Interpretation on the definition of WARP]
			If $x$ is \emph{revealed} at least as good as $y$, then $y$ cannot be revealed preferred to $x$.
		\end{remark}
		
\end{document}




















































