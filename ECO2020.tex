\documentclass{article}
\usepackage{spikey}
\usepackage{amsmath}
\usepackage{mathrsfs}
\usepackage{amssymb}
\usepackage{soul}
\usepackage{float}
\usepackage{graphicx}
\usepackage{hyperref}
\usepackage{xcolor}
\usepackage{chngcntr}
\usepackage{centernot}
\usepackage[shortlabels]{enumitem}
\usepackage[margin=1truein]{geometry}
\usepackage{tkz-graph}
\usepackage{dsfont}

\counterwithin{equation}{section}
\counterwithin{figure}{section}

\usepackage[
    type={CC},
    modifier={by-nc},
    version={4.0},
]{doclicense}

\title{ECO2020 Microeconomic Theory I (PhD) \\ \small Individual Decision Making, Market Equilibrium, Market Failure, and Other Topics.}
\date{\today}
\author{Tianyu Du}
\begin{document}
	\maketitle
	\doclicenseThis
	\begin{itemize}
		\item GitHub: \url{https://github.com/TianyuDu/Spikey_UofT_Notes}
		\item Website: \url{TianyuDu.com/notes}
	\end{itemize}
	\tableofcontents
	\newpage
	
	\section{Chapter 1. Preference and Choice}
	\subsection{Preference Relations}
	
		\begin{definition} \quad
			\begin{enumerate}[(i)]
				\item The \textbf{strict preference} relation, $\succ$, is defined by
					\begin{equation}
						x \succ y \iff x \succsim y \land \neg (y \succsim x)
					\end{equation}
				\item The \textbf{indifference} relation, $\sim$, is defined by
					\begin{equation}
						x \sim y \iff x \succsim y \land y \succsim x
					\end{equation}
			\end{enumerate}
		\end{definition}
	
		\begin{definition}[1.B.1]
			The preference relation $\succsim$ is \textbf{rational} if it possesses the following two properties
			\begin{enumerate}[(i)]
				\item \emph{Completeness} 
					\begin{equation}
						\forall x, y \in X,\ x \succsim y \lor y \succsim x
					\end{equation}
				\item \emph{Transitivity}
					\begin{equation}
						\forall x, y, z \in X,\ x \succsim y \land y \succsim z \implies x \succsim z
					\end{equation}
			\end{enumerate}
		\end{definition}
	
		\begin{proposition}[1.B.1]
			If $\succsim$ is rational, then
			\begin{enumerate}[(i)]
				\item $\succ$ is both \textbf{reflexive} ($\neg\ x \succ x$) and \textbf{transitive} ($x \succ y \land y \succ z \implies x \succ z$);
				\item $\sim$ is both \textbf{reflexive} and \textbf{transitive};
				\item $x \succ y \succsim z \implies x \succ z$.
			\end{enumerate}
		\end{proposition}
		
		\begin{example}
			Typical scenarios when transitivity of preference is violated:
			\begin{enumerate}[(i)]
				\item \emph{Just perceptible differences};
				\item \emph{Framing problem};
				\item \emph{Observed preference might from the result of the interaction of several more primitive rational preferences (Condorcet paradox)};
				\item \emph{Change of tastes}.
			\end{enumerate}
		\end{example}
		
		\begin{definition}[1.B.2]
			A function $u: X \to \R$ is a \textbf{utility function representing preference relation} $\succsim$ if
			\begin{equation}
				\forall x, y \in X,\ x \succsim y \iff u(x) \geq u(y)
			\end{equation}
		\end{definition}
		
		\begin{proposition}[1.B.2]
			If a preference relation $\succsim$ can be represented by a utility function, then $\succsim$ is rational.
		\end{proposition}
	
	\subsection{Choice Rules}
		\begin{definition}
			A \textbf{choice structure}, $(\mathscr{B}, C(\cdot))$, is a tuple consists of
			\begin{enumerate}[(i)]
				\item The collection of \textbf{budget sets} $\mathscr{B}$, which is a set of nonempty subsets of $X$.
				\item The \textbf{choice rule}, $C(B) \subset B$, is a \emph{correspondence} for every $B \subset \mathscr{B}$ denotes the individual's choice from among the alternatives in $B$. If $C(B)$ is not a singleton, it can be interpreted as the \emph{acceptable alternatives} in $B$, which the individual would actually chosen if the decision-making process is run repeatedly. 
			\end{enumerate}
		\end{definition}
		
		\begin{definition}[1.C.1]
			The choice structure $(\mathscr{B}, C(\cdot))$ satisfies the \textbf{weak axiom of revealed preference} if
			\begin{equation}
				\Big(\underbrace{
					\exists B \in \mathscr{B}\ s.t.\ x, y \in B \land x \in C(B)
					}_{\tx{$x \succsim y$ revealed.}}
				\Big)
				\implies 
				\Big(
					\forall B' \in \mathscr{B}\ s.t.\ x, y \in B',\ y \in C(B') \implies x \in C(B')
				\Big)
			\end{equation}
		\end{definition}
		
		\begin{definition}
			Given a choice structure $(\mathscr{B}, C(\cdot))$, the \textbf{revealed preference relation} $\succsim^*$ is defined as
			\begin{equation}
				x \succsim^* y \iff \exists B \in \mathscr{B}\ s.t.\ x, y \in B \land x \in C(B)
			\end{equation}
		\end{definition}
		
		\begin{remark}[Interpretation on the definition of WARP]
			If $x$ is \emph{revealed} at least as good as $y$, then $y$ cannot be revealed preferred to $x$.
		\end{remark}
	
	\subsection{The Relationship between Preference Relations and Choice Rules}
		\begin{definition}
			Given \ul{rational} preference relation $\succsim$ on $X$, the \textbf{preference-maximizing choice rule} is defined as 
			\begin{equation}
				C^*(B, \succsim) := \{x \in B: x \succsim y\ \forall y \in B\}\ \forall B \in \mathscr{B}
			\end{equation}
			We say the \ul{rational} preference relation \textbf{generates} the choice structure $(\mathscr{B}, C^*(\cdot, \succsim))$.
		\end{definition}
		
		\begin{assumption}
			Assume $C^*(B, \succsim) \neq \varnothing$ for all $B \in \mathscr{B}$.
		\end{assumption}
		
		\begin{proposition}[1.D.1 (\hl{Rational $\to$ WARP})]
			Suppose that $\succsim$ is a \ul{rational} preference relation. Then the choice structure generated by $\succsim$, $(\mathscr{B}, C^*(\cdot, \succsim))$, satisfies the weak axiom.
		\end{proposition}
		
		\begin{definition}[1.D.1]
			Given choice structure $(\mathscr{B}, C(\cdot))$, we say that the \ul{rational preference relation $\succsim$ \textbf{rationalizes} $C(\cdot)$ relative to $\mathscr{B}$} if
			\begin{equation}
				C(B) = C^*(B, \succsim)\ \forall B \in \mathscr{B}
			\end{equation}
			That is, \emph{$\succsim$ generates the choice structure $(\mathscr{B}, C(\cdot))$}.
		\end{definition}
		
		\begin{remark}
			In general, for a given choice structure $(\mathscr{B}, C(\cdot))$, there may be more than one rational preference relation $\succsim$ rationalizing it.
		\end{remark}
		
		\begin{proposition}[1.D.2 (\hl{WARP $\to$ Rational})]
			If $(\mathscr{B}, C(\cdot))$ is a choice structure such that
			\begin{enumerate}[(i)]
				\item The weak axiom is satisfied;
				\item $\mathscr{B}$ includes all subsets of $X$ up to three elements.
			\end{enumerate}
			Then there is a rational preference relation $\succsim$ that rationalizes $C(\cdot)$ relative to $\mathscr{B}$.
		\end{proposition}
	
	\section{Chapter 2. Consumer Choice}
		\subsection{Commodities}
			\begin{definition}
				Assume the number of \textbf{commodities} is finite and equal to $L$. In general, a \textbf{commodity vector} or \textbf{commodity bundle} is an element in a \textbf{commodity space}, typically $\R^L$.
				\begin{equation}
					\ve{x} := \begin{bmatrix}
						x_1 \\ \vdots \\ x_L
					\end{bmatrix} \in \R^L
				\end{equation}
 			\end{definition}
 			
 			\begin{remark}[Time Aggregation]
 				The time/location of commodity matters in some scenarios, and can be built into the definition of a commodity.
 			\end{remark}
 			
 			\begin{remark}
 				We should also note that in some contexts it becomes convenient, and even necessary, to expand the set of commodities to include goods and services that may potentially be available for purchase but are not actually so and even some that may be available by means other than market exchange.
 			\end{remark}
 		
 		\subsection{The Consumption Set}
 			\begin{definition}
 				The \textbf{consumption set} is a subset of the commodity space $\R^L$, denoted by $X \subset \R^L$, whose elements are the consumption bundles that the individual can conceivably consume given the physical constraints imposed by his environment.
 			\end{definition}
 			
 			\begin{assumption}
 				For simplicity, we assume the consumption set to be $\R_+^L$, which is \emph{convex}.
 				\begin{equation}
 					X := \R_+^L = \{\ve{x} \in \R^L: x_{\ell} \geq 0,\ \forall \ell \in [L]\}
 				\end{equation}
 			\end{assumption}
 			
 		\subsection{Competitive Budgets}
 			\begin{definition}
 				A \textbf{price vector} is defined as
 				\begin{equation}
 					\ve{p} := \begin{bmatrix} p_1 \\ \vdots \\ p_L \end{bmatrix} \in \R^L
 				\end{equation}
 				For simplicity, here we always assume 
 				\begin{enumerate}[(i)]
 					\item \emph{Positive price}: $\ve{p} \gg \ve{0}$;
 					\item \emph{Price-taking assumption}: $\ve{p}$ is beyond the influence of the consumer.
 				\end{enumerate}
 			\end{definition}
 			
 			\begin{definition}[2.D.1]
 				The \textbf{Walrasian}, or \textbf{competitive budget set} is defined as
 				\begin{equation}
 					B_{\ve{p}, w} := \{
 						\ve{x} \in \R^L_+: \ve{p} \cdot \ve{x} \leq w
 					\}
 				\end{equation}
 				where $w$ is the \emph{wealth} of consumer, and assumed to be positive.
 			\end{definition}
 			
 			\begin{definition}
 				The \textbf{consumer's problem} is choosing a consumption bundle $\ve{x} \in B_{\ve{p}, w}$, for each given $(\ve{p}; w) \in \R^L_{++}$.
 			\end{definition}
 			
 			\begin{definition}
 				The set $\{\ve{x} \in \R^L_+: \ve{p} \cdot \ve{x} = w\}$ is called the \textbf{budget hyperplane}.
 			\end{definition}
 			
 			\begin{proposition}
 				The price vector $\ve{p}$ is orthogonal to the budget hyperplane.
 			\end{proposition}
 			
 			\begin{proposition}
 				The Walrasian budget set $B_{\ve{p}, w}$ is a \emph{convex} set.
 			\end{proposition}
\end{document}



















































