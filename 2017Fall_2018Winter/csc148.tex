\documentclass{article}
\usepackage{amsmath}
\usepackage{amssymb}
\usepackage{pgfplots}
\usepackage{graphicx}
\usepackage{enumitem}
\usepackage{hyperref}
\usepackage{fancyhdr}
\usepackage{float}
\pagestyle{fancy}
\lhead{Notes by T.Du}
\usepackage[
	type={CC},
	modifier={by-nc-sa},
	version={3.0},
]
{doclicense}
\title{CSC148 Notes}
\author{Tianyu Du}
\date{\today}

\begin{document}
	\maketitle
	\doclicenseThis
	\tableofcontents
	\section{Lecture 1. Jan. 10 2017}
	\paragraph{Outlines}
	\begin{enumerate}
		\item Construct solutions to real world problems.
		\item Abstract data types.
		\item Recursion.
		\item Exceptions.
		\item Design.
		\item Efficiency.
	\end{enumerate}
	\subsection{Object}
	\begin{verbatim}
	>>> s1 = 'word'
	>>> s2 = 'sword'[1:]
	>>> s1 == s2
	True
	>>> s1 is s2 # s1 and s2 are different objects.
	False
	>>> n1 = 255
	>>> n2 = 255
	>>> n1 == n2
	True
	>>> n1 is n2
	True
	>>> n3 = 257
	>>> n4 = 257
	>>> n3 is n4
	False
	\end{verbatim}
	\paragraph{Object} Components of object:
	\begin{itemize}
		\item Identifier.
		\item Type.
		\item Value. 
	\end{itemize}
	
	\subsection{Review function design recipe}
	\paragraph{\emph{Repeated} function}

	\underbar{Check list:}
	\begin{enumerate}
		\item Header.
		\item Type contract.
		\item Doc string.
		\item Function body.
		\item Test.
	\end{enumerate}
	\paragraph{Design recipe}
	\begin{verbatim}
	from typing import List

	def repeated(word: str, n: int) -> List[str]:
		'''
		Repeated - return a list of word n times.

		>>> repeated('a', 2)
		['a', 'a']

		>>> repeated('a', 0)
		[]
		'''
		return [word] * n
	\end{verbatim}
	\subsection{Point}
	\begin{verbatim}
	class Point:
		'''
		Represent a two-dimensional point.

		x - horizontal position.
		y - vertical position.
		'''

		x: float
		y: float

		def __init__(self, x, y) -> None:
			'''

			'''
			self.x, self.y = x, y

		def distance_to_origin(self):
			return (self.x **2 + self.y ** 2) ** .5

		def __eq__(self, other: Any) -> bool:
			'''
			Return whether self is equivalent to other
			>>> Point(3, 5) == Point(3.0, 5.0)
			True
			>>> Point(3, 5) == Point(5, 3)
			False
			>>> Point(3, 5) == 7
			False
			'''
			return (type(self) == type(other)
					and self.x == other.x
					and self.y == other.y)

		def __str__(self) -> str:
			'''
			Return a string representing this point itself.

			>>> print(Point(3, 5))
			(3.0, 5.0)
			'''
			return ''({}, {})''.format(self.x, self.y)

	\end{verbatim}
	\subsection{Build API}
	\paragraph{Define a class API:}
	\begin{enumerate}
		\item Choose a class name and write a brief description in the class doc string.
		\begin{verbatim}
			Point
		\end{verbatim}
		\item Write some examples of client code that uses you class.
		\begin{verbatim}
			p = Point(3, 4)
			p.distance_to_origin()
		\end{verbatim}
	\end{enumerate}

	\section{Lecture 2. Jan. 17 2017}
	\paragraph{Rational numbers Class}
	\paragraph{Module} A name space contain names. \begin{verbatim} module.name \end{verbatim} 
	

	\section{Lecture Fri Jan. 19 2018}
	\paragraph{}
\end{document}





















