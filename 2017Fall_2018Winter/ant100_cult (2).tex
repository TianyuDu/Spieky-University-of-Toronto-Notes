\documentclass{article}

\title{ANT10 0 Notes Cult.}
\author{Tianyu Du}
\date{\today}


\begin{document}
	\maketitle
	\tableofcontents

	\section{Lecture 1 Jan.11 2018}
	\subsection{Culture: universal ability and particular culture}
	\paragraph{} being able to \emph{have culture} is \emph{programmed} by our genes the \emph{program} enables culture=specific ways of acting,
	living and thinking:
	\begin{itemize}
		\item Forming precreative relationships.
		\item Recognizing hierarchy.
		\item Dealing with violence.
		\item Sharing and giving.
		\item Education.
		\item Speaking: language, how you see it.
	\end{itemize}

	\subsection{Cultural universals and particulars}
	\paragraph{}Cultures are significantly different, but not infinitely different.
	\newline Human cultures have more in common than not.

	\subsection{Language}
	\paragraph{Language} is a distinctively human resource and central element of culture, used for:
	\begin{itemize}
		\item Communication and cooperation.
		\item Adapting to the natural and social environment.
		\item Signification. \emph{Making meanings}.
		\item Structuring our imagination.
		\item How we see the world.
		\item Each other.
		\item Ourselves.
	\end{itemize}

	\paragraph{Language and languages} Like cultures, languages have more in common than not.
	\newline \textbf{Word universals:}
	\begin{itemize}
		\item Nouns and verbs.
		\item Importance of word order.
	\end{itemize}

	\subsection{Recap: Universals and Divisions}
	\paragraph{} Language and Culture are human universals, but specific language and specific cultures are human particulars.
	\newline Universals(language, culture) are innate. \emph{Transmitted via gene (through sex)}
	\newline ...

	\subsection{Adaptive Value of Social Transmission}
	\paragraph{} Social transmission is much more flexible than genetic transmission:
	\begin{itemize}
		\item Major changes can occur within a generation or two.
		\item Major changes can occur without a change of species.
	\end{itemize}
	\underbar{Language and culture allow us to adapt to new situations.}

	\subsection{Why the differences?}
	\paragraph{} Specific languages and cultures develop to cope with specific environment and social contexts. \textbf{niches}.
	\paragraph{Cultures:} potlatch, brides-wealth / dowry.
	\newline Culture can change the niche itself.
	\newline Result: Adaptation without becoming a new species.

	\subsection{Difference. Conflict. Prejudice}
	\paragraph{} Within all species, groups can come into conflict, often over \emph{resources}.
	\newline The resources may be \emph{natural} or \emph{socials}.
	\newline In \emph{homo-sapiens} these groups may have different cultures and languages.
	\newline and this may be accompanied by prejudices about the other.
	\newline \underbar{Prejudice does \textbf{not} come from difference but from its \textbf{context}}

	\subsection{Physical Type and Prejudice}
	\paragraph{Race} a \emph{folk} notion, not a scientific notion. Human \textbf{race} are not scientifically but \emph{popularly} defined.
	\newline Does races exist?
	\newline \underbar{Yes! But not as a scientific category. And race is not really based (only) on physical appearance.}

	\paragraph{Different humans do look different.} and human looks can correlate with some other genetic traits. But this is \textbf{not} consistently enough to justify \emph{scientifically} the notion of human races.
	\newline There is more variation (other than skin color, and some other exceptions) across than within races.
	\newline The popular definition of race are \emph{imprecise}.

	\subsection{The One-Drop Rule}
	\paragraph{} If you have any \emph{black blood}, then you're black.
	\newline Also word for other non-while groups: if you have any non-white blood, then you're not white.

	\paragraph{Brown} a new racial category. Cover people who are neither black nor white.
	\paragraph{Lesson from the One-Drop Rule} Humans do differ by appearance, but their \emph{classification} by difference is not given nature,
	\newline It is given by \textbf{language}.
	\newline If there were no words for the different races, then there would not be races.
	\newline Possible test question: How does the one-drop rule prove that race is not a classification given by nature.

	\subsection{Social Construction and Nature}
	\paragraph{Social construction} is the source of ideas and arrangements that are not given by nature.
	\newline Social construction is not part of nature, but does works with natural materials. Race works with real, natural evidence, yet is determined by society, not nature.

	\paragraph{Invented} Ideas that are the result of social construction.

	\subsection{Naturalization}
	\paragraph{Naturalization} The process that people come to think of what was socially constructed as if it were given by nature.

	\paragraph{Conclusion} \underbar{Race is real: invented but naturalized.}
	\newline Other examples: national identities, gender roles.

	\subsection{Talking About Race}
	\paragraph{} Silence and color-blindness is \textbf{not} the solution.
	\newline Social science \emph{describe} before it can (if it ever can) prescribe.
	\newline \textbf{Political correctness} aims to protect others.
	\newline but it can be a hindrance to understanding.
	\newline if it prevents us from \emph{talking} about what we see as happening,
	\newline or ...

	\subsection{The Role of Anthropology}
	\paragraph{} Unravel the social constructedness of invented (but real) categories like race, national identity, or gender.
	\newline Discover their \textbf{genealogy}.
	\newline This work must precede any advice anthropologists can given on what it right or \emph{correct}.
	\paragraph{Genealogy} What history makes them possible.

	\paragraph{Race and Power Relations: The Genealogy of Black and White} \quad
	\newline Racial classification are made in the context of relations of power within society.
	\newline People have always perceived differences of skin color ...
	\newline but did not classify people into distinct categories of \emph{race} ...
	\newline Race meat imagined common descent, and include English or Irish ...
	\newline The \emph{black race} and \emph{white race} were \emph{invented} along with slavery in the Americas.
	\newline \emph{Whiteness} means freedom and need to be protected.
\end{document}






























