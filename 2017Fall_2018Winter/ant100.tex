\documentclass{article}
\usepackage{amsmath}
\usepackage{pgfplots}
\usepackage{enumitem}
\author{Tianyu Du}
\date{Since Sep 2017}
\title{Notes on ANT100, Introduction to Anthropology}
\begin{document}
	\maketitle
	\tableofcontents
	\section{Sep. 7 Course Descriptions}
	\subsection{Tutorial}
	\begin{itemize}
		\item Must attend. \textbf{Attendance taken}.
		\item Starts from \textbf{Sep.18-22}.
		\item Deadline for signing up: \textbf{Sep. 26}.
		\item Called \textbf{group} at blackboard.
	\end{itemize}
	\subsection{Intro.}
	\paragraph{What's Anthropology?} Holistic study of humans,\emph{Homo Sapiens}, past and present that draws and builds upon knowledge from the social sciences, biological sciences, humanities and the natural sciences.
	\paragraph{First Section: Evolutionary Anthropology}
	\begin{enumerate}
		\item Historical development, mechanisms, and outcomes of biological evolution.
		\item Diversity of life and the natural processes that produced diversity.
		\item The \textbf{primate} fossil record, with a basic understanding of patterns and processes that evolved in the \textbf{hominin} branch.
		\item The basic ecology, behaviour, and conservation biology of extant primates.
		\item \textbf{Medical anthropology} How evolutionary anthropologists apply biological concepts in their research on human health, disease, and forensics.
	\end{enumerate}
	\paragraph{Archaeology Lecture Topic Outline}
	\begin{enumerate}
		\item Introduction to archaeology.
		\begin{enumerate}
			\item ...
		\end{enumerate}
		\item The archaeological record.
		\begin{enumerate}
			\item What survives from the past and how can we interpret it.
			\item Fieldwork.
			\item Dating. 
		\end{enumerate}
		\item Analysis and interpretation.
		\begin{enumerate}
			\item Archaeological data.
			\item Interpretation.
		\end{enumerate}
		\item The earliest races of human behaviour. \emph{e.g. Stone Tools.}
		\item Origin and spread of modern humans. \emph{From African origins to entire globe.}
		\item From food production to early states.
		\begin{enumerate}
			\item Origin of agriculture.
			\item Origin of urban, state-level society, \emph{civilization}.
		\end{enumerate}
	\end{enumerate}
	\section{Sep. 22 2017}
	\subsection{Lecture Objectives}
	\begin{enumerate}
		\begin{itemize}
			\item Genetic basis of inheritance and biological evolution.
			\item Population genetics.
			\item Natural selection.
			\item Adaptation	
		\end{itemize}
	\end{enumerate}
	\subsection{Modern Synthesis of Evolution}
	\begin{itemize}
		\item $DNA \rightarrow RNA \rightarrow Protein$
		\item Microevolution.
		\item Macroevolution.
	\end{itemize}
	\paragraph{Genetics}
	\begin{itemize}
		\item Somatic Cells. \emph{most cells in body, except sex cells}
		\item Gametes. \emph{sex cells}
		\item Cytoplasm. \emph{complex mix of membranes, molecules and tiny structures}
		\item Nucleus.
	\end{itemize}
	\paragraph{Chromosomes} Paired rod-shaped structures in cell nucleus containing DNA.
	\paragraph{DNA (Deoxyribonucleic Acid)} Storing genetic information.
	\newline \textbf{Four Bases of DNA:}
	\begin{enumerate}
		\item \textbf{Adenine(A)}
		\item \textbf{Guanine(G)}
		\item \textbf{Cytosine(C)}
		\item \textbf{Thymine(T)}
	\end{enumerate}
	\paragraph{RNA()Ribonucleic Acid}
	\begin{enumerate}
		\item Dictate synthesis of proteins that perform a wide variety of functions in body.
		\item Regulate expression of other genes.
		\item Work with structures in cell.
	\end{enumerate}
	\paragraph{Proteins}
	\paragraph{DNA \& Protein Production}
	\begin{enumerate}
		\item (DNA) Replication.
			\begin{itemize}
				\item messengerRNA.
			\end{itemize}
		\item Transcription.
		\item Translation.
			\begin{itemize}
				\item transfer RNA. (tRNA)
			\end{itemize}
		\item Proteins.
	\end{enumerate}
	\paragraph{Codons} To avoid \textbf{copying errors}.
	\paragraph{Genetics \& Heredity}
	\begin{itemize}
		\item \textbf{Gene} chemical unit of heredity.
		\item \textbf{Phenotype} may or may not reflect genotype.
		\item \textbf{Genotype} represents the copy of genes 
		\item \textbf{Alleles} one number of a pair of genes.
	\end{itemize}
	\begin{itemize}
		\item \textbf{Homozygous} processing two \emph{identical} genes or alleles in the pair.
		\item \textbf{Heterozygous} different genes or alleles.
	\end{itemize}
	\paragraph{Dominant Alleles} \emph{Always} phenotypically expressed in \emph{heterozygous} form.
	\paragraph{Recessive Alleles} Only be represented in \emph{homozygous} form.
	\subsection{Mutation}
	\paragraph{} Error or change in a nucleotide sequence. Randomly occurring process. Somatic cell mutations vs. germ cell mutations in terms of relevance to evolutionary anthropology.
	\paragraph{} Can be neutral, harmful, or beneficial for organism.
	\newline \textbf{Result of four thing.}
	\begin{itemize}
		\item Copying errors in cell division.
		\item Exposure to radiation.
		\item Exposure to mutagens
		\item Exposure to viruses.
	\end{itemize}
	\paragraph{}Ultimate source of new genetic materials.
	\subsection{Population Genetics: Genetic Drift}
\end{document}




















