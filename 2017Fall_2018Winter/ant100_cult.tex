\documentclass{article}

\title{ANT10 0 Notes Cult.}
\author{Tianyu Du}
\date{\today}


\begin{document}
	\maketitle
	\tableofcontents

	\section{Lecture 1 Jan.11 2018}
	\subsection{Culture: universal ability and particular culture}
	\paragraph{} being able to \emph{have culture} is \emph{programmed} by our genes the \emph{program} enables culture=specific ways of acting,
	living and thinking:
	\begin{itemize}
		\item Forming precreative relationships.
		\item Recognizing hierarchy.
		\item Dealing with violence.
		\item Sharing and giving.
		\item Education.
		\item Speaking: language, how you see it.
	\end{itemize}

	\subsection{Cultural universals and particulars}
	\paragraph{}Cultures are significantly different, but not infinitely different.
	\newline Human cultures have more in common than not.

	\subsection{Language}
	\paragraph{Language} is a distinctively human resource and central element of culture, used for:
	\begin{itemize}
		\item Communication and cooperation.
		\item Adapting to the natural and social environment.
		\item Signification. \emph{Making meanings}.
		\item Structuring our imagination.
		\item How we see the world.
		\item Each other.
		\item Ourselves.
	\end{itemize}

	\paragraph{Language and languages} Like cultures, languages have more in common than not.
	\newline \textbf{Word universals:}
	\begin{itemize}
		\item Nouns and verbs.
		\item Importance of word order.
	\end{itemize}

	\subsection{Recap: Universals and Divisions}
	\paragraph{} Language and Culture are human universals, but specific language and specific cultures are human particulars.
	\newline Universals(language, culture) are innate. \emph{Transmitted via gene (through sex)}
	\newline ...

	\subsection{Adaptive Value of Social Transmission}
	\paragraph{} Social transmission is much more flexible than genetic transmission:
	\begin{itemize}
		\item Major changes can occur within a generation or two.
		\item Major changes can occur without a change of species.
	\end{itemize}
	\underbar{Language and culture allow us to adapt to new situations.}

	\subsection{Why the differences?}
	\paragraph{} Specific languages and cultures develop to cope with specific environment and social contexts. \textbf{niches}.
	\paragraph{Cultures:} potlatch, brides-wealth / dowry.
	\newline Culture can change the niche itself.
	\newline Result: Adaptation without becoming a new species.

	\subsection{Difference. Conflict. Prejudice}
	\paragraph{} Within all species, groups can come into conflict, often over \emph{resources}.
	\newline The resources may be \emph{natural} or \emph{socials}.
	\newline In \emph{homo-sapiens} these groups may have different cultures and languages.
	\newline and this may be accompanied by prejudices about the other.
	\newline \underbar{Prejudice does \textbf{not} come from difference but from its \textbf{context}}

	\subsection{Physical Type and Prejudice}
	\paragraph{Race} a \emph{folk} notion, not a scientific notion. Human \textbf{race} are not scientifically but \emph{popularly} defined.
	\newline Does races exist?
	\newline \underbar{Yes! But not as a scientific category. And race is not really based (only) on physical appearance.}

	\paragraph{Different humans do look different.} and human looks can correlate with some other genetic traits. But this is \textbf{not} consistently enough to justify \emph{scientifically} the notion of human races.
	\newline There is more variation (other than skin color, and some other exceptions) across than within races.
	\newline The popular definition of race are \emph{imprecise}.

	\subsection{The One-Drop Rule}
	\paragraph{} If you have any \emph{black blood}, then you're black.
	\newline Also word for other non-while groups: if you have any non-white blood, then you're not white.

	\paragraph{Brown} a new racial category. Cover people who are neither black nor white.
	\paragraph{Lesson from the One-Drop Rule} Humans do differ by appearance, but their \emph{classification} by difference is not given nature,
	\newline It is given by \textbf{language}.
	\newline If there were no words for the different races, then there would not be races.
	\newline Possible test question: How does the one-drop rule prove that race is not a classification given by nature.

	\subsection{Social Construction and Nature}
	\paragraph{Social construction} is the source of ideas and arrangements that are not given by nature.
	\newline Social construction is not part of nature, but does works with natural materials. Race works with real, natural evidence, yet is determined by society, not nature.

	\paragraph{Invented} Ideas that are the result of social construction.

	\subsection{Naturalization}
	\paragraph{Naturalization} The process that people come to think of what was socially constructed as if it were given by nature.

	\paragraph{Conclusion} \underbar{Race is real: invented but naturalized.}
	\newline Other examples: national identities, gender roles.

	\subsection{Talking About Race}
	\paragraph{} Silence and color-blindness is \textbf{not} the solution.
	\newline Social science \emph{describe} before it can (if it ever can) prescribe.
	\newline \textbf{Political correctness} aims to protect others.
	\newline but it can be a hindrance to understanding.
	\newline if it prevents us from \emph{talking} about what we see as happening,
	\newline or ...

	\subsection{The Role of Anthropology}
	\paragraph{} Unravel the social constructedness of invented (but real) categories like race, national identity, or gender.
	\newline Discover their \textbf{genealogy}.
	\newline This work must precede any advice anthropologists can given on what it right or \emph{correct}.
	\paragraph{Genealogy} What history makes them possible.

	\paragraph{Race and Power Relations: The Genealogy of Black and White} \quad
	\newline Racial classification are made in the context of relations of power within society.
	\newline People have always perceived differences of skin color ...
	\newline but did not classify people into distinct categories of \emph{race} ...
	\newline Race meat imagined common descent, and include English or Irish ...
	\newline The \emph{black race} and \emph{white race} were \emph{invented} along with slavery in the Americas.
	\newline \emph{Whiteness} means freedom and need to be protected.
	
	\section{Lecture 3 Jan. 25 2018}
	\paragraph{Lecture outline}
	\begin{itemize}
    	\item Real and not real
    	\item Reality as a construct
    	\item \underbar{Whorf hypothesis} how different languages construct different realities.
    	\item Self as a construct
    	\item \underbar{Lacan's stages of development} the real, the imaginary and the symbolic.
	\end{itemize}
	\subsection{The importance of making up things}
	\paragraph{} Some signified are real and others are not.
	\paragraph{} Irony, lies, fantasy, plans.
	\paragraph{} Making up thing is an act of \textbf{signifying} - using signs it is an adaptive advantage of \emph{homo sapiens}. Together with inventing categories of understanding.
	\paragraph{} \textbf{Signs reflect but also make our world.}
	
	\subsection{What do we mean by "real"}
	\paragraph{} Studied by philosophers, psychologists, political scientists sociologists, and anthropology.
	\paragraph{} Overlapping disciplines.
	\paragraph{} Anthropologists focus on \underbar{small-group constructs and interactive practice}.
	\paragraph{Philosophers} \textbf{John Wisdom}(1904-1993) and \textbf{John Austin}(1911-1960)
	
	\subsection{Fake News!}
	\paragraph{} Real, imaginary, fake.
	
	\subsection{No contradiction}
	\paragraph{} Reality is constructed
	\paragraph{} Yet we can distinguish reality from irreality (truth from lies).
	
	\subsection{Reality as a construct}
	\paragraph{Reality}
	\begin{itemize}
	    \item What is verifiable.
	    \item The world as \emph{it makes sense to us}.
	\end{itemize}
	\paragraph{Construct} formed by people in society. Most of reality comes across to us through the filter of signs and language.
	\newline It is socially constructed. It may not always be the same as what \emph{really exists}: \textbf{the real}.
	
	\subsection{Jargon v.s. ordinary language}
	\paragraph{} In ordinary language: \underbar{reality is what is real}.
	\paragraph{} In social science/ humanities jargon: \underbar{reality is how we understand the real}, it is "\emph{our reality}".
	\paragraph{} For the most part, reality seems pretty real and we are well advised to live in it.
	\paragraph{} But a lot of the \emph{real} is not our reality: we do not understand it.
	
	\subsection{Species-specific vision}
	\paragraph{} Different animals see the same scene in different ways.
	\paragraph{} Their eyes construct the image of the scene they see.
	\paragraph{} What is the real, unconstructed scene?
	
	\subsection{The "Whorf Hypothesis"}
	\paragraph{Benjamin Lee Whorf} (1897-1941)
	\paragraph{Linguistic relativity} Each language decisively influence the way its speakers think, different languages construct different reality.
	
	\subsection{Colour as a social construct}
	\paragraph{Example} Colours: Green and blue.
	
	\subsection{Perception and interpretation}
	\paragraph{} The point is not that old-time Chinese \emph{could not} see green and blue, but they did not \emph{understand} them as essentially different.
	\paragraph{} So our understanding of how the world is constructed by signs, and especially languages.
	
	\subsection{The self as a construct}
	\paragraph{Jacques Lacan}(1901-1981) Reinterpreted Freud.
	\paragraph{Stages of How the self develops:}
	\begin{itemize}
	    \item Real.
	    \item Imaginary (mirror stage)
	    \item Symbolic (accomplish through "\emph{language}")
	\end{itemize}
	
	\subsection{The Real(stage) - ego not yet formed.}
	\begin{itemize}
    	\item The Real is undifferentiated.
    	\item No signs
    	\item Uncategorized experience.
	\end{itemize}
	
	\subsection{The Imaginary(Mirror) Stage: The ego forms.}
	\paragraph{To remember about the imaginary stage}
	\begin{itemize}
	    \item It corresponds to the \underbar{icon}: images rather than words. (the most typical image sign is the icon)
	    \item In the imaginary stage, the world is perceived \underbar{without words}.
	    \item The ego image is supported by the authority of Mother/ Father/ Society.
	    \item Roughly: this is the image v.s. the word stage.
	\end{itemize}
	
	\subsection{The symbolic stage}
	\begin{itemize}
	    \item Language appears (language is mostly symbols).
	    \item Language is learned from parents/ society.
	    \item Language is a complicated system.
	    \item In this stage the world is differentiated into categories marked by signifiers (e.g. words).
	    \item Ego is called "\emph{I}".
	    \item Ego learns to understand "\emph{I}" as part of a society that is in relation to "you" and "he/she/they".
	\end{itemize}
	
	\subsection{"I"}
	\begin{itemize}
	    \item "I" is a symbol.
	    \item Each occurrence of $I$ is a signifier, whose meaning is developed from relations to other signifiers. From other times and I said "I" and from times I heard others say "you".
	\end{itemize}
	
	\subsection{Inner conversation}
	\paragraph{} We are both "I" and "you" to ourselves.
	\paragraph{} In inner conversation, one party coaches the other.
	\paragraph{} This \emph{coach} is more influenced by the society, it \emph{represents} society.
	\paragraph{} Freud: \textbf{superego}
	\paragraph{} Our self is a conversation, and it includes a representation of society(the superego).
	
	\subsection{Conclusion}
	\paragraph{So } Language and signs
	\begin{itemize}
	   \item  make sense of the work - they construct it.
	   \item help us function in society - to communicate.
	   \item but also help society function within us - our self is socially constructed.   
	\end{itemize}

	\section{Religion}
	\emph{What's Reality Real}

	\subsection{Lecture Themes}
	\paragraph{} Anthropology and Religion.
	\paragraph{Religion}
	\begin{itemize}
		\item A link beyond reality to the real.
		\item A link beyond culture.
		\item A link beyond society.
	\end{itemize}
	\paragraph{Ritual}

	\subsection{The anthropological attitude to religion}
	\paragraph{} The goal is not to judge or to establish truth or falsehood.
	\paragraph{} Recognize the nature and role of a religion in its cultural and social context.
	\paragraph{} Discover what \emph{religion} might mean as a general characteristic of human society.

	\subsection{What Religion is Not Necessary}
	\paragraph{} To some extent, what is or is not religion is a matter of arbitrary definition. It may be that the very concept of \emph{religion} is a modern western one.
	\paragraph{} None of the following exists in all religions.
	\begin{itemize}
		\item A holy text.
		\item Dogma (\emph{Principles that are authorized as true.})
		\item Natural history (\emph{origin and evolution of Nature.})
		\item ``Faith'' 
		\item ``God''
		\item ``The victory of good over evil''
	\end{itemize}
\end{document}






























