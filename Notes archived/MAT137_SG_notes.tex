\documentclass{article}
\author{Tianyu Du}
\date{\today}
\usepackage{amsmath}
\usepackage{amssymb}
\title{Notes on MAT 137 study guide}
\begin{document}
	\maketitle
	\tableofcontents
	\section{Part 1. Sets and functions}
	\paragraph{Functions}
	\[
	f: A \rightarrowtail B.
	\]
	\begin{itemize}
		\item A: \textbf{domain} of $f$.
		\item B: \textbf{codomain} of $f$.
		\item $f(A)$: \textbf{range} of $f$.
		\item \textbf{Onto/surjective} if and only if codomain = range.
		\item \textbf{One-to-one/injective} if and only if $f(x) = f(y) \implies x = y$.
		\item \textbf{Bijective} if and only if a function is both surjective and injective.
		\item \textbf{Even}: $f(x) = f(-x) \forall x \in A$.
		\item \textbf{Odd}: $f(x) = -f(-x) \forall x \in A$.
		\item \textbf{Increasing[Decreasing]}: $x < y \implies f(x) < f(y) [f(x) > f(y)]$.
	\end{itemize}
	\paragraph{Operations on functions}
	\begin{itemize}
		\item $(f \pm g)(x) = f(x) \pm g(x)$.
		\item $(f \cdot g)(x) = f(x) \cdot g(x)$.
		\item $(\frac{f}{g})(x) = \frac{f(x)}{g(x)}$.
	\end{itemize}
	\paragraph{Composite of functions} for functions 
	\[ f:A\rightarrow B
	\]\[
	g:C\rightarrow D
	\]
	$f(g(x))$ is defined when $D \subset A$.
	\paragraph{Exponential and logarithms} for $x,y,n \in \mathbb{R}, x,y > 0$
	\begin{itemize}
		\item $b^{x+y} = b^x + b^y$.
		\item $b^{x-y} = \frac{b^x}{b^y}$.
		\item $\log_b(xy) = \log_b(x) + \log_b(y)$.
		\item $\log_b(\frac{x}{y}) = \log_b(x) - \log_b(y)$.
		\item $\log_b(x^n) = n \log_b(x)$.
		\item $\log_y(b) \log_b(x) = \log_y(x)$.
	\end{itemize}
	\paragraph{The supremum and infimum} A set S of real numbers is said to be \textbf{bounded above[below]} if there is a $s \in \mathbb{R}$ such that $s \geq x$[$s \leq x$] for all $s \in A$.
	\paragraph{Definition} A number is a \textbf{least upper bound(supremum)} of S if both:
	\begin{itemize}
		\item s is an upper bound of S.
		\item $\forall x$: upper bound of S, then $s \leq x$.
	\end{itemize}
	\section{Part 2. Limits \& Continuity}
	\paragraph{Definition} We say that the limit of a function $f(x)$ as x approaches a is L($\lim_{x\rightarrow a}f(x)=L$) if:
	\[\forall \epsilon >0, \exists b > 0, s.t. 0<\lvert x-a\rvert \implies \lvert f(x) - L \rvert < \epsilon
	\]
	\newline $\lim_{x\rightarrow a}f(x)=L$ exists if and only if the left-hand and right-hand limits both exist and are equal.
\end{document}












