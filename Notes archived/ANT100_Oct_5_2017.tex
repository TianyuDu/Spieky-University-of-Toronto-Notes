\documentclass{article}
\author{Tianyu Du}
\date{\today}
\title{Lecture notes on ANT100}

\begin{document}
\maketitle
\tableofcontents

\section{Oct.5.2017 @CH Lecture4: Primate \& Human Evolution}
\subsection{Outlines}
\begin{itemize}
	\item General patterns of morphology and phylogenetics for fossil primates and hominins.
	\item Hominin is in terms of taxonomy.
	\item Morphological trends in hominin evolution.
	\begin{itemize}
		\item Bipedalism
		\item Exapnsion of brain size.
		\item in dental / cramial features.
	\end{itemize}
\end{itemize}
\subsection{Epochs during Tertiary Period}
\subsubsection{Paleocenne Primates}
	\paragraph{} Different geography and climate.
	\paragraph{} Different conditions.
	\paragraph{} Hotter, more humid. \textbf{About 25 degree C}.
	\paragraph{Body size} Tiny, \emph{size of small dog}.
	\paragraph{Niche} Likely solitary, nocturnal, quadrupeds, well deveoped sense of \emph{smell}.
	\paragraph{Diet} Insects and seeds.
	\paragraph{} Have \emph{primate-like teeth and limbs} that are adapted for \emph{arboreal} lifestyle.  
	\paragraph{Recent: Plesiadapids NOT primates}
		\begin{itemize}
			\item No post-orbital bar.
			\item Claws instead of nails.
			\item Eyes placed on sides of head.
			\item Enlarged incisors.
		\end{itemize}
	\paragraph{More Recent: Plesiadapids \& Others ARE Primates} Analyzing the \emph{gene} of species.

\subsubsection{Eocene}
	\paragraph{Geography \& Climate} Increase in global average temperature, average \textbf{30 degree C}.
	\paragraph{Adapidae}
	\begin{itemize}
		\item Body size: 100g \~ 6900g.
		\item Diurnal and nocturnal forms.
		\item Mainly arboreal quadrupeds, some were specialized leapers.
		\item Smaller adapids ate mostly \textbf{fruit} and \textbf{insects}, larger forms ate more \textbf{fruit} and \textbf{leaves}. \emph{Decompositing leaves requires special bacteria.}
		\item Lead to \emph{lemurs}?
	\end{itemize}
	\paragraph{Omomyidae}
	\begin{itemize}
		\item Body size: 45g \~ 2500g.
		\item Some nocturnal others diurnal.
		\item Omomyids thought to been specialized leaps.
		\item Teeth: adapted for eating insects and soft fruits, only few species were leaf-eaters.
		\item Led to \emph{tarsiers}?
	\end{itemize}

\subsubsection{Oligocene}
	\paragraph{Geography \& Climate} Increase in temperature.
	\paragraph{Oligocene Primates} 3 \textbf{haplorhine} features:
	\begin{itemize}
		\item Fused forntal bones.
		\item Full postorbital closure.
		\item Fused mandibular symphasis.
	\end{itemize}
	\paragraph{Three taxonomic Groups}
	\begin{itemize}
		\item Paraphithecidae.
		\item Proliopithecidae.
		\item Platyrrhini.
	\end{itemize 	\paragraph{} Monkeys found in South America the same time as in Africa.
	\paragraph{Hypothesis} May have rafted over to South America from Africa.
\subsubsection{Micoene}
	\paragraph{Geography \& Climate}
	\paragraph{Early} (23.0 \~ 16.0 MYA) Monkeys \& Apes, apparently confined to Aferica.
	\paragraph{Middle} (16.0 \~ 11.6 MYA) Ape-like catarrhines widespread and diverse in Europe and Asia.
	\paragraph{Late} (11.6 \~ 5.3 MYA) Apes became rarer as woodlands and forests replaced by drier and more open habitats.
\subsubsection{Pliocence}
	\paragraph{Weather \& Geography} Land masses still on move, connection between north and south America opended via Panama.
	\newline Fluctuations in global temperatures.
	\newline Mediterranean sea droed up and end of Miocene and filled up again in Pliocene.
	\paragraph{Pliocene Primates} Two main taxa.
	\begin{itemize}
		\item Fossil \textbf{Cercophithecinae}.
		\item Fossil \textbf{Colobinae}.
	\end{itemize}

\subsection{Transitional Forms(Apes:Humans)}
\begin{itemize}
	\item Modifications of postcranial skeleton for bipedal locomotion.
	\item Shape and size of canines, especially in males, changes so not pointy or blade-like. Reduction in level of sexual dimorphism in canine size.
	\item Expansion of brain.
\end{itemize}
\subsection{What's a Hominin?}
\paragraph{Hominin} modern humans, chinmpanzees, and fossil species more closely related to each other than to any other living species.
\paragraph{Morphological Trends in Hominin Evolution}
\begin{itemize}
	\item \textbf{Mosaic Evolution} Major evolutionary changes tend to take place \emph{in stages}, not all at once.
 	\item Bipedalism.
 	\item Increased brain size.
 	\begin{itemize}
 		\item Intelligence.
 		\item Relative size v.s. Absolute size.(e.g. Blue Whale)
 	\end{itemize}
 \end{itemize} 
 \paragraph{Quadrupedalism v.s. Bipedalism}
 \begin{itemize}
 	\item \textbf{Foramen magrnum} position, under / back.
 	\item \textbf{Crinial capacity}
 \end{itemize}
\subsection{Hominin Fossils}
\paragraph






\end{document}