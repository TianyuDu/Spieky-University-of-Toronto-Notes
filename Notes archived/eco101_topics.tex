\documentclass{article}
\author{Tianyu Du}
\date{Oct. 2. 2017}
\title{Topic Outline for ECO101}
\begin{document}
	\maketitle
	\tableofcontents
	\section{What is Economics?}
	\subsection{Basic ideas.}
	\paragraph{Scarcity} Our inability to get everything.
	\paragraph{Economics} is the \emph{social science} that \emph{studies the choices} that individuals, businesses governments , and entire societies make as they cope with \emph{scarcity} and the \emph{incentives} that influence and reconcile those choices.
	\paragraph{Central problems}
	\begin{enumerate}
		\item \emph{What} to produce.
		\item \emph{How}: allocation of \textbf{factors of production}
		\begin{itemize}
			\item Land earns \textbf{rent}
			\item Labour earns \textbf{wages}
			\item Capital earns \textbf{interest}
			\item Entrepreneurship earns \textbf{profit}
		\end{itemize}
		\item \emph{For whom}: allocation of products among agents.
	\end{enumerate}
	\subsection{Economics as Social Science and Policy Tool}
	\paragraph{Positive Statements} is about what \emph{is}, and can be tested.
	\paragraph{Normative Statements} is about what \emph{ought to be}, and can \textbf{not} be tested.
	\section{The Economic Problem}
	\paragraph{Production Possibility Frontier(PPF)} is the boundary between those combinations of goods and services that can be produced and those that cannot. PPF is concave due to \emph{increasing of opportunity cost}.
	\paragraph{Production Efficiency} produce at the \emph{lowest} possible cost.
	\paragraph{Allocative efficiency} When goods and services are produced at the lowest possible cost and in the quantities that provide the greatest possible benefit.\textbf{MC=MB}.
	\paragraph{Opportunity Cost} of an action is the \emph{highest-valued} alternative forgone.
	\paragraph{Marginal Benefit} \emph{willingness to pay}. 
	\paragraph{The Cost of Economic Growth}
	\paragraph{Comparative advantage} advantages on \emph{opportunity cost}.
	\paragraph{Absolute advantage} advantages on \emph{actual cost}.
	\paragraph{Economic Coordination} firms, markets, property rights, money.
	\section{Demand and Supply}
	\subsection{Demand}
	\paragraph{The Law of Demand} Other things remaining the same, the higher the price of a good, the smaller is the quantity demanded; and the lower the price of a good, the greater is the quantity demanded.\footnote{Graph required?}
	\begin{itemize}
		\item Substitution effect.
		\item Income effect.	
	\end{itemize}
	\paragraph{Factors affecting demand}
	\begin{itemize}
		\item The prices of related-goods.
		\item Expected future prices.
		\item Income.
		\item Expected future income and credit.
		\item Population.
		\item Preferences.
	\end{itemize}
	\subsection{Supply}
	\paragraph{Law of supply} Other things remaining the same, the higher the price of a good, the greater is the quantity supplied; and the lower the price of a good, the smaller is the quantity suppled.
	\paragraph{Factors affecting supply}
	\begin{itemize}
		\item The prices of factors of production.
		\item The prices of related goods produced.
		\item Expected future pries.
		\item The number of suppliers.
		\item Technology.
		\item The state of nature.
	\end{itemize}
	\paragraph{$\star$ Shifting of demand and supply: diagrams}.
	\section{Elasticity}
	\paragraph{Calculating PED}
	\begin{itemize}
		\item By definition.
		\item Mid-point.
		\item PED at a point.
	\end{itemize}
	\paragraph{Factors influencing PED}
	\paragraph{Total revenue and PED}
	\paragraph{XED}
	\paragraph{PES}
	\paragraph{YED}\footnote{See page 97 for a compact table}
	\section{Efficiency and Equity}
	\paragraph{Consumer/Producer Surplus}
	\paragraph{Marginal social benefit and marginal social cost}
	\paragraph{Source of market failure}
	\begin{itemize}
		\item Price/quantity regulations.
		\item Taxes and subsidies.
		\item Externalities.\footnote{diagram!}
		\item Public Goods and Common Resources.
		\item Monopoly.
		\item High transactions costs.
	\end{itemize}
	\paragraph{Fairness?} \emph{Fair-results} v.s. \emph{Fair-rules}
	\begin{itemize}
		\item \textbf{Fair-results} require income transfers from the rich to the poor.
		\item \textbf{Fair-rules} require property rights and voluntary exchange.
	\end{itemize}
	\section{Government Actions in Markets}
	\paragraph{Examples}\footnote{diagram!}
	\begin{itemize}
		\item A housing market with a \emph{rent ceiling}.
		\item A labour market with a \emph{minimum wage}.
	\end{itemize}
	\paragraph{Taxes/Subsidies}
	\begin{itemize}
		\item Diagram.
		\item Tax incidence: related to \emph{relative elasticity}.
	\end{itemize}
	\paragraph{Illegal Goods.} \emph{black market}.
	\section{Global Markets in Action}
	\paragraph{Comparative advantages} drives international trade. Consider domestic price and international price(trade price) to calculate the \emph{fair price} for international trade.
	\paragraph{Rent seeking} is the lobbying for special treatment by the government to create economic profit or to divert consumer surplus or producer surplus away from others.
	\paragraph{Protection} methods:
	\begin{itemize}
		\item import quota.
		\item tariff.
		\item subsidies to domestic producers.
	\end{itemize}
	\paragraph{Protection} reasons:
	\begin{itemize}
		\item Infant industry.
		\item Jobs.
		\item Dumping?
	\end{itemize}
	\section{Utility and Demand}
	\paragraph{Consumption choice}
	\begin{itemize}
		\item Preferences.
		\item Budget line.
		\item Diminishing in marginal utility.
	\end{itemize}
	\paragraph{Utility-Maximizing Choice}
	\paragraph{New ideas}:
	\begin{itemize}
		\item Behavioural economics.
		\item Bounded rationality.
		\item Neuro-economics.
	\end{itemize}
	\section{Possibilities, Preferences, and Choices}
	\paragraph{Consumption Possibilities}
	\begin{itemize}
		\item Budget.
		\item Indifference curve.
		\item Marginal rate of substitution.
		\item Price effect. \emph{Substitution effects} and \emph{Income effect}.
	\end{itemize}
	\section{Organizing Production} 
	\paragraph{Goal} Maximizing \emph{economic profit}.
	\paragraph{Normal profit} the opportunity cost of entrepreneurship and is part of the firm's opportunity cost.
	\paragraph{Technologically efficient} when a firm uses the least amount of inputs to produce a given output.
	\paragraph{Economies of scale}
	\paragraph{Economies of scope}
	\paragraph{Four-firm concentration ratio}
	\paragraph{Herfindahl-Hirschman Index}
	\paragraph{Implicit rental rate} the firm's opportunity cost of using the capital it owns.
	\section{Output and Costs}
	\paragraph{Diagrams !!!!}
	\begin{itemize}
		\item Short Run v.s. Long run.
		\item A(T)FC, A(T)VC, ATC, TC.
		\item Increasing/Decreasing/Constant returns to scale.
		\item Minimum efficient scale.
		\item Marginal cost/product.
	\end{itemize}
	\section{Perfect Competition}
	\paragraph{Diagrams !!!}
	\paragraph{Equilibrium} MC=MR=P=D
	\begin{itemize}
		\item MR
		\item Short run supply
		\item Shutdown point
		\item Total revenue.
	\end{itemize}
	\section{Monopoly}
	\paragraph{Causes of monopoly}
	\paragraph{Diagrams !!!} MR=MC.
	\paragraph{Price discrimination}
	\paragraph{Monopoly regulation}
	\section{Monopolistic competition}
	\paragraph{} Downwards sloping demand. \emph{Product differentiation}
	\paragraph{Long run} entry and exit causes zero economic profit.
	\section{Oligopoly}
	\paragraph{Anti-combine Law}
	\paragraph{Game theory analysis}
	\paragraph{Nash equilibrium}
	\section{Externalities}
	\paragraph{}Difference between social benefit(cost) and private benefit(cost).
	\paragraph{Diagrams !!!}
	\section{Public Goods and Common Resources}
	\paragraph{} Analyzing with the externalities.
	\paragraph{Public good} is \emph{non-rival} and \emph{non-excludable}, so it creates \emph{free-rider problem}.
	\paragraph{Tragedy of commons} lack of private incentives.
	\section{Markets for Factors of Production}
	\paragraph{Bilateral monopoly}
	\paragraph{Derived demand} for factors of production.
	\paragraph{Hoteling Principle} Traders expect the price of a non-renewable natural resource to rise at a rate equal to the interest rate.
	\paragraph{Monopsony}
	\paragraph{Value of Marginal Product}
\end{document}




















