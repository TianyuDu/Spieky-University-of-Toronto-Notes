\documentclass{article}

\title{ECO206 Lecture Note}
\date{Jun. 12 2018}
\author{Tianyu Du}


\usepackage{amsmath}
\usepackage{amssymb}


\newcommand{\pd}[2]{\frac{\partial{#1}}{\partial{#2}}}




\begin{document}

\maketitle

\section{Tophat Question}

	\paragraph{} $MP_{k}$ in Canada is lower than that in Inida. Because of diminishing in marginal return.


\section{Firm's Problem}

	\paragraph{Objective} Economic Profit.
	\[
		\pi (x) = p(x)x - C(x, w, r)
	\]

	\paragraph{Topic} \emph{Find the cheapest way to produce certain amount of output.}

	\subsection{Types of Costs}
		\paragraph{Opportunity Cost}
		\begin{itemize}
			\item Fixed Cost, \textbf{Definition} independent to output level.
			\item Sunk Cost, not \textbf{avoidable}
			\item Variable Cost
			\item Marginal Cost $\pd{C}{x}$
			\item Average Cost
		\end{itemize}


	\subsection{Short Run Cost Minimization}
		\emph{Find the cheapest way of producing any quantity $x$}.
		\paragraph{Note} \underbar{Production function} $f: \text{Input Bundle} \rightarrow \text{\textbf{maximum} Output}$
		\paragraph{Example} if $k$ is fixed in the short run,
		\[
			\min_{l} wl + r\overline{k}\ s.t.\ f(l, \overline{k}) = \overline{x}
		\]
		Or, if $l$ is fixed in the short run,
		\[
			\min_{k} w\overline{l} + rk\ s.t. f(\overline{l}, k) = \overline{x}
		\]

	\subsection{Long Run Cost Minimization}
		\emph{All inputs variable.}

		\paragraph{Isocost} Contains all of the input bundles that cost the same amount
		\[
			\{(l, k) \vert wl + rk = \overline{C}\}
		\]
		\paragraph{Optimization} cost minimization
		\[
			min_{k,j} wl + rk\ s.t.\ f(l,k) = \overline{x}
		\]
		\emph{Interior solution} $-\frac{w}{r} = -\frac{\pd{f}{k}}{\pd{f}{l}} = MRTS$

		Consider \textbf{conditional} input demands.
		\[
			l(\overline{x}, w, r), k(\overline{x}, w, r)
		\]

		\paragraph{Note} Non-Tangency Cases.

		\subsubsection{Cost Curve}
		\paragraph{Example} $C(x) = 2x + F$, then 
		\[
			MC(x) = \pd{C}{x} = 2
		\]
		\[
			AC(x) = \frac{C}{x} = 2 + \frac{F}{x}
		\]
		\paragraph{LR vs SR on Choice Graph}


\end{document}
















