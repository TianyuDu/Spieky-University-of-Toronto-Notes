\documentclass[11pt]{article}

\title{ECO326 Advanced Microeconomic Theory \\ \small A Course in Game Theory}

\author{Tianyu Du}
\date{\today}

\usepackage{spikey}
\usepackage{amsmath}
\usepackage{amssymb}
\usepackage{soul}
\usepackage{float}
\usepackage{graphicx}
\usepackage{hyperref}
\usepackage{xcolor}
\usepackage{chngcntr}

\counterwithin{equation}{section}

\usepackage[
    type={CC},
    modifier={by-nc},
    version={4.0},
]{doclicense}

\begin{document}
	\maketitle
	\doclicenseThis
	\texttt{Github Page} \url{https://github.com/TianyuDu/Spikey_UofT_Notes}\\
	\texttt{Note Page} \url{TianyuDu.com/notes}
	\paragraph{Readme} this note is based on the course content of \emph{ECO326 Advanced Microeconomics - Game Theory}, this note contains all materials covered during lectures and mentioned in the course syllabus. However, notations, statements of theorems and proofs are following the book \emph{A Course in Game Theory} by Osborne and Rubinstein, so they might be, to some extent, more mathematical than the required text for ECO326, \emph{An Introduction to Game Theory}.
	
	\tableofcontents
	\section{Lecture 1. Games and Dominant Strategies}
		\begin{assumption}[pg.4]
			Assume that each decision-maker is \emph{rational} in the sense that he is aware of his alternatives, forms expectation about any unknowns, has clear preferences, and chooses his action deliberately after some process of optimization.
		\end{assumption}
		
		\begin{definition}[pg.4]
			A model of \textbf{rational choice} consists
			\begin{itemize}
				\item A set $A$ of \emph{actions}.
				\item A set $C$ of \emph{consequences}.
				\item A \emph{consequence function} $g: A \to C$.
				\item A \emph{preference relation} $\succsim$ on $C$.
			\end{itemize}
		\end{definition}
		
		\begin{definition}[pg.7]
			A \textbf{preference relation} is a \ul{complete reflexive and transitive} binary relation.
		\end{definition}
\begin{definition}[11.1]
			A \textbf{strategic game} consists of
			\begin{itemize}
				\item a \ul{finite} set of \textbf{players} $N$.
				\item for each player $i \in N$, an \textbf{actions} $A_i \neq \emptyset$.
				\item for each player $i \in N$, a \textbf{preference relation} $\succsim_i$ defined on $A \equiv \prod_{i\in N}A_i$.
			\end{itemize}
			and can be written as a triple $\langle N, (A_i), (\succsim_i) \rangle$.
		\end{definition}
		
		\begin{definition}[pg.11]
			A strategic game $\langle N, (A_i), (\succsim_i) \rangle$ is \textbf{finite} if 
			\[
				|A_i| < \aleph_0\ \forall i \in N
			\]
		\end{definition}
		
	\section{Lecture 2. Iterated Elimination and Rationalizability}
		\subsection{Iterated Elimination of Strictly Dominated Strategies(Actions)}
			\begin{definition}[60.2]
				The set $X \subseteq A$ of outcomes of a \ul{finite} strategic game $\langle N, (A_i), (u_i) \rangle$ \textbf{survives iterated elimination of \ul{strictly} dominated actions} if $X = \times_{j \in N} X_j$ and there is a collection $((X_j^t)_{j \in N})_{t=0}^T$ of sets that satisfies the following conditions for each $j \in N$.
				\begin{itemize}
					\item $X_j^0 = A_j$ and $X_j^T = X_j$.
					\item $X_j^{t+1} \subseteq X_j^t$ for each $t = 0, \dots, T - 1$.
					\item For each $t = 0, \dots, T-1$ every action of player $j$ in $X_j^t \backslash X_j^{t+1}$ is \ul{strictly dominated} in the game $\langle N, (X_i^t), (u_i^t) \rangle$, where $u_i^t$ for each $i \in N$ is the function $u_i$ restricted to $\times_{j \in N} X_j^t$.
					\item No action in $X_t^T$ is strictly dominated in game $\langle N, (X_i^T), (u_i^T) \rangle$.
				\end{itemize}
			\end{definition}
			
			\begin{proposition}[61.2]
				If $X = \times_{j \in N}X_j$ survives iterated elimination of strictly dominated actions in a \ul{finite} strategic game $\langle N, (A_i), (u_i) \rangle$ then $X_j$ is the set of player $j$'s rationalizable actions for each $j \in N$.
			\end{proposition}
			
		\subsection{Rationalizability}
			\begin{definition}[59.1]
				An action of player $i$ in a strategic game is a \textbf{never best response} if it is not a best response to any \ul{belief} of player $i$.
			\end{definition}
			
			\begin{definition}[59.2]
				The action $a_i \in A_i$ of player $i$ in the strategic game $\langle N, (A_i), (u_i) \rangle$ is \textbf{strictly dominated} if there is a mixed strategy $\alpha_i$ of player $i$ such that 
				\[
					U_i(a_{-i}, \alpha_i) > u_i(a_{-i}, a_i)
				\]
				for all $a_{-i} \in A_{-i}$, where $U_i(a_{-i}, \alpha_i)$ is the payoff of player $i$ if he uses the mixed strategy $\alpha_i$ and the other players' vector of actions is $a_{-i}$.
			\end{definition}
		
		
	\section{Lecture 3. Nash Equilibrium}
		\begin{definition}[14.1]
			A \textbf{Nash equilibrium of a strategic game} $\langle N, (A_i), (\succsim_i) \rangle$ is a profile $a^* \in A$ of actions with property that for every player $i \in N$
			\[
				(a_i^*, a^*_{-i}) \succsim_i (a_i, a^*_{-i})\ \forall a_i \in A_i
			\]
		\end{definition}
		
		\begin{definition}[pg.15]
			The \textbf{best-response function} for a player $i$ is defined as
			\[
				B_i(a_{-i}) = \{a_i \in A_i : (a_i, a_{-i}) \succsim_i (a_i', a_{-i})\ \forall a_i' \in A_i \}
			\]
		\end{definition}
		
		\begin{remark}
			The best-response of $a_{-i}$ can be written as 
			\[
				B_i(a_{-i}) = \bigcap_{a_i' \in A_i} \{a_i \in A_i : (a_i, a_{-i}) \succsim_i (a_i', a_{-i})\}
			\]
			where each of them is the upper contour set of $a_i'$. \\
			Thus, if $\succsim_i$ is quasi-concave, then $B_i(a_{-i})$ is an intersection of convex sets and therefore itself convex.
		\end{remark}
		
		\begin{remark}[pg.15]
			So a Nash equilibrium is a profile $a^* \in A$ such that
			\[
				a^*_i \in B_i(a^*_{-i})\ \forall i \in N
			\]
		\end{remark}
		
		\begin{lemma}[pg.19]
			A strategic game $\langle N, (A_i), (\succsim_i) \rangle$ has a Nash equilibrium if equivalent to the following statement:\\
			Define set-valued function $B: A \to A$ by 
			\[
				B(a) = \prod_{i\in N} B_i (a_{-i})
			\]
			and there exists $a^* \in A$ such that $a^* \in B(a^*)$.
		\end{lemma}
	
		\begin{lemma}[20.1 Kakutani's fixed point theorem]
			Let $X$ be a \ul{compact convex subset} of $\R^n$ and let $f: X \to X$ be a set-valued function for which
			\begin{itemize}
				\item for all $x \in X$ the set $f(x)$ is non-empty and convex.
				\item the graph of $f$ is closed. \emph{(i.e. for all sequences $\{x_n\}$ and $\{y_n\}$ such that $y_n \in f(x_n)$ for all $n$, $x_n \to x$ and $y_n \to y$ then $y \in f(x)$)}
			\end{itemize}
			Then there exists $x^* \in X$ such that $x^* \in f(x^*)$.
		\end{lemma}
		
		\begin{definition}[pg.20]
			A preference relation $\succsim_i$ over $A$ is quasi-concave on $A_i$ if for every $a^* \in A$ the upper contour set over $a^*_i$, given other players' strategies
			\[
				\{a_i \in A_i: (a^*_{-i}, a_i) \succsim_i a^*\}
			\]
			is convex.
		\end{definition}
		
		\begin{proposition}[20.3]
			The strategic game $\langle N, (A_i), (\succsim_i) \rangle$ has a Nash equilibrium if for all $i \in N$,
			\begin{itemize}
				\item the set $A_i$ of actions of player $i$ is a nonempty \ul{compact convex} subset of a Euclidian space
			\end{itemize}
			and the preference relation $\succsim_i$ is
			\begin{itemize}
				\item continuous
				\item quasi-concave on $A_i$.
			\end{itemize}
		\end{proposition}
		
		\begin{proof}
			Let $B: A \to A$ be a correspondence defined as 
			\[
				B(a) := \prod_{i \in N} B_i(a_{-i})
			\]
			Note that for each $a \in A$ and for each $i \in N$, \\
			$B_i(a_{-i}) \neq \emptyset$ since preference $\succsim_i$ is continuous and $A_i$ is compact (EVT). \\
			Also $B_i(a_{-i})$ is convex since it's basically an intersection of  upper contour sets and each of those upper contour is convex since $\succsim_i$ is quasi-concave. \\
			So the Cartesian product of the finite collection of $B_i$ is non-empty and convex. \\
			Also the graph $B$ is closed since $\succsim_i$ is continuous. \\
			So there exists $a^* \in A$ such that $a^* \in B(a^*)$. \\
			So Nash equilibrium presents.
		\end{proof}

%	\section{Lecture 4. Nash Equilibrium: Examples}
%	\section{Lecture 5. Mixed Strategies}
%	\section{Lecture 6. Extensive Form Games and Subgame Perfection}
%		\subsection{Extensive Form Game}
%		\subsection{Subgame Perfection}
%	\section{Lecture 7. Extensive Form Games: Examples}
%	\section{Lecture 8. Repeated Games}
%	\section{Lecture 9. Game with Incomplete Information}
%	\section{Lecture 10. Game with Incomplete Information II}
%	\section{Lecture 11. Auctions}
\end{document}
















