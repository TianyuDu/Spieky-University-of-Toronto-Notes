\documentclass[11pt]{article}

\title{ECO326 Advanced Microeconomic Theory \\ \small A Course in Game Theory}

\author{Tianyu Du}
\date{\today}

\usepackage{spikey}
\usepackage{amsmath}
\usepackage{amssymb}
\usepackage{soul}
\usepackage{float}
\usepackage{graphicx}
\usepackage{hyperref}
\usepackage{xcolor}
\usepackage{chngcntr}
\usepackage{centernot}
\usepackage[shortlabels]{enumitem}
\usepackage[margin=1truein]{geometry}
\usepackage{sgame}

\counterwithin{equation}{section}
\counterwithin{figure}{section}

\newcommand{\floor}[1]{\lfloor #1 \rfloor}

\usepackage[
    type={CC},
    modifier={by-nc},
    version={4.0},
]{doclicense}

\begin{document}
	\maketitle
	\doclicenseThis
	\texttt{Github Page:} \url{https://github.com/TianyuDu/Spikey_UofT_Notes}\\
	\texttt{Note Page:} \url{TianyuDu.com/notes}
	\paragraph{Readme} this note is based on the course content of \emph{ECO326 Advanced Microeconomics - Game Theory}, this note contains all materials covered during lectures and mentioned in the course syllabus. However, notations, statements of theorems and proofs are following the book \emph{A Course in Game Theory} by Osborne and Rubinstein, so they might be, to some extent, more mathematical than the required text for ECO326, \emph{An Introduction to Game Theory}.
	
	\tableofcontents
	\newpage
	
	\section{Lecture 1. Jan. 7 2019\\Games and Dominant Strategies}
		\paragraph{Game Theory} Choice environment where individual choices impact others.
		
		\begin{example}
			\begin{figure}[h]
				\centering
				  \begin{tabular}{c|c|c}
				    & W & S\\
				    \hline
				    W & $(1-c, 1-c)$ & $(\red{1-c}, \red{1})$ \\
				    \hline
				    S & $(\red{1}, \red{1-c})$ & $(0, 0)$
				  \end{tabular}
				  \caption{Payoff Matrix for Example 1}
			\end{figure}
		\end{example}
		Suppose $c \in (0, 1)$. In this game,
		\begin{enumerate}[i]
			\item $N = \{i, j\}$,
			\item $A_i = A_j = \{W, S\}$,
		\end{enumerate}
		
		\begin{definition}[pg.7]
			A \textbf{preference relation} is a \ul{complete reflexive and transitive} binary relation.
		\end{definition}
		
		\begin{definition}[11.1, lec.1]
			A \textbf{(strategic) game} consists of
			\begin{enumerate}[i]
				\item a \ul{finite} set of \textbf{players} $N$, with $|N| \geq 2$.
				\item for each player $i \in N$, an \textbf{actions} $A_i \neq \emptyset$.
				\item for each player $i \in N$, a \textbf{preference relation} $\succsim_i$ defined on $A \equiv \prod_{i\in N}A_i$.
			\end{enumerate}
			and can be written as a triple $\langle N, (A_i), (\succsim_i) \rangle$.
		\end{definition}

		\begin{definition}
			An \textbf{action profile} is a $n$-tuple of actions $a_i \in A_i$ for each player $i \in N$ and denoted as 
				\[
					(a_i)_{i \in N} \tx{ or } (a_i)
				\]
			The \textbf{action profile space} is defined as 
				\[
					A \equiv \prod_{i \in N} A_i
				\]
		\end{definition}
		
		\begin{definition}[Equivalent definition of game]
			For each player $i \in N$ we can define a \ul{utility function}, $u_i: A \to \R$ such that
			\begin{equation}
				\forall (a_i), (a_i)' \in A,\ u_i((a_i)) \geq u_i((a_i)') \iff (a_i) \succsim_i (a_i)'
			\end{equation}
			So the game can be defined as a triple $\langle N, (A_i), (u_i) \rangle$.
		\end{definition}
		
		\begin{definition}
			Action $a_i \in A_i$ is \textbf{strictly dominated} by action $\tilde{a}_i \in A_i$ if
			\[
				\forall a_{-i} \in A_{-i},\ u_i(a_i, a_{-i}) < u_i(\tilde{a}_i, a_{-i})
			\]
			And $a_i$ is \textbf{weakly dominated} by $\tilde{a}_i$ if
			\[
				\forall a_{-i} \in A_{-i},\ u_i(a_i, a_{-i}) \leq u_i(\tilde{a}_i, a_{-i})
			\]
			\ul{and}
			\[
				\exists a_{-i} \in A_{-i},\ u_i(a_i, a_{-i}) < u_i(\tilde{a}_i, a_{-i})
			\]
		\end{definition}
		
		\begin{corollary}[Consequence of RCT]
			It is irrational to play strictly dominated actions. So rational choice theory suggests a player would never play \red{strictly} dominated strategies.
		\end{corollary}
		
		\begin{definition}
			Action $a_i \in A_i$ is \textbf{strictly dominant} if it strictly dominates \red{all} other actions.
		\end{definition}
		
		\begin{definition}
			Action $a_i \in A_i$ is \textbf{weakly dominant} if it weakly dominates \red{all} other actions.
		\end{definition}
		
		\begin{definition}
			Action $a_i \in A_i$ is \textbf{weakly/strictly dominated} if \red{there exists} another strategy weakly/strictly dominates $a_i$.
		\end{definition}
		
		\begin{example}[Prisoner Dilemma]
			\begin{figure}[h]
				\centering
				\begin{tabular}{c|c|c}
					 & S & C \\
					\hline
					S & (-1, -1) & (-10, \red{0})\\
					\hline
					C & (\red{0}, -10) & (\red{-5}, \red{-5})
				\end{tabular}
				\caption{Payoff matrix for example 2}
			\end{figure}
			Note that S is strictly dominated by C. Therefore C is strictly dominant for both players.
		\end{example}
		
		\begin{example}
			\begin{figure}[h]
				\centering
				\begin{tabular}{c|c|c|c}
					 & L & C & R\\
					\hline
					U & (2, \red{2}) & (\red{5}, 0) & (\red{3}, 0)\\
					\hline
					M & (2, \red{7}) & (2, 5) & (2, 6)\\
					\hline 
					D & (\red{5}, \red{3}) & (4, 2) & (\red{3}, 1)
				\end{tabular}
				\caption{Payoff matrix for example 2}
			\end{figure}
			So in this game, for player 2, L is strictly dominant. \\For player 1, M is strictly dominated by D. And M is weakly dominated by U.
		\end{example}
		
		\begin{example}
			There are three candidates, $\{A, B, C\}$. And there are 50 players (voters, note that $\emptyset \notin A_i$ since they must vote).
			And 
			\[
				\forall i \in N,\ A_i = \{A, B, C\}
			\]
			Each individual has strictly preference over $A, B, C$. If tie is encountered, randomization would be taken.
			\begin{enumerate}[i]
				\item $A \succ B \succ C$,
				\item $A \succ AC_{tie} \succ C$
			\end{enumerate}
			\paragraph{Claim 1:}There are no weakly or strictly dominant actions. 
			\begin{proof}
				Let $a_i \in \{V_A, V_B, V_C\}$ denote the action taken by player $i \in N$, \\
				Note that weak dominance is a necessary condition for strict dominance, \\
				So above claim is reduced to \emph{there are no weakly dominant actions}. \\
				The reduced claim is equivalent to the following statement, \\
				
				\begin{gather*}
					\forall a_i \in A_i,\ \exists \tilde{a}_i \in A_i\ s.t.\ a_i \neq \tilde{a}_i\\
					s.t.\ \exists a_{-i} \in A_{-i}\ s.t.\ u_i(a_i, a_{-i}) > u_i(\tilde{a}_i, a_{-i}) \lor \forall a_{-i} \in A_{-i},\ u_i(a_i, a_{-i}) = u_i(\tilde{a}_i, a_{-i})
				\end{gather*}
				Let $n_{-i}^j$ denote the number of voters other than $i$ voting for candidate $j$. Clearly each $a_{-i} \in A_{-i}$ would induce an outcome as a triple $(n_{-i}^A, n_{-i}^B, n_{-i}^C)$.\\
				Consider action $V_A$, and $a_{-i}$ induces 
				\[
					(n_{-i}^A, n_{-i}^B, n_{-i}^C) = (1, 24, 24)
				\]
				then 
				\[
					(V_B, a_{-i}) \succ_i (V_A, a_{-i})
				\]
				So $V_A$ failed to be a dominant strategy of any kind. \\
				Similarly, consider action $V_B$, if $a_{-i}$ induces
				\[
					(n_{-i}^A, n_{-i}^B, n_{-i}^C) = (24, 1, 24)
				\]
				then 
				\[
					(V_A, a_{-i}) \succsim_i (V_B, a_{-i})
				\]
				So $V_B$ failed to be a dominant strategy. \\
				Similarly, consider action $V_C$, if $a_{-i}$ induces
				\[
					(n_{-i}^A, n_{-i}^B, n_{-i}^C) = (24, 24, 1)
				\]
				then 
				\[
					(V_A, a_{-i}) \succsim_i (V_C, a_{-i})
				\]
				So $V_B$ failed to be a dominant strategy.
			\end{proof}
			\paragraph{Claim 2:}Only voting for your least preferred candidate is weakly dominated.
			\begin{proof}
				We are going to show there exists a strategy (voting for $B$) weakly dominates voting for $C$.
				\begin{figure}[H]
					\centering
					\begin{tabular}{c|c|c}
						Vote A & Cases & Vote C \\
						\hline
						A & $n_{-i}^A > n_{-i}^B, n_{-i}^C$ & A, AC \\
						B & $n_{-i}^B > n_{-i}^A, n_{-i}^C$ & B, BC \\
						C, BC & $n_{-i}^C > n_{-i}^A, n_{-i}^B$ & C\\
						B & $n_{-i}^A = n_{-i}^B > n_{-i}^C$ & AB\\
						A & $n_{-i}^A = n_{-i}^C > n_{-i}^B$ & C\\
						BC & $n_{-i}^C = n_{-i}^B > n_{-i}^A$ & C
					\end{tabular}
					\caption{Voting for A versus Voting for C}
				\end{figure}
			\end{proof}
		\end{example}
		
		\begin{definition}
			A strategic game $\langle N, (A_i), (\succsim_i) \rangle$ is \textbf{finite} if for every $i \in N$, $A_i$ is finite.
		\end{definition}
	
	\section{Lecture 2. Jan. 14 2019\\Iterated Elimination and Rationalizability}
		\begin{example}[Bubble Game]
			Consider a strategic game
			\begin{equation}
				\langle N, (A_i), (u_i) \rangle
			\end{equation}
			where 
			\begin{equation}
				A_i = \{0,\dots,100\},\ \forall i
			\end{equation}
			and 
			\begin{gather}
				u_i(a_i, a_{-i}) = a_i - penalty_i(a_i, a_{-i}) \\
				penalty_i = \begin{cases}
					0 \tx{ if } a_i < \max_{j \neq i} a_j - 1 \\
					10(a_i - \max_{j \neq i} a_j + 1) \tx{ if } a_i \geq \max_{j \neq i} - 1
				\end{cases}
			\end{gather}
		\end{example}	
		
		\subsection{Iterated Elimination of Strictly Dominated Strategies(Actions)}
			\begin{definition}[IESD]
				Given the initial game,
				\[
					G_0 = \langle N, (A^0_i), (u_i) \rangle
				\] 
				At stage $k \in \N$, 
				\[
					G_k = \langle N, (A^k_i), (u_i) \rangle
				\]
				In stage $k$, for all $i \in N$, find the set of \ul{strictly} dominated actions, $D_i^k \subsetneq A_i^k$.
				\begin{enumerate}[i)]
					\item If $\forall i \in N\ s.t.\ D_i^k = \emptyset$, conclude the profile
					\[
						(A_i^k)
					\]
					to be the \textbf{set of action profiles survive from IESD}.
					\item If $\exists i \in N\ s.t.\ D_i^k \neq \emptyset$, define
					\[
						\forall i \in N,\ A^{k+1}_i \leftarrow A^k_i \backslash D_i^k
					\]
				\end{enumerate}
			\end{definition}
			
			\begin{example}
				Action profile $(M, R)$ survives the IESD.
				\begin{proof}
					\begin{gather*}
						k=0,\ A_1^0 = \{U, M, D\},\ 
						A_2^0 = \{L, R\} \\
						k=1,\ A_1^1 = \{U, M\},\ 
						A_2^1 = \{L, R\} \\
						k=2,\ A_1^2 = \{U, M\},\ 
						A_2^2 = \{R\} \\
						k=3,\ A_1^3 = \{M\},\ 
						A_2^3 = \{R\} \\
					\end{gather*}
				\end{proof}
			\end{example}
			\begin{figure*}[h]
				\centering
				\begin{tabular}{l|cc}
				  & L & R \\
				  \hline
				  U & 4,0 & 2,2 \\
				  M & 1, 2& \red{5,3} \\
				  D & 0,5 & 1,4 \\
				\end{tabular}
				\caption{Game for Example 2.1}
			\end{figure*}
			
			\begin{example}[Hotelling Model of Politics]
				Players maximize their votes by choosing where to stand along a natural number line.
				\begin{itemize}
					\item Player $N = \{1,2\}$
					\item Action set $A_i = \{1, \dots, M\}$, with $2 \centernot | M$ and $M > 3$.
					\item Payoff
					\begin{equation}
						u_i(a_i, a_{-i}) = \begin{cases}
							a_i + \frac{1}{2}(a_{-i} - a_{i} - 1) \tx{ if } a_i < a_{-i} \\
							\frac{M}{2} \tx{ if } a_i = a_{-i} \\
							M - [a_{-i} + \frac{1}{2}(a_{i} - a_{-i} - 1)] \tx{ if } a_i > a_{-i}
						\end{cases}
					\end{equation}
				\end{itemize}
				\paragraph{Claim i.} $a_i=1$ is strictly dominated by $a_i=2$.
				\begin{proof}
					\begin{gather}
						u_i(a_i=1, a_{-i}) =
						\begin{cases}
							\frac{M}{2} \tx{ if } a_{-i} = 1 \\
							\frac{a_{-i}}{2} \tx{ if } a_{-i} > 1
						\end{cases} \\
						u_i(a_i=2, a_{-i}) = 
						\begin{cases}
							M - 1 \tx{ if } a_{-i} = 1 \\
							\frac{M}{2} \tx{ if } a_{-i} = 2 \\
							\frac{a_{-i}}{2} + \frac{1}{2} \tx{ if } a_{-i} > 2
						\end{cases}
					\end{gather}
				\end{proof}
				\paragraph{Claim ii.} $\floor{\frac{n}{2}} + 1$ is the only action survives.
				\begin{proof}
					Similarly, we can eliminate all edge-values iteratively.
				\end{proof}
			\end{example}
			
			\begin{definition}
				For each $i \in N$, the \textbf{best-response function} of this player is a correspondence $Br_i: A_{-i} \rightrightarrows A_i$ defined as
				\begin{equation}
					Br_i(a_{-i}) \equiv \{a_i \in A_i : u_i(a_i, a_{-i}) \geq u_i(a_i', a_{-i})\ \forall a_i' \in A_i \}
				\end{equation}
			\end{definition}
			
			\begin{definition}
				A \textbf{belief} of player $i$ (about the actions of the other players) is a \ul{probability measure}, $\alpha_i$, on $A_{-i}=\prod_{j \in N \backslash \{i\}} A_j$. \\
				$\alpha_i$ is a mapping such that
				\begin{itemize}
					\item $\alpha_i: A_{-i} \to [0,1]$.
					\item $\alpha_i(A_{-i}) = 1$.
					\item For all countable piece-wise \ul{disjoint} collection 
					\[
						\{E_j\}_{j\in \mc{J}} \in \mc{P}(A_{-i})
					\]
					$\alpha_i$ satisfies the \emph{countable additivity property}:
					\[
						\alpha_i(\bigcup_{i \in I} E_i) = \sum_{i \in I}\alpha_i(E_i)
					\]
				\end{itemize}
			\end{definition}
				
			\begin{definition}
				$a_i$ is a \ul{\textbf{best response} to the belief $\alpha_i$} if
				\begin{equation}
					\forall a_i' \in A_i,\ \sum_{a_{-i}} u_i(a_i, a_{-i}) \alpha_i(a_{-i}) \geq \sum_{a_{-i}} u_i(a_i', a_{-i}) \alpha_i(a_{-i})
				\end{equation}
				or, more generally,
				\begin{equation}
					\forall a'_i \in A_i, \expect{u_i(a_i, a_{-i})|\alpha_i} \geq \expect{u_i(a_i', a_{-i})|\alpha_i}
				\end{equation}
			\end{definition}
			
			\begin{definition}
				$a_i \in A_i$ is a \textbf{never best response} if it is not a best response given any belief $\alpha_i$.
			\end{definition}
			
			\begin{corollary}
				\emph{Iterative Elimination of Never Best Response}: same procedures but $D_i^k$ is the set of never best responses for player $i$ at game $G^k$.
			\end{corollary}
			
			\begin{example}
				For player 1, $D$ is not strictly dominated, but it is a never best response.
				\begin{proof}
					Let $\alpha$ be a probability measure on $\{L, R\}$ such that $\alpha(L) = p \in [0, 1]$.\\
					\begin{gather}
						\expect{u_1|U,\alpha} = 10p \\
						\expect{u_1|M,\alpha} = 10 - 10p \\
						\expect{u_1|D,\alpha} = 1
					\end{gather}
					\textbf{Case i}
					\begin{equation}
						p \geq 0.5 \implies \expect{u_1|U,\alpha} \geq 5
					\end{equation}
					\textbf{Case ii}
					\begin{equation}
						p < 0.5 \implies \expect{u_1|M,\alpha} > 5
					\end{equation}
					Therefore, for any belief $\alpha$, $D$ cannot be a best response. So $D$ is a never best response.
				\end{proof}
			\end{example}
			\begin{figure}[h]
				\centering
				\begin{tabular}{l|cc}
				  & L & R\\
				  \hline
				  U & 10,0 & 0,0 \\
				  M & 0,0 & 10,0\\
				  D & 1,0 & 1,0
				\end{tabular}
			\end{figure}
			
			\begin{definition}
				An action $a_i \in A_i$ is \textbf{rationalizable} if it survives \emph{iterative elimination of never best responses}.
			\end{definition}
			
			\begin{lemma}[i385.3]
				In a \ul{two player} game, $a_i$ is strictly dominated 
				\ul{if and only if} it is a never best response.
			\end{lemma}
			
			\begin{assumption}[Common knowledge rationality]
				We assume our players of game all acknowledge that other players are playing the game in rational ways.
			\end{assumption}
		
	\section{Lecture 3. Jan. 21 2019}
		\begin{definition}
			A \textbf{pure strategy Nash equilibrium} is a \ul{strategy profile} $(a_i)$ such that 
			\begin{equation}
				\forall i \in N,\ a_i \in Br_i(a_{-i})
			\end{equation}
		\end{definition}
			
		\begin{remark}
			That's, a NE is a situation that if player $i$ knows what other players do, the action given by the NE profile is still a best response.
		\end{remark}
		
		\begin{remark}[Interpretations]
			A Nash equilibrium is 
			\begin{enumerate}[i)]
				\item An action profile induces a \ul{stable outcome},
				\item A \ul{creditable agreement}, such that no player has incentive to break the agreement.
			\end{enumerate}
		\end{remark}
		
		\begin{example}[Cournot Duopoly]
			Consider a game with
			\begin{enumerate}[i)]
				\item Player $N = \{1,2\}$
				\item Action set $A_i = [0, \infty)\ \forall i \in N$
			\end{enumerate}
			And revenue $R_i$ defined by 
			\begin{equation}
				R_i = p_i q_i
			\end{equation}
			where price is linear in quantity supplied,
			\begin{equation}
				p_i = \alpha - (q_i + q_{-i}),\ \alpha \in \R
			\end{equation}
			and firms face fixed cost $c \in \R$, with the assumption that $\alpha > c$. So the cost function is given by
			\begin{equation}
				C_i(q_i) = c q_i 
			\end{equation}
			The profit function is given by
			\begin{gather}
				\Pi_i(q_i, q_{-i}) = (\alpha - (q_i + q_{-i}) - c) q_i \\
				= (\alpha-c-q_{-i})q_i - q_i^2
			\end{gather}
			Given $q_{-i} \in A_{-i}$, the best response is given by
			\begin{gather}
				Br_i(q_{-i}) = \tx{argmax}_{q_i \in [0, \infty)} \Pi_i(q_i, q_{-i}) \\
				= \max\{0, \frac{\alpha - c - q_{-i}}{2}\}\ \forall i \in N
			\end{gather}
			Considering the case that both players are producing positive quantities, we can solve $q_i^*$ by
			\begin{gather}
				Br_i \circ Br_{-i}(q_i) = \frac{\alpha - c - \frac{\alpha - c - q_i}{2}}{2} = q_i \\
				\implies 2 q_i - \frac{q_i}{2} = \frac{\alpha - c}{2} \\
				\implies q_i^* = \frac{\alpha - c}{3}
			\end{gather}
			
			\begin{remark}
				If fixed cost presents, even if the game is symmetric, the Nash equilibrium could be asymmetric. (e.g. one firm is out of market and the other firm produces the monopoly amount)
			\end{remark}
		\end{example}
		
		\begin{remark}
			Nash equilibrium induces an \emph{individual level optimality} instead of the common wealth optimality.
		\end{remark}
		
		\begin{example}[Continue Cournot Duopoly]
			Note that in the Cournot Duopoly game, for each $i \in N$, the Nash equilibrium profit is
			\begin{gather}
				\Pi_{NE}^* = (\alpha - c - \frac{2(\alpha - c)}{3}) \frac{\alpha - c}{3} \\
				= \frac{(\alpha - c)^2}{9}
			\end{gather}
			So the total profit for the two firms is $\frac{2(\alpha - c)^2}{9}$. \\
			Now considering if the two firms form a Cartel, the \emph{aggregate} quantity produced is
			\begin{gather}
				Q^* = \tx{argmax}_{Q \in \R_{\geq 0}} (\alpha - c - Q) Q \\
				= \frac{\alpha - c}{2} \\
				\implies \Pi_{Cartel}^* = (\alpha - c - \frac{\alpha - c}{2}) \frac{\alpha - c}{2} \\
				= \frac{(\alpha - c)^2}{4} > \frac{2(\alpha - c)^2}{9}
			\end{gather}
			The fact $\Pi_{Cartel}^* > 2 \times \Pi_{NE}^*$ suggests the Nash equilibrium action profile did not induce the optimal common wealth outcome. However, the Cartel action profile is not a stable outcome since every player has incentive to increase their production level.
		\end{example}
		
		\begin{example}[Prisoner's Dilemma]
			The Nash equilibrium(Confess, Confess) is \ul{not} the best outcome for the two players as a group. The optimal action profile \ul{for a group} is (Silent, Silent).
		\end{example}
		
		\begin{proposition}
			No strategy that is eliminated during iterated elimination of \emph{never best response} can be played in a Nash equilibrium.
		\end{proposition}
	
	\section{Lecture 4. Jan. 28. 2019}
	\begin{example}[From last lecture]
		Consider the payoff matrix
		\begin{figure*}[h]
			\centering
			\begin{tabular}{c|c|c}
				 & A & B \\
				\hline
				A & (\red{1},\red{1}) & (\red{0},0) \\
				B & (0,\red{0}) & (\red{0},\red{0}) \\
			\end{tabular}
		\end{figure*}
		Both $(A,A)$ and $(B,B)$ are Nash equilibria. But in the former NE, for each $i \in N$,
		\begin{equation}
			Br_i(a_{-i}=A) = \{A\}
		\end{equation}
		which is a singleton. \\
		In the second NE, for each $i \in N$,
		\begin{equation}
			Br_i(a_{-i}=B) = \{A, B\}
		\end{equation}
		We call Nash equilibria of the first type \emph{strict Nash equilibria} and the later one \emph{weak Nash equilibria}. More formal definitions of these two types of Nash equilibria are given below.
	\end{example}
	
	\begin{definition}
		A \textbf{strict Nash equilibrium} is an action profile $(a_i)$ such that
		\[
			\forall i \in N,\ |Br_i(a_{-i})| = 1
		\]
	\end{definition}
	
	\begin{definition}
		A \textbf{weak Nash equilibrium} is a Nash equilibrium that is not strict. That's, a weak Nash equilibrium is an action profile $(a_i)$ such that
		\[
			\exists i \in N,\ |Br_i(a_{-i})| > 1
		\]
	\end{definition}
	
	\begin{example}[Cournot with $n$ firms]
		Consider the game 
		\begin{equation}
			\langle N, (\R_{\geq 0}), (\pi_i) \rangle
		\end{equation}
		Where $|N| = n$ and each firm picks a quantity $q_i \in A_i \equiv \R_{\geq 0}$. \\
		Define 
		\begin{equation}
			Q \equiv \sum_j q_j
		\end{equation}
		For each $i \in N$, define 
		\begin{equation}
			Q_{-i} \equiv \sum_{j\neq i} q_j
		\end{equation}
		And the market has \emph{linear demand curve}
		\begin{equation}
			P(\{q_i\}_{i\in N}) = \alpha - \sum_{j}q_j = \alpha - Q
		\end{equation}
		And firms face fixed cost
		\begin{equation}
			\forall i \in N,\ C(q_i) = c q_i \tx{ where } 0 < c < \alpha
		\end{equation}
		The profit function is
		\begin{equation}
			\forall i \in N,\ \pi_i(q_i, Q_{-i}) = (\alpha - c - Q_{-i})q_i - q_i^2
		\end{equation}
		\paragraph{Question}What are the Nash equilibria in this environment? \\
		For each firm $i \in N$, the best response correspondence is
		\begin{gather}
			Br_i(Q_{-i}) = \max\{0, \argmax_{q_i \in \nnr} \pi_{i}(q_i, Q_{-i}) \} \\
			= \max\{0, \frac{\alpha - c - Q_{-i}}{2}\}
		\end{gather}
		\paragraph{Assume}We have a symmetric Nash equilibrium, 
		\begin{gather}
			\forall i \in N,\ q^*_i = q^* = \frac{Q}{n} \\
			\implies q^* = Br_i(\frac{n-1}{n}Q) \\
			\implies 2 q^* = \alpha - c - (n-1) q^* \\
			\implies q^* = \frac{\alpha - c}{n + 1}
		\end{gather}
		\paragraph{Check}Then check the validity of symmetric Nash equilibrium by asserting for every player, if all other players are playing the action suggested by the symmetric Nash equilibrium, then this player should also play it. That's
		\begin{gather}
			q^* = Br_i(\{q_j\}_{j\neq i}) \\
			= \frac{\alpha - c}{2} - \frac{1}{2} \frac{n-1}{n+1} (\alpha - c) \\
			= \frac{1}{2}((\alpha - c) - \frac{n-1}{n+1}(\alpha - c)) \\
			= \frac{1}{2} \frac{2}{n+1}(\alpha - c) \\
			= \frac{\alpha - c}{n + 1} 
			= q^*
		\end{gather}
		\paragraph{Uniqueness} We are going to show the symmetric Nash equilibrium is the only possible equilibrium action profile. \\
		\begin{proof}
			Suppose there exists some non-symmetric Nash equilibrium. \\
			For concreteness, assuming there exists $\epsilon > 0$ such that
			\begin{equation}
				\exists i, j \in N,\ q_i = q_j + \epsilon
			\end{equation}
			Define $Q_{-ij} \equiv Q - q_i - q_j$.\\
			For firm $i$,
			\begin{gather}
				q_i = Br_i(Q_{-i}) = \frac{1}{2} (\alpha - c - Q_{-i}) \\
				= \frac{1}{2}(\alpha - c - Q_{-ij} - q_{j}) \\
				= \frac{1}{2}(\alpha - c - Q_{-ij} - q_i + \epsilon) \\
				\implies 3 q_i = \alpha - c - Q_{-ij} + \epsilon \\
				\implies 3 q_i = \alpha - c - Q_{-j} + q_i + \epsilon \\
				\implies 2 q_i - \epsilon = \alpha - c - Q_{-j} \\
				\implies  q_i - \frac{\epsilon}{2} = Br(Q_{-j}) = q_j \\
				= q_i - \epsilon \\
				\implies \epsilon = 2 \epsilon
			\end{gather}
			which contradicts the assumption that $\epsilon > 0$. \\
			Therefore we conclude that the symmetric Nash equilibrium is the only Nash equilibrium in this environment.
		\end{proof}
	\end{example}
	
	\begin{example}[Cournot duopoly with fixed cost]
		Consider the game 
		\begin{equation}
			\mc{G} = \langle N = (1,2), (A_i = \nnr), (\pi_i) \rangle
		\end{equation}
		Where the cost function is defined as
		\begin{equation}
			C_i(q_i) = \begin{cases}
				c q_i + f &\tx{ if } q_i > 0 \\
				0 &\tx{ if } q_i = 0
			\end{cases}
		\end{equation}
		where $\alpha > c > 0$. \\
		For each firm $i \in N$, the best response function, conditioned on $q_i > 0$, is
		\begin{equation}
			q_i = Br_i(q_{-i}) = \frac{\alpha - c - q_{-i}}{2}
		\end{equation}
		We have to \textbf{check the profit} to assert the profit is non-negative while firm $i$ is producing above quantity. Because, otherwise, this firm could always derivate to $q_i = 0$ to avoid loss (earning zero profit).
		\begin{gather}
			\pi_i(Br_{q_{-i}}, q_{-i}) = (\alpha - q_{-i} - \frac{\alpha - c - q_{-i}}{2} - c)\frac{\alpha - c - q_{-i}}{2} - f \\
			= \frac{(\alpha - c - q_{-i})^2}{4} - f \\
			\pi_i(q_{-i}) \geq 0 \iff \alpha - c - q_{-i} \geq 2\sqrt{f} \\
			\iff q_{-i} \leq \alpha - c - 2\sqrt{f}
		\end{gather}
		Therefore we can modify our \textbf{best response correspondence} to
		\begin{gather}
			Br_i(q_{-i}) = 
			\begin{cases}
				\frac{\alpha - c - q_{-i}}{2} &\tx{ if }  q_{-i} \leq \alpha - c - 2\sqrt{f} \\
				0 &\tx{ if } q_{-i} \geq \alpha - c - 2\sqrt{f}
			\end{cases}
		\end{gather}
		\textbf{Monopoly}: the monopoly amount, conditioned on the firm decides to produce, is $\frac{\alpha - c}{2}$. In the monopoly case, suppose the NE action profile is (the opposite case can be shown by symmetry)
		\begin{equation}
			(\frac{\alpha - c}{2}, 0)
		\end{equation}
		We have to assert both 
		\begin{gather}
			\begin{cases}
				\frac{\alpha - c}{2} \in Br_1(0) \\
				0 \in Br_2(\frac{\alpha - c}{2})
			\end{cases} \\
			\implies
			\begin{cases}
				0 \leq \alpha - c - 2\sqrt{f} &\tx{ for firm 1 to produce positive amount.}\\
				\frac{\alpha - c}{2} \geq \alpha - c - 2\sqrt{f} &\tx{ for firm 2 to produce zero.}
			\end{cases} \\
			\implies 0 \leq \alpha - c - 2\sqrt{f} \leq \frac{\alpha - c}{2} \\
			\implies -(\alpha - c) \leq - 2\sqrt{f} \leq -\frac{\alpha - c}{2} \\
			\implies \red{\sqrt{f} \in [\frac{\alpha-c}{4}, \frac{\alpha - c}{2}]}
		\end{gather}
		\textbf{Positive symmetric equilibrium}: we've proven, in the general case, it's impossible for both firms to produce positive but different amounts. Therefore we have to assert 
		\begin{gather}
			\frac{\alpha - c}{3} \in Br_i(\frac{\alpha - c}{3})\ \forall i \in N \\
			\implies \frac{\alpha - c}{3} \leq \alpha - c - 2\sqrt{f} \\
			\implies \sqrt{f} \leq \frac{1}{3} (\alpha - c)
		\end{gather}
		\textbf{Zero symmetric equilibrium}: we have to assert
		\begin{gather}
			0 \in Br_i(0)\ \forall i \in N\\
			\implies 0 \geq \alpha - c - 2\sqrt{f} \\
			\implies \sqrt{f} \geq \frac{\alpha - c}{2}
		\end{gather}
	\end{example}
	
	\begin{example}[Discrete price Bertrand duopoly]
		Consider the following game
		\begin{gather}
			\mc{G} = \langle N=\{1,2\}, (A_i = \{k \epsilon: k \in \Z_{\geq 0}\}), (\pi_i) \rangle
		\end{gather}
		The profit function can be derived as
		\begin{gather}
			\pi_i(p_i, p_{-i}) = \begin{cases}
				(\alpha - p_i)(p_i - c)&\tx{ if } p_i < p_{-i} \\
				\frac{\alpha - p_i}{2}(p_i - c) &\tx{ if } p_i = p_{-i} \\
				0 &\tx{ if } p_i > p_{-i}
			\end{cases}
		\end{gather}
		\textbf{Claim} the only Nash equilibria are $(c, c)$ and $(c+\epsilon, c+\epsilon)$. \\
		\textbf{Justify} $(c, c)$, consider any firm $i \in N$, currently $\pi_i = 0$ 
		\begin{gather}
			\uparrow p_i \implies \pi_i \leftarrow \pi_i' = 0 \tx{ \xmark} \\
			\downarrow p_i \implies \pi_i \leftarrow \pi_i' < 0 \tx{ \xmark}
		\end{gather}
		So no firm has incentive to derivate from this action profile, so $(c, c)$ is a Nash equilibrium by definition. \\
		Consider the action profile $(c+\epsilon, c+\epsilon)$, both firms are earning a positive profit $\pi_i > 0$. For any firm $i \in N$,
		\begin{gather}
			\uparrow p_i \implies \pi_i \leftarrow \pi_i' = 0 \tx{ \xmark} \\
			\downarrow p_i \implies \pi_i \leftarrow \pi_i' = 0 \tx{ \xmark}
		\end{gather}
		So $(c+\epsilon, c+\epsilon)$ is a Nash equilibrium by definition. \\
		\textbf{No other Nash equilibrium} \\
		\textbf{Claim}: $(p_1, p_2)$ cannot be a Nash equilibrium if any $p_i < c$. \\
		Obviously, set $p_i < c$ would induce negative profit and firm $i$ do better off by setting $p_i \leftarrow c$. \\
		\textbf{Claim}: the symmetric profit $(p, p)$ with $p > c + \epsilon$ cannot be a Nash equilibrium. For both firms, the current profit is
		\begin{equation}
			\pi_1 = \pi_2 = \frac{1}{2}(\alpha - c - k\epsilon) k \epsilon
		\end{equation}
		And reducing price by $\epsilon$ leads to a profit of
		\begin{equation}
			\pi_i ' = (\alpha - c - k \epsilon + \epsilon) (k-1)\epsilon > \pi_i
		\end{equation}
		so such action profile cannot be Nash equilibrium. \\
		\textbf{Claim} $(p_1, p_2)$ with $p_1 \neq p_2$ and $p_1, p_2 > c$ cannot be a Nash equilibrium. The firm charges higher price can always reduce it's price to the price charged by the other firm and gain a positive profit. \\
		\textbf{Claim} $(c, p_2)$ with $p_2 \geq c + \epsilon$ cannot be a Nash equilibrium. The firm charging $p = c $ can always increase its price to $c + \epsilon$ to earn positive profit. \\
		Therefore there's no Nash equilibrium other than $(c,c)$ and $(c+\epsilon, c+\epsilon)$.
	\end{example}
	
	\begin{example}[Bertrand duopoly with differentiated products]
		Consumers are \ul{uniformly distributed} on a preference line $[0,1]$.\\
		For a firm $i \in N$, let $x_i \in [0,1]$ measure consumer's preference towards firm $i$'s products. \\
		Define 
		\begin{equation}
			x_{-i} \equiv 1 - x_i \sim Unif(0,1)
		\end{equation}
		Consumer buy product $i$ if 
		\begin{gather}
			x_i - p_i \geq x_{-i} - p_{-i}
		\end{gather}
		and purchases product 1 if
		\begin{gather}
			x_i - p_i \leq x_{-i} - p_{-i}
		\end{gather}
		Then solve the boundary $x^* \in [0, 1]$ such that consumers at $x^*$ are indifferent between  two products.
		\begin{gather}
			x + p_i = (1 - x) + p_{-i} \\
			\implies x^* = \frac{1 - p_i + p_{-i}}{2}
		\end{gather}
		So any consumer with $x_i > x^*$ would choose firm $i$'s product, also because consumers are uniformly distributed on $x_i \in [0,1]$. So the portion of consumers buying firm $i$'s product is
		\begin{equation}
			1 - x_i = \frac{1 + p_i - p_{-i}}{2}
		\end{equation}
		And, clearly, 
		The demand function for firm $i$ is
		\begin{gather}
			D_i(q_i, q_{-i}) = \begin{cases}
				0 &\tx{ if } p_i \geq p_{-i} + 1 \\
				\frac{1 + p_i - p_{-i}}{2} &\tx{ if } p_i \in [p_{-i}-1, p_{-i}+1]\\
				1 &\tx{ if } p_i \leq p_{-i} - 1\\
			\end{cases}
		\end{gather}
		Consider the case where $|p_i - p_{-i}| \leq 1$, 
		\begin{gather}
			\pi_i = D_i(p_i) (p_i - c) = \frac{1 + p_i - p_{-i}}{2} (p_i - c)
		\end{gather}
		Take the first order condition
		\begin{gather}
			\pd{\pi_i}{p_i} = 0 \\
			\implies p_i = \frac{p_{-i} + c - 1}{2}
		\end{gather}
	\end{example}
	
	\section{Lecture 5 Feb. 7 2019}
		\subsection{Lecture Content}
		\begin{example}Matching pennies: an example of game without pure strategy Nash equilibrium.
			\begin{figure}[h]
				\centering
				\begin{game}{2}{2}
					& $H$ & $T$ \\
					$H$ & (\red{1},-1) & (-1,\red{1}) \\
					$L$ & (-1,\red{1}) & (\red{1},-1) \\
				\end{game}
			\end{figure}
		\end{example}
		
		\begin{definition}
			Suppose player $i$ has a \emph{finite} set of pure strategies $A_i$, then a \textbf{mixed strategy} $\sigma_i \in \Delta (A_i)$ is a probability distribution over $A_i$.
		\end{definition}
		
		\begin{definition}
			The \textbf{support} of $\sigma_i$ is defined as
			\begin{equation}
				\mc{S}(\sigma_i) \equiv \{a \in A_i: \sigma_i(a) > 0\}
			\end{equation}
		\end{definition}
		
		\begin{notation}
			Given action set $A_i \equiv \{a_i\}$, a mixed strategy can be denoted as $A_i^\alpha$ where $\alpha$ is a multi-index.
		\end{notation}
		
		\begin{remark}
			A pure strategy $a_i \in A_i$ is a mixed strategy with 
			\begin{equation}
				\sigma_i(a_i) = 1
			\end{equation}
			So mixed strategy is a generalization of pure strategy.
		\end{remark}
		
		\begin{proposition}
			In a finite game, given the independence of randomization, the probability of the action profile $a = (a_i)$ to be realized given mixed strategy profile $(\sigma_i)$ is
			\begin{equation}
				\sigma (a) \equiv \prob{(a_i)|(\sigma_i)} = \prod_{i \in N} \sigma_i(a_i)
			\end{equation}
			and for player $i$, the \textbf{expected payoff} from mixed strategy profile $(\sigma_i)$ is 
			\begin{equation}
				U_i((\sigma_{i})) = 
				\sum_{a \in A} \big[\prod_{j \in N} \sigma_j(a_j)\big] u_i(a) = \expect{u_i(a)|(\sigma_i)}
			\end{equation}
		\end{proposition}
		
		\begin{proposition}
			The \textbf{expected payoff} from mixed strategy profile $(\sigma_i) \equiv (\sigma_i, \sigma_{-i})$ is
			\begin{equation}
				U_i(\sigma_i, \sigma_{-i}) \equiv \expect{u_i(a)|(\sigma_i)} = \sum_{a_i \in A_i} \sum_{a_{-i} \in A_{-i}} u_i(a_i, a_{-i}) \sigma_{-i}(a_{-i}) \sigma_i(a_i)
			\end{equation}
		\end{proposition}
		
		\begin{definition}[Osborne 2030 lec2]
			The \textbf{mixed extension} of the strategic game $\langle N, (A_i)_{i\in N}, (u_i)_{i\in N} \rangle$ is the following strategic game:
			\begin{enumerate}[(i)]
				\item \textbf{Players} $N$
				\item \textbf{Action sets} $\Delta(A_i)$ for player $i$
				\item \textbf{Preferences} Represented by $U_i: \prod_{j \in N} \Delta(A_j) \to \R$, where $U_i(\alpha)$ is the \ul{expected value} under $u_i$ of the lottery over $\prod_{j \in N} \Delta(A_j)$ induced by $\alpha$.
			\end{enumerate}
		\end{definition}
		
		\begin{definition}
			A pure strategy $a_i \in A_i$ is \textbf{strictly dominated} by a mixed strategy $\sigma_i \in \Delta(A_i)$ if
			\begin{equation}
				\forall a_{-i} \in A_{-i}\ u_i(a_i, a_{-i}) < U_i(\sigma_i, a_{-i})
			\end{equation}
		\end{definition}
		
		\begin{proposition}
			We can also define the strict dominance by replacing $\forall a_{-i} \in A_{-i}$ with $\forall \sigma_{-i} \in \Delta A_{-i}$ since
			\begin{equation}
				\forall a_{-i} \in A_{-i}\ u_i(a_i, a_{-i}) < U_i(\sigma_i, a_{-i}) \iff \forall \sigma_{-i} \in \Delta(A_{-i})\ U_i(a_i, \sigma_{-i}) < U_i(\sigma_i, \sigma_{-i})
			\end{equation}
			\begin{proof}
				($\implies$) Suppose 
				\begin{equation}
					\forall a_{-i} \in A_{-i}\ u_i(a_i, a_{-i}) < U_i(\sigma_i, a_{-i})
				\end{equation}
				Let $\sigma_{-i} \in \Delta(A_{-i})$,
				\begin{gather}
					U_i(a_i, \sigma_{-i}) = \sum_{a_{-i} \in A_{-i}} u_i(a_i, a_{-i}) \sigma_{-i}(a_{-i}) \\
					< \sum_{a_{-i} \in A_{-i}} U_i(\sigma_i, a_{-i}) \sigma_{-i}(a_{-i}) \\
					= U_i(\sigma_i, \sigma_{-i})
				\end{gather}
				Thus 
				\begin{equation}
					\forall \sigma_{-i} \in \Delta(A_{-i})\ U_i(a_i, \sigma_{-i}) < U_i(\sigma_i, \sigma_{-i})
				\end{equation} \\
				($\impliedby$) Suppose
				\begin{equation}
					\forall \sigma_{-i} \in \Delta(A_{-i})\ U_i(a_i, \sigma_{-i}) < U_i(\sigma_i, \sigma_{-i})
				\end{equation}
				Let $a_{-i} \in A_{-i}$, consider $\sigma_{-i}$ defined as $\sigma^*_{-i}(x) \equiv \mathbb{I}(x = a_{-i})$
				so that
				\begin{gather}
					U_i(a_i, \sigma^*_{-i}) < U_i(\sigma_i, \sigma^*_{-i}) \\
					\implies u_i(a_i, a_{-i}) < U_i(\sigma_i, a_{-i})
				\end{gather}
				The equivalence is shown.
			\end{proof}
		\end{proposition}
		
		\begin{definition}
			The \textbf{best response} to a mixed strategy $\sigma_{-i} \in \Delta{A_{-i}}$ is defined as
			\begin{equation}
				Br_i(\sigma_{-i}) \equiv \{\sigma_i \in \Delta(A_i): 
					\forall \tilde{\sigma}_i \in \Delta(A_i),\ U_i(\sigma_i, \sigma_{-i}) \geq U_i(\tilde{\sigma}_i, \sigma_{-i})
				\}
			\end{equation}
		\end{definition}
		
		\begin{proposition}
			Let $(\sigma_i, \sigma_{-i}) \in \Delta(A)$ then $\sigma_i \in Br_i(\sigma_{-i})$ \emph{if and only if}
			\begin{enumerate}
				\item $\forall a_j, a_k \in \mc{S}(\sigma_i),\ a_j \sim_i a_k$,
				\item \ul{and} $\forall a_j \in \mc{S}(\sigma_i)\ a_k \notin \mc{S}(\sigma_i),\ a_j \succsim_i a_k$
			\end{enumerate}
			\begin{proof}
				($\implies$) case 1: suppose (1) is false, for concreteness, assume
				\begin{equation}
					\exists a_j, a_k \in \mc{S}(\sigma_i)\ s.t.\ U_i(a_j, \sigma_{-i}) > U_i(a_k, \sigma_{-i})
				\end{equation}\\
				Define
				\begin{gather}
					\mc{U}_j \equiv U_i(a_j, \sigma_{-i}) = \sum_{a_{-i} \in A_{-i}} u_i(a_j, a_{-i}) \sigma_{-i}(a_{-i}) \\
					\mc{U}_k \equiv U_i(a_k, \sigma_{-i}) = \sum_{a_{-i} \in A_{-i}} u_i(a_k, a_{-i}) \sigma_{-i}(a_{-i})
				\end{gather}
				And $\mc{U}_j > \mc{U}_k$ by assumption. Consider $\sigma_i'$ defined with 
				\begin{equation}
					\sigma_i'(a) = 
					\begin{cases}
						\sigma_i(a_j) + \sigma_i(a_k) &\tx{ if } a = a_j \\
						0 &\tx{ if } a = a_k \\
						\sigma_i(a) &\tx{ otherwise}
					\end{cases}
				\end{equation}
				Therefore 
				\begin{gather}
					U_i(\sigma_i', \sigma_{-i}) \equiv \sum_{a_i \in A_i} \sum_{a_{-i} \in A_{-i}} u_i(a_i, a_{-i}) \sigma_i'(a_i) \sigma_{-i}(a_{-i}) \\
					= \sum_{a_i \in A_i\backslash \{a_j, a_k\}} \sum_{a_{-i} \in A_{-i}} u_i(a_i, a_{-i}) \sigma_i'(a_i) \sigma_{-i}(a_{-i}) + \sigma_i'(a_j) \mc{U}_j + \sigma_i'(a_k) \mc{U}_k \\
					= \sum_{a_i \in A_i\backslash \{a_j, a_k\}} \sum_{a_{-i} \in A_{-i}} u_i(a_i, a_{-i}) \sigma_i(a_i) \sigma_{-i}(a_{-i}) + \sigma_i(a_j) \mc{U}_j + \sigma_i(a_k) \mc{U}_j \\
					> \sum_{a_i \in A_i\backslash \{a_j, a_k\}} \sum_{a_{-i} \in A_{-i}} u_i(a_i, a_{-i}) \sigma_i(a_i) \sigma_{-i}(a_{-i}) + \sigma_i(a_j) \mc{U}_j + \sigma_i(a_k) \mc{U}_k \\
				= \sum_{a_i \in A_i} \sum_{a_{-i} \in A_{-i}} u_i(a_i, a_{-i}) \sigma_i(a_i) \sigma_{-i}(a_{-i})
				\equiv U_i(\sigma_i, \sigma_{-i})
				\end{gather}
				So that $\sigma_i'$ strictly dominates $\sigma_i$, so $\sigma_i \notin Br_i(\sigma_{-i})$. \\
				case 2: suppose (2) is false, for concreteness, assume
				\begin{equation}
					\exists a_j \in \mc{S}(\sigma_i),\ a_k \notin \mc{S}(\sigma_i)\ s.t.\ U_i(a_k, \sigma_{-i}) > U_i(a_j, \sigma_{-i})
				\end{equation}
				Consider mixed strategy $\sigma'_i$ defined as
				\begin{equation}
					\sigma_i'(a) = \begin{cases}
						0 &\tx{ if } a = \sigma_j \\
						\sigma_i(a_j) &\tx{ if } a = \sigma_k \\
						\sigma_i(a) &\tx{ otherwise}
					\end{cases}
				\end{equation}
				By previous definitions, $\mc{U}_k > \mc{U}_j$.\\
				Then
				\begin{gather}
					U_i(\sigma_i', \sigma_{-i}) \equiv \sum_{a_i \in A_i} \sum_{a_{-i} \in A_{-i}} u_i(a_i, a_{-i}) \sigma_i'(a_i) \sigma_{-i}(a_{-i}) \\
					= \sum_{a_i \in A_i\backslash \{a_j, a_k\}} \sum_{a_{-i} \in A_{-i}} u_i(a_i, a_{-i}) \sigma_i'(a_i) \sigma_{-i}(a_{-i}) + \sigma_i'(a_j) \mc{U}_j + \sigma_i'(a_k) \mc{U}_k\\
					= \sum_{a_i \in A_i\backslash \{a_j, a_k\}} \sum_{a_{-i} \in A_{-i}} u_i(a_i, a_{-i}) \sigma_i(a_i) \sigma_{-i}(a_{-i}) + 0\  \mc{U}_j + \sigma_i(a_j) \mc{U}_k \\
					> \sum_{a_i \in A_i\backslash \{a_j, a_k\}} \sum_{a_{-i} \in A_{-i}} u_i(a_i, a_{-i}) \sigma_i(a_i) \sigma_{-i}(a_{-i}) + \sigma_i(a_j) \mc{U}_j \\
					= U_i(\sigma_i, \sigma_{-i})
				\end{gather}
				Therefore $\sigma_i \notin Br_i(\sigma_{-i})$, \emph{modus tollens}. \\
				($\impliedby$) suppose $\sigma_i \notin Br_i(\sigma_{-i})$, that's
				\begin{equation}
					\exists \sigma'_i \in \Delta(A_i)\ s.t.\ U_i(\sigma_i', \sigma_{-i}) > U_i(\sigma_i, \sigma_{-i})
				\end{equation}
				Therefore
				\begin{gather}
					\exists \bar{a}_i, \underline{a}_i \in \mc{S}(\sigma_i)\ s.t.\ U_i(\bar{a}_i, \sigma_{-i}) \geq U_i(\sigma_i, \sigma_{-i}) \geq U_i(\underline{a}_i, \sigma_{-i})\\
					\land \exists \bar{a}'_i, \underline{a}'_i \in \mc{S}(\sigma_i')\ s.t.\ U_i(\bar{a}_i', \sigma_{-i}) \geq U_i(\sigma_i', \sigma_{-i}) \geq U_i(\underline{a}_i', \sigma_{-i})\\
					\implies U_i(\bar{a}_i', \sigma_{-i}) \geq U_i(\sigma_i', \sigma_{-i}) > U_i(a_i, \sigma_{-i}) \geq U_i(\underline{a}_i, \sigma_{-i}) \\
					\implies U_i(\bar{a}_i', \sigma_{-i}) > U_i(\underline{a}_i, \sigma_{-i}) \\
					\implies \bar{a}_i' \succ_i \underline{a}_i \in \mc{S}(\sigma_i)
				\end{gather}
				In both cases, $\overline{a}_i' \in \mc{S}(\sigma_i)$ and $\overline{a}_i' \notin \mc{S}(\sigma_i)$, the two requirements are falsified, \emph{modus tollens}.
			\end{proof}
		\end{proposition}
		
		\begin{remark}
			Procedure to solve Nash equilibria for two player finite games,
			\begin{enumerate}[(i)]
				\item Solve for pure strategy Nash equilibria.
				\item Eliminate \ul{strictly dominated} strategies.
				\item For player $i$, solve a mixed strategy scheme $\sigma_j$ so that player $i$ is indifferent between actions in a non-singleton subset of his/her action set, $A_i^*$.
				\item Player $i$ would play \ul{any} mixed strategy with support $A_i^*$.
				\item Repeat (iii) and (iv) for player $j$.
			\end{enumerate}
		\end{remark}
	\subsection{Some Propositions and Proofs}
		\begin{proposition}[exer.120.3]
			A mixed strategy that assigns positive probability of a \ul{strictly} dominated action is strictly dominated. That's, suppose $\tilde{a}_i \in A_i$ is strictly dominated, then for any $\sigma_i \in \Delta(A_i)$,
			\begin{equation}
				\tilde{a}_i \in \mc{S}(\sigma_i) \implies \sigma_i \tx{ strictly dominated}
			\end{equation}
			Note that the converse is \textbf{not} true.
		\end{proposition}
	
		\begin{proposition}[Nash]
			Every strategic game in which each player has \ul{finitely many actions} has a mixed strategy Nash equilibrium.
		\end{proposition}
		\begin{proposition}
			Let $\sigma_{-i} \in \Delta(A_{-i})$, if a pure strategy $a_i \in A_i$ is \ul{strictly dominated} by another \ul{pure strategy} $\tilde{a}_i \in A_i$, then
			\begin{equation}
				\forall \sigma_{i} \in Br_i(\sigma_{-i})\ a_i \notin \mc{S}(\sigma_i)
			\end{equation}
			\begin{proof}
				
			\end{proof}
		\end{proposition}
		
		\begin{proposition}
			Let $\sigma_{-i} \in \Delta(A_{-i})$, if a pure strategy $a_i \in A_i$ is \ul{strictly dominated} by another \ul{mixed strategy} $\tilde{\sigma}_i \in \Delta(A_i)$, then
			\begin{equation}
				\forall \sigma_{i} \in Br_i(\sigma_{-i})\ a_i \notin \mc{S}(\sigma_i)
			\end{equation}
			\begin{proof}
				
			\end{proof}
		\end{proposition}
		
		\begin{proposition}
			For pure strategy $a_i \in A_i$, if there exists $a_{-i} \in A_{-i}$ such that
			\begin{equation}
				u_i(a_i, a_{-i}) \geq u_i(\tilde{a}_i, a_{-i})\ \forall \tilde{a}_i \in A_i
			\end{equation}
			then $a_i$ is \textbf{not} strictly dominated (by pure strategies or mixed strategies).
			\begin{proof}
				
			\end{proof}
		\end{proposition}
		
		\begin{proposition}
			Let $a_i \in A_i$, then the following statements are equivalent
			\begin{enumerate}[(i)]
				\item \begin{equation}
					\exists \sigma_i \in \Delta(A_i)\ s.t.\ a_i \in \mc{S}(\sigma_i) \land 
					\forall \sigma_{-i} \in \Delta(A_{-i}),\ U_i(a_i, \sigma_{-i}) < U_i(\sigma_i, \sigma_{-i})
				\end{equation}
				\item \begin{equation}
					\exists \tilde{\sigma}_i \in \Delta(A_i)\ s.t.\ a_i \notin \mc{S}(\tilde{\sigma}_i) \land
					\forall \sigma_{-i} \in \Delta(A_{-i}),\ U_i(a_i, \sigma_{-i}) < U_i(\tilde{\sigma}_i, \sigma_{-i})
				\end{equation}
			\end{enumerate}
			\begin{proof}
				($\implies$) Suppose there exists $\sigma_i$ satisfies (i), \\
				Note that $\sigma_i(a_i) \neq 1$, define $\tilde{\sigma}_i$ as 
				\begin{equation}
					\tilde{\sigma}_i(a) = \begin{cases}
						\frac{\sigma_i(a)}{1 - \sigma_i(a_i)} &\tx{ if } a \in \mc{S}(\sigma_i) \backslash \{a_i\} \\
						0 &\tx{ otherwise}
					\end{cases}
				\end{equation}
				And $\mc{S}(\tilde{\sigma}_i) = \mc{S}(\sigma_i) \backslash \{a_i\}$.\\
				Then, by (i), 
				\begin{gather}
					U_i(a_i, \sigma_{-i}) < \sum_{a_j \in \mc{S}(\sigma_i)} U_i(a_j, \sigma_{-i})\sigma_i(a_j) \\
					\implies (1 - \sigma_i(a_i)) U_i(a_i, \sigma_{-i}) < \sum_{a_j \in \mc{S}(\sigma_i) \backslash \{a_i\}} U_i(a_j, \sigma_{-i})\sigma_i(a_j) \\
					\implies U_i(a_i, \sigma_{-i}) < \frac{1}{1 - \sigma_i(a_i)} \sum_{a_j \in \mc{S}(\tilde{\sigma}_i)} U_i(a_j, \sigma_{-i})\sigma_i(a_j) \\
					\implies U_i(a_i, \sigma_{-i}) < \sum_{a_j \in \mc{S}(\tilde{\sigma}_i)} U_i(a_j, \sigma_{-i}) \frac{\sigma_i(a_j)}{1 - \sigma_i(a_i)} \\
					\implies U_i(a_i, \sigma_{-i}) < \sum_{a_j \in \mc{S}(\tilde{\sigma}_i)} U_i(a_j, \sigma_{-i}) \tilde{\sigma}_i(a_j) \equiv U_i(\tilde{\sigma}_i, \sigma_{-i})
				\end{gather}
				($\impliedby$) Suppose there exists $\tilde{\sigma}_i$ satisfies (ii).
				Take $\eta \in \R_+$, define $\sigma_i$ as
				\begin{equation}
					\sigma_i(a) = \begin{cases}
						\frac{\tilde{\sigma}_i(a)}{1+\eta} &\tx{ if } a \in \mc{S}(\tilde{\sigma}_i) \\
						\frac{\eta}{1+\eta} &\tx{ if } a = a_i \\
						0 &\tx{ otherwise}
					\end{cases}
				\end{equation}
				Note that $\mc{S}(\sigma_i) = \mc{S}(\tilde{\sigma}_i) \cup \{a_i\}$.\\
				By (ii), 
				\begin{gather}
					U_i(a_i, \sigma_{-i}) < \sum_{a_j \in \mc{S}(\tilde{\sigma}_i)} U_i(a_j, \sigma_{-i}) \tilde{\sigma}_i(a_j) \\
					\implies (1+\eta) U_i(a_i, \sigma_{-i}) < \sum_{a_j \in \mc{S}(\tilde{\sigma}_i)} U_i(a_j, \sigma_{-i}) \tilde{\sigma}_i(a_j) + U_i(a_i, \sigma_{-i}) \eta \\
					\implies U_i(a_i, \sigma_{-i}) < \sum_{a_j \in \mc{S}(\tilde{\sigma}_i)} U_i(a_j, \sigma_{-i}) \frac{\tilde{\sigma}_i(a_j)}{1+\eta} + \frac{\eta}{1+\eta} U_i(a_i, \sigma_{-i}) \\
					= \sum_{a_j \in \mc{S}(\sigma_i)} U_i(a_j, \sigma_{-i} \sigma_i(a_j) \equiv U_i(\sigma_i, \sigma_{-i})
				\end{gather}
			\end{proof}
		\end{proposition}
		\begin{remark}
			With above proposition, while we are checking if $a_i \in A_i$ is strictly dominated by some mixed strategies, we don't have to include $a_i$ in the support of potential $\sigma_i$.
		\end{remark}
		
		\begin{proposition}[Osborne 2030 lec2]
			Let $\mc{G} = \langle N, (A_i), (u_i) \rangle$ and $\mc{G}' = \langle N, (\Delta(A_i)), (U_i) \rangle$. \\
			Then any Nash equilibrium of $\mc{G}$ is a mixed strategy Nash equilibrium(in which each player's strategy is pure) of $\mc{G}'$.
		\end{proposition}
		
		\begin{remark}
			Hence mixed strategy may do \emph{as well as} a pure strategy but can never \emph{do strictly better} than \emph{all} pure strategy. Consequently pure strategy remains optimal when mixed strategies are allowed.
		\end{remark}
		
		\begin{proposition}[Osborne 2030 lec2]
			For any mixed strategy Nash equilibrium of $\mc{G}'$ in which each player's strategy is pure is a Nash equilibrium.
		\end{proposition}
		
		\begin{remark}
			If player optimally chooses a pure strategy when she is allowed to randomize, then when she is prohibited from randomizing the pure strategy remains optimal.
		\end{remark}
		
		\begin{proposition}[Osborne 2030 lec2, lemma 33.2]
			The following statements are equivalent:
			\begin{enumerate}[(i)]
				\item $(\sigma^*_i)$ is a mixed strategy Nash equilibrium.
				\item $\forall i \in N,\ \sigma^*_i \in Br_i(\sigma_{-i}^*)$.
				\item $\forall i \in N$, every action in support of $\sigma^*_i$ is a best response to $\sigma_{-i}^*$.
			\end{enumerate} 
		\end{proposition}
\end{document}


























