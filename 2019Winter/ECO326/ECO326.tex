\documentclass[11pt]{article}

\title{ECO326 Advanced Microeconomic Theory \\ \small A Course in Game Theory}

\author{Tianyu Du}
\date{\today}

\usepackage{spikey}
\usepackage{amsmath}
\usepackage{amssymb}
\usepackage{soul}
\usepackage{float}
\usepackage{graphicx}
\usepackage{hyperref}
\usepackage{xcolor}
\usepackage{chngcntr}
\usepackage{centernot}
\usepackage[shortlabels]{enumitem}
\usepackage[margin=1truein]{geometry}
\usepackage{sgame}

\counterwithin{equation}{section}
\counterwithin{figure}{section}

\newcommand{\floor}[1]{\lfloor #1 \rfloor}

\usepackage[
    type={CC},
    modifier={by-nc},
    version={4.0},
]{doclicense}

\begin{document}
	\maketitle
	\doclicenseThis
	\texttt{Github Page} \url{https://github.com/TianyuDu/Spikey_UofT_Notes}\\
	\texttt{Note Page} \url{TianyuDu.com/notes}
	\paragraph{Readme} this note is based on the course content of \emph{ECO326 Advanced Microeconomics - Game Theory}, this note contains all materials covered during lectures and mentioned in the course syllabus. However, notations, statements of theorems and proofs are following the book \emph{A Course in Game Theory} by Osborne and Rubinstein, so they might be, to some extent, more mathematical than the required text for ECO326, \emph{An Introduction to Game Theory}.
	
	\tableofcontents
	\newpage
	
	\section{Lecture 1. Jan. 7 2019\\Games and Dominant Strategies}
		\paragraph{Game Theory} Choice environment where individual choices impact others.
		
		\begin{example}
			\begin{figure}[h]
				\centering
				  \begin{tabular}{c|c|c}
				    & W & S\\
				    \hline
				    W & $(1-c, 1-c)$ & $(\red{1-c}, \red{1})$ \\
				    \hline
				    S & $(\red{1}, \red{1-c})$ & $(0, 0)$
				  \end{tabular}
				  \caption{Payoff Matrix for Example 1}
			\end{figure}
		\end{example}
		Suppose $c \in (0, 1)$. In this game,
		\begin{enumerate}[i]
			\item $N = \{i, j\}$,
			\item $A_i = A_j = \{W, S\}$,
		\end{enumerate}
		
		\begin{definition}[pg.7]
			A \textbf{preference relation} is a \ul{complete reflexive and transitive} binary relation.
		\end{definition}
		
		\begin{definition}[11.1, lec.1]
			A \textbf{(strategic) game} consists of
			\begin{enumerate}[i]
				\item a \ul{finite} set of \textbf{players} $N$, with $|N| \geq 2$.
				\item for each player $i \in N$, an \textbf{actions} $A_i \neq \emptyset$.
				\item for each player $i \in N$, a \textbf{preference relation} $\succsim_i$ defined on $A \equiv \times_{i\in N}A_i$.
			\end{enumerate}
			and can be written as a triple $\langle N, (A_i), (\succsim_i) \rangle$.
		\end{definition}
		
		\begin{definition}[Equivalent definition of game]
			For each player $i \in N$ we can define a \ul{utility function}, $u_i: A \to \R$ such that
			\begin{equation}
				\forall (a_i), (a_i)' \in A,\ u_i((a_i)) \geq u_i((a_i)') \iff (a_i) \succsim_i (a_i)'
			\end{equation}
			So the game can be defined as a triple $\langle N, (A_i), (u_i) \rangle$.
		\end{definition}
		
		\begin{definition}[lec.1]
			An \textbf{action profile} is a $n$-tuple of actions $a_i \in A_i$ for each player $i \in N$ and denoted as 
				\[
					(a_i)_{i \in N} \tx{ or } (a_i)
				\]
			The \textbf{action profile space} is defined as 
				\[
					A \equiv \times_{i \in N} A_i
				\]
		\end{definition}
		
		\begin{definition}[lec.1]
			Action $a_i \in A_i$ is \textbf{strictly dominated} by action $\tilde{a}_i \in A_i$ if
			\[
				\forall a_{-i} \in A_{-i},\ u_i(a_i, a_{-i}) < u_i(\tilde{a}_i, a_{-i})
			\]
			And $a_i$ is \textbf{weakly dominated} by $\tilde{a}_i$ if
			\[
				\forall a_{-i} \in A_{-i},\ u_i(a_i, a_{-i}) \leq u_i(\tilde{a}_i, a_{-i})
			\]
			\ul{and}
			\[
				\exists a_{-i} \in A_{-i},\ u_i(a_i, a_{-i}) < u_i(\tilde{a}_i, a_{-i})
			\]
		\end{definition}
		
		\begin{corollary}[Consequence of RCT]
			It is irrational to play strictly dominated actions. So rational choice theory suggests a player would never play \red{strictly} dominated strategies.
		\end{corollary}
		
		\begin{definition}
			Action $a_i \in A_i$ is \textbf{strictly dominant} if it strictly dominates \red{all} other actions.
		\end{definition}
		
		\begin{definition}
			Action $a_i \in A_i$ is \textbf{weakly dominant} if it weakly dominates \red{all} other actions.
		\end{definition}
		
		\begin{definition}
			Action $a_i \in A_i$ is \textbf{weakly/strictly dominated} if \red{there exists} another strategy weakly/strictly dominates $a_i$.
		\end{definition}
		
		\begin{example}[Prisoner Dilemma]
			\begin{figure}[h]
				\centering
				\begin{tabular}{c|c|c}
					 & S & C \\
					\hline
					S & (-1, -1) & (-10, \red{0})\\
					\hline
					C & (\red{0}, -10) & (\red{-5}, \red{-5})
				\end{tabular}
				\caption{Payoff matrix for example 2}
			\end{figure}
			Note that S is strictly dominated by C. Therefore C is strictly dominant for both players.
		\end{example}
		
		\begin{example}
			\begin{figure}[h]
				\centering
				\begin{tabular}{c|c|c|c}
					 & L & C & R\\
					\hline
					U & (2, \red{2}) & (\red{5}, 0) & (\red{3}, 0)\\
					\hline
					M & (2, \red{7}) & (2, 5) & (2, 6)\\
					\hline 
					D & (\red{5}, \red{3}) & (4, 2) & (\red{3}, 1)
				\end{tabular}
				\caption{Payoff matrix for example 2}
			\end{figure}
			So in this game, for player 2, L is strictly dominant. \\For player 1, M is strictly dominated by D. And M is weakly dominated by U.
		\end{example}
		
		\begin{example}
			There are three candidates, $\{A, B, C\}$. And there are 50 players (voters, note that $\emptyset \notin A_i$ since they must vote).
			And 
			\[
				\forall i \in N,\ A_i = \{A, B, C\}
			\]
			Each individual has strictly preference over $A, B, C$. If tie is encountered, randomization would be taken.
			\begin{enumerate}[i]
				\item $A \succ B \succ C$,
				\item $A \succ AC_{tie} \succ C$
			\end{enumerate}
			\paragraph{Claim 1:}There are no weakly or strictly dominant actions. 
			\begin{proof}
				Let $a_i \in \{V_A, V_B, V_C\}$ denote the action taken by player $i \in N$, \\
				Note that weak dominance is a necessary condition for strict dominance, \\
				So above claim is reduced to \emph{there are no weakly dominant actions}. \\
				The reduced claim is equivalent to the following statement, \\
				
				\begin{gather*}
					\forall a_i \in A_i,\ \exists \tilde{a}_i \in A_i\ s.t.\ a_i \neq \tilde{a}_i\\
					s.t.\ \exists a_{-i} \in A_{-i}\ s.t.\ u_i(a_i, a_{-i}) > u_i(\tilde{a}_i, a_{-i}) \lor \forall a_{-i} \in A_{-i},\ u_i(a_i, a_{-i}) = u_i(\tilde{a}_i, a_{-i})
				\end{gather*}
				Let $n_{-i}^j$ denote the number of voters other than $i$ voting for candidate $j$. Clearly each $a_{-i} \in A_{-i}$ would induce an outcome as a triple $(n_{-i}^A, n_{-i}^B, n_{-i}^C)$.\\
				Consider action $V_A$, and $a_{-i}$ induces 
				\[
					(n_{-i}^A, n_{-i}^B, n_{-i}^C) = (1, 24, 24)
				\]
				then 
				\[
					(V_B, a_{-i}) \succ_i (V_A, a_{-i})
				\]
				So $V_A$ failed to be a dominant strategy of any kind. \\
				Similarly, consider action $V_B$, if $a_{-i}$ induces
				\[
					(n_{-i}^A, n_{-i}^B, n_{-i}^C) = (24, 1, 24)
				\]
				then 
				\[
					(V_A, a_{-i}) \succsim_i (V_B, a_{-i})
				\]
				So $V_B$ failed to be a dominant strategy. \\
				Similarly, consider action $V_C$, if $a_{-i}$ induces
				\[
					(n_{-i}^A, n_{-i}^B, n_{-i}^C) = (24, 24, 1)
				\]
				then 
				\[
					(V_A, a_{-i}) \succsim_i (V_C, a_{-i})
				\]
				So $V_B$ failed to be a dominant strategy.
			\end{proof}
			\paragraph{Claim 2:}Only voting for your least preferred candidate is weakly dominated.
			\begin{proof}
				We are going to show there exists a strategy (voting for $B$) weakly dominates voting for $C$.
				\begin{figure}[H]
					\centering
					\begin{tabular}{c|c|c}
						Vote A & Cases & Vote C \\
						\hline
						A & $n_{-i}^A > n_{-i}^B, n_{-i}^C$ & A, AC \\
						B & $n_{-i}^B > n_{-i}^A, n_{-i}^C$ & B, BC \\
						C, BC & $n_{-i}^C > n_{-i}^A, n_{-i}^B$ & C\\
						B & $n_{-i}^A = n_{-i}^B > n_{-i}^C$ & AB\\
						A & $n_{-i}^A = n_{-i}^C > n_{-i}^B$ & C\\
						BC & $n_{-i}^C = n_{-i}^B > n_{-i}^A$ & C
					\end{tabular}
					\caption{Voting for A versus Voting for C}
				\end{figure}
			\end{proof}
		\end{example}
		
		\begin{definition}[pg.11]
			A strategic game $\langle N, (A_i), (\succsim_i) \rangle$ is \textbf{finite} if 
			\[
				|A_i| < \aleph_0\ \forall i \in N
			\]
		\end{definition}
	
	\section{Lecture 2. Jan. 14 2019\\Iterated Elimination and Rationalizability}
		\begin{example}[Bubble Game]
			Consider a player game
			\begin{equation}
				\langle N, (A_i), (u_i) \rangle
			\end{equation}
			where 
			\begin{equation}
				A_i = \{0,\dots,100\},\ \forall i
			\end{equation}
			and 
			\begin{gather}
				u_i(a_i, a_{-i}) = a_i - penalty_i(a_i, a_{-i}) \\
				penalty_i = \begin{cases}
					0 \tx{ if } a_i < \max_{j \neq i} a_j - 1 \\
					10(a_i - \max_{j \neq i} a_j + 1) \tx{ if } a_i \geq \max_{j \neq i} - 1
				\end{cases}
			\end{gather}
		\end{example}	
		
		\subsection{Iterated Elimination of Strictly Dominated Strategies(Actions)}
			\begin{definition}[IESD]
				Given the initial game,
				\[
					G_0 = \langle N, (A^0_i), (u_i) \rangle
				\] 
				At stage $k \in \N$, 
				\[
					G_k = \langle N, (A^k_i), (u_i) \rangle
				\]
				In stage $k$, for all $i \in N$, find the set of \ul{strictly} dominated actions, $D_i^k \subsetneq A_i^k$.
				\begin{enumerate}[i)]
					\item If $\forall i \in N\ s.t.\ D_i^k = \emptyset$, conclude the profile
					\[
						(A_i^k)
					\]
					to be the set of action profiles survive from IESD.
					\item If $\exists i \in N\ s.t.\ D_i^k \neq \emptyset$, define
					\[
						\forall i \in N,\ A^{k+1}_i := A^k_i \backslash D_i^k
					\]
				\end{enumerate}
			\end{definition}
			
			\begin{example}
				Action profile $(M, R)$ survives the IESD.
				\begin{proof}
					\begin{gather*}
						k=0,\ A_1^0 = \{U, M, D\},\ 
						A_2^0 = \{L, R\} \\
						k=1,\ A_1^1 = \{U, M\},\ 
						A_2^1 = \{L, R\} \\
						k=2,\ A_1^2 = \{U, M\},\ 
						A_2^2 = \{R\} \\
						k=3,\ A_1^3 = \{M\},\ 
						A_2^3 = \{R\} \\
					\end{gather*}
				\end{proof}
			\end{example}
			\begin{figure*}[h]
				\centering
				\begin{tabular}{l|cc}
				  & L & R \\
				  \hline
				  U & 4,0 & 2,2 \\
				  M & 1, 2& \red{5,3} \\
				  D & 0,5 & 1,4 \\
				\end{tabular}
				\caption{Game for Example 2.1}
			\end{figure*}
			
			\begin{example}[Hotelling Model of Politics]
				Players maximize their votes by choosing where to stand along a natural number line.
				\begin{itemize}
					\item Player $N = \{1,2\}$
					\item Action set $A_i = \{1, \dots, M\}$, with $2 \centernot | M$ and $M > 3$.
					\item Payoff
					\begin{equation}
						u_i(a_i, a_{-i}) = \begin{cases}
							a_i + \frac{1}{2}(a_{-i} - a_{i} - 1) \tx{ if } a_i < a_{-i} \\
							\frac{M}{2} \tx{ if } a_i = a_{-i} \\
							M - [a_{-i} + \frac{1}{2}(a_{i} - a_{-i} - 1)] \tx{ if } a_i > a_{-i}
						\end{cases}
					\end{equation}
				\end{itemize}
				\paragraph{Claim i.} $a_i=1$ is strictly dominated by $a_i=2$.
				\begin{proof}
					\begin{gather}
						u_i(a_i=1, a_{-i}) =
						\begin{cases}
							\frac{M}{2} \tx{ if } a_{-i} = 1 \\
							\frac{a_{-i}}{2} \tx{ if } a_{-i} > 1
						\end{cases} \\
						u_i(a_i=2, a_{-i}) = 
						\begin{cases}
							M - 1 \tx{ if } a_{-i} = 1 \\
							\frac{M}{2} \tx{ if } a_{-i} = 2 \\
							\frac{a_{-i}}{2} + \frac{1}{2} \tx{ if } a_{-i} > 2
						\end{cases}
					\end{gather}
				\end{proof}
				\paragraph{Claim ii.} $\floor{\frac{n}{2}} + 1$ is the only action survives.
				\begin{proof}
					Similarly, we can eliminate all edge-values iteratively.
				\end{proof}
			\end{example}
			
			\begin{definition}
				For each $i \in N$, the \textbf{best-response function} of this player is a correspondence $B_i: A_{-i} \rightrightarrows A_i$ defined as
				\begin{equation}
					B_i(a_{-i}) := \{a_i \in A_i : u_i(a_i, a_{-i}) \geq u_i(a_i', a_{-i})\ \forall a_i' \in A_i \}
				\end{equation}
			\end{definition}
			
			\begin{definition}
				A \textbf{belief} of player $i$ (about the actions of the other players) is a \ul{probability measure}, $\alpha_i$, on $A_{-i}=\times_{j \in N \backslash \{i\}} A_j$. \\
				$\alpha_i$ is a mapping such that
				\begin{itemize}
					\item $\alpha_i: A_{-i} \to [0,1]$.
					\item $\alpha_i(A_{-i}) = 1$.
					\item For all countable piece-wise \ul{disjoint} collection 
					\[
						\{E_j\}_{j\in \mc{J}} \in \mc{P}(A_{-i})
					\]
					$\alpha_i$ satisfies the \emph{countable additivity property}:
					\[
						\alpha_i(\bigcup_{i \in I} E_i) = \sum_{i \in I}\alpha_i(E_i)
					\]
				\end{itemize}
			\end{definition}
				
			\begin{definition}
				$a_i$ is a \ul{\textbf{best response} to the belief $\alpha_i$} if
				\begin{equation}
					\forall a_i' \in A_i,\ \sum_{a_{-i}} u_i(a_i, a_{-i}) \alpha_i(a_{-i}) \geq \sum_{a_{-i}} u_i(a_i', a_{-i}) \alpha_i(a_{-i})
				\end{equation}
				or, more generally,
				\begin{equation}
					\forall a'_i \in A_i, \expect{u_i(a_i, a_{-i})|\alpha_i} \geq \expect{u_i(a_i', a_{-i})|\alpha_i}
				\end{equation}
			\end{definition}
			
			\begin{definition}
				$a_i \in A_i$ is a \textbf{never best response} if it is not a best response given any belief $\alpha_i$.
			\end{definition}
			
			\begin{corollary}
				\emph{Iterative Elimination of Never Best Response}: same procedures but $D_i^k$ is the set of never best responses for player $i$ at game $G^k$.
			\end{corollary}
			
			\begin{example}
				For player 1, $D$ is not strictly dominated, but it is a never best response.
				\begin{proof}
					Let $\alpha$ be a probability measure on $\{L, R\}$ such that $\alpha(L) = p \in [0, 1]$.\\
					\begin{gather}
						\expect{u_1|U,\alpha} = 10p \\
						\expect{u_1|M,\alpha} = 10 - 10p \\
						\expect{u_1|D,\alpha} = 1
					\end{gather}
					\textbf{Case i}
					\begin{equation}
						p \geq 0.5 \implies \expect{u_1|U,\alpha} \geq 5
					\end{equation}
					\textbf{Case ii}
					\begin{equation}
						p < 0.5 \implies \expect{u_1|M,\alpha} > 5
					\end{equation}
					Therefore, for any belief $\alpha$, $D$ cannot be a best response. So $D$ is a never best response.
				\end{proof}
			\end{example}
			\begin{figure}[h]
				\centering
				\begin{tabular}{l|cc}
				  & L & R\\
				  \hline
				  U & 10,0 & 0,0 \\
				  M & 0,0 & 10,0\\
				  D & 1,0 & 1,0
				\end{tabular}
			\end{figure}
			
			\begin{definition}
				An action $a_i \in A_i$ is \textbf{rationalizable} if it survives \emph{iterative elimination of never best responses}.
			\end{definition}
			
			\begin{lemma}[i385.3]
				In a \ul{two player} game, $a_i$ is strictly dominated 
				\ul{if and only if} it is a never best response.
			\end{lemma}
			
			\begin{assumption}[Common knowledge rationality]
				We assume our players of game all acknowledge that other players are playing the game in rational ways.
			\end{assumption}
		
	\section{Lecture 3. Jan. 21 2019}
		\begin{definition}
			A \textbf{pure strategy Nash equilibrium} is a \ul{strategy profile} $(a_i)$ such that 
			\begin{equation}
				\forall i \in N,\ a_i \in Br_i(a_{-i})
			\end{equation}
		\end{definition}
			
		\begin{remark}
			That's, a NE is a situation that if player $i$ knows what other players do, the action given by the NE profile is still a best response.
		\end{remark}
		
		\begin{remark}[Interpretations]
			A Nash equilibrium is 
			\begin{enumerate}[i)]
				\item An action profile induces a \ul{stable outcome},
				\item A \ul{creditable agreement}, such that no player has incentive to break the agreement.
			\end{enumerate}
		\end{remark}
		
		\begin{example}[Cournot Duopoly]
			Consider a game with
			\begin{enumerate}[i)]
				\item Player $N = \{1,2\}$
				\item Action set $A_i = [0, \infty)\ \forall i \in N$
			\end{enumerate}
			And revenue $R_i$ defined by 
			\begin{equation}
				R_i = p_i q_i
			\end{equation}
			where price is linear in quantity supplied,
			\begin{equation}
				p_i = \alpha - (q_i + q_{-i}),\ \alpha \in \R
			\end{equation}
			and firms face fixed cost $c \in \R$, with the assumption that $\alpha > c$. So the cost function is given by
			\begin{equation}
				C_i(q_i) = c q_i 
			\end{equation}
			The profit function is given by
			\begin{gather}
				\Pi_i(q_i, q_{-i}) = (\alpha - (q_i + q_{-i}) - c) q_i \\
				= (\alpha-c-q_{-i})q_i - q_i^2
			\end{gather}
			Given $q_{-i} \in A_{-i}$, the best response is given by
			\begin{gather}
				Br_i(q_{-i}) = \tx{argmax}_{q_i \in [0, \infty)} \Pi_i(q_i, q_{-i}) \\
				= \max\{0, \frac{\alpha - c - q_{-i}}{2}\}\ \forall i \in N
			\end{gather}
			Considering the case that both players are producing positive quantities, we can solve $q_i^*$ by
			\begin{gather}
				Br_i \circ Br_{-i}(q_i) = \frac{\alpha - c - \frac{\alpha - c - q_i}{2}}{2} = q_i \\
				\implies 2 q_i - \frac{q_i}{2} = \frac{\alpha - c}{2} \\
				\implies q_i^* = \frac{\alpha - c}{3}
			\end{gather}
			
			\begin{remark}
				If fixed cost presents, even if the game is symmetric, the Nash equilibrium could be asymmetric. (e.g. one firm is out of market and the other firm produces the monopoly amount)
			\end{remark}
		\end{example}
		
		\begin{remark}
			Nash equilibrium induces an \emph{individual level optimality} instead of the common wealth optimality.
		\end{remark}
		
		\begin{example}[Continue Cournot Duopoly]
			Note that in the Cournot Duopoly game, for each $i \in N$, the Nash equilibrium profit is
			\begin{gather}
				\Pi_{NE}^* = (\alpha - c - \frac{2(\alpha - c)}{3}) \frac{\alpha - c}{3} \\
				= \frac{(\alpha - c)^2}{9}
			\end{gather}
			So the total profit for the two firms is $\frac{2(\alpha - c)^2}{9}$. \\
			Now considering if the two firms form a Cartel, the \emph{aggregate} quantity produced is
			\begin{gather}
				Q^* = \tx{argmax}_{Q \in \R_{\geq 0}} (\alpha - c - Q) Q \\
				= \frac{\alpha - c}{2} \\
				\implies \Pi_{Cartel}^* = (\alpha - c - \frac{\alpha - c}{2}) \frac{\alpha - c}{2} \\
				= \frac{(\alpha - c)^2}{4} > \frac{2(\alpha - c)^2}{9}
			\end{gather}
			The fact $\Pi_{Cartel}^* > 2 \times \Pi_{NE}^*$ suggests the Nash equilibrium action profile did not induce the optimal common wealth outcome. However, the Cartel action profile is not a stable outcome since every player has incentive to increase their production level.
		\end{example}
		
		\begin{example}[Prisoner's Dilemma]
			The Nash equilibrium(Confess, Confess) is \ul{not} the best outcome for the two players as a group. The optimal action profile \ul{for a group} is (Silent, Silent).
		\end{example}
		
		\begin{proposition}
			No strategy that is eliminated during iterated elimination of \emph{never best response} can be played in a Nash equilibrium.
		\end{proposition}
	
	\section{Lecture 4. Jan. 28. 2019}
	\begin{example}[From last lecture]
		Consider the payoff matrix
		\begin{figure*}[h]
			\centering
			\begin{tabular}{c|c|c}
				 & A & B \\
				\hline
				A & (\red{1},\red{1}) & (\red{0},0) \\
				B & (0,\red{0}) & (\red{0},\red{0}) \\
			\end{tabular}
		\end{figure*}
		Both $(A,A)$ and $(B,B)$ are Nash equilibria. But in the former NE, for each $i \in N$,
		\begin{equation}
			Br_i(a_{-i}=A) = \{A\}
		\end{equation}
		which is a singleton. \\
		In the second NE, for each $i \in N$,
		\begin{equation}
			Br_i(a_{-i}=B) = \{A, B\}
		\end{equation}
		We call Nash equilibria of the first type \emph{strict Nash equilibria} and the later one \emph{weak Nash equilibria}. More formal definitions of these two types of Nash equilibria are given below.
	\end{example}
	
	\begin{definition}
		A \textbf{strict Nash equilibrium} is an action profile $(a_i)$ such that
		\[
			\forall i \in N,\ |Br_i(a_{-i})| = 1
		\]
	\end{definition}
	
	\begin{definition}
		A \textbf{weak Nash equilibrium} is a Nash equilibrium that is not strict. That's, a weak Nash equilibrium is an action profile $(a_i)$ such that
		\[
			\exists i \in N,\ |Br_i(a_{-i})| > 1
		\]
	\end{definition}
	
	\begin{example}[Cournot with $n$ firms]
		Consider the game 
		\begin{equation}
			\langle N, (\R_{\geq 0}), (\pi_i) \rangle
		\end{equation}
		Where $|N| = n$ and each firm picks a quantity $q_i \in A_i \equiv \R_{\geq 0}$. \\
		Define 
		\begin{equation}
			Q \equiv \sum_j q_j
		\end{equation}
		For each $i \in N$, define 
		\begin{equation}
			Q_{-i} \equiv \sum_{j\neq i} q_j
		\end{equation}
		And the market has \emph{linear demand curve}
		\begin{equation}
			P(\{q_i\}_{i\in N}) = \alpha - \sum_{j}q_j = \alpha - Q
		\end{equation}
		And firms face fixed cost
		\begin{equation}
			\forall i \in N,\ C(q_i) = c q_i \tx{ where } 0 < c < \alpha
		\end{equation}
		The profit function is
		\begin{equation}
			\forall i \in N,\ \pi_i(q_i, Q_{-i}) = (\alpha - c - Q_{-i})q_i - q_i^2
		\end{equation}
		\paragraph{Question}What are the Nash equilibria in this environment? \\
		For each firm $i \in N$, the best response correspondence is
		\begin{gather}
			Br_i(Q_{-i}) = \max\{0, \argmax_{q_i \in \nnr} \pi_{i}(q_i, Q_{-i}) \} \\
			= \max\{0, \frac{\alpha - c - Q_{-i}}{2}\}
		\end{gather}
		\paragraph{Assume}We have a symmetric Nash equilibrium, 
		\begin{gather}
			\forall i \in N,\ q^*_i = q^* = \frac{Q}{n} \\
			\implies q^* = Br_i(\frac{n-1}{n}Q) \\
			\implies 2 q^* = \alpha - c - (n-1) q^* \\
			\implies q^* = \frac{\alpha - c}{n + 1}
		\end{gather}
		\paragraph{Check}Then check the validity of symmetric Nash equilibrium by asserting for every player, if all other players are playing the action suggested by the symmetric Nash equilibrium, then this player should also play it. That's
		\begin{gather}
			q^* = Br_i(\{q_j\}_{j\neq i}) \\
			= \frac{\alpha - c}{2} - \frac{1}{2} \frac{n-1}{n+1} (\alpha - c) \\
			= \frac{1}{2}((\alpha - c) - \frac{n-1}{n+1}(\alpha - c)) \\
			= \frac{1}{2} \frac{2}{n+1}(\alpha - c) \\
			= \frac{\alpha - c}{n + 1} 
			= q^*
		\end{gather}
		\paragraph{Uniqueness} We are going to show the symmetric Nash equilibrium is the only possible equilibrium action profile. \\
		\begin{proof}
			Suppose there exists some non-symmetric Nash equilibrium. \\
			For concreteness, assuming there exists $\epsilon > 0$ such that
			\begin{equation}
				\exists i, j \in N,\ q_i = q_j + \epsilon
			\end{equation}
			Define $Q_{-ij} \equiv Q - q_i - q_j$.\\
			For firm $i$,
			\begin{gather}
				q_i = Br_i(Q_{-i}) = \frac{1}{2} (\alpha - c - Q_{-i}) \\
				= \frac{1}{2}(\alpha - c - Q_{-ij} - q_{j}) \\
				= \frac{1}{2}(\alpha - c - Q_{-ij} - q_i + \epsilon) \\
				\implies 3 q_i = \alpha - c - Q_{-ij} + \epsilon \\
				\implies 3 q_i = \alpha - c - Q_{-j} + q_i + \epsilon \\
				\implies 2 q_i - \epsilon = \alpha - c - Q_{-j} \\
				\implies  q_i - \frac{\epsilon}{2} = Br(Q_{-j}) = q_j \\
				= q_i - \epsilon \\
				\implies \epsilon = 2 \epsilon
			\end{gather}
			which contradicts the assumption that $\epsilon > 0$. \\
			Therefore we conclude that the symmetric Nash equilibrium is the only Nash equilibrium in this environment.
		\end{proof}
	\end{example}
	
	\begin{example}[Cournot duopoly with fixed cost]
		Consider the game 
		\begin{equation}
			\mc{G} = \langle N = (1,2), (A_i = \nnr), (\pi_i) \rangle
		\end{equation}
		Where the cost function is defined as
		\begin{equation}
			C_i(q_i) = \begin{cases}
				c q_i + f &\tx{ if } q_i > 0 \\
				0 &\tx{ if } q_i = 0
			\end{cases}
		\end{equation}
		where $\alpha > c > 0$. \\
		For each firm $i \in N$, the best response function, conditioned on $q_i > 0$, is
		\begin{equation}
			q_i = Br_i(q_{-i}) = \frac{\alpha - c - q_{-i}}{2}
		\end{equation}
		We have to \textbf{check the profit} to assert the profit is non-negative while firm $i$ is producing above quantity. Because, otherwise, this firm could always derivate to $q_i = 0$ to avoid loss (earning zero profit).
		\begin{gather}
			\pi_i(Br_{q_{-i}}, q_{-i}) = (\alpha - q_{-i} - \frac{\alpha - c - q_{-i}}{2} - c)\frac{\alpha - c - q_{-i}}{2} - f \\
			= \frac{(\alpha - c - q_{-i})^2}{4} - f \\
			\pi_i(q_{-i}) \geq 0 \iff \alpha - c - q_{-i} \geq 2\sqrt{f} \\
			\iff q_{-i} \leq \alpha - c - 2\sqrt{f}
		\end{gather}
		Therefore we can modify our \textbf{best response correspondence} to
		\begin{gather}
			Br_i(q_{-i}) = 
			\begin{cases}
				\frac{\alpha - c - q_{-i}}{2} &\tx{ if }  q_{-i} \leq \alpha - c - 2\sqrt{f} \\
				0 &\tx{ if } q_{-i} \geq \alpha - c - 2\sqrt{f}
			\end{cases}
		\end{gather}
		\textbf{Monopoly}: the monopoly amount, conditioned on the firm decides to produce, is $\frac{\alpha - c}{2}$. In the monopoly case, suppose the NE action profile is (the opposite case can be shown by symmetry)
		\begin{equation}
			(\frac{\alpha - c}{2}, 0)
		\end{equation}
		We have to assert both 
		\begin{gather}
			\begin{cases}
				\frac{\alpha - c}{2} \in Br_1(0) \\
				0 \in Br_2(\frac{\alpha - c}{2})
			\end{cases} \\
			\implies
			\begin{cases}
				0 \leq \alpha - c - 2\sqrt{f} &\tx{ for firm 1 to produce positive amount.}\\
				\frac{\alpha - c}{2} \geq \alpha - c - 2\sqrt{f} &\tx{ for firm 2 to produce zero.}
			\end{cases} \\
			\implies 0 \leq \alpha - c - 2\sqrt{f} \leq \frac{\alpha - c}{2} \\
			\implies -(\alpha - c) \leq - 2\sqrt{f} \leq -\frac{\alpha - c}{2} \\
			\implies \red{\sqrt{f} \in [\frac{\alpha-c}{4}, \frac{\alpha - c}{2}]}
		\end{gather}
		\textbf{Positive symmetric equilibrium}: we've proven, in the general case, it's impossible for both firms to produce positive but different amounts. Therefore we have to assert 
		\begin{gather}
			\frac{\alpha - c}{3} \in Br_i(\frac{\alpha - c}{3})\ \forall i \in N \\
			\implies \frac{\alpha - c}{3} \leq \alpha - c - 2\sqrt{f} \\
			\implies \sqrt{f} \leq \frac{1}{3} (\alpha - c)
		\end{gather}
		\textbf{Zero symmetric equilibrium}: we have to assert
		\begin{gather}
			0 \in Br_i(0)\ \forall i \in N\\
			\implies 0 \geq \alpha - c - 2\sqrt{f} \\
			\implies \sqrt{f} \geq \frac{\alpha - c}{2}
		\end{gather}
	\end{example}
	
	\begin{example}[Discrete price Bertrand duopoly]
		Consider the following game
		\begin{gather}
			\mc{G} = \langle N=\{1,2\}, (A_i = \{k \epsilon: k \in \Z_{\geq 0}\}), (\pi_i) \rangle
		\end{gather}
		The profit function can be derived as
		\begin{gather}
			\pi_i(p_i, p_{-i}) = \begin{cases}
				(\alpha - p_i)(p_i - c)&\tx{ if } p_i < p_{-i} \\
				\frac{\alpha - p_i}{2}(p_i - c) &\tx{ if } p_i = p_{-i} \\
				0 &\tx{ if } p_i > p_{-i}
			\end{cases}
		\end{gather}
		\textbf{Claim} the only Nash equilibria are $(c, c)$ and $(c+\epsilon, c+\epsilon)$. \\
		\textbf{Justify} $(c, c)$, consider any firm $i \in N$, currently $\pi_i = 0$ 
		\begin{gather}
			\uparrow p_i \implies \pi_i \leftarrow \pi_i' = 0 \tx{ \xmark} \\
			\downarrow p_i \implies \pi_i \leftarrow \pi_i' < 0 \tx{ \xmark}
		\end{gather}
		So no firm has incentive to derivate from this action profile, so $(c, c)$ is a Nash equilibrium by definition. \\
		Consider the action profile $(c+\epsilon, c+\epsilon)$, both firms are earning a positive profit $\pi_i > 0$. For any firm $i \in N$,
		\begin{gather}
			\uparrow p_i \implies \pi_i \leftarrow \pi_i' = 0 \tx{ \xmark} \\
			\downarrow p_i \implies \pi_i \leftarrow \pi_i' = 0 \tx{ \xmark}
		\end{gather}
		So $(c+\epsilon, c+\epsilon)$ is a Nash equilibrium by definition. \\
		\textbf{No other Nash equilibrium} \\
		\textbf{Claim}: $(p_1, p_2)$ cannot be a Nash equilibrium if any $p_i < c$. \\
		Obviously, set $p_i < c$ would induce negative profit and firm $i$ do better off by setting $p_i \leftarrow c$. \\
		\textbf{Claim}: the symmetric profit $(p, p)$ with $p > c + \epsilon$ cannot be a Nash equilibrium. For both firms, the current profit is
		\begin{equation}
			\pi_1 = \pi_2 = \frac{1}{2}(\alpha - c - k\epsilon) k \epsilon
		\end{equation}
		And reducing price by $\epsilon$ leads to a profit of
		\begin{equation}
			\pi_i ' = (\alpha - c - k \epsilon + \epsilon) (k-1)\epsilon > \pi_i
		\end{equation}
		so such action profile cannot be Nash equilibrium. \\
		\textbf{Claim} $(p_1, p_2)$ with $p_1 \neq p_2$ and $p_1, p_2 > c$ cannot be a Nash equilibrium. The firm charges higher price can always reduce it's price to the price charged by the other firm and gain a positive profit. \\
		\textbf{Claim} $(c, p_2)$ with $p_2 \geq c + \epsilon$ cannot be a Nash equilibrium. The firm charging $p = c $ can always increase its price to $c + \epsilon$ to earn positive profit. \\
		Therefore there's no Nash equilibrium other than $(c,c)$ and $(c+\epsilon, c+\epsilon)$.
	\end{example}
	
	\begin{example}[Bertrand duopoly with differentiated products]
		Consumers are \ul{uniformly distributed} on a preference line $[0,1]$.\\
		For a firm $i \in N$, let $x_i \in [0,1]$ measure consumer's preference towards firm $i$'s products. \\
		Define 
		\begin{equation}
			x_{-i} \equiv 1 - x_i \sim Unif(0,1)
		\end{equation}
		Consumer buy product $i$ if 
		\begin{gather}
			x_i - p_i \geq x_{-i} - p_{-i}
		\end{gather}
		and purchases product 1 if
		\begin{gather}
			x_i - p_i \leq x_{-i} - p_{-i}
		\end{gather}
		Then solve the boundary $x^* \in [0, 1]$ such that consumers at $x^*$ are indifferent between  two products.
		\begin{gather}
			x + p_i = (1 - x) + p_{-i} \\
			\implies x^* = \frac{1 - p_i + p_{-i}}{2}
		\end{gather}
		So any consumer with $x_i > x^*$ would choose firm $i$'s product, also because consumers are uniformly distributed on $x_i \in [0,1]$. So the portion of consumers buying firm $i$'s product is
		\begin{equation}
			1 - x_i = \frac{1 + p_i - p_{-i}}{2}
		\end{equation}
		And, clearly, 
		The demand function for firm $i$ is
		\begin{gather}
			D_i(q_i, q_{-i}) = \begin{cases}
				0 &\tx{ if } p_i \geq p_{-i} + 1 \\
				\frac{1 + p_i - p_{-i}}{2} &\tx{ if } p_i \in [p_{-i}-1, p_{-i}+1]\\
				1 &\tx{ if } p_i \leq p_{-i} - 1\\
			\end{cases}
		\end{gather}
		Consider the case where $|p_i - p_{-i}| \leq 1$, 
		\begin{gather}
			\pi_i = D_i(p_i) (p_i - c) = \frac{1 + p_i - p_{-i}}{2} (p_i - c)
		\end{gather}
		Take the first order condition
		\begin{gather}
			\pd{\pi_i}{p_i} = 0 \\
			\implies p_i = \frac{p_{-i} + c - 1}{2}
		\end{gather}
	\end{example}
	
	\newpage
	\section*{\red{CONTENTS NOT COVERED IN LECTURE}}
	\begin{definition}[60.2]
				The set $X \subseteq A$ of outcomes of a \ul{finite} strategic game $\langle N, (A_i), (u_i) \rangle$ \textbf{survives iterated elimination of \ul{strictly} dominated actions} if $X = \times_{j \in N} X_j$ and there is a collection $((X_j^t)_{j \in N})_{t=0}^T$ of sets that satisfies the following conditions for each $j \in N$.
				\begin{itemize}
					\item $X_j^0 = A_j$ and $X_j^T = X_j$.
					\item $X_j^{t+1} \subseteq X_j^t$ for each $t = 0, \dots, T - 1$.
					\item For each $t = 0, \dots, T-1$ every action of player $j$ in $X_j^t \backslash X_j^{t+1}$ is \ul{strictly dominated} in the game $\langle N, (X_i^t), (u_i^t) \rangle$, where $u_i^t$ for each $i \in N$ is the function $u_i$ restricted to $\times_{j \in N} X_j^t$.
					\item No action in $X_t^T$ is strictly dominated in game $\langle N, (X_i^T), (u_i^T) \rangle$.
				\end{itemize}
			\end{definition}
			
			\begin{proposition}[61.2]
				If $X = \times_{j \in N}X_j$ survives iterated elimination of strictly dominated actions in a \ul{finite} strategic game $\langle N, (A_i), (u_i) \rangle$ then $X_j$ is the set of player $j$'s rationalizable actions for each $j \in N$.
			\end{proposition}
			
		\subsection{Rationalizability}
			\begin{definition}[pg.15]
				The \textbf{best-response function} for a player $i$ is defined as
				\[
					B_i(a_{-i}) = \{a_i \in A_i : (a_i, a_{-i}) \succsim_i (a_i', a_{-i})\ \forall a_i' \in A_i \}
				\]
			\end{definition}
			
			\begin{remark}
				The best-response of $a_{-i}$ can be written as 
				\[
					B_i(a_{-i}) = \bigcap_{a_i' \in A_i} \{a_i \in A_i : (a_i, a_{-i}) \succsim_i (a_i', a_{-i})\}
				\]
				where each of them is the upper contour set of $a_i'$. \\
				Thus, if $\succsim_i$ is quasi-concave, then $B_i(a_{-i})$ is an intersection of convex sets and therefore itself convex.
			\end{remark}
			\begin{definition}[pg.54]
				A \textbf{belief} of player $i$ (about the actions of the other players) is a \ul{probability measure}, $\mu_i$, on $A_{-i}=\times_{j \in N \backslash \{i\}} A_j$. \\
				$\mu_i$ is a mapping such that
				\begin{itemize}
					\item $\mu_i: A_{-i} \to [0,1]$.
					\item $\mu_i(A_{-i}) = 1$.
					\item For all countable piece-wise \ul{disjoint} collection $\{E_i\}_{i\in I}$, it satisfies the \emph{countable additivity property}:
					\[
						\mu_i(\bigcup_{i \in I} E_i) = \sum_{i \in I}\mu_i(E_i)
					\]
				\end{itemize}
			\end{definition}
			
			\begin{definition}[lec.2]
				For a player $i \in N$, $a_i^* \in A_i$ is the \textbf{best response to belief $\mu_i$} in a strategic game $\langle N, (A_i), (u_i) \rangle$ if and only if 
				\[
					\forall a_i \in A_i,\ \sum_{a_{-i} \in A_{-i}} u_i(a_i^*, a_{-i}) \mu_i(a_{-i}) \geq \sum_{a_{-i} \in A_{-i}} u_i(a_i, a_{-i}) \mu_i(a_{-i})
				\]
				Equivalently,
				\[
					\forall a_i \in A_i,\ \expect{u_i(a_i^*, a_{-i})|\mu_i} \geq \expect{u_i(a_i, a_{-i})|\mu_i}
				\]
			\end{definition}
			
			\begin{definition}[59.1]
				An action of player $i$ in a strategic game is a \textbf{never best response} if it is not a best response to any \ul{belief} of player $i$.
			\end{definition}
			
			\begin{definition}[lec.2]
				For player $i \in N$, action $a_i \in A_i$ is \textbf{rationalizable} if it survives from the iterated elimination of never best responses.
			\end{definition}
			
			\begin{definition}[59.2]
				The action $a_i \in A_i$ of player $i$ in the strategic game $\langle N, (A_i), (u_i) \rangle$ is \textbf{strictly dominated} if there is a mixed strategy $\alpha_i$ of player $i$ such that 
				\[
					U_i(a_{-i}, \alpha_i) > u_i(a_{-i}, a_i)
				\]
				for all $a_{-i} \in A_{-i}$, where $U_i(a_{-i}, \alpha_i)$ is the payoff of player $i$ if he uses the mixed strategy $\alpha_i$ and the other players' vector of actions is $a_{-i}$.
			\end{definition}
		
		
	\section{Lecture 3. Nash Equilibrium}
		\begin{definition}[14.1]
			A \textbf{Nash equilibrium of a strategic game} $\langle N, (A_i), (\succsim_i) \rangle$ is a profile $a^* \in A$ of actions with property that for every player $i \in N$
			\[
				(a_i^*, a^*_{-i}) \succsim_i (a_i, a^*_{-i})\ \forall a_i \in A_i
			\]
		\end{definition}
		
		\begin{proposition}[pg.15, equivalent definition of Nash equilibrium]
			So a Nash equilibrium is a profile $a^* \in A$ such that
			\[
				a^*_i \in B_i(a^*_{-i})\ \forall i \in N
			\]
		\end{proposition}
		
		\begin{proposition}[lec.3]
			No strategy that is eliminated during iterated deletion of never best response can be played in Nash equilibrium.
		\end{proposition}
		
		\begin{lemma}[pg.19]
			A strategic game $\langle N, (A_i), (\succsim_i) \rangle$ has a Nash equilibrium if equivalent to the following statement:\\
			Define set-valued function $B: A \to A$ by 
			\[
				B(a) = \times_{i\in N} B_i (a_{-i})
			\]
			and there exists $a^* \in A$ such that $a^* \in B(a^*)$.
		\end{lemma}
	
		\begin{lemma}[20.1 Kakutani's fixed point theorem]
			Let $X$ be a \ul{compact convex subset} of $\R^n$ and let $f: X \to X$ be a set-valued function for which
			\begin{itemize}
				\item for all $x \in X$ the set $f(x)$ is non-empty and convex.
				\item the graph of $f$ is closed. \emph{(i.e. for all sequences $\{x_n\}$ and $\{y_n\}$ such that $y_n \in f(x_n)$ for all $n$, $x_n \to x$ and $y_n \to y$ then $y \in f(x)$)}
			\end{itemize}
			Then there exists $x^* \in X$ such that $x^* \in f(x^*)$.
		\end{lemma}
		
		\begin{definition}[pg.20]
			A preference relation $\succsim_i$ over $A$ is quasi-concave on $A_i$ if for every $a^* \in A$ the upper contour set over $a^*_i$, given other players' strategies
			\[
				\{a_i \in A_i: (a^*_{-i}, a_i) \succsim_i a^*\}
			\]
			is convex.
		\end{definition}
		
		\begin{proposition}[20.3]
			The strategic game $\langle N, (A_i), (\succsim_i) \rangle$ has a Nash equilibrium if for all $i \in N$,
			\begin{itemize}
				\item the set $A_i$ of actions of player $i$ is a nonempty \ul{compact convex} subset of a Euclidian space
			\end{itemize}
			and the preference relation $\succsim_i$ is
			\begin{itemize}
				\item continuous
				\item quasi-concave on $A_i$.
			\end{itemize}
		\end{proposition}
		
		\begin{proof}
			Let $B: A \to A$ be a correspondence defined as 
			\[
				B(a) := \times_{i \in N} B_i(a_{-i})
			\]
			Note that for each $a \in A$ and for each $i \in N$, \\
			$B_i(a_{-i}) \neq \emptyset$ since preference $\succsim_i$ is continuous and $A_i$ is compact (EVT). \\
			Also $B_i(a_{-i})$ is convex since it's basically an intersection of  upper contour sets and each of those upper contour is convex since $\succsim_i$ is quasi-concave. \\
			So the Cartesian product of the finite collection of $B_i$ is non-empty and convex. \\
			Also the graph $B$ is closed since $\succsim_i$ is continuous. \\
			So there exists $a^* \in A$ such that $a^* \in B(a^*)$. \\
			So Nash equilibrium presents.
		\end{proof}
		
		\begin{definition}[lec.3]
			A \textbf{strict Nash equilibrium} is an action profile $a^* \in A$ where all players are playing their \ul{unique} best response. That is, for every player $i \in N$, the image of their best response $B_i(a_{-i}^*)$ is singleton,
			\[
				\forall i \in N\ B_i(a_{-i}^*) = \{a_i^*\}
			\]
		\end{definition}
		
		\begin{definition}[lec.3]
			Otherwise, a Nash equilibrium is a \textbf{weak Nash equilibrium}.
		\end{definition}

	\section{Lecture 4. Nash Equilibrium: Examples}
	\section{Lecture 5. Mixed Strategies}
		\begin{notation}[pg.32]
			Let $\Delta(A_i)$ denote the \ul{set of probability measures/distributions} on set $A_i$.
		\end{notation}
		\begin{definition}[lec.5]
			For player $i \in N$, a \textbf{mixed strategy} $\sigma_i$ is a member in $\Delta(A_i)$ and it is a \ul{probability distribution} over $A_i$.
		\end{definition}
		
		\begin{remark}[lec.5]
			A pure strategy $a_i \in A_i$ is a mixed strategy with 
			\[
				\sigma_i(a_i) = 1
			\]
			So mixed strategy is a generalization of pure strategy.
		\end{remark}
		
		\begin{definition}[pg.32]
			A profile $(\sigma_j)_{j\in N}$ of mixed strategies induces a probability distribution over the set $A$.
		\end{definition}
		
		\begin{proposition}[pg.32]
			In a finite game,(i.e., each $A_i$ is finite), then given the independence of randomization, the probability of the action profile $a = (a_j)_{j \in N}$ to be realized given mixed strategy profile $(\sigma_j)_{j\in N}$ is
			\[
				Pr((a_j)_{j \in N}) = \prod_{j \in N} \sigma_j(a_j)
			\]
			and for player $i$, the \textbf{expected payoff} on profile $(\sigma_j)_{j\in N}$ is 
			\[
				U_i((\sigma_{j})_{j \in N}) = \sum_{a \in A} (\prod_{j \in N} \sigma_j(a_j)) u_i(a) = \expect{u_i(a)|(\sigma_j)_{j \in N}}
			\]
		\end{proposition}
		
		\begin{proposition}[lec.5, equivalent]
			The \textbf{expected payoff} from mixed strategy profile $(\sigma_i) \equiv (\sigma_i, \sigma_{-i})$ is
			\[
				U_i(\sigma_i, \sigma_{-i}) \equiv \expect{u_i(a)|(\sigma_i)} = \sum_{a_i \in A_i} \sum_{a_{-i} \in A_{-i}} u_i(a_i, a_{-i}) \sigma_{-i}(a_{-i}) \sigma_i(a_i)
			\]
		\end{proposition}
		
		\begin{definition}[32.1]
			The \textbf{mixed extension} of the strategic game $\langle N, (A_i), (u_i) \rangle$ is the strategic game $\langle N, (\Delta(A_i)), (U_i) \rangle$ in which $\Delta(A_i)$ is the set of probability distributions over $A_i$ and $U_i: \times_{j \in N} \Delta(A_i) \to \R$ assigns to each $(\sigma_i)_{i \in N} \in \times_{j \in N} \Delta(A_i)$ the \ul{expected value} under $u_i$ of the lottery over $A$ that is induced by $(\sigma_i)_{i \in N}$.
		\end{definition}
		
		\begin{remark}[pg.32, notes on above definition]
			If the game is finite, that is, for each $i \in N$, the set $A_i$ is finite, then
			\[
				U_i(\sigma) = \sum_{a \in A} (\prod_{j \in N} \sigma_j(a_j)) u_i(a)
			\]
		\end{remark}
		
		\begin{definition}[32.3]
			A \textbf{mixed strategy Nash equilibrium of a strategic game} is a Nash equilibrium of its mixed extension.
		\end{definition}
		
		\begin{proposition}[33.1]
			Every \ul{finite} strategic game has a mixed strategy Nash equilibrium.
		\end{proposition}
		
		\begin{lemma}[33.2]
			Let $G = \langle N, (A_i), (u_i) \rangle$ be a \ul{finite} strategic game. Then $\sigma^* \in \times_{i \in N}\Delta(A_i)$ is a mixed strategy Nash equilibrium of $G$ is and only if for every player $i \in N$ every pure strategy in the \ul{support} of $\sigma_i^*$ is a best response to $\sigma_{-i}^*$
		\end{lemma}
		
		\begin{assumption}[lec.5]
			Assuming all agents follows Von-Neumann Morgenstern theorem. 
		\end{assumption}
		
		\begin{definition}[lec.5]
			An action $a_i$ is \textbf{strictly dominated} by mixed strategy $\sigma_i$ if and only if
			\[
				\forall a_{-i} \in A_{-i} \ u_i(a_i, a_{-i}) < U_i(\sigma_i, a_{-i})
			\]
			where $\sigma_i$ could be a pure strategy.
		\end{definition}
		
		\begin{definition}[lec.5]
			A mixed strategy $\sigma_i$ is a \textbf{best response} to $\sigma_{-i}$ if and only if
			\[
				\forall \sigma_i' \in \Delta(A_i)\ U_i(\sigma_i, \sigma_{-i}) \geq U_i(\sigma_i', \sigma_{-i})
			\]
		\end{definition}
		
		\begin{definition}[lec.5]
			The \textbf{support} of a mixed strategy $\sigma_i \in \Delta(A_i)$ is the set
			\[
				supp(\sigma_i) = \{a_i \in A_i:\sigma_i(a_i) \neq 0\}
			\]
		\end{definition}
		
		\begin{proposition}[lec.5]
			A mixed strategy $\sigma_i$ is a \textbf{best response} to an strategy profile $\sigma_{-i}$ if and only if 
			\begin{enumerate}[(a)]
				\item Player $i$ is indifferent between all $a_i$ in the support of $\sigma_i$,
				\[
					\forall a_j, a_j' \in supp(\sigma_i)\quad a_j \sim_i a_j'
				\]
				\item \ul{and} player $i$ weakly prefers all actions in the support of $\sigma_i$ to those not in the support of $\sigma_i$. That's
				\[
					\forall a_j \in supp(\sigma_i),\ \forall a_j' \notin supp(\sigma_i)\quad a_j \succsim_i a_j'
				\]
			\end{enumerate}
		\end{proposition}
		
		\begin{proof}
			($\implies$) show the if parts by proving it's contraposition. \\
			Suppose (a) is not true, then
			\[
				\exists a_i, a_j \in supp(\sigma_i)\ s.t.\ a_i \centernot \sim_i a_j
			\]
			WLOG, suppose
			\[
				u_i(a_i, \sigma_{-i}) > u_i(a_j, \sigma_{-i})
			\]
			then $\sigma_i$ would not be the best response since we can refine it by assigning 
			\[
				\begin{cases}
					\sigma_i' (a_i) = \sigma_i(a_i) + \sigma_i(a_j) \\
					\sigma_i' (a_j) = 0 \\
					\sigma_i' (a_k) = \sigma_i(a_k)\ \tx{otherwise}
				\end{cases}
			\]
			and $\sigma_i'$ would provides higher expected payoff. \\
			Suppose (b) does not hold, \\
			\[
				\exists a_i \notin supp(\sigma_i)\ s.t.\ \exists a_j \in supp(\sigma_i)\ s.t\ u_i(a_i, \sigma_{-i}) > u_i(a_j, \sigma_{-i})
			\]
			Then $\sigma_i$ could not be a best response since we can construct another mixed strategy $\sigma_i'$ strictly dominating $\sigma_i$ by setting
			\[
				\begin{cases}
					\sigma_i' (a_j) = 0 \\
					\sigma_i' (a_i) = \sigma_i(a_j) \\
					\sigma_i' (a_k) = \sigma_i(a_k)\ \tx{otherwise}
				\end{cases}
			\] \\
			($\impliedby$) Assuming $\sigma_i$ is not a best response towards $\sigma_{-i}$,\\
			then there exists $\sigma_i' \in \Delta(A_i)$ such that\\
			\begin{gather*}
				U_i(\sigma_i', \sigma_{-i}) > U_i(\sigma_i, \sigma_{-i}) \\
				\iff \expect{u_i(a)|(\sigma_i', \sigma_{-i})} > \expect{u_i(a)|(\sigma_i, \sigma_{-i})} \\
				\iff \sum_{a_i \in A_i} \sum_{a_{-i} \in A_{-i}} u_i(a_i, a_{-i}) \sigma_i'(a_i) \sigma_{-i}(a_{-i}) 
				> \sum_{a_i \in A_i} \sum_{a_{-i} \in A_{-i}} u_i(a_i, a_{-i}) \sigma_i(a_i) \sigma_{-i}(a_{-i})
			\end{gather*}
			Probability measures $\sigma_i$ and $\sigma_i'$ could only be different in two aspects, their supports and the values assigned on elements in their supports, this fails assumption (a). \\
			\textcolor{red}{The following argument needs to be revised.}\\
			\textbf{Case 1} suppose $supp(\sigma_i) = supp(\sigma_i')$, then the strictly inequality in expected payoffs implies redistributing probabilities does affect the expected payoffs.\\
			So player $i$ cannot be indifferent between any two actions in the support. \\
			\textbf{Case 2} suppose $supp(\sigma_i) \neq supp(\sigma_i')$ and $supp(\sigma_i') \centernot \subseteq supp(\sigma_i)$. That's
			\[
				\exists a_i \in supp(\sigma_i') \land \notin supp(\sigma_i)
			\]
			Then extending the support to $a_i$ of $\sigma_i$ gives higher expected payoff, this fails the assumption (b). \\
			\textbf{Case 3} suppose $supp(\sigma_i') \subsetneq supp(\sigma_i)$. Then the expected payoff can be strictly increased by eliminating actions in $supp(\sigma_i) \backslash supp(\sigma_i')$. Then those actions eliminated must be strictly dominated by actions in $supp(\sigma_i')$. This fails assumption (a).
		\end{proof}
		
		
		\begin{proposition}[lec.5 equivalent proposition]
			All actions in the support are best responses. (i.e.\emph{best response mixed strategy is a mixture of best response pure actions})
		\end{proposition}
		
		\begin{remark}[lec.5 Intuition of proposition]
			If the requirements of above proposition are not satisfied, the player can reduce the probability assigned to the non-best-response pure action and better off.
		\end{remark}
		
		\begin{theorem}[lec.5 Nash's Theorem]
			Any player $i \in N$ in finite game $\langle N, (A_i), (\succsim_i) \rangle$ has a mixed strategy Nash equilibrium.
		\end{theorem}
		
	\section{Lecture 6. Extensive Form Games and Subgame Perfection}
		\subsection{Extensive Form Game}
			\begin{definition}[89.1]
				An \textbf{extensive game with perfect information} has the following components.
				\begin{itemize}
					\item A set $N$ of \textbf{players}.
					\item A set $H$ of sequences (finite or infinite) of \textbf{histories} with properties:
						\begin{itemize}
							\item $\emptyset \in H$.
							\item For all $L < K$, $(a^k)_{k=1,2,\dots,K} \in H \implies (a^k)_{k=1,2,\dots,L} \in H$.
							\item For infinite sequence $(a^k)_{k=1}^\infty$,\\ $(a^k)_{k=1,2,\dots,L} \in H,\ \forall L \in \Z_{++} \implies (a^k)_{k=1}^\infty \in H$.
						\end{itemize} And each component of history $h \in H$ is an \textbf{action} taken by a player.
					\item A function $P: H \backslash Z \to N$, where for $h \in H$, $P(h) \in N$ is defined by the player who takes an action \ul{after} the history $h$.
					\item For each player $i \in N$ a \textbf{preference relation} $\succsim_i$ \ul{defined on $Z$}.
				\end{itemize}
			\end{definition}
			
			\begin{notation}[pg.90]
				An extensive game with perfect information can be represented by a 4-tuple, $\langle N, H, P, (\succsim_i) \rangle$. \emph{Sometimes it is convenient to specify the structure of an extensive game without specifying the players' preference, as $\langle N, H, P \rangle$}.
			\end{notation}
			
			\begin{definition}[pg.90]
				A history $(a^k)_{k=1,2,\dots,K} \in H$ is \textbf{terminal} if
				\begin{enumerate}
					\item it is infinite,
					\item \ul{or} (i.e. it cannot be extended to another valid history sequence)
					\[
						\forall a^{K+1},\ (a^k)_{k=1,2,\dots,K+1} \notin H
					\]
				\end{enumerate}
				The set of terminal histories is denoted by $Z$.
			\end{definition}
			
			\begin{notation}[pg.90, the action set]
				After any nonterminal history, $h \in H \backslash Z$, the player $P(h)$ chooses an action from set
				\[
					A(h) = \{a: (h, a) \in H\}
				\]
			\end{notation}
			
			\begin{remark}
				Note that all player function, action set and player preference relation are defined on $H$. Thus, unlike a normal form game, which was \emph{player oriented}, we'd better consider an extensive form game as \emph{history oriented}.
			\end{remark}
			
			\begin{definition}[pg.90]
				We refer to the empty set, which is required to be an element of $H$, as the \textbf{initial history}.
			\end{definition}
			
			\begin{definition}[92.1]
				A \textbf{strategy of player} $i \in N$, $s_i$, in an extensive game with perfect information $\langle N, H, P, (\succsim_i) \rangle$ is a function that assigns an action in $A(h)$ to each nonterminal history $h \in H \backslash Z$ for which $P(h) = i$.
			\end{definition}
			
			\begin{remark}[pg.92]
				A strategy specifies the action chosen by a player for \emph{every} history after which it is his turn to move, \emph{even for histories that is, if the strategy is followed, are never reached}.
			\end{remark}
			
			\begin{definition}[pg.93]
				For each strategy profile $s = (s_i)_{i \in N}$ in the extensive game $\langle N, H, P, (\succsim_i) \rangle$, the \textbf{outcome} of $s$, $O(s)$, is defined as the \ul{terminal history} that results when each player $i \in N$ follows the precepts of $s_i$. That is, $O(s)$ is the (possibly infinite) history
				\[
					(a^1, \dots, a^K) \in Z
				\]
				such that
				\[
					\forall k \in \{0, 1, \dots K-1\},\ s_{P(a^1, \dots, a^k)}(a^1, \dots, a^k) = a^{k+1}
				\]
			\end{definition}
			
			\begin{definition}[lec.6]
				A extensive game $\Gamma = \langle N, H, P, (\succsim_i) \rangle$ is finite if and only if 
				\begin{enumerate}[(a)]
					\item $N$ is finite.
					\item $(A_i)$ are all finite.
					\item All $h \in H$ reach the terminal state with finite length.
				\end{enumerate}
			\end{definition}
			
			\begin{definition}[93.1] A \textbf{Nash equilibrium of an extensive game with perfect information} $\langle N, H, P, (\succsim_i) \rangle$ is a strategy profile $s^*$ such that for every player $i \in N$ we have
			\[
				\forall s_i \in S_i,\ O(s_{-i}^*, s_i^*) \succsim_i O(s_{-i}^*, s_i)
			\]
			\end{definition}
			
			\begin{definition}[94.1] The \textbf{strategic form of the extensive game with perfect information}, $\Gamma=\langle N, H, P, (\succsim_i) \rangle$, is the strategic game $\langle N, (S_i), (\succsim_i') \rangle$ in which for each player $i \in N$
				\begin{itemize}
					\item $S_i$ is the \textbf{set of strategies} of player $i$ in $\Gamma$.
					\item $\succsim_i'$ is defined on $\times_{i \in N}S_i$ and defined by
					\[
						\forall s, s' \in \times_{i \in N}S_i,\ s \succsim_i' s' \iff O(s) \succsim_i O(s')
					\]
				\end{itemize}
			\end{definition}
			
			\begin{definition}[pg.94]
				A \textbf{reduced strategy} of player $i$ is defined to be a function $f_i$ whose domain is a \emph{subset} of $\{h \in H: P(h) = i\}$ and has the following properties
				\begin{enumerate}
					\item it associates with every history $h$ in the domain of $f_i$ an action in $A(h)$.
					\item a history $h$ with $P(h) = i$ is in the domain of $f_i$ if and only if all the actions of player $i$ in $h$ are those dictated by $f_i$. (i.e., for any $h = (a^k)$ and for any $h' = (a^k)_{k=1}^L$ as a subsequence of $h$ such that $P(h')=i$, $f_i(h') = a^{L+1}$.)
				\end{enumerate}
			\end{definition}
			\begin{remark}[pg.94]
				Each \textbf{reduced strategy} of player $i$ corresponds to a \ul{set of strategies of player $i$}, such that for each vector of strategies of the \ul{other players} each strategy in this set yields the \ul{same outcome}. (strategies in the same set are \textbf{outcome-equivalent}.) \\
				That's, for each strategy $s_i \in S_i$, its reduced strategy can be defined with an outcome equivalence class, $[s_i]$,
				\[
					[s_i] \equiv \{
					s'_i \in S_i: \forall s_{-i} \in \times_{j \in N \backslash \{i\}} S_j,\ O(s_{-i}, s_i) \textcolor{red}{=} O(s_{-i}, s_i')
					\}
				\]
				But in some other game, the definition of outcome-equivalence is more general and defined by \ul{generating the same payoff} (through possibly difference outcomes), then the reduced strategy is defined as
				\[
					[s_i] \equiv \{
					s'_i \in S_i: \forall s_{-i} \in \times_{j \in N \backslash \{i\}} S_j,\ \forall j \in N,\ O(s_{-i}, s_i) \textcolor{red}{\sim_j} O(s_{-i}, s_i')
					\}
				\]
			\end{remark}
			
			\begin{definition}[95.1.1]
				Let $\Gamma = \langle N, H, P, (\succsim_i) \rangle$ be an extensive game with perfect information and let $\langle N, (S_i), (\succsim_i') \rangle$ be its strategic form. For any $i \in N$ define the strategies $s_i, s_i' \in S_i$ to be \textbf{equivalent} if 
				\[
					\forall s_{-i} \in S_{-i},\ \forall j \in N,\ (s_{-i}, s_i) \sim_j' (s_{-i}, s_i')
				\]
			\end{definition}
			
			\begin{definition}[95.1.2]
				The \textbf{reduced strategic form of} $\Gamma$ is the strategic game $\langle N, (S_i'), (\succsim_i'') \rangle$ in which for each $i \in N$ each set $S_i'$ contains one member of each set of equivalent strategies in $S_i$ and $\succsim_{i}''$ is the preference ordering over $\times_{j \in N} S_j'$ induced by $\succsim_i'$.
			\end{definition}
		\subsection{Subgame Perfection}
			\begin{definition}[97.1]
				The \textbf{subgame of extensive game with perfect information $\Gamma = \langle N, H, P, (\succsim_i) \rangle$ that follows the history $h$} is the extensive  game $\Gamma(h) = \langle N, H|_h, P|_h, (\succsim_i|_h) \rangle$ where
				\begin{itemize}
					\item $H|_h$ is the set of sequences $h'$ such that $(h, h') \in H$.
					\item $P|_h$ is defined by $P|_h(h') = P(h, h')$ for each $h' \in H|_h$.
					\item $\succsim_i|_h$ is defined by $h' \succsim_i|_h h'' \iff (h, h') \succsim_i (h, h'') \in Z$.
				\end{itemize}
			\end{definition}
			
			\begin{notation}[pg.97]
				Given strategy $s_i \in S_i$ and $h \in H \in \Gamma$, $s_i|_h$ represents the \textbf{strategy that $s_i$ induces in the subgame $\Gamma(h)$}. That's, for each $h' \in H_h$
				\[
					s_i|_h (h') \equiv s_i(h, h')
				\]
			\end{notation}
			
			\begin{notation}
				Let $O_h$ denote the \textbf{outcome function of $\Gamma(h)$}, that's, for all $h' \in H|_h$,
				\[
					O_h(h') \equiv O(h, h')
				\]
			\end{notation}
			
			\begin{definition}[97.2]
				A \textbf{subgame perfect equilibrium of an extensive game with perfect information $\Gamma = \langle N, H, P, (\succsim_i) \rangle$} is a strategy profile $s^*$ such that for every player $i \in N$ and every nonterminal history $h \in H \backslash Z$ for which $P(h) = i$ we have
				\[
					O_h (s_{-i}^*|_h, s_{i}^*|_h) \succsim_i|_h O_h (s_{-i}^*|_h, s_{i}|_h)
				\]
				for every strategy $s_i$ of player $i$ in the subgame $\Gamma(h)$.
			\end{definition}
			
			\begin{definition}[pg.97]Equivalently, define SPNE to be a strategy profile $s^*$ in $\Gamma$ for which for any history $h \in H$ the strategy profile $s^*|_h$ is a Nash equilibrium of the subgame $\Gamma(h)$.
				
			\end{definition}
			
			\begin{remark}[pg. 97]
				The notion of SPNE requires the action prescribed by each player's strategy to be optimal, given other players' strategies, after \emph{every} history.
			\end{remark}
			
			\begin{proposition}[99.2]
				Every finite extensive game with perfect information has a subgame perfect equilibrium.
			\end{proposition}
	\section{Lecture 7. Extensive Form Games: Examples}
	\section{Lecture 8. Repeated Games}
	\section{Lecture 9. Game with Incomplete Information}
	\section{Lecture 10. Game with Incomplete Information II}
	\section{Lecture 11. Auctions}
\end{document}
















