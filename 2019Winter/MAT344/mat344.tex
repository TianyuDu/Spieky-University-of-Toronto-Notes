\documentclass{article}
\usepackage{spikey}
\usepackage{amsmath}
\usepackage{amssymb}
\usepackage{soul}
\usepackage{float}
\usepackage{graphicx}
\usepackage{hyperref}
\usepackage{xcolor}
\usepackage{chngcntr}
\usepackage{centernot}
\usepackage[shortlabels]{enumitem}
\usepackage[margin=1truein]{geometry}
\usepackage{tkz-graph}
%\GraphInit[vstyle = Shade]

\counterwithin{equation}{section}
\counterwithin{figure}{section}

\def\Z{{\mathbb Z}}
\def\Q{{\mathbb Q}}
\def\R{{\mathbb R}}
\def\C{{\mathbb C}}

\newcommand{\bi}[2]{\begin{pmatrix}{#1}\\{#2}\end{pmatrix}}
%\counterwithin{equation}{section}


\title{MAT 344 Lecture Notes}
\date{\today}
\author{Tianyu Du}
\begin{document}
	\maketitle
	\tableofcontents
	\newpage
	
	\section{Strings, Sets, and Binomial Coefficients}
		\subsection{Strings and Sets}
			\begin{notation}
				Let $n \in \Z_{++}$, and we use $[n]$ to denote the $n$-element set $\{1,2,\dots,n\}$.
			\end{notation}
			
			\begin{definition}
				Let $X$ be a set, then an $X$-\textbf{string of length} (or a \textbf{word}/\textbf{array}) $n$ is a function $s:[n] \to X$, and $X$ is called the \textbf{alphabet} of the string, and each $x \in X$ is called a \textbf{character} or \text{letter}.
			\end{definition}
			
			\begin{remark}
				An $X$-string defined by $s: [n] \to X$ with length $n$ can be equivalently defined as a \textbf{sequence} consisting elements in $X$.
				\begin{equation}
					s(1)s(2)\dots s(n)
				\end{equation}
			\end{remark}
			
			\begin{definition}
				In the case $X = \{0, 1\}$, strings generated from $X$ are called \textbf{binary strings}. When $X = \{0,1,2\}$, strings are called \textbf{ternary strings}.
			\end{definition}
			
			\begin{definition}
				Let $X$ be a \emph{finite} set and let $n \in \Z_{++}$. An $X$-string $s = x_1 x_2 \dots x_n$ is a \textbf{permutation} of size $m$ if $x_i \neq x_j\ \forall x_i, x_j \in s$.
			\end{definition}
			
			\begin{proposition}
				If $X$ is an $m$-element set and $m \geq n \in \Z_{++}$, then the number of $X$-strings of length $n$ that are permutations is 
				\begin{equation}
					P(m, n) \equiv \frac{m!}{(m-n)!}
				\end{equation}
			\end{proposition}
			
			\begin{definition}
				Let $X$ be a \emph{finite} set and let $0 \leq k \leq |X|$. Then $S \subseteq X$ with $|S| = k$ is a \textbf{combination} of size $k$.
			\end{definition}
			
			\begin{proposition}
				Let $n, k \in \Z$ such that $0 \leq k \leq n$, then the number of combinations is
				\begin{equation}
					\bi{n}{k} \equiv \frac{P(n, k)}{n!} = \frac{n!}{k!(n-k)!}
				\end{equation}
			\end{proposition}
			
			\begin{proposition}
				For all integers $n$ and $k$ with $0 \leq k \leq n$
				\begin{equation}
					\bi{n}{k} = \bi{n}{n-k}
				\end{equation}
			\end{proposition}
			
			\begin{example}
				Binomial coefficients can be used to find the number of integer solutions of
				\begin{equation}
					\sum_{i=1}^k x_i \leq N
				\end{equation}
				given appropriate integers $k, N \in \Z$.
				\begin{enumerate}[(i)]
					\item $x_i > 0\ \forall i \in [k]$ and equality holds, then $C(N-1, k-1)$.
					\item $x_i \geq 0\ \forall i \in [k]$ and equality holds, then $C(N+k-1, k-1)$.\footnote{Simulate choosing $x_i + 1$ instead of $x_i$.}
					\item $x_i > 0\ \forall i \neq j, x_j = Z$ and equality holds, then $C(N-Z+k-2,k-2)$.
					\item $x_i > 0\ \forall i \in [k]$ and strict inequality holds, then $C(N-1, k)$.\footnote{Image there is a placeholder $x_{k+1} > 0$.}
					\item $x_i \geq 0\ \forall i \in [k]$ and strict inequality holds, then $C(N+k-1, k)$.
					\item $x_i \geq 0\ \forall i \in [k]$ and \emph{weak} inequality holds, $C(N+k, k)$ \footnote{This can be calculated by adding case (ii) and case (v) together, and apply Pascal's identity}.
					\begin{gather}
						\bi{N+k-1}{k-1} + \bi{N+k-1}{k} = \bi{N+k}{k}
					\end{gather}
				\end{enumerate}
			\end{example}
			
		\begin{definition}
			Define a \textbf{plane} as $\Z^2$, then a \textbf{lattice path} in the plane is a \emph{sequence} of elements in $\Z^2$
			\begin{equation}
				((x_i, y_i))_{i=1}^t
			\end{equation}
			such that for every $i \in \{1,\dots,t-1\}$, either
			\begin{enumerate}[(i)]
				\item (\emph{Horizontal move}) $x_{i+1} = x_i + 1 \land y_{i+1} = y_i$
				\item Or (\emph{vertical move}) $x_{i+1} = x_i \land y_{i+1} = y_i + 1$
			\end{enumerate}
		\end{definition}
		
		\begin{lemma}
			Let $(p, q), (m, n) \in \Z^2$, then the number of lattice paths from $(p, q)$ to $(m, n)$ is 
			\begin{equation}
				\bi{(p-m) + (q-n)}{p-m}
			\end{equation}
			\begin{proof}
				The lattice is isomorphic to a ${H, V}$-string with length $(p-m)+(q-n)$. There are exactly $p-m$ horizontal moves as well as exactly $q-n$ vertical moves.
			\end{proof}
		\end{lemma}
		
		\begin{theorem}
			Given $n \in \Z_+$, the number of lattice paths from $(0,0)$ to $(n,n)$ which \emph{never go above the diagonal line} is the \textbf{Catalan number}
			\begin{equation}
				C(n) \equiv \frac{1}{n+1} \bi{2n}{n}
			\end{equation}
			\begin{proof}
				Omitted
			\end{proof}
		\end{theorem}
		
		\begin{theorem}[Binomial Theorem]
			Let $x, y \in \R$, then $\forall n \in \Z_+$
			\begin{equation}
				(x + y)^n = \sum_{i=0}^n \bi{n}{i} x^{n-i} y^i
			\end{equation}
		\end{theorem}
		
		\begin{theorem}[Multinomial Theorem]
			Let $r \in \Z_+$, $\{x_i\}_{i=1}^r \in \mc{P}(\R)$. Then for every $n \in \Z_+$,
			\begin{equation}
				(\sum_{i=1}^r x_i)^n = \sum_{|\alpha|\red{=}n} \bi{n}{\alpha} (x_i)^\alpha
			\end{equation}
			where $\alpha \equiv (\alpha_i)_{i=1}^r,\ \alpha_i \in \Z_{++}\ \forall i$ is a \textbf{multi-index}, and 
			\begin{gather}
				(x_i)^\alpha \equiv \sum_{i=1}^r x_i^{\alpha_i} \\
				|\alpha| \equiv \sum_{i=1}^r \alpha_i \\
				\bi{n}{\alpha} \equiv \frac{n!}{\alpha_1!\alpha_2!\dots\alpha_r!}
			\end{gather}
		\end{theorem}
		
	\section{Induction}
		\begin{theorem}[Well-Ordering Principle]
			Every non-empty set of $Z_{++}$ has a least element.
			\begin{proof}
				Prove using principle of mathematical induction and contradiction.
			\end{proof}
		\end{theorem}
		
		\begin{definition}
			\textbf{Recursive definition}
		\end{definition}
		
		\begin{theorem}[The Principle of Mathematical Induction]
			If $S$ is any set of natural numbers with properties that
			\begin{enumerate}
				\item 1 is in $S$, and
				\item $k+1$ is in $S$ whenever $k$ is any number in $S$.
			\end{enumerate}
			then $S = \Z_+$.
		\end{theorem}
		
		\begin{remark}
			\textbf{Recursive definitions} can also be recast as \textbf{inductive definitions}.
		\end{remark}
		
		\begin{definition}[Summation]
			Summation operator beginning with index 1, $\sum: \mc{F}_1 \times Z_{++} \to \R$, where $\mc{F}_1$ is the set of unary real-valued functions, is defined inductively as
			\begin{align}
				\sum_{i=1}^1 f(i) &\equiv f(1) \\
				\sum_{i=1}^{k+1} f(i) &\equiv \sum_{i=1}^k f(i) + f(k+1)
			\end{align}
		\end{definition}
		
		
		\begin{theorem}[The Principle of Complete Mathematical Induction]
			If $S$ is any set of natural numbers with the properties that
			\begin{enumerate}
				\item $1 \in S$, and
				\item $\{1, 2, \dots, k\} \subset S \implies k+1 \in S$,
			\end{enumerate}
			then $S = \Z_+$.
		\end{theorem}
	
	\section{Pigeon Hole Principle and Complexity}
		\subsection{Pigeon Hole Principle}
			\begin{theorem}
				Let $f: X \to Y$ be a function, then 
				\begin{equation}
					f \tx{ injective} \implies |X| \leq |Y|
				\end{equation}
			\end{theorem}
			
			\begin{theorem}[Pigeon Hole Principle]
				Let $f: X \to Y$, and suppose $|X| > |Y|$, then $f$ is not injective, that's
				\begin{equation}
					\exists x_1 \neq x_2 \in X\ s.t.\ f(x_1) = f(x_2)
				\end{equation}
				\begin{proof}
					Contrapositive form of the theorem 3.1
				\end{proof}
			\end{theorem}
			
			\begin{theorem}[Erods/Szekeres]
				Let $m, n \in \Z_+$, then any sequence of $mn + 1$ \emph{distinct} real numbers either
				\begin{enumerate}[(i)]
					\item has an increasing subsequence of $m+1$ terms,
					\item or it has a decreasing subsequence of $n+1$ terms.
				\end{enumerate}
				\begin{proof}
					Let $\sigma = (x_1, x_2, \dots, x_{mn+1})$ be a sequence with length $mn+1$ consisting of distinct reals.\\
					For each $i \in [mn+1]$ define $a_i$ as the maximum length of an increasing subsequence of $\sigma$ \emph{beginning with} $x_i$. \\
					Define $b_i$ as the maximum length of a decreasing subsequence of $\sigma$ \emph{ending with} $x_i$. \\
					\textbf{Case (i)}
					\begin{gather}
						\exists i \in [mn+1]\ s.t.\ a_i \geq m+1 \lor b_i \geq n+1
					\end{gather}
					then the theorem is proven. \\
					\textbf{Case (ii)} Suppose otherwise
					\begin{gather}
						\forall i \in [mn+1]\ a_i \leq m \land b_i \leq n
					\end{gather}
					construct function $f: [mn+1] \to [m]\times[n]$ defined as
					\begin{equation}
						f(i) \equiv (a_i, b_i)
					\end{equation}
					Note that $|[mn+1]| > |[m]\times[n]|$ so $f$ cannot be injective, so there exists $j\neq k \in [mn+1]$ such that $(a_j, b_j) = (a_k, b_k)$. \\
					WLOG, assume $j < k$.\\
					Since all elements in $\sigma$ are distinct, $j \neq k \implies x_j \neq x_k$. \\
					\textbf{Sub-case (i)} $x_j < x_k$, then any increasing subsequence beginning with $x_k$ can be extended by prepending $x_j$, so $a_j > a_k$. \\
					\textbf{Sub-case (ii)} $x_j > x_k$, then any decreasing subsequence ending with $x_j$ can be extended by appending $x_k$, so $b_k > b_j$. \\
					Either sub-case leads to a contradiction, so \textbf{case (ii)} is impossible.
				\end{proof}
			\end{theorem}
		\subsection{Complexity}
			\begin{definition}
				Let $f, g: \N \to \R$ be a function, then the \textbf{big oh} $\mc{O}(f)$ is a collection of functions such that, for every $g \in \mc{O}(f)$
				\begin{gather}
					\exists c \in \R, n^* \in \N\ s.t.\ \forall n \in \N, n \geq n^* \implies g(n) \leq c f(n)
				\end{gather}
			\end{definition}
			
			\begin{definition}
				Let $f, g: \N \to \R$ be a function. If $f(n) > 0\ \forall n \in \N$,
				then the \textbf{little oh} $o(f)$ is the collection of functions such that, for every $g \in o(f)$, 
				\begin{equation}
					\lim_{n \to \infty} \frac{g(n)}{f(n)} = 0
				\end{equation}
			\end{definition}
			
			\begin{definition}
				Let $f, g: \N \to \R$, then the \textbf{little oh}, $o(f)$ is defined as the collection of functions such that $g \in o(f)$ if and only if
				\begin{gather}
					\exists c \in \R, n^* \in \N,\ s.t.\ \forall n \in \N, n \geq n^* \implies |g(n)| < c|f(n)|
				\end{gather}
			\end{definition}
			
			\begin{definition}
				Define $\pi: \Z_{++} \to \Z_+$ as $\pi(n) \equiv$ \emph{the number of primes among the first $n$ positive integers}.
			\end{definition}
			
			\begin{theorem}[Prime Number Theorem] $\pi(n)$ grows at a rate the same as $\frac{n}{\ln(n)}$. That's
				\begin{equation}
					\lim_{n \to \infty} \pi(n) \frac{\ln(n)}{n} = 1
				\end{equation}
			\end{theorem}
			
			\begin{definition}
				The class of \textbf{polynomial time} problems, denoted as $\mc{P}$, is the set of decision problems for which there exists one polynomial run time algorithm as the solution.
			\end{definition}
			
			\begin{definition}
				The class of \textbf{nondeterministic polynomial time} problems, denoted as $\mc{NP}$, is the set of decision problems for which there is a certificate for a yes answer whose correctness can be verified in polynomial time.
			\end{definition}
\end{document}





































