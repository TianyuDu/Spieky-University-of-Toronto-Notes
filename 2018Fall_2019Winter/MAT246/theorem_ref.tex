\documentclass[11pt]{article}

\author{Tianyu Du}
\date{\today}
\title{MAT246: Concepts in Abstract Mathematics: \\ \small Theorem Quick Reference Sheet}

\usepackage{amsmath}
\usepackage{amssymb}
\usepackage{float}
\usepackage{soul}
\usepackage{spikey}
\usepackage{xcolor}
\usepackage{centernot}
\usepackage{txfonts}
\usepackage{graphicx}

\usepackage[
    type={CC},
    modifier={by-nc},
    version={4.0},
]{doclicense}

\begin{document}
	\maketitle
	\doclicenseThis
	\tableofcontents
	
	\section{Introduction to the Natural Numbers}
		\begin{lemma}[1.1.1]
			Every natural number greater than 1 has a prime divisor.
		\end{lemma}
		\begin{proof}
			Decompose iteratively if composite.
		\end{proof}
		
		\begin{theorem}[1.1.2]
			There is no largest prime number.
		\end{theorem}
		\begin{proof}
			Let $S$ be the finite set containing all primes. \\
			Consider $M=p_1 p_2 \dots p_n + 1 \notin S$ has no prime divisor, contradiction.
		\end{proof}
	
	\section{Mathematical Induction}
		\begin{theorem}[The Principle of Mathematical Induction 2.1.1]
			If $S$ is any set of natural numbers with properties that
			\begin{enumerate}
				\item 1 is in $S$, and
				\item $k+1$ is in $S$ whenever $k$ is any number in $S$.
			\end{enumerate}
			then $S$ is the set of all natural numbers.
		\end{theorem}
		\begin{proof}
			Let $T = S^c$ and suppose $T \neq \emptyset$. By WOP, let $t = \min T$. \\
			Then by definition of minimum, $t - 1 \notin T$, i.e. $t - 1 \in S$. \\
			By assumption of PMI, $t-1+1=t \in S$, contradiction. \\
			$T = \emptyset \land S = \mathbb{N}$.
		\end{proof}
		
		\begin{theorem}[The Well-Ordering Principle 2.1.2]
			Every set of natural numbers that contains at least one element has a smallest element in it.
		\end{theorem}
		\begin{proof}
			Let $T \neq \emptyset$ and $T$ has no minimal element. \\
			Let $S = T^c \subseteq \mathbb{N}$. Clearly $1 \notin T$. \\
			i.e. $1 \in S$. And suppose $1,2, \dots k \notin T$, then $k+1 \notin T$. \\
			By principle of complete induction, $S = \mathbb{N}$, i.e. $T = \emptyset$. \\
			Contradiction, thus $T$ has a smallest element.
		\end{proof}
		
		\begin{theorem}[The Generalized Principle of Mathematical Induction 2.1.4]
			Let $m$ be a natural number. If $S$ is a set of natural numbers with the properties that
			\begin{enumerate}
				\item $m$ is in $S$, and
				\item $k+1$ is in $S$ whenever $k$ is in $S$ and it greater than or equal to $m$.
			\end{enumerate}
			then $S$ contains every natural number greater than or equal to $m$.
		\end{theorem}
		\begin{proof}
			Prove using PMI.
		\end{proof}
		
		\begin{theorem}[The Principle of Complete Mathematical Induction 2.2.1]
			If $S$ is any set of natural numbers with the properties that
			\begin{enumerate}
				\item $1 \in S$, and
				\item $\{1, 2, \dots, k\} \subset S \implies k+1 \in S$,
			\end{enumerate}
			then $S$ is the set of all natural numbers.
		\end{theorem}
		
		\begin{theorem}[The Generalized Principle of Complete Mathematical Induction 2.2.2]
			If $S$ is any set of natural numbers with the properties that
			\begin{enumerate}
				\item $m \in S$, and
				\item $\{m, m+1, \dots, k\} \subset S \implies k+1 \in S$,
			\end{enumerate}
			then $S$ contains all natural numbers greater than or equal to $m$.
		\end{theorem}
		
		\begin{theorem}[2.2.4]
			Every natural number other than $1$ is a product of prime numbers.
		\end{theorem}
		\begin{proof}
			Case 1: $n \in \mathbb{P}$. \\
			Case 2: $n \notin \mathbb{P} \implies n = a\times b$, proven by GPCI.
		\end{proof}
		
	\section{Modular Arithmetic}[3.1.2]
		\begin{theorem}
			If $a \equiv b \mod m$ and $b \equiv c \mod m$, then $a \equiv c \mod m$.
		\end{theorem}
		
		\begin{theorem}[3.1.3]
			When $a$ and $b$ are nonnegative integers, the relationship $a \equiv b \mod m$ is equivalent to $a$ and $b$ leaving equal reminders upon division by $m$.
		\end{theorem}
		
		\begin{theorem}[3.1.4]
			For a given modulus $m$, each integer is congruent to exactly one of the numbers in the set $\{0, 1, \dots, m-1\}$.
		\end{theorem}
		
		\begin{theorem}[3.2.1]
			Every natural number $d_n \dots d_2 d_1 d_0$ is congruent to the sum of its digits modulo $9$. In particular, a natural number is divisible by $9$ if and only if the sum of its digits is divisible by $9$.
			\[
				\sum_{i=0}^n {10^i d_i} \equiv \sum_{i=0}^n{d_i} \mod 9
			\]
		\end{theorem}
		\begin{proof}
			Note that $10^i \equiv 1 \mod 9,\ \forall i \geq 0$.
		\end{proof}
		
	\section{The Fundamental Theorem of Arithmetic}
		\begin{theorem}[The Fundamental Theorem of Arithmetic 4.1.1]
			Every natural number greater than $1$ can be written as a product of primes, and the expression of a number as a product of primes is unique except for the order of the factors
		\end{theorem}
	
		\begin{corollary}[4.1.3]
			 If $p$ is a prime number and $a$ and $b$ are natural numbers such that p divides $ab$, then $p$ divides at least one of $a$ and $b$. (That is, if a prime divides a product, then it divides at least one of the factors.)
			 \[
			 	p | ab \implies p | a \lor p | b 
			 \]
		\end{corollary}
	
	\section{Fermat's Theorem and Wilson's Theorem}
		\begin{theorem}[5.1.1]
			If $p$ is a prime and $a$ is not divisible by $p$, and if $ab \equiv ac \mod p$, then $b \equiv c \mod p$.
		\end{theorem}
		
		\begin{theorem}[Fermat's Theorem 5.1.2]
			If $p$ is a prime number and $a$ is any natural \textcolor{red}{not divisible by $p$}, then 
			\[
				a^{p-1} \equiv 1 \mod p
			\]
		\end{theorem}
		
		\begin{corollary}[5.1.3]
			If $p$ is a prime number and $a$ is any natural number, then 
			\[
				a^p \equiv a \mod p
			\]
		\end{corollary}
		
		\begin{definition}[5.1.4]
			A \textbf{multiplicative inverse modulo $p$} for a natural number $a$ is a natural number $b$ such that $ab \equiv 1 \mod p$.
		\end{definition}
		
		\begin{corollary}[5.1.5]
			If $p$ is a prime and $a$ is a natural number that is not divisible by $p$, then there exists a natural number $x$ such that
			\[
				ax \equiv 1 \mod p
			\]
		\end{corollary}
		\begin{proof}
			Using Fermat's Theorem and take $x = a^{p-2}$.
		\end{proof}
		
		\begin{lemma}[5.1.6]
			If $a$ and $c$ have the same multiplicative inverse modulo $p$, then $a$ is congruent to $c$ modulo $p$.
		\end{lemma}
		\begin{proof}
			Suppose $ab \equiv 1 \mod p$ and $cb \equiv 1 \mod p$, \\
			then $abc \equiv c \mod p$, which implies $a \equiv c \mod p$.
		\end{proof}
		
		\begin{theorem}[5.1.7]
			Let $p \in \mathbb{P}$, and $x \in \mathbb{Z}$ satisfying $x^2 \equiv 1 \mod p$, then $x \equiv 1 \mod p$ or $x \equiv -1 \mod p$.
		\end{theorem}
		\begin{proof}
			$x^2 \equiv 1 \mod p \iff p|x^2-1 \iff p|(x-1)(x+1) \implies p|(x-1) \lor p | (x+1)$.
		\end{proof}
		
		\begin{theorem}[Wilson's Theorem 5.2.1]
			If $p$ is a prime number, then
			\[
				(p-1)! \equiv -1 \mod p
			\]
		\end{theorem}
		
		\textcolor{red}{
		\begin{theorem}[5.2.2]
			If $m$ is a composite number larger than $4$, then
			\[
				(m-1)! \equiv 0 \mod m
			\]
		\end{theorem}}
		
		\begin{theorem}[Extended version of Wilson's theorem 5.2.3]
			If $m$ is a natural number other than $1$, then $(m-1)! \equiv -1 \mod m$ \ul{if and only if} $m \in \mathbb{P}$.
		\end{theorem}
	
	\section{Sending and Receiving Secret Messages}
		\begin{theorem}[6.1.2]
			Let $N=pq$, where $p$ and $q$ are distinct prime numbers, and let $\phi(N) = (p-1)(q-1)$. If $k$ and $a$ are any natural natural numbers, then 
			\[
				a \cdot a^{k \phi(N)} \equiv a \mod N
			\]
		\end{theorem}
		
	\section{The Euclidean Algorithm and Applications}
	\par RSA encryption procedure(7.2.5):
		\begin{enumerate}
			\item Phase 1 (Receiver)
			\begin{enumerate}
				\item pick large $p, q \in \mathbb{P}$ such that $p \neq q$. 
				\item compute $N=pq$ and $\phi(N) = (p-1)(q-1)$.
				\item pick $e$ relatively prime to $\phi(N)$.
				\item announce $N, e$.
			\end{enumerate}
			\item Phase 2 (Sender)
			\begin{enumerate}
				\item pick message $M < N$.
				\item compute encoded message $R$ from $M^e \equiv R \mod N$.
				\item announce $R$.
			\end{enumerate}
			\item Phase 3 (Receiver)
			\begin{enumerate}
				\item compute decoder $d > 0$ from $de + k\phi(N) = 1$.
				\item compute decoded message $M$ from $R^d \equiv 1 \mod N$.
			\end{enumerate} 
		\end{enumerate}
		
		\begin{lemma}[7.2.2]
			If a prime number divides the product of two natural numbers, then it divides at least one of the numbers.
		\end{lemma}
		
		\begin{lemma}[Extended version of lemma 7.2.2, 7.2.3]
			For any natural number $n$, if a prime divides the product of $n$ natural numbers, then it divides at least one of the numbers.
		\end{lemma}
		\begin{proof}
			Using lemma 7.2.2 and PMI.
		\end{proof}
		
		\begin{theorem}[7.2.8]
			The \emph{Diophantine} equation $ax+by=c$, with $a$, $b$, and $c$ integers, has integral solutions if and only if $\gcd(a,b)$ divides $c$.
		\end{theorem}
		
		\begin{definition}[7.2.12]
			For any natural number $m$, the \textbf{Euler $\phi$ function}, $\phi(m)$, is defined to be the number of numbers in $\{1,2,\dots,m-1\}$ that are relatively prime to $m$. (Note that $1$ is relatively prime to every natural number)
		\end{definition}
		
		\begin{theorem}[7.2.14]
			If $p$ is prime, then $\phi(p) = p-1$.
		\end{theorem}
		\begin{proof}
			Directly form the definition of Euler-$\phi$ function.
		\end{proof}
		
		\begin{theorem}[7.2.15]
			If $p$ and $q$ are distinct primes, then $\phi(pq) = (p-1)(q-1)$.
		\end{theorem}
		\begin{proof}
			Consider the multiples of $p$ and $q$ in set $\{1, 2, \dots, pq - 1\}$. \\
			There would be $p-1$ multiples of $q$ and $q-1$ multiples of $p$. \\
			Total number of multiples is $(p-1) + (q-1) = p+q-2$. \\
			Any number other than the multiples above will be relatively prime to $pq$.\\
			There would be $pq - 1 - p - q + 2 = pq - p - q + 1= (p-1)(q-1)$.  
		\end{proof}
		
		\begin{theorem}[unnumbered, result from Euclidean algorithm]
			Let $a, b \in \mathbb{N}$, then there exists integers $z_1,z_2$ such that
			\[
				z_1 a + z_2 b = \gcd(a,b)
			\]
		\end{theorem}
		
		\begin{theorem}
			If $a$ is relatively prime to $m$ and $ax \equiv ay \mod m$, then $x \equiv y \mod m$.
		\end{theorem}
		
		\begin{theorem}[Euler's Theorem 7.2.17]
			If $m$ is a natural number greater than $1$ and $a$ is a natural number that is relatively prime to $m$, then 
			\[
				a^{\phi(m)} \equiv 1 \mod m
			\]
		\end{theorem}
		
		\textcolor{red}{
		\begin{theorem}[7.3.Q27]
			Let $n \in \mathbb{N}$, and suppose $n$ can be factorized into $p_1^{k_1} p_2^{k_2} \cdots p_m^{k_m}$ then
			\[
				\phi(n) = (p_1^{k_1} - p_1^{k_1 - 1})(p_2^{k_2} - p_2^{k_2 - 1}) \cdots (p_m^{k_m} - p_m^{k_m - 1})
			\]
		\end{theorem}
		}
	
	\section{Rational Numbers and Irrational Numbers}
		\begin{theorem}[The Rational Roots Theorem 8.1.9]
			If $\frac{m}{n}$ is a rational root of the polynomial
			\[
				a_k x^k + a_{k-1} x^{k-1} + \cdots + a_1 x + a_0
			\]
			where $a_j$ are integers and \ul{$m$ and $n$ are relatively prime}, then $m | a_0$ and $n | a_k$.
		\end{theorem}
		
		\begin{theorem}[8.2.6]
			If $p$ is a prime number, then $\sqrt{p}$ is irrational.
		\end{theorem}
		
		\begin{theorem}[8.2.8]
			If the square root of a natural number is rational, then the square root is an integer.
		\end{theorem}
		
		\begin{theorem}[Extended 8.2.8]
			Let $n \in \mathbb{N}$, then $\sqrt{n} \in \mathbb{Q}$ \ul{if and only if} $n$ is a perfect square.
		\end{theorem}
		
		\begin{theorem}[Extended 8.2.8]
			Let $n \in \mathbb{N}$, then $\sqrt[3]{n} \in \mathbb{Q}$ \ul{if and only if} $n$ is a perfect cube.
		\end{theorem}
		
		\begin{remark}
			As immediate result from (8.2.8), we can conclude that the square or cubic root is integer.
			\begin{gather*}
				\sqrt{n} \in \Q \implies \sqrt{n} \in \Z \\
				\sqrt[3]{n} \in \Q \implies \sqrt[3]{n} \in \Z
			\end{gather*}
		\end{remark}
		
	
	\section{The Complex Numbers}
		\begin{theorem}[9.2.3]
			The modulus of the product of two complex numbers is the product of their moduli. The argument of the product of two complex numbers is the sum of their arguments.
		\end{theorem}
		
		\begin{theorem}[9.2.5]
			Every complex number has a complex square root.
		\end{theorem}
		
		\begin{theorem}[De Moivre's Theorem 9.2.6]
			For every natural number $n$
			\[
				(r(\cos \theta + i \sin \theta))^n = r^n (\cos n\theta + i \sin n \theta)
			\]
		\end{theorem}
		
		\begin{theorem}[The Fundamental Theorem of Algebra 9.3.1]
			Every non-constant polynomial with complex coefficients has a complex root.
		\end{theorem}
		
		\begin{theorem}[The Factor Theorem 9.3.6]
			The complex number $r$ is a root of polynomial $p(z)$ if and only if $z-r$ is a factor of $p(z)$. That's 
			\[
				\exists f(z),\ p(z) = f(z)(z - r)
			\]
		\end{theorem}
		
		\begin{theorem}[9.3.8]
			\textcolor{red}{
				A polynomial of degree $n$ has at most $n$ complex roots; if "multiplicities" are counted, it has exactly $n$ roots.
			}
		\end{theorem}
	
	\section{Sizes of Infinite Sets}
		\begin{definition}[10.1.8]
			The sets $\mc{S}$ and $\mc{T}$ have the \textbf{same cardinality} if there is a bijective function $f: \mc{S} \to \c{T}$.
		\end{definition}
		
		\begin{theorem}[10.1.13]
			\[
				|\N| = |\Q^+|
			\]
		\end{theorem}
		
		\begin{definition}[10.2.1]
			A set is \textbf{countable} if it is either finite or has the same cardinality as the set of natural numbers.
		\end{definition}
		
		\begin{theorem}[10.2.2]
			\[
				|[0,1]| > \aleph_0
			\]
		\end{theorem}
		
		\begin{theorem}[10.2.4]
			Let $a, b \in \R$ and $a < b$, then
			\[
				|[a,b]| = |[0,1]|
			\]
		\end{theorem}
		
		\begin{theorem}[Unnumbered, generalized]
			All open, half-open and closed intervals in $\R$ have the same cardinality $c$.
		\end{theorem}
		
		\begin{theorem}[10.2.7]
			If $|\mc{S}| = |\mc{T}|$ and $|\mc{T}| = |\mc{U}|$, then $|\mc{S}| = |\mc{U}|$.
		\end{theorem}
		
		\begin{theorem}[10.2.10]
			The union of a countable number of countable sets is countable.
		\end{theorem}
		
		\begin{theorem}[The Cantor-Bernstein Theorem 10.3.5] If $\mc{S}$ and $\mc{T}$ are sets such that $|\mc{S}| \leq |\mc{T}|$ and $|\mc{T}| \leq |\mc{S}|$, then $|\mc{S}| = |\mc{T}|$.
		\end{theorem}
		
		\begin{corollary}[10.3.6]
			If $\mc{S}$ is a subset of $\mc{T}$ and there exists a function $f: \mc{T} \to \mc{S}$ that is injective, then $\mc{S}$ and $\mc{T}$ have the same cardinality.
		\end{corollary}
		
		\begin{theorem}
		    A subset of a countable set is countable.
		\end{theorem}
		
		\begin{corollary}[10.3.10]
		    Is $\mc{S}$ is any set and there exists a injective function $f: \mc{S} \to \N$, then $\mc{S}$ is countable.
		\end{corollary}
		
		\begin{theorem}[10.3.12]
		    The set of all finite sequences of natural numbers countable.
		\end{theorem}
		
		\begin{remark}
		    All sequences are countable.
		\end{remark}
		
		\begin{definition}[10.3.14]
		    Let $\mc{S}$ and $\mc{T}$ by any sets. We will say that $\mc{T}$ \textbf{can be labelled} by the set $\mc{S}$ if there is way of assigning a \ul{finite} sequence of elements of $\mc{S}$ to each element of $\mc{T}$ so that each finite sequence corresponds to at most one element $\mc{T}$.
		\end{definition}
		
		\begin{theorem}[The Enumeration Principle 10.3.16]
		    Every set that can be labelled by a \ul{countable set} is countable.
		\end{theorem}
		
		\begin{definition}[10.3.19]
		    The real number of $x_0$ is said to be \textbf{algebraic} if it is the root of a polynomial with \ul{integer coefficients}. The real number $x_0$ is said to be \textbf{transcendental} if there is no polynomial with integer coefficients that $x_0$ as a root.
		\end{definition}
		
		\begin{theorem}[Theorem 10.3.20]
		    \[
		        |\mc{A}| \leq \aleph_0
		    \]
		\end{theorem}
		
		\begin{corollary}[10.3.21]
		    There exist transcendental numbers.
		\end{corollary}
		
		\begin{theorem}[10.3.24]
		    If $\mc{S}$ is an infinite set, then $\aleph_0 \leq |\mc{S}|$.
		\end{theorem}
		
		\begin{theorem}[10.3.27]
		    For every set $\mc{S}$, then 
		    \[
		        |\mc{S}| \textcolor{red}{<} |\mc{P}(\mc{S})|
		    \]
		\end{theorem}

        \begin{theorem}[10.3.28]
            The cardinality of the set of all sets of natural numbers is the same as the cardinality of the set of real numbers. That is,
            \[
                |\mc{P}(\N)| = 2^{\aleph_0} = c
            \]
        \end{theorem}
        
        \begin{theorem}[10.3.30 and extended]
            The cardinality of the unit square and unit cube are $c$.
        \end{theorem}
        
        \begin{theorem}[Generalized 10.3.30]
            Let $n \in \N$, then 
            \[
                |\R^n| = c
            \]
        \end{theorem}
        
        \begin{definition}[10.3.31]
            If $S$ is a set and $S_0$ is a subset of $S$, then the \textbf{characteristic function} of $S_0$ as a subset of $S$ is the function $f$, with domain $S$, defined by
            \[
                f(s) = \begin{cases}
                    1 \tx{ if } s \in S_0 \\
                    0 \tx{ if } s \notin S_0
                \end{cases}
            \]
        \end{definition}
        	
        	\setcounter{section}{11}
    \section{Constructibility}
        \begin{definition}[12.2.2]
            A real number is \textbf{constructible} if the point corresponding to it one the number line can be obtained from marked points 0 and 1 by performing a \ul{finite sequence of constructions} using only a straightedge and compass.
        \end{definition}
        
        \begin{theorem}[12.2.9]
            If $\mc{F}$ is any subfield of $\R$, then $\mc{F}$ contains all rational numbers.
        \end{theorem}
        
        \begin{theorem}[12.2.10]
            The set of constructible number is a subfield of $\R$.
        \end{theorem}
        
        \begin{theorem}[12.2.12\& 12.2.13]
            \textcolor{red}{Let $\mc{F}$ be any subfield of $\R$ and suppose that $0 < r \in \mc{F}$ and $\sqrt{r} \notin \mc{F}$, then the field obtained by adjoining $\sqrt{r}$ to $\mc{F}$ is defined as 
            \[
                \mc{F}(\sqrt{r}) = \{a + b \sqrt{r}: a, b \in \mc{F}\}
            \]
            is called the extension of $\mc{F}$ by $\sqrt{r}$. And then $\mc{F}(\sqrt{r})$ is a subfield of $\R$.
            }
        \end{theorem}
        
        \begin{theorem}[12.2.15]
            If $r$ is a positive constructible number, then $\sqrt{r}$ is constructible.
        \end{theorem}
        
        \begin{definition}[12.2.16]
            A \textbf{tower of fields} is a \ul{finite} sequence $\mc{F}_0, \mc{F}_1, \mc{F}_2,\dots,\mc{F}_n$ of subfields of $\R$ such that $\mc{F}_0 = \Q$ and, for each $i$ from $1$ to $n$, there is a positive number $r_i$ in $\mc{F}_{i-1}$ such that $\sqrt{r_i}$ is not in $\mc{F}_{i-1}$ and $\mc{F}_i = \mc{F}_{i-1}(\sqrt{r_i})$.
        \end{definition}
        
        \begin{definition}[12.3.1]
            A \textbf{surd} is a number that is in some field that is in a tower. That is, $x$ is a surd if there exists a tower:
            \[
                \mc{F}_0 \subset \mc{F}_1 \subset \mc{F}_2 \subset \cdots \subset \mc{F}_n
            \]
        \end{definition}
        
        \begin{theorem}[12.3.2]
            The set of all surds is a subfield of $\R$. Moreover, if $r$ is a positive surd, then $\sqrt{r}$ is a surd.
        \end{theorem}
        
        \begin{theorem}[12.3.3]
            Every surd is constructible.
        \end{theorem}
        
        \begin{theorem}[12.3.10]
            The points of intersection of a line that has an equation with surd coefficients and a circle that has an equation with surd coefficients lie in the surd plane.
        \end{theorem}
        
        \begin{theorem}[12.3.11]
            The points of intersection of two distinct circles that have equations with surd coefficients lie in the surd plane.
        \end{theorem}
        
        \begin{theorem}[12.3.12]
            The field of constructible numbers is the same as the field of surds.
        \end{theorem}
        
        \begin{theorem}[12.3.13]
            \[
                \theta \in C \iff \cos \theta \in C
            \]
        \end{theorem}
        
        \begin{theorem}[12.3.16]
            \[
                \cos 3\theta = 4 \cos^2 \theta - 3\cos \theta
            \]
        \end{theorem}
        
        \begin{theorem}[12.3.21]
            If $a + b \sqrt{r}$ is in $\mc{F}(\sqrt{r})$ and is a root of a polynomial with rational coefficients, then $a - b \sqrt{r}$ is also a root of the polynomial.
        \end{theorem}
        
        \begin{theorem}[12.3.22]
            If a \ul{cubic} equation with rational coefficients has a constructible root, then the equation has a rational root. 
        \end{theorem}
        
		\section*{References}
			Rosenthal, D., Rosenthal, D., \& Rosenthal, P. (2014). A Readable Introduction to Real Mathematics. Springer.
\end{document}
































