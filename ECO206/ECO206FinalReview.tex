\documentclass[]{article}

\usepackage{spikey}
\usepackage{amsmath}
\usepackage{amssymb}
\usepackage{float}
\usepackage{soul}
\usepackage{xcolor}

\usepackage[
    type={CC},
    modifier={by-nc},
    version={4.0},
]{doclicense}

\title{ECO206 Microeconomic Theory \\ \small Final Preparation Notes}
\author{Tianyu Du}

\begin{document}
    \maketitle
    \doclicenseThis
    \tableofcontents

    \section{Lecture 1 Budget Constraints}
    
        \subsection{Notation-Bundle}
            \begin{definition}
                If we have $n$ goods, then vector $(x_1^A, x_2^A, \dots, x_n^A) \in \R^n_+$ represents a \textbf{bundle}, where $x_1^A$ represents a quantity($x$) of good 1 in bundle A. 
            \end{definition}
            
        \subsection{Types of Income}
            \paragraph{Exogenous}independent to price of goods.
            \paragraph{Endogenous}bundle of goods endowed, value dependents on price of goods.
            
        \subsection{Budget Set}
            \begin{definition}
                \textbf{Budget Set} is the set of all affordable consumption \emph{bundles}. Let $\vec{e}=(\omega_1,\omega_2,\dots,\omega_n) \in \R^n_+$ denote the endowment of consumer and $I$ denote the exogenous income, and consumers are price-takers facing price vector $\vec{p}$. Then
                \[
                    \mathcal{B} = \big\{\vec{x}\in \R^n_+ \vert \vec{x}\cdot\vec{p} \leq \vec{e}\cdot\vec{p} + I  \big\}
                \]
            \end{definition}
            
        \subsection{Opportunity Cost}
            \par Opportunity cost is the rate at which one good can be traded for another though the \emph{market}. It captures what the consumer is \textbf{able} to trade at market price, not what the consumer is \emph{willing} to trade.
            \par Consider budget constraint $\vec{x}\cdot\vec{p}=I$. Take the total differential,
            \[
                \sum_i \pd{\vec{x}\cdot\vec{p}}{x_i} d{x_i} = 0
            \]
            And the opportunity cost for one unit of good $x$ is shown to be 
            \[
                \frac{dy}{dx} = -\frac{p_x}{p_y}
            \]
            
        \subsection{Income and Price changes}
        \par \ul{Pure Income Changes: } \emph{parallel shift} of budget line.
        \par \ul{Price changes: } \emph{rotation} of budget line around the \emph{invariant point} (i.e. the consumption bundle that is not affected by the price change).
            \begin{enumerate}
                \item Exogenous income: Invariant point $\impliedby$ $\{x_1, \dots, x_i=0, \dots, x_n\}$ with $p_i$ changes. 
                \item Endogenous income: Invariant point $\impliedby$ endowment point $\vec{e}$.
            \end{enumerate}
            
    \section{Lecture 2 Preferences and Utility}
        \subsection{Tastes as Binary Relations}
            \begin{definition}
                \begin{enumerate}
                    \item Bundle A $(x_1^A, x_2^A)$ is \textbf{strictly preferred to} bundle B $(x_1^B, x_2^B)$.
                        \[
                            (x_1^A, x_2^A) \succ (x_1^B, x_2^B)
                        \]
                    \item Bundle A is \textbf{at least as good as} bundle B
                        \[
                            (x_1^A, x_2^A) \succcurlyeq (x_1^B, x_2^B)
                        \]
                    \item The consumer is indifferent between bundle A and bundle B
                        \[
                            (x_1^A, x_2^A) \sim (x_1^B, x_2^B)
                        \]
                \end{enumerate}
            \end{definition}
        
        \subsection{Rationality Assumptions}
            \begin{definition}
                \textbf{Complete: } For all bundles $A=(x_1^A, x_2^A)$ and $B=(x_1^B, x_2^B)$.
                \[
                    A \succcurlyeq B \lor B \succcurlyeq A
                \]
            \end{definition}
            
            \begin{definition}
                \textbf{Transitive: } for any three bundles, $A, B, C$
                \[
                    A \succcurlyeq B \land B \succcurlyeq C \implies A \succcurlyeq C
                \]
            \end{definition}
        
        \subsection{Convenience Assumptions}
            \begin{definition}
                \textbf{Monotonic: }
                \[
                    A \succcurlyeq B\ \impliedby \ x_i^A \geq x_i^B \ \forall \ i
                \]
                \[
                    A \succ B\ \impliedby \ x_i^A > x_i^B \ \forall \ i
                \]
            \end{definition}
            
            \begin{definition}
                \textbf{Convexity: } Suppose $A = (x_1^A, x_2^A) \sim B = (x_1^B, x_2^B)$
                \[
                    \alpha A + (1 - \alpha) B \succcurlyeq A,\ \forall \ \alpha \in [0, 1]
                \]
            \end{definition}
            
            \begin{definition}
                \textbf{Continuity: } (no sudden jumps)
            \end{definition}
        
        \subsection{Utility Function and Indifference Curve}
            \begin{definition}
                Let $\mathcal{X}$ denote the consumption set and a \textbf{utility function} is a real-valued function $u:\mathcal{X} \rightarrow \R_+$ so that
                \[
                    A \succcurlyeq B \iff u(A) \geq u(B)
                \]
                and
                \[
                    A \succ B \iff u(A) > u(B)
                \]
            \end{definition}
            
            \begin{remark}
                The \ul{positive monotone transformations} of utility function of a preference also captures the same preference. 
            \end{remark}
            
            \begin{definition}
                The \textbf{indifference curve} corresponding to utility level $\overline{u}$ is a set of consumption bundles
                \[
                    \{\vec{x} \in \R^n_+\ \vert\ u(\vec{x}) = \overline{u} \}
                \]
            \end{definition}
            
        \subsection{Marginal Rate of Substitution}
            \begin{definition}
                The \textbf{MRS} is the number of good $y$ that a consumer is \textbf{willing} to given up in order to get one more unit of good $x$.
                \\Take total differential of equation $u(\vec{x}) = \overline{u}$,
                \[
                    \sum_{i}{
                        \pd{u(\vec{x})}{x_i} d{x_i} = 0
                        }
                    \implies
                    \frac{dy}{dx} = -\frac{MU_x}{MU_y}
                \]
                And MRS is a function of bundle $\vec{x}$.
            \end{definition}
            
            \paragraph{Diminishing MRS} is when the \textbf{absolute MRS} decreases as more good $x$ is consumed along the indifference curve.
            
        \subsection{Different Types of Tastes}
            \begin{itemize}
                \item \textbf{Perfect Substitutes} MRS constant at every bundle.
                \item \textbf{Perfect Complements} Unwilling to substitute.
                \item \textbf{Homothetic} MRS constant on a ray ($\beta \vec{x_o},\ \beta > 0$) through origin.
                \item \textbf{Quasi-linear} MRS depends on $\vec{x_{-i}}$ where $x_i$ is excluded.
            \end{itemize}
            
    \section{Lecture 3 Choice}
        \paragraph{Diminishing MRS in words} Comparing two bundles on the \emph{same} indifferent curve. At each bundle consider how much of $y$ we are willing to given up for an additional unit of $x$. \emph{If we have relatively \ul{more} $x$ in one bundle, then we are \ul{less} willing to give up $y$ at that bundle compared to the other bundle}.
        
        \begin{remark}
            Perfect Substitutes (linear in every commodity) $\implies$ quasi-linear.
        \end{remark}
        
        \subsection{Lagrangian Multiplier}
            \paragraph{Sufficiency} FOCs from Lagrangian multiplier is both necessary and sufficient if and only if
                \begin{enumerate}
                    \item There are no "flat spots" on indifference curves.
                    \item All goods are essential (\ul{interior solution})
                    \item Both choice set (characterized by budget constraint) and tastes (characterized by preference and utility function, quasi-concave).
                \end{enumerate}
            \paragraph{Uniqueness} Assuming \textbf{convexity} of preferences and choice set guarantees \emph{uniqueness} of optimal choice. (Reducing problem to a convex optimization problem).
        
        \subsection{Problem Solving}
            \begin{enumerate}
                \item Determine preference type.
                \item Check expectation of interior solution.
                \begin{itemize}
                    \item If interior solution (e.g. Cobb-Douglas): setup and solve Lagrangian multiplier, check sufficiency.
                    \item If not:
                    \begin{enumerate}
                        \item Perfect Substitute: Corner Solution if $MRS \neq rel.Price$.
                        \item Perfect Complement: Solution on path $x_1=\gamma x_2$.
                        \item Quasi-linear: interior solutions found for some parameter values only. \emph{Use LM to find expression of solution and check the condition for positiveness}.
                    \end{enumerate}
                \end{itemize}
                \item Check for \emph{multiple optimal solutions}.
            \end{enumerate}
            
    \section{Lectuer 4 Demand and Income Effects}
        \subsection{Demand}
            \begin{definition}
                A \textbf{demand function} is an equation that expresses \ul{bundle choices} as a \ul{function} of prices and incomes.
            \end{definition}
            
        \subsection{Income Effects}
            \begin{definition}
                \textbf{Income effect} is the change in behavior arising from just a change in income and leads to a parallel shift in the budget constraint.
            \end{definition}
            
            \begin{remark}
                With quasi-linear preference with numeraire $x_1$, then the demand for all \ul{other} commodities other than $x_1$ have no income effect. We often refer to quasi-linear goods as the borderline goods between normal and inferior.
            \end{remark}
            
            \begin{remark}
                With pure income effect and homothetic preference, the ratios of commodities before and after are the same.
            \end{remark}
            
            \begin{definition}
                A \textbf{Engel curve} $x_i^D(I, \vec{p})$ captures the relationship between \ul{demand} and \ul{income}. Note: put $I$ on y-axis.
            \end{definition}
            
            \begin{itemize}
                \item \textbf{Normal goods} $\pd{x}{I} > 0$
                \item \textbf{Inferior goods} $\pd{x}{I} < 0$
            \end{itemize}
            
        \subsection{Decomposition of Substitution Effect and Income Effect}
            \begin{itemize}
                \item \textbf{Substitution Effect} changing \ul{relative prices} holding the \ul{utility level constant}.
                \item \textbf{Income Effect} changing \ul{purchasing power} of the consumer's income holding 
                \ul{relative prices constant}.
            \end{itemize}
            
    \section{Lecture 5 Income and Substitution Effects}
        \begin{definition}
            \textbf{Hicksian substitution effect} derived by compensating consumer enough income so that the consumer can reach the \ul{original utility level} after the price changes.
        \end{definition}
        \begin{definition}
            \textbf{Slutsky substitution effect} derived by compensating consumer enough income so that the consumer can afford the \ul{original optimal bundle} after the price changes.
        \end{definition}
        
        \subsection{Expenditure Minimization}
            \begin{definition}
                With \ul{new prices}, given consumer enough (hypothetical) income to just reach the \ul{original indifference curve}. Mathematically,
                \[
                    min_{x_1,x_2} p_1^{final} x_1 + p_2^{final} x_2 \ s.t.\ u(x_1, x_2) = U^{initial} = u(x_1^{initial}, x_2^{initial})
                \]
            \end{definition}
            
            \par Solving the expenditure using Lagrangian multiplier method 
                \[
                    \mc{L}(\vec{x}, \lambda) = \vec{x} \cdot \vec{p} + \lambda\{\overline{u} - u(\vec{x}) \}
                \]
                Let $\vec{x}^{SE} = (x_1^{SE}, x_2^{SE}, \dots, x_N^{SE}) = argmin_{\vec{x}}\{\vec{p} \cdot \vec{x}\ s.t.\ u(\vec{x}) = \overline{u}\}$ and $\vec{p}^{new} \cdot \vec{x}$ denote the \textbf{compensated income}.
            
            \begin{definition}
                From expenditure minimization, we define the \textbf{expenditure function} $e(\vec{p}, 
                \overline{u}): \R^{n+1}_+ \rightarrow \R_+$ as a function of price vector $\vec{p}$ and utility level $\overline{u}$.
                \[
                    e(\vec{p}, \overline{u}) := min_{\vec{x}} \{\vec{p} \cdot \vec{x} \ s.t. \ u(\vec{x}) = \overline{u} \}
                \]
            \end{definition}
            
        \subsection{Income Effect}
            \begin{remark} Decomposing substitution and income effects.
                \begin{enumerate}
                    \item The utility value achieved is $\overline{u} = u(\vec{x}^*_1)$.
                    \item Given prices $\vec{p}_1, \vec{p}_2$, use \ul{utility maximization} to find optimal bundles $\vec{x}^*_1, \vec{x}^*_2$.
                    \item Use \ul{expenditure minimization} with respect to \ul{new price} and \ul{original utility level} to find $\vec{x}_{SE}(\vec{p}_2, \overline{u})$.
                    \item Substitution effect is $\vec{x}_{SE} - \vec{x}^*_1$.
                    \item Income effect is $\vec{x}^*_2 - \vec{x}_{SE}$.
                \end{enumerate}
            \end{remark}
            \begin{figure}[h]
                \centering
                % \includegraphics{}
                \caption{Decomposition of two effects}
            \end{figure}
        
        \subsection{Compensated Demand Curve}
            \begin{remark}
                \textbf{Regular demand (Marshallian)} changes in $x_1$ as $p_1$ changes when \ul{income held fixed}. (On graph: initial to final bundles)
            \end{remark}
            
            \begin{definition}
                \textbf{Compensated demand (Hicksian)} captures the changes in $x_1$ as $p_1$ changes, when \ul{utility level is held fixed}(hitting the same IC) by compensating consumer additional income. (On graph: initial to SE)
                \\\textbf{Notation}:
                \\Given prices $p_i$ and prices of other goods, $\vec{p}_{-i}$ and original utility $\overline{u}$, compensated demand for $x_i$ is denoted as
                \[
                    h_i(p_i, \vec{p}_{-1}, \overline{u}): \R^n_+ \rightarrow \R_+
                \]
            \end{definition}
            \begin{remark}
                \begin{itemize}
                    \item \ul{Change in prices of other goods / changes in original utility level} Movements of the compensated demand curve.
                    \item \ul{Own-price changes} Movement \emph{along} the curve.
                \end{itemize}
            \end{remark}
            
        \subsection{Slutsky Equation (Optional)}
        \begin{theorem}
            \[
                \pd{x_i}{p_j} = \pd{h_i}{p_j} - h_j \pd{x_i}{y}
            \]
        \end{theorem}
        \begin{proof}
            Notice that at the optimal, $x_i(\vec{p}, e(\vec{p}, u)) = h_i(\vec{p}, u)$, take derivative with respect to $p_j$ we have
            \[
                \pd{x_i}{p_j} + \pd{x_i}{e} \pd{e}{p_j} = \pd{h_i}{p_j}
            \]
            By Sheppard's Lemma $\pd{e(\vec{p}, u)}{p_j} = h_j(\cdot)$, 
            \[
                \pd{x_i}{p_j} = \pd{h_i}{p_j} - h_j \pd{x_i}{y}
            \]
        \end{proof}
    
    \section{Lecture 6 Labor Supply and Elasticities}
        \subsection{Labor Supply}
            \begin{itemize}
                \item Endogenous income: Leisure endowment ($L$) sold at real wage rate $w$.
                \item Exogenous income: Non-labor income ($M$) 
            \end{itemize}
            \paragraph{Setup}
            \quad \\
            \emph{Budget constraint:}
            \[
                c = w(L-\ell) + M
            \]
            \\
            \emph{Optimization problem:}
            \begin{gather*}
                \max_{c, \ell} u(c,\ell)\\ s.t.\ c \leq w(L - \ell) + M
            \end{gather*}
            Let $c^*(w), \ell^*(w)$ denote the maximizers and \textbf{labor supply} is
            \[
                h(w) = L - \ell(w)
            \]
            \begin{example}
                For wage rate $w$ increases:
                \begin{itemize}
                    \item Substitution effect: $\ell \downarrow$ and $h \uparrow$
                    \item Income effect (positive):
                        \begin{itemize}
                            \item (Inferior leisure) $\ell \downarrow$ and $h \uparrow$
                            \item (Normal leisure) $\ell \uparrow$ and $h \downarrow$
                        \end{itemize}
                \end{itemize}
            \end{example}
            
            \begin{remark}
                Labor supply is \ul{upwards sloping} if \ul{substitution effect dominates income effect} or leisure is considered as \ul{inferior}.
            \end{remark}
        
        \subsection{Taxes and Labor Supply}
            \begin{remark}
                \emph{Proportional income tax} is effective a change in real wage $w$ faced by consumers. Consider both \ul{income effect} (and notice leisure could be inferior or normal) and \ul{substitution effect}. 
            \end{remark}
            
            \subsubsection{Laffer Curve (Optional)}
                \begin{definition}
                    \textbf{Laffer curve} captures the relation between \ul{tax revenue} and \ul{marginal tax rate}. For proportional income tax, tax revenue $T(\tau)$ is
                    \[
                        T(\tau) = \tau*w*h(\tau)
                    \]
                    and marginal tax revenue depends on the \ul{elasticity of labor supply}.
                \end{definition}
                
        \subsection{Elasticity}
            \begin{definition}
                The \textbf{own price elasticity of demand} for commodity $i$ at coordinate $(x_i, p_i)$ on its demand curve is 
                \[
                    \epsilon_i = \frac{
                        \frac{dx_i}{x_i}
                        }{
                        \frac{dp_i}{p_i}
                        }
                    = \frac{dx_i}{dp_i} \frac{p_i}{x_i}
                \]
            \end{definition}
            
            \begin{definition}
                \textbf{Cross price elasticity of demand} for commodity $i$ with respect to price of commodity $j$ is 
                \[
                    \epsilon_{i,j} = \frac{dx_i}{dp_j}\frac{p_j}{x_i}
                \]
            \end{definition}
            
            \begin{definition}
                The \textbf{income elasticity of demand} for commodity $i$ is
                \[
                    \eta_{i} = \frac{dx_i}{dI}\frac{I}{x_i}
                \]
            \end{definition}
            
            \begin{definition} Elasticity $\epsilon$ is classified as 
                \begin{itemize}
                    \item \textbf{Elastic} if $|\epsilon| > 1$.
                    \item \textbf{Inelastic} if $|\epsilon| < 1$.
                    \item \textbf{Unit elastic} if $|\epsilon| = 1$.
                    \item \textbf{Perfectly inelastic} if $|\epsilon| = 0$.
                    \item \textbf{Perfectly elastic} if $|\epsilon| = \infty$.
                \end{itemize}
            \end{definition}
            
        \subsection{Change in Revenue}
            \par Total revenue $TR(p) = p*x(p)$, take derivative with respect to $p$ on both sides
            \begin{gather*}
                \pd{TR(p)}{p} = x(p) + p\pd{x(p)}{p} \\
                \implies \pd{TR(p)}{p} = x(p) \{ 1 + \pd{x(p)}{p} \frac{p}{x(p)} \} = x(p) \{1 - |\epsilon| \}\\
                \implies \pd{TR(p)}{p} > 0 \iff |\epsilon| < 1 \\
            \end{gather*}
        
    \section{Lecture 7 Elasticity and Consumer Surplus}
        \subsection{Consumer Surplus}
            \par Relating the \ul{regular demand}, \ul{choice diagram} and \ul{compensated demand}.
        
            \begin{definition}
                \textbf{Consumer surplus} is the difference between what you are \ul{willing to pay (MRS)} and what you have to pay for \ul{each unit} in your chosen bundle \ul{quantified in dollars}.
            \end{definition}
            
            \paragraph{Calculating Consumer Surplus}
            \begin{enumerate}
                \item Identify the original bundle chosen and utility level $\overline{u}$.
                \item Calculate $h_1(p_1, \textcolor{red}{p_2=1},\overline{u})$.
                \item $CS = \int_{p^*}^{\infty}{h_1(p_1,p_2=1,\overline{u})}\ dp_1$
            \end{enumerate}
            
            \begin{remark}
                If the preference is \textbf{quasi-linear}, then for \ul{non-numeraire} commodities, consumer surplus calculated from compensated and uncompensated demand curves are the same.
            \end{remark}
        
        \subsection{Expenditure Function}
            \begin{definition}
                \textbf{Expenditure function} is the value function for consumer expenditure minimization problem.
                \[
                    e(\vec{p}, \overline{u}) = \min_{\vec{x}} \{\vec{x} \cdot \vec{p}\ s.t.\ u(\vec{x}) \geq \overline{u}\} = \vec{p} \cdot \vec{h}^*(\vec{p}, \overline{u})
                \]
            \end{definition}
        
        \subsection{Compensating Variation}
            \begin{definition}
                \textbf{Compensating Variation(CV)} is the amount of compensation consumer needs to achieve it's \ul{original utility level} with the \ul{new price}.
                \[
                    |CV| = e(
                    \textcolor{red}{
                        \vec{p}_{final}, \overline{u}_{initial}
                        }) - I
                \]
            \end{definition}
            
            \begin{remark}
                Consider a change on $p_i$ from $p_i^1$ to $p_i^2$, the prices for other commodities $\overline{p}_{-i}$ are unchanged. By Sheppard's lemma, $\pd{e(\cdot)}{p_i} = h_i$.
                \[
                    |CV| = e(p_i^2, \overline{p}_{-i}, \overline{u}_{initial}) - e(p_i^1, \overline{p}_{-i}, \overline{u}_{initial}) = \int_{p_i^1}^{p_i^2}{h_i(p_i, \overline{u}_{initial})\ dp_i}
                \]
            \end{remark}
        
        \subsection{Equivalent Variation}
            \begin{definition}
                \textbf{Equivalent Variation (EV)} is the amount of compensation consumer needs at \ul{initial price} to make him as well off as \ul{after} the price change.
                \[
                    |EV| = e(
                    \textcolor{red}{
                        \vec{p}_{initial}, \overline{u}_{final}
                        }) - I 
                \]
            \end{definition}
        
        \begin{remark}
            Consider a change on $p_i$ from $p_i^1$ to $p_i^2$, the prices for other commodities $\overline{p}_{-i}$ are unchanged.
            \[
                |EV| = e(p_i^1, \overline{p}_{-i}, \overline{u}_{final}) - e(p_i^2, \overline{p}_{-i}, \overline{u}_{final}) = \int_{p_i^2}^{p_i^1}{h_i(p_i, \overline{u}_{final})\ dp_i}
            \]
        \end{remark}
        
        \paragraph{Exit Question}
            CV and EV are derived from integral of compensated demand $h_i$ at difference utility levels.
            \textcolor{blue}{For quasi-linear preferences, $|CV| = |EV|$}
            , since there is no income effect, compensated demand curves at any utility level coincides the uncompensated demand curve. Therefore both CV and EV are derived from integral compensated demand and they have the \ul{absolute absolute value} \emph{but opposite sign}.
    
    \section{Lecture 8 Dead Weight Loss and Uncertainty}
        \begin{remark}
            In general, the compensated and uncompensated demand curves intersect at one point (where the compensated demand curve is constructed). As we move away from that point by changing prices, the quantity on CD and Reg.D will differ because the CD captures only SE but regular demand captures both IE and SE.
        \end{remark}
        
        \subsection{Finding $\Delta CS$ on Graph}
            \begin{figure}[h]
                \centering
                % \includegraphics{}
                \caption{CV, EV and CS on graph for normal goods}
            \end{figure}
            
        \subsection{Dead Weight Loss(DWL)}
            \begin{definition}
                \textbf{Dead weight loss} is the \ul{loss in surplus} that could have been captured by someone in the economy but isn't.
            \end{definition}
            
            \begin{remark}
                DWL is due to a \textbf{substitution effect} - changing relative prices causes people to move away from relatively more expensive good, lowering tax revenue.
            \end{remark}
            
            \begin{remark}
                Lump-sum tax causes pure income effect and therefore no dead weight loss. We calculate the DWL from taxation using 
                \[
                    DWL = |T - L|
                \]
                where $T$ is the actual tax revenue.
            \end{remark}
            
            \begin{figure}[h]
                \centering
                % \includegraphics{}
                \caption{Lump-sum revenue, tax revenue and dead weight loss on graph}
            \end{figure}
            
            \subsubsection{Calculating DWL from taxation}
                \begin{enumerate}
                    \item Put composite good on y-axis of the choice diagram.
                    \item Find \ul{post tax} choice $\vec{x}_t$ and utility level $\overline{u}_t$.
                    \item Calculate $EV = I - e(\vec{p}_{initial}, \overline{u}_t)$. And notice $EV$ is the maximum amount of lump sum tax could be charged leaving consumer at least as well as charging them proportional tax $t$ on commodity.
                    \item Calculate $T = t*p_x*x^t$.
                    \item $DWL = |T - L|$. (\ul{focus on the absolute value})
                \end{enumerate}
            
        \subsection{Duality}
            \begin{remark}
                Notice at the optimal, 
                \[
                    v(\vec{p}, e(\vec{p}, u)) = u
                \]
                and
                \[
                    e(\vec{p}, v(\vec{p}, I)) = I
                \]
                Indirect utility function $v(\vec{p}, I)$ and expenditure function $e(\vec{p}, u)$ are invert of each other.
            \end{remark}
    
        \subsection{Risk and Uncertainty}
            \subsubsection{Setup}
                \paragraph{Axes} Accident state ($x_A$) consumption on x-axis and safe state consumption ($x_S$) on y-axis.
                
                \begin{figure}[h]
                    \centering
                    % \includegraphics{}
                    \caption{Coordinate used to analyze uncertainty}
                \end{figure}
                
                \begin{definition}
                    A \textbf{gamble} is a bundle giving payoff in different states, and each state has its probability of being realized.
                    \[
                        (x_A, x_S) =
                        \begin{cases}
                            x_A \tx{ with probability } \delta \\
                            x_S \tx{ with probability} 1 - \delta \\
                        \end{cases}
                    \]
                \end{definition}
            
            \subsubsection{Budget Constraint}
                \begin{remark} 
                    \textbf{Endowed income} here is the consumption in each state when there are no interventions.
                \end{remark}
                
                \begin{definition}
                    A general form of \text{insurance} is paying \textbf{premium} $p$ in \ul{both} states and get \textbf{benefit} $b$ only in a certain state.
                \end{definition}
                
                \begin{example}
                    With endowed income $(e_A, e_S)$, an insurance costs premium $p$ in both states and pays benefit $b$ in accident state only transforms the consumption bundle
                    \[
                        \begin{cases}
                            x_A = e_A \\
                            x_S = e_S \\
                        \end{cases} \rightarrow
                        \begin{cases}
                            x_A = e_A - p + b \\
                            x_S = e_S - p
                        \end{cases}
                    \]
                \end{example}
                
                \begin{remark}
                    We usually consider two types of insurance contracts:
                    \begin{enumerate}
                        \item \textbf{Buy or not:} Comparing the \ul{expected utility level} with and without insurance.
                        \item \textbf{How much to buy:} \textcolor{red}{negative amount allowed}.
                    \end{enumerate}
                \end{remark}
                
                \par Consider the insurance contract
                \[
                    b = \gamma p
                \]
                Then the budget constraint between $x_A$ and $x_S$ becomes
                \[
                    x_S = e_S + \frac{\gamma}{1 - \gamma}(e_A - x_A)
                \]
                Therefore we can find the opportunity cost for each unit of $x_A$ is 
                \[
                    \frac{dx_S}{dx_A} = - \frac{\gamma}{1 - \gamma}
                \]
            \paragraph{Exit Question}
            \textcolor{blue}{When insurance gets more expensive ($\gamma \uparrow$), how does the budget constraint change?} The budget constraint \textbf{rotates} around the endowment point. It's getting steeper (with accident consumption on x-axis and safe consumption on y-axis) as more safe state consumption are needed to be given up ($p$) for one more unit of accident state consumption ($b - p = (1 - \gamma)*b$).
\end{document}