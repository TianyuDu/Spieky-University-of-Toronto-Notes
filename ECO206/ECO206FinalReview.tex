\documentclass[]{article}

\usepackage{spikey}
\usepackage{amsmath}
\usepackage{amssymb}
\usepackage{float}
\usepackage{soul}

\title{ECO206 Final Review Notes}
\author{Tianyu Du}

\begin{document}
    \maketitle
    \tableofcontents

    \section{Lecture 1 Budget Constraints}
    
        \subsection{Notation-Bundle}
            \begin{definition}
                If we have $n$ goods, then vector $(x_1^A, x_2^A, \dots, x_n^A) \in \R^n_+$ represents a \textbf{bundle}, where $x_1^A$ represents a quantity($x$) of good 1 in bundle A. 
            \end{definition}
            
        \subsection{Types of Income}
            \paragraph{Exogenous}independent to price of goods.
            \paragraph{Endogenous}bundle of goods endowed, value dependents on price of goods.
            
        \subsection{Budget Set}
            \begin{definition}
                \textbf{Budget Set} is the set of all affordable consumption \emph{bundles}. Let $\vec{e}=(\omega_1,\omega_2,\dots,\omega_n) \in \R^n_+$ denote the endowment of consumer and $I$ denote the exogenous income, and consumers are price-takers facing price vector $\vec{p}$. Then
                \[
                    \mathcal{B} = \big\{\vec{x}\in \R^n_+ \vert \vec{x}\cdot\vec{p} \leq \vec{e}\cdot\vec{p} + I  \big\}
                \]
            \end{definition}
            
        \subsection{Opportunity Cost}
            \par Opportunity cost is the rate at which one good can be traded for another though the \emph{market}. It captures what the consumer is \textbf{able} to trade at market price, not what the consumer is \emph{willing} to trade.
            \par Consider budget constraint $\vec{x}\cdot\vec{p}=I$. Take the total differential,
            \[
                \sum_i \pd{\vec{x}\cdot\vec{p}}{x_i} d{x_i} = 0
            \]
            And the opportunity cost for one unit of good $x$ is shown to be 
            \[
                \frac{dy}{dx} = -\frac{p_x}{p_y}
            \]
            
        \subsection{Income and Price changes}
        \par \ul{Pure Income Changes: } \emph{parallel shift} of budget line.
        \par \ul{Price changes: } \emph{rotation} of budget line around the \emph{invariant point} (i.e. the consumption bundle that is not affected by the price change).
            \begin{enumerate}
                \item Exogenous income: Invariant point $\impliedby$ $\{x_1, \dots, x_i=0, \dots, x_n\}$ with $p_i$ changes. 
                \item Endogenous income: Invariant point $\impliedby$ endowment point $\vec{e}$.
            \end{enumerate}
            
    \section{Lecture 2 Preferences and Utility}
        \subsection{Tastes as Binary Relations}
            \begin{definition}
                \begin{enumerate}
                    \item Bundle A $(x_1^A, x_2^A)$ is \textbf{strictly preferred to} bundle B $(x_1^B, x_2^B)$.
                        \[
                            (x_1^A, x_2^A) \succ (x_1^B, x_2^B)
                        \]
                    \item Bundle A is \textbf{at least as good as} bundle B
                        \[
                            (x_1^A, x_2^A) \succcurlyeq (x_1^B, x_2^B)
                        \]
                    \item The consumer is indifferent between bundle A and bundle B
                        \[
                            (x_1^A, x_2^A) \sim (x_1^B, x_2^B)
                        \]
                \end{enumerate}
            \end{definition}
        
        \subsection{Rationality Assumptions}
            \begin{definition}
                \textbf{Complete: } For all bundles $A=(x_1^A, x_2^A)$ and $B=(x_1^B, x_2^B)$.
                \[
                    A \succcurlyeq B \lor B \succcurlyeq A
                \]
            \end{definition}
            
            \begin{definition}
                \textbf{Transitive: } for any three bundles, $A, B, C$
                \[
                    A \succcurlyeq B \land B \succcurlyeq C \implies A \succcurlyeq C
                \]
            \end{definition}
        
        \subsection{Convenience Assumptions}
            \begin{definition}
                \textbf{Monotonic: }
                \[
                    A \succcurlyeq B\ \impliedby \ x_i^A \geq x_i^B \ \forall \ i
                \]
                \[
                    A \succ B\ \impliedby \ x_i^A > x_i^B \ \forall \ i
                \]
            \end{definition}
            
            \begin{definition}
                \textbf{Convexity: } Suppose $A = (x_1^A, x_2^A) \sim B = (x_1^B, x_2^B)$
                \[
                    \alpha A + (1 - \alpha) B \succcurlyeq A,\ \forall \ \alpha \in [0, 1]
                \]
            \end{definition}
            
            \begin{definition}
                \textbf{Continuity: } (no sudden jumps)
            \end{definition}
        
        \subsection{Utility Function and Indifference Curve}
            \begin{definition}
                Let $\mathcal{X}$ denote the consumption set and a \textbf{utility function} is a real-valued function $u:\mathcal{X} \rightarrow \R_+$ so that
                \[
                    A \succcurlyeq B \iff u(A) \geq u(B)
                \]
                and
                \[
                    A \succ B \iff u(A) > u(B)
                \]
            \end{definition}
            
            \begin{remark}
                The \ul{positive monotone transformations} of utility function of a preference also captures the same preference. 
            \end{remark}
            
            \begin{definition}
                The \textbf{indifference curve} corresponding to utility level $\overline{u}$ is a set of consumption bundles
                \[
                    \{\vec{x} \in \R^n_+\ \vert\ u(\vec{x}) = \overline{u} \}
                \]
            \end{definition}
            
        \subsection{Marginal Rate of Substitution}
            \begin{definition}
                The \textbf{MRS} is the number of good $y$ that a consumer is \textbf{willing} to given up in order to get one more unit of good $x$.
                \\Take total differential of equation $u(\vec{x}) = \overline{u}$,
                \[
                    \sum_{i}{
                        \pd{u(\vec{x})}{x_i} d{x_i} = 0
                        }
                    \implies
                    \frac{dy}{dx} = -\frac{MU_x}{MU_y}
                \]
                And MRS is a function of bundle $\vec{x}$.
            \end{definition}
            
            \paragraph{Diminishing MRS} is when the \textbf{absolute MRS} decreases as more good $x$ is consumed along the indifference curve.
            
        \subsection{Different Types of Tastes}
            \begin{itemize}
                \item \textbf{Perfect Substitutes} MRS constant at every bundle.
                \item \textbf{Perfect Complements} Unwilling to substitute.
                \item \textbf{Homothetic} MRS constant on a ray ($\beta \vec{x_o},\ \beta > 0$) through origin.
                \item \textbf{Quasi-linear} MRS depends on $\vec{x_{-i}}$ where $x_i$ is excluded.
            \end{itemize}
            
    \section{Lecture 3 Choice}
        \paragraph{Diminishing MRS in words} Comparing two bundles on the \emph{same} indifferent curve. At each bundle consider how much of $y$ we are willing to given up for an additional unit of $x$. \emph{If we have relatively \ul{more} $x$ in one bundle, then we are \ul{less} willing to give up $y$ at that bundle compared to the other bundle}.
        
        \begin{remark}
            Perfect Substitutes (linear in every commodity) $\implies$ quasi-linear.
        \end{remark}
        
        \subsection{Lagrangian Multiplier}
            \paragraph{Sufficiency} FOCs from Lagrangian multiplier is both necessary and sufficient if and only if
                \begin{enumerate}
                    \item There are no "flat spots" on indifference curves.
                    \item All goods are essential (\ul{interior solution})
                    \item Both choice set (characterized by budget constraint) and tastes (characterized by preference and utility function, quasi-concave).
                \end{enumerate}
            \paragraph{Uniqueness} Assuming \textbf{convexity} of preferences and choice set guarantees \emph{uniqueness} of optimal choice. (Reducing problem to a convex optimization problem).
        
        \subsection{Problem Solving}
            \begin{enumerate}
                \item Determine preference type.
                \item Check expectation of interior solution.
                \begin{itemize}
                    \item If interior solution (e.g. Cobb-Douglas): setup and solve Lagrangian multiplier, check sufficiency.
                    \item If not:
                    \begin{enumerate}
                        \item Perfect Substitute: Corner Solution if $MRS \neq rel.Price$.
                        \item Perfect Complement: Solution on path $x_1=\gamma x_2$.
                        \item Quasi-linear: interior solutions found for some parameter values only. \emph{Use LM to find expression of solution and check the condition for positiveness}.
                    \end{enumerate}
                \end{itemize}
                \item Check for \emph{multiple optimal solutions}.
            \end{enumerate}
            
    \section{Lectuer 4 Demand and Income Effects}
        \subsection{Demand}
            \begin{definition}
                A \textbf{demand function} is an equation that expresses \ul{bundle choices} as a \ul{function} of prices and incomes.
            \end{definition}
            
        \subsection{Income Effects}
            \begin{definition}
                \textbf{Income effect} is the change in behavior arising from just a change in income and leads to a parallel shift in the budget constraint.
            \end{definition}
            
            \begin{remark}
                With quasi-linear preference with numeraire $x_1$, then the demand for all \ul{other} commodities other than $x_1$ have no income effect. We often refer to quasi-linear goods as the borderline goods between normal and inferior.
            \end{remark}
            
            \begin{remark}
                With pure income effect and homothetic preference, the ratios of commodities before and after are the same.
            \end{remark}
            
            \begin{definition}
                A \textbf{Engel curve} $x_i^D(I, \vec{p})$ captures the relationship between \ul{demand} and \ul{income}. Note: put $I$ on y-axis.
            \end{definition}
            
            \begin{itemize}
                \item \textbf{Normal goods} $\pd{x}{I} > 0$
                \item \textbf{Inferior goods} $\pd{x}{I} < 0$
            \end{itemize}
            
        \subsection{Decomposition of Substitution Effect and Income Effect}
            \begin{itemize}
                \item \textbf{Substitution Effect} changing \ul{relative prices} holding the \ul{utility level constant}.
                \item \textbf{Income Effect} changing \ul{purchasing power} of the consumer's income holding 
                \ul{relative prices constant}.
            \end{itemize}
            
    \section{Lecture 5 Income and Substitution Effects}
        \begin{definition}
            \textbf{Hicksian substitution effect} derived by compensating consumer enough income so that the consumer can reach the \ul{original utility level} after the price changes.
        \end{definition}
        \begin{definition}
            \textbf{Slutsky substitution effect} derived by compensating consumer enough income so that the consumer can afford the \ul{original optimal bundle} after the price changes.
        \end{definition}
        
        \subsection{Expenditure Minimization}
            \begin{definition}
                With \ul{new prices}, given consumer enough (hypothetical) income to just reach the \ul{original indifference curve}. Mathematically,
                \[
                    min_{x_1,x_2} p_1^{final} x_1 + p_2^{final} x_2 \ s.t.\ u(x_1, x_2) = U^{initial} = u(x_1^{initial}, x_2^{initial})
                \]
            \end{definition}
            
            \par Solving the expenditure using Lagrangian multiplier method 
                \[
                    \mc{L}(\vec{x}, \lambda) = \vec{x} \cdot \vec{p} + \lambda\{\overline{u} - u(\vec{x}) \}
                \]
                Let $\vec{x}^{SE} = (x_1^{SE}, x_2^{SE}, \dots, x_N^{SE}) = argmin_{\vec{x}}\{\vec{p} \cdot \vec{x}\ s.t.\ u(\vec{x}) = \overline{u}\}$ and $\vec{p}^{new} \cdot \vec{x}$ denote the \textbf{compensated income}.
            
            \begin{definition}
                From expenditure minimization, we define the \textbf{expenditure function} $e(\vec{p}, 
                \overline{u}): \R^{n+1}_+ \rightarrow \R_+$ as a function of price vector $\vec{p}$ and utility level $\overline{u}$.
                \[
                    e(\vec{p}, \overline{u}) := min_{\vec{x}} \{\vec{p} \cdot \vec{x} \ s.t. \ u(\vec{x}) = \overline{u} \}
                \]
            \end{definition}
            
        \subsection{Income Effect}
            \begin{remark} Decomposing substitution and income effects.
                \begin{enumerate}
                    \item The utility value achieved is $\overline{u} = u(\vec{x}^*_1)$.
                    \item Given prices $\vec{p}_1, \vec{p}_2$, use \ul{utility maximization} to find optimal bundles $\vec{x}^*_1, \vec{x}^*_2$.
                    \item Use \ul{expenditure minimization} with respect to \ul{new price} and \ul{original utility level} to find $\vec{x}_{SE}(\vec{p}_2, \overline{u})$.
                    \item Substitution effect is $\vec{x}_{SE} - \vec{x}^*_1$.
                    \item Income effect is $\vec{x}^*_2 - \vec{x}_{SE}$.
                \end{enumerate}
            \end{remark}
            \begin{figure}
                \centering
                \includegraphics{}
                \caption{Decomposition of two effects}
                \label{fig:my_label}
            \end{figure}
        
        \subsection{Compensated Demand Curve}
            \begin{remark}
                \textbf{Regular demand (Marshallian)} changes in $x_1$ as $p_1$ changes when \ul{income held fixed}. (On graph: initial to final bundles)
            \end{remark}
            
            \begin{definition}
                \textbf{Compensated demand (Hicksian)} captures the changes in $x_1$ as $p_1$ changes, when \ul{utility level is held fixed}(hitting the same IC) by compensating consumer additional income. (On graph: initial to SE)
                \\\textbf{Notation}:
                \\Given prices $p_i$ and prices of other goods, $\vec{p}_{-i}$ and original utility $\overline{u}$, compensated demand for $x_i$ is denoted as
                \[
                    h_i(p_i, \vec{p}_{-1}, \overline{u}): \R^n_+ \rightarrow \R_+
                \]
            \end{definition}
\end{document}