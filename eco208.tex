\documentclass[11pt]{article}

\title{ECO208 Macroeconomic Theory \\ Lecture Note}
\author{Tianyu Du}
\date{\today}

\usepackage{amsmath}
\usepackage{amssymb}
\usepackage{pgfplots}
\usepackage{graphicx}
\usepackage{enumitem}
\usepackage{hyperref}
\usepackage{fancyhdr}
\usepackage{perpage}
\usepackage{float}

\lhead{Notes by T.Du}
\usepackage[
	type={CC},
	modifier={by-sa},
	version={3.0},
]
{doclicense}

\begin{document}
\maketitle
\doclicenseThis
\tableofcontents

\section{Lecture 1 May. 8 2018}
\paragraph{Note} No notes for this lecture.


\section{Lecture 2 May. 9 2018}

\subsection{Measuring GDP, three measures}
	\paragraph{Gross Domestic Product (GDP)} dollar value of goods and services produced during a given period of time \underbar{within the borders of a country}.
	
	\paragraph{Product/Value-added approach} Sum of \underbar{value-added} to goods and services in production across all productive units in the economy.
	
	\paragraph{Expenditure approach} Sum of all spending on goods and services in the economy.
	
	\paragraph{Income approach} Sum of all incomes received by economic agents contributing to production.

	\paragraph{Note} All three measures must add up to the same value.

\subsection{Other measures}
	\paragraph{Gross National Product (GNP)} Dollar value of goods and services produced during a given period of time \underbar{by the residents or citizens of a country}.
	
	\paragraph{Net Factor Payments (NFP)} Income paid to domestic factors of production by the rest of the world, minus income paid to foreign factors of production by the domestic economy.

	\[
		GNP = GDP + NFP
	\]
	
\subsection{Problems with Measures of GDP}
\begin{itemize}
	\item Inequality
	\item Non-market activity/home production
	\item Underground economy
	\item Value-added for government services (e.g. military, education)
\end{itemize}

\subsection{Nominal GDP and Real GDP}
	\subsubsection{Real GDP at Constant Price}
		\paragraph{Computation} Let there be $N$ goods in economy, nominal GDP at time $t$ is 
		\[
			GDP_t = \sum_{n=1}^N p_n^t q_n^t
		\]
		and Real GDP at constant prices at $t$ using based year $t = b$ is 
		\[
			RGDP_t^b = \sum_{n=1}^N p_n^b q_n^t
		\]
	\subsubsection{Chain-weighted Real GDP}
		\paragraph{Definition} Real GDP using chain=weighted method between year $t$ and year $t+1$ is 
		\[
			RGDP_{t+1}^{c, t} = GDP_t \times (1+g_c)
		\] where 
		\[
			1 + g_c = \sqrt{(1+g_t) \times (1 + g_{t+1}))}
		\] and 
		\[
			1 + g_t = \frac{RGDP_{t+1}^t}{RGDP_{t}^t},\ 1 + g_{t+1} = \frac{RGDP_{t+1}^{t+1}}{RGDP_{t}^{t+1}}
		\] where $g_i$ is the growth rate of real GDP using $t=i$ as base year.
		

\subsection{Price Level}
	\paragraph{(General) Price level} A \underbar{hypothetical} measure of overall prices for some set of goods and services in the economy.
	\[
		\text{Implicit GDP Price Deflator} = \frac{\text{Nominal GDP}}{\text{Real GDP}} \times 100
	\]
	\paragraph{Consumer Price Index (CPI)}is the other most commonly used price level measure.
	\[
		CPI_t^b = \frac{\sum_{n=1}^N p_n^t q_n^b}{\sum_{n=1}^N p_n^b q_n^b} \times 100
	\]

\subsection{Saving, Wealth and Capital}
	\paragraph{Flow} is a quantity measured per unit of time.
	\paragraph{Stock} is a quantity measured at a moment of time.
	\paragraph{Private disposable income}
	\[
		Y^d = Y + NFP + TR + INT - T
	\]
	\paragraph{Private sector saving}
	\[
		S^p = Y^d - C
	\]
	\paragraph{Government saving}
	\[
		S^g = T - TR - INT - G
	\]
	\paragraph{National Saving}
	\[
		S = C + I + G + NX + NFP - C - G
	\]
	
\subsection{Labour Market Measurement}
	\paragraph{Employed} Worked full-time or part-time during the \underbar{last week}.
	\paragraph{Unemployed} Not employed during the last week, but \underbar{actively searched for work} during the past 4 weeks.
	\paragraph{Out of the labour force}
	
\section{Lecture 3 May. 10 2018}
	\subsection{Representative Consumer: Preference}
	\paragraph{Good 1: Physical good} The average bundle of goods purchased and consumed the representative consumer. Also represents \underbar{consumption} $C$.
	\paragraph{Good 2: Leisure} time spent not working in the market.
	
	\paragraph{Indifference curve} is a set of points in the space of consumption bundles among which the consumer is indifferent.
	
	\paragraph{Utility function}
	\[
		U(C, I)
	\]
	
	\subsection{Budget constraint}
	\paragraph{Constraint}
	\[
		w N^s + \pi - T \geq C
	\]
\end{document}



















