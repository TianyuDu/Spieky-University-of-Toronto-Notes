\documentclass[11pt]{article}
\usepackage{spikey}
\usepackage{amsmath}
\usepackage{amssymb}
\usepackage{soul}
\usepackage{float}
\usepackage{graphicx}
\usepackage{hyperref}
\usepackage{xcolor}
\usepackage{chngcntr}
\usepackage{centernot}
\usepackage{datetime}
\usepackage[shortlabels]{enumitem}

\usepackage[margin=1truein]{geometry}
\usepackage{setspace}
\linespread{1.15}

\title{Introduction to Real Analysis}
\date{\today}
\author{Tianyu Du}

\usepackage[
    type={CC},
    modifier={by-nc},
    version={4.0},
]{doclicense}

\counterwithin{equation}{section}
\counterwithin{theorem}{section}
\counterwithin{lemma}{section}
\counterwithin{corollary}{section}
\counterwithin{proposition}{section}
\counterwithin{remark}{section}
\counterwithin{example}{section}
\counterwithin{definition}{section}

\begin{document}
	\maketitle
	\doclicenseThis
	\tableofcontents
	\newpage
	\section{The Axiom of Completeness}
	\subsection{Preliminaries}
	\begin{definition}
		A set $A \subset \R$ is \textbf{bounded above} if
		\begin{align}
			\exists u \in \R\ s.t.\ \forall a \in A,\ u \geq a
		\end{align} 
		It is said to be \textbf{bounded below} if
		\begin{align}
			\exists l \in \R\ s.t.\ \forall a \in A,\ l \leq a
		\end{align}
	\end{definition}
	
	\begin{example}
		The set of integers, $\Z$, is neither bounded from above nor below. Sets $\{1, 2, 3\}$ and $\{\frac{1}{n}: n \in \N \}$ are bounded from both above and below.
	\end{example}
	
	\begin{notation}
		Let $A \subset \R$, we use $A^\uparrow$ and $A^\downarrow$ to denote collections of upper bounds of $A$ and lower bounds of $A$. Note that $A^\uparrow$ and $A^\downarrow$ are potentially empty.
	\end{notation}
	
	\begin{definition}
		A real number $s \in \R$ is the \textbf{least upper bound(supremum)} for a set $A \subset \R$ if $s \in A^\uparrow$ and $\forall u \in A^\uparrow,\ s \leq u$. Such $s$ is denoted as $s := \sup A$.
	\end{definition}
	
	\begin{definition}
		A real number $f \in \R$ is the \textbf{greatest lower bound (infimum)} for $A$ if $f \in A^\downarrow$ and $\forall l \in A^\downarrow,\ l \leq f$. Such $f$ is often written as $f := \inf A$.
	\end{definition}
	
	\begin{axiom}[The Axiom of Completeness/Least Upper Bounded Property]
		$\forall \es \neq A \subset \R$ such that $A^\uparrow \neq \es$, $\exists \sup A$.
	\end{axiom}
	
	\begin{definition}
		Let $\es \neq A \subset \R$, $a_0 \in A$ is the \textbf{maximum} of $A$ if $\forall a \in A, a_0 \geq a $; $a_1 \in A$ is the \textbf{minimum} of $A$ if 
		$\forall a \in A, a_1 \leq a$.
	\end{definition}
	
	\begin{example}
		$\Q \subset \R$ does not satisfy the axiom of completeness.
	\end{example}
	
	\begin{proposition}
		Let $\es \neq A \subset \R$ bounded above, and $c \in \R$. Define $c + A := \{a + c: a \in A\}$. Then
		\begin{align}
			\sup (c + A) = c + \sup A
		\end{align}
	\end{proposition}
	
	\begin{proof}
		\emph{Step 1: Show $c + \sup A \in (c + A)^\uparrow$:} \\
		Let $x \in c + A$, $\exists\ a \in A\ s.t.\ x = c + a$. Then, $x = c + a \leq c + \sup A$. Therefore, $x \leq c + \sup A\ \forall x \in A$, which implies what desired. \\
		\emph{Step 2: Show $\forall u \in (c + A)^\uparrow,\ c + \sup A \leq u$: }\\
		Let $u \in (c + A)^\uparrow$, then $u \geq c + a\ \forall a \in A \implies u - c \geq a\ \forall a \in A \implies u - c \in A\uparrow \implies u - c \geq \sup A \implies u \geq c + \sup A$. \\
		Hence, $\sup (c + A) = c + \sup A$.
	\end{proof}
	
	\begin{lemma}[Alternative Definition of Supremum]
		Let $s \in A^\uparrow$ for some nonempty $A \subset \R$. The following statements are equivalent:
		\begin{enumerate}[(i)]
			\item $s = \sup A$;
			\item $\forall \varepsilon, \exists a \in A,\ s.t.\ a > s - \varepsilon$ (i.e. $s - \varepsilon \notin A^\uparrow$).
		\end{enumerate}
	\end{lemma}
	
	\begin{proof}
		Immediately.
	\end{proof}
	
	\begin{theorem}[Nested Interval Property]
		Let $(I_n)_n$ be a sequence of closed intervals $I_n := [a_n, b_n]$ such that these intervals are \emph{nested} in a sense that
		\begin{align}
			I_{n+1} \subset I_n\ \forall n \in \N
		\end{align}
		Then,
		\begin{align}
			\bigcap_{n \in \N} I_n \neq \es
		\end{align}
	\end{theorem}
	
	\begin{proof}
		Note that the sequence $(a_n)_{n \in \N}$ is bounded above by any $b_k$, by the completeness axiom, there exists $a^* := \sup_{n \in \N} a_n$. Since $a^* \in (a_n)^\uparrow$, $a^* \geq a_n\ \forall n \in \N$. Further, because $a^*$ is the \emph{least} upper bound, then for every upper bound $b_n$, it must be $a^* \leq b_n\ \forall n \in \N$. Therefore, $x^* \in [a_n, b_n]\ \forall n \in \N$. That is, $x^* \in \bigcap_{n \in \N} I_n$.
	\end{proof}
	
	Note that NIP requires all intervals to be closed. One instance when this fails to hold: $\bigcap_{n \in \N} \left(0, \frac{1}{n} \right) = \es$.
	
	\begin{theorem}[Archimedean Property] \quad
		\begin{enumerate}[(i)]
			\item $\forall x \in \R,\ \exists n \in \N\ s.t.\ n > x$;
			\item $\forall y \in \R_{++},\ \exists n \in \N\ s.t. \frac{1}{n} < y$.
		\end{enumerate}
	\end{theorem}
	
	Archimedean property \emph{of natural numbers} can be interpreted as \emph{there is no real number that bounds $\N$}. This interpretation can be seen by considering the negations of above statements:
	\begin{enumerate}[(i)]
		\item $\exists x \in \R\ s.t.\ \forall n \in \N,\ n \leq x$;
		\item $\exists y \in \R_{++}\ s.t.\ \forall n \in \N,\ y \leq \frac{1}{n}$.
	\end{enumerate}
	
	\begin{proof}[Proof of (i) by Contradiction.]
		Suppose the negated statement (i) is true, $\N$ is bounded above. By the completeness axiom, there exists $a^* := \sup \N$. $\exists n \in \N\ s.t.\ a^* - 1< n$. In this case, $a^* < n + 1 \in \N$, which means $a^* \notin \N^\uparrow$ and leads to a contradiction.
	\end{proof}
	
	\begin{proof}[Proof of (ii).]
		Let $y^* \in \R_{++}$, take $x = \frac{1}{y}$. By statement (i), there exists $n^* \in \N$ such that $n >	\frac{1}{y}$. Because $y > 0$, $\frac{1}{n} < y$.
	\end{proof}
	
	\subsection{Density of Rational Numbers}
	\begin{theorem}
		For every $a, b \in \R$ such that $a < b$, there exists $r \in \Q$ such that $a < r < b$.
	\end{theorem}
	
	The above theorem says $\Q$ is in fact \textbf{dense} in $\R$. More generally, one says a set $A \subset X$ is dense whenever the closure of $A$, $\overline{A} = X$.
	
	\begin{proof}
		\emph{Step 1:} Since $b - a > 0$, by the first Archimedean property, there exists $n \in \N$ such that $n > \frac{1}{b - a}$. Such natural number satisfies $\frac{1}{n} < b - a$.\\
		\emph{Step 2:} Let $m$ be smallest integer such that $m > an$. That is, $m - 1 \leq an < m$. Obviously, $a < \frac{m}{n}$ since $n > 0$. Further, since $m \leq an + 1$, with results from step (i), $m < bn - 1 + 1 = bn$, and $\frac{m}{n} < b$. Therefore $\frac{m}{n} \in (a, b)$.
	\end{proof}
	
	\begin{theorem}
		$\exists \alpha \in \R\ s.t. \alpha^2 = 2$.
	\end{theorem}
	
	\begin{proof}
		Let $\Omega := \{t \in \R: t^2 < 2\}$, which is obviously a set in $\R$ bounded from above. By the completeness axiom, $\Omega$ possesses a supremum, and we claim $\alpha := \sup \Omega$ satisfies $\alpha^2 = 2$. Suppose $\alpha^2 > 2$, then there exists $\varepsilon > 0$ such that $\alpha^2 - 2 \alpha \varepsilon + \varepsilon^2 > 2$. Therefore, $\alpha > \alpha - \varepsilon \in \Omega^\uparrow$, which contradicts the fact that $\alpha$ is the least upper bound. Suppose $\alpha^2 < 2$, then there exists some $\varepsilon > 0$ such that $\alpha + \varepsilon \in \Omega$, which contradicts the assumption that $\alpha$ is an upper bound. Hence, it must be the case that $\alpha^2 = 2$.
	\end{proof}
\end{document}























