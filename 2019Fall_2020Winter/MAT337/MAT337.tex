\documentclass[11pt]{article}
\usepackage{spikey}
\usepackage{amsmath}
\usepackage{amssymb}
\usepackage{soul}
\usepackage{float}
\usepackage{graphicx}
\usepackage{hyperref}
\usepackage{xcolor}
\usepackage{chngcntr}
\usepackage{centernot}
\usepackage{datetime}
\usepackage[shortlabels]{enumitem}

\usepackage[margin=1truein]{geometry}

\title{Elements of Real Analysis \\ \small Based on Lecture Notes for MAT337: Introduction to Real Analysis (2019Winter)}
\date{\today}
\author{Tianyu Du}

\usepackage[
    type={CC},
    modifier={by-nc},
    version={4.0},
]{doclicense}

\counterwithin{equation}{section}
\counterwithin{theorem}{section}
\counterwithin{lemma}{section}
\counterwithin{corollary}{section}
\counterwithin{proposition}{section}
\counterwithin{remark}{section}
\counterwithin{example}{section}
\counterwithin{definition}{section}

\newcommand{\seq}[2]{({#1}_{#2})_{{#2}=1}^\infty}
\newcommand{\ser}[2]{\sum_{{#2}=1}^\infty {#1}_{#2}}

\begin{document}
	\maketitle
	\doclicenseThis
	\begin{itemize}
		\item GitHub: \url{https://github.com/TianyuDu/Spikey_UofT_Notes}
		\item Website: \url{TianyuDu.com/notes}
	\end{itemize}
	\tableofcontents
	\newpage
	
    \section{Real Numbers}
        \subsection{Definitions}
        \begin{definition}
            Reals are \ul{proper initial segments} of $\Q$ with no maximum.
        \end{definition}
    
        \begin{definition}
            A subset $A \subset \Q$ is an \textbf{initial segment} if 
            \begin{equation}
                y \in A, x \in \Q, x < y \implies x \in A
            \end{equation}
        \end{definition}
	   
        \begin{definition}
            $A$ is \textbf{proper} if $A \neq \Q$.
        \end{definition}
        
        \begin{definition}
            $A$ has no maximal elements if 
            \begin{equation}
                \forall x \in A,\ \exists y \in A\ s.t.\ y > x
            \end{equation}
        \end{definition}
        
        \begin{example}
            \begin{align}
                \sqrt{2} &\approx A_{\sqrt{2}} := \{q \in \Q: q < \sqrt{2}\} \equiv \{q \in \Q: q \leq 0 \lor q^2 < 2\} \\
                x &\approx A_x := \{q \in \Q: q < x\}
            \end{align}
        \end{example}
        
        \subsection{The Axiom of Completeness}
            \begin{axiom}[Axiom of Completeness]
                Every non-empty subset $B \subset \R$ that is bounded has a supremum (i.e. the least upper bound). That's
                \begin{equation}
                    \forall B \subset \R,\ s.t.\ B \neq \varnothing\ \exists b \in \R\ s.t.\ 
                    \begin{cases}
                        \forall x \in B,\ x \leq b \tx{ (upper bound)} \\
                        \forall c \in \R (\forall x \in B, x \leq c) \implies b \leq c \tx{ (least upper bound)}
                    \end{cases}
                \end{equation}
            \end{axiom}
            
            \begin{theorem}
                $\Q$ is \emph{dense} in $\R$, that's
                \begin{equation}
                    \forall x < y \in \R,\ \exists q \in \Q\ s.t.\ x < q < y
                \end{equation}
            \end{theorem}
            
            \begin{theorem}[Cardinality]
                Let $A, B$ be non-empty subsets of $\R$, then the following statements are equivalent:
                \begin{enumerate}[(i)]
                    \item $\exists h: A \to B$ such that $h$ is bijective;
                    \item $\exists f: A \to B$ and $g: B \to A$ such that both $f$ and $g$ are injective.
                \end{enumerate}
            \end{theorem}
            
            \begin{proof}
                (i) is the definition for sets $A$ and $B$ to have the same cardinality. And the existence of injection from $A$ to $B$ implies the cardinality of $A$ cannot be greater than the cardinality of $B$. Similarly, the existence of injection from $B$ to $A$ implies the cardinality of $B$ cannot be greater than the cardinality of $A$. Therefore $A$ and $B$ share the same cardinality.
            \end{proof}
            
            \begin{theorem}[Nested Intervals]
                Let $(I_n)$ be a sequence of closed and non-empty intervals in $\R$ such that
                \begin{equation}
                    I_0 \supset I_1 \supset I_2 \supset \cdots \supset I_n \supset \cdots
                \end{equation}
                then 
                \begin{equation}
                    \bigcap_{n \in \N} I_n \neq \varnothing
                \end{equation}
            \end{theorem}
            
            \begin{proof}
                Claim:
                \begin{equation}
                    x := \sup\{\min(I_n): n \in \N\} \in \bigcap_{n \in \N} I_n
                \end{equation}
                Let $n \in \N$, then $x \geq \min(I_n)$. Now show $x \leq \max(I_n)\ \forall n \in \N$. Suppose not, then $\exists k \in \N$ such that $x > \max(I_k)$. Then by the definition of supremum, there exists $j \in \N$ such that $\max(I_k) < \min(I_j)$. Note that if $k=j$, this leads to a contradiction. If $k < j$, then because $I_k\supset I_j$, $\max(I_k) \geq \max(I_j) \geq \min(I_j) \geq \min(I_k)$, this leads to a contradiction. If $k > j$, then $I_k \subset I_j$, thus $\min(I_j) \leq \min(I_k) \leq \max(I_k) \leq \max(I_j)$, which also leads to a contradiction.\\
                Therefore we conclude
                \begin{equation}
                    \min(I_n) \leq x \leq \max(I_n)\ \forall n \in \N
                \end{equation}
                therefore $x \in I_n\ \forall n \in \N$, so $x \in \bigcap_{n \in \N}I_n$.
            \end{proof}
            
            \begin{theorem}
                There exists no injection from $\R$ to $\N$.
            \end{theorem}
            
            \begin{proof}
                $\R$ has cardinality $c$ but $\N$ has cardinality $\aleph_0$.
            \end{proof}
        
    \section{Sequences and Series}
        \begin{definition}
            A sequence $\seq{a}{n}$ of real numbers \textbf{converges} to a real number $a$ if
            \begin{equation}
                \forall \varepsilon > 0\ \exists N \in \N\ s.t.\ \forall n > N\ |a_n - a| < \varepsilon
            \end{equation}
            If there does not exist such $a$, we conclude $\seq{a}{n}$ is \textbf{divergent}.
        \end{definition}
        
        \begin{theorem}
            Every convergent sequence is bounded.
        \end{theorem}
        
        \begin{proof}
            Let $\seq{a}{n}$ be a convergent sequence in $\R$ with limit point $a$. Then take $\varepsilon=1$, there exists $N \in \N$ such that $n > N \implies |a_n - a| < 1 \implies |a_n| < |a| + 1$. Take 
            \begin{equation}
                M := \max\{\max_{n \leq N}\{|a_n|\}, |a| +1 \}
            \end{equation}
            and the sequence is bounded by $M$.
        \end{proof}
        
        \begin{definition}
            Let $\seq{a}{n}$ be a sequence, then a sub-sequence of $(a_n)$ is any sequence in the form $(a_{n_k})_{k=1}^\infty$ such that $n_1 < n_2 < \cdots < n_k < \cdots$.
        \end{definition}
        
        \begin{remark}
            A sub-sequence can be generated with a strictly increasing function defined on $\N$ and a sequence $(a_n)$.
        \end{remark}
        
        \begin{theorem}[Bolzano-Weierstrass]
            Every bounded sequence has a convergent sub-sequence.
        \end{theorem}
        
        \begin{proof}
            Let $\seq{a}{n}$ be a bounded sequence bounded by $M > 0$. Define
            \begin{align}
                I_0 &:= [-M, M] \\
                J^0 &:= [-M, 0] \\
                J^1 &:= [0, M] \\
                X^0 &:= \{n \in \N: a_n \in J^0\} \\
                X^1 &:= \{n \in \N: a_n \in J^1\}
            \end{align}
            therefore $\N = X^0 \cup X^1$. Thus at least one of $X^0$ and $X^1$ is infinite. If $X^0$ is infinite, define $I_1 := J^0$, otherwise, define $I_1 := J^1$. Let 
            \begin{equation}
                A := \{x \in \R: \{n \in \N: x < a_n\} \tx{ is infinite} \}
            \end{equation}
            which is the lower bound of selected infinite half intervals. And define $a:= \inf(A)$. we can construct a sub-sequence, for each $n \in \N$, take $a_n \in I_n$. And by the nested interval theorem, the intersection of all those selected intervals is non-empty. And $a$ is the limit point of the constructed sequence. So a convergent sub-sequence exists.
        \end{proof}
        
        \begin{definition}
            A sequence $\seq{a}{n}$ is a \textbf{Cauchy} sequence if 
            \begin{equation}
                \forall \varepsilon > 0\ \exists N \in \N\ s.t.\ \forall m, n > N,\ |a_n - a_m| < \varepsilon
            \end{equation}
        \end{definition}
        
        \begin{theorem}[Convergent $\implies$ Cauchy]
            Every convergent sequence is a Cauchy sequence.
        \end{theorem}
        
        \begin{proof}
            Let $(a_n)$ be a convergent sequence, fix $\varepsilon > 0$. Suppose $(a_n) \to a$, take $\varepsilon^* = \varepsilon / 2$. Thus, there exists $N \in \N$ such that $\forall n > N, |a_n - a| < \varepsilon^* = \varepsilon / 2$. By taking such $N$, $\forall n, m > N$, both $|a_n - a|$ and $|a_m - a| < \varepsilon / 2$. By triangle inequality, $|a_n - a_m| \leq |a_n - a| + |a_m - a| < \varepsilon / 2 + \varepsilon / 2 = \varepsilon$. Hence, we've shown that for an arbitrary $\varepsilon > 0$, there exists $N \in \N$ such that $\forall m, n > N, |a_n - a_m| < \varepsilon$. Therefore $(a_n)$ is Cauchy.
        \end{proof}
        
        \begin{theorem}[Cauchy $\implies$ Convergent]
            Every Cauchy sequence is convergent.
        \end{theorem}
        
        \begin{proof}
            Let $(a_n)$ be a Cauchy sequence. \\
            Claim: $(a_n)$ is bounded.
            \begin{proof}[Proof. Bounded]
                Take $\varepsilon=1$, then $\exists N \in \N$ such that $\forall m, n > N, |a_n - a_m| < 1$. Take $m = N+1$ and define $a^* := a_m$. Then we have $\forall n > N$, $|a_n - a^*| < 1$, which implies $|a_n| < |a^*| + 1$. Define 
                \begin{equation}
                    M := \max\{
                        \max\{a_n: n \leq N\}, |a^*| + 1
                    \}
                \end{equation}
                So $(a_n)$ is bounded by $M$.
            \end{proof}
            Then by Bolzano-Weierstrass Theorem, there exists a sub-sequence $(a_{n_k})_{k=1}^\infty$ converges to some limit point $a \in \R$. We are going to show $(a_n) \to a$. Fix $\varepsilon>0$, by the convergence of the sub-sequence
            \begin{equation}
                \exists N_1 \in \N\ s.t.\ \forall n \geq N_1,\ |a_{n_k} - a| < \frac{\varepsilon}{2}
            \end{equation}
            Also since the sequence itself is Cauchy,
            \begin{equation}
                \exists N_2 \in \N,\ s.t.\ \forall m, n \geq N_2,\ |a_n - a| < \frac{\varepsilon}{2}
            \end{equation}
            Take $N^* := \max\{N_1, N2_\}$. Show $|a_n - a| < \varepsilon\ \forall n \geq N^*$. Note that 
            \begin{align}
                |a_n - a| &= |(a_n - a_{n_k}) + (a_{n_k} - a)| \\
                &\leq |a_n - a_{n_k}| + |a_{n_k} - a| \\
                &<\frac{\varepsilon}{2} + \frac{\varepsilon}{2} = \varepsilon
            \end{align}
            since $n_k \geq n$ by the definition of sub-sequences.
        \end{proof}
    
        \begin{corollary}
            A sequence is Cauchy if and only if it is convergent.
        \end{corollary}
        
        \begin{proof}
            Let $(a_n)$ denote a Cauchy sequence, then suppose a 
        \end{proof}
\end{document}