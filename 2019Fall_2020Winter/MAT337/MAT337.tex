\documentclass[11pt]{article}
\usepackage{spikey}
\usepackage{amsmath}
\usepackage{amssymb}
\usepackage{soul}
\usepackage{float}
\usepackage{graphicx}
\usepackage{hyperref}
\usepackage{xcolor}
\usepackage{chngcntr}
\usepackage{centernot}
\usepackage{datetime}
\usepackage[shortlabels]{enumitem}

\usepackage[margin=1truein]{geometry}
\usepackage{setspace}
\linespread{1.15}

\title{Introduction to Real Analysis}
\date{\today}
\author{Tianyu Du}

\usepackage[
    type={CC},
    modifier={by-nc},
    version={4.0},
]{doclicense}

\counterwithin{equation}{section}
\counterwithin{theorem}{section}
\counterwithin{lemma}{section}
\counterwithin{corollary}{section}
\counterwithin{proposition}{section}
\counterwithin{remark}{section}
\counterwithin{example}{section}
\counterwithin{definition}{section}

\begin{document}
	\maketitle
	\doclicenseThis
	\tableofcontents
	\newpage
	\section{The Axiom of Completeness}
	\subsection{Preliminaries}
	\begin{definition}
		A set $A \subset \R$ is \textbf{bounded above} if
		\begin{align}
			\exists u \in \R\ s.t.\ \forall a \in A,\ u \geq a
		\end{align} 
		It is said to be \textbf{bounded below} if
		\begin{align}
			\exists l \in \R\ s.t.\ \forall a \in A,\ l \leq a
		\end{align}
	\end{definition}
	
	\begin{example}
		The set of integers, $\Z$, is neither bounded from above nor below. Sets $\{1, 2, 3\}$ and $\{\frac{1}{n}: n \in \N \}$ are bounded from both above and below.
	\end{example}
	
	\begin{notation}
		Let $A \subset \R$, we use $A^\uparrow$ and $A^\downarrow$ to denote collections of upper bounds of $A$ and lower bounds of $A$. Note that $A^\uparrow$ and $A^\downarrow$ are potentially empty.
	\end{notation}
	
	\begin{definition}
		A real number $s \in \R$ is the \textbf{least upper bound(supremum)} for a set $A \subset \R$ if $s \in A^\uparrow$ and $\forall u \in A^\uparrow,\ s \leq u$. Such $s$ is denoted as $s := \sup A$.
	\end{definition}
	
	\begin{definition}
		A real number $f \in \R$ is the \textbf{greatest lower bound (infimum)} for $A$ if $f \in A^\downarrow$ and $\forall l \in A^\downarrow,\ l \leq f$. Such $f$ is often written as $f := \inf A$.
	\end{definition}
	
	\begin{axiom}[The Axiom of Completeness/Least Upper Bounded Property]
		$\forall \es \neq A \subset \R$ such that $A^\uparrow \neq \es$, $\exists \sup A$.
	\end{axiom}
	
	\begin{definition}
		Let $\es \neq A \subset \R$, $a_0 \in A$ is the \textbf{maximum} of $A$ if $\forall a \in A, a_0 \geq a $; $a_1 \in A$ is the \textbf{minimum} of $A$ if 
		$\forall a \in A, a_1 \leq a$.
	\end{definition}
	
	\begin{example}
		$\Q \subset \R$ does not satisfy the axiom of completeness.
	\end{example}
	
	\begin{proposition}
		Let $\es \neq A \subset \R$ bounded above, and $c \in \R$. Define $c + A := \{a + c: a \in A\}$. Then
		\begin{align}
			\sup (c + A) = c + \sup A
		\end{align}
	\end{proposition}
	
	\begin{proof}
		\emph{Step 1: Show $c + \sup A \in (c + A)^\uparrow$:} \\
		Let $x \in c + A$, $\exists\ a \in A\ s.t.\ x = c + a$. Then, $x = c + a \leq c + \sup A$. Therefore, $x \leq c + \sup A\ \forall x \in A$, which implies what desired. \\
		\emph{Step 2: Show $\forall u \in (c + A)^\uparrow,\ c + \sup A \leq u$: }\\
		Let $u \in (c + A)^\uparrow$, then $u \geq c + a\ \forall a \in A \implies u - c \geq a\ \forall a \in A \implies u - c \in A\uparrow \implies u - c \geq \sup A \implies u \geq c + \sup A$. \\
		Hence, $\sup (c + A) = c + \sup A$.
	\end{proof}
	
	\begin{lemma}[Alternative Definition of Supremum]
		Let $s \in A^\uparrow$ for some nonempty $A \subset \R$. The following statements are equivalent:
		\begin{enumerate}[(i)]
			\item $s = \sup A$;
			\item $\forall \varepsilon, \exists a \in A,\ s.t.\ a > s - \varepsilon$ (i.e. $s - \varepsilon \notin A^\uparrow$).
		\end{enumerate}
	\end{lemma}
	
	\begin{proof}
		Immediately.
	\end{proof}
	
	\begin{theorem}[Nested Interval Property]
		Let $(I_n)_n$ be a sequence of closed intervals $I_n := [a_n, b_n]$ such that these intervals are \emph{nested} in a sense that
		\begin{align}
			I_{n+1} \subset I_n\ \forall n \in \N
		\end{align}
		Then,
		\begin{align}
			\bigcap_{n \in \N} I_n \neq \es
		\end{align}
	\end{theorem}
	
	\begin{proof}
		Note that the sequence $(a_n)_{n \in \N}$ is bounded above by any $b_k$, by the completeness axiom, there exists $a^* := \sup_{n \in \N} a_n$. Since $a^* \in (a_n)^\uparrow$, $a^* \geq a_n\ \forall n \in \N$. Further, because $a^*$ is the \emph{least} upper bound, then for every upper bound $b_n$, it must be $a^* \leq b_n\ \forall n \in \N$. Therefore, $x^* \in [a_n, b_n]\ \forall n \in \N$. That is, $x^* \in \bigcap_{n \in \N} I_n$.
	\end{proof}
	
	Note that NIP requires all intervals to be closed. One instance when this fails to hold: $\bigcap_{n \in \N} \left(0, \frac{1}{n} \right) = \es$.
	
	\begin{theorem}[Archimedean Property] \quad
		\begin{enumerate}[(i)]
			\item $\forall x \in \R,\ \exists n \in \N\ s.t.\ n > x$;
			\item $\forall y \in \R_{++},\ \exists n \in \N\ s.t. \frac{1}{n} < y$.
		\end{enumerate}
	\end{theorem}
	
	Archimedean property \emph{of natural numbers} can be interpreted as \emph{there is no real number that bounds $\N$}. This interpretation can be seen by considering the negations of above statements:
	\begin{enumerate}[(i)]
		\item $\exists x \in \R\ s.t.\ \forall n \in \N,\ n \leq x$;
		\item $\exists y \in \R_{++}\ s.t.\ \forall n \in \N,\ y \leq \frac{1}{n}$.
	\end{enumerate}
	
	\begin{proof}[Proof of (i) by Contradiction.]
		Suppose the negated statement (i) is true, $\N$ is bounded above. By the completeness axiom, there exists $a^* := \sup \N$. $\exists n \in \N\ s.t.\ a^* - 1< n$. In this case, $a^* < n + 1 \in \N$, which means $a^* \notin \N^\uparrow$ and leads to a contradiction.
	\end{proof}
	
	\begin{proof}[Proof of (ii).]
		Let $y^* \in \R_{++}$, take $x = \frac{1}{y}$. By statement (i), there exists $n^* \in \N$ such that $n >	\frac{1}{y}$. Because $y > 0$, $\frac{1}{n} < y$.
	\end{proof}
	
	\subsection{Density of Rational Numbers}
	\begin{theorem}
		For every $a, b \in \R$ such that $a < b$, there exists $r \in \Q$ such that $a < r < b$.
	\end{theorem}
	
	The above theorem says $\Q$ is in fact \textbf{dense} in $\R$. More generally, one says a set $A \subset X$ is dense whenever the closure of $A$, $\overline{A} = X$.
	
	\begin{proof}
		\emph{Step 1:} Since $b - a > 0$, by the first Archimedean property, there exists $n \in \N$ such that $n > \frac{1}{b - a}$. Such natural number satisfies $\frac{1}{n} < b - a$.\\
		\emph{Step 2:} Let $m$ be smallest integer such that $m > an$. That is, $m - 1 \leq an < m$. Obviously, $a < \frac{m}{n}$ since $n > 0$. Further, since $m \leq an + 1$, with results from step (i), $m < bn - 1 + 1 = bn$, and $\frac{m}{n} < b$. Therefore $\frac{m}{n} \in (a, b)$.
	\end{proof}
	
	\begin{theorem}
		$\exists \alpha \in \R\ s.t.\ \alpha^2 = 2$.
	\end{theorem}
	
	\begin{proof}
		Let $\Omega := \{t \in \R: t^2 < 2\}$, which is obviously a set in $\R$ bounded from above. By the completeness axiom, $\Omega$ possesses a supremum, and we claim $\alpha := \sup \Omega$ satisfies $\alpha^2 = 2$. Suppose $\alpha^2 > 2$, then there exists $\varepsilon > 0$ such that $\alpha^2 - 2 \alpha \varepsilon + \varepsilon^2 > 2$. Therefore, $\alpha > \alpha - \varepsilon \in \Omega^\uparrow$, which contradicts the fact that $\alpha$ is the least upper bound. Suppose $\alpha^2 < 2$, then there exists some $\varepsilon > 0$ such that $\alpha + \varepsilon \in \Omega$, which contradicts the assumption that $\alpha$ is an upper bound. Hence, it must be the case that $\alpha^2 = 2$.
	\end{proof}
	
	\section{Sequences}
	\begin{theorem}[Triangle Inequality]
		Let $a, b \in \R$, then $|a + b| \leq |a| + |b|$.
	\end{theorem}
	
	\begin{corollary}
		Let $a, b \in \R$, then
		\begin{align}
			\large| |a| - |b| \large| \leq |a - b|
		\end{align}
	\end{corollary}
	
	\begin{proof}
		Note that $|a| = |a - b + b| \leq |a - b| + |b|$, which implies $|a| - |b| \leq |a - b|$. And $|b| = |b - a + a| \leq |b - a| + |a| = |a - b| + |a|$, which implies $|b| - |a| \leq |a - b|$. Therefore, by taking the absolute value, $||a| - |b|| \leq |a - b|$.
	\end{proof}
	
	\begin{definition}
		A sequence $(a_n) \subset \R$ \textbf{converges} to $a \in \R$ if 
		\begin{align}
			\forall \varepsilon > 0,\ \exists N \in \N,\ n \geq N \implies |a_n - a| < \varepsilon
		\end{align}
	\end{definition}
	
	Let $a \in \R$ and $\varepsilon > 0$, the open ball centred at $a$ with radius $\varepsilon$ is denoted as 
	\begin{align}
		V_\varepsilon(a) := \left\{x \in \R : |x - a| < \varepsilon \right\}
	\end{align}
	
	\begin{theorem}
		The limit of any convergent sequence is unique.
	\end{theorem}
	
	\begin{proof}
		Let $(a_n)$ be a convergent sequence, assume, for contradiction, that $(a_n) \to L_1$ and $(a_n) \to L_2$ such that $L_1 \neq L_2$. Let $\varepsilon = \frac{|L_1 - L_2|}{3}$, because $(a_n) \to L_1$, there exists $N\in\N$ such that $n \geq N \implies |a_n - L_1| < \frac{|L_1 - L_2|}{3}$. Therefore, for every $n \geq N$,
		\begin{align}
			|a_n - L_2| &= |a_n - L_1 - (L_2 - L_1)| \\ 
			&\geq ||a_n - L_1| - |L_2 - L_1|| \\
			&= ||L_1 - L_2| - |a_n - L_1|| \\
			&= 3\varepsilon - |a_n - L_1| \\
			&> 2 \varepsilon
		\end{align}
		Therefore, there does not exist any $N' \in \N$ such that $|a_n - L_2| < \varepsilon$ for every $n \geq \N$.
	\end{proof}
	
	\begin{definition}
		A sequence $(a_n)$ is \textbf{divergent} if it does not converge.
	\end{definition}
	
	\begin{example}
		The sequence $(a_n) := (1, -1/2, 1/3, 1/4, -1/5, 1/5, -1/5, 1/5, \cdots)$ is divergent.
	\end{example}
	
	\begin{proof}
		Let $\varepsilon := \frac{2}{5 \times 3}$, assume, for contradiction, that $(a_n) \to L$ for some $L \in \R$. Then there exists $N \in \N$ such that for every $n \geq N$, $|a_n - L| < \frac{2}{15}$. Since the sequence is alternating, it must be the case that $\left|L - \frac{1}{5} \right| < \frac{2}{15}$. Similarly, 
		\begin{align}
			\left| - \frac{1}{5} - L\right| &= \left|\frac{1}{5} + L \right|\\
			&= \left|\frac{1}{5} + L - \frac{1}{5} + \frac{1}{5} \right| \\
			&= \left|(L - \frac{1}{5}) - (- \frac{2}{5})\right| \\
			&\geq \left | \left | L - \frac{1}{5} \right | - \frac{6}{15} \right|\\
			&= \frac{6}{15} - \left|L - \frac{1}{5}\right| \\
			&> \frac{4}{15} \\
			&> \varepsilon
		\end{align}
		the strict inequality suggests there cannot be a $M \in \N$ such that $|a_n - L| < \varepsilon$ for every $n \geq M$.
	\end{proof}
	
	\begin{proof}[Alternative Proof.]
		If $(a_n)$ is convergent, then all of its subsequences must converge to the same limit. Obviously, there are subsequences of $(a_n)$ converging to $\frac{1}{5}$ and $-\frac{1}{5}$ respectively, this leads to a contradiction.
	\end{proof}
	
	\begin{definition}
		A sequence is \textbf{bounded} if $\exists M \in \R$ such that $\forall n \in \N,\ |a_n| < M$.
	\end{definition}
	
	\begin{theorem}
		Every convergent sequence is bounded.
	\end{theorem}
	
	\begin{proof}
		Let $(a_n) \to L$, take $\varepsilon = 1$, then there exists $N \in \N$ such that $|a_n - L| < 1$ for every $n > N$. Note that $|a_n| - |L| \leq ||a_n| - |L|| \leq |a_n - L| < \varepsilon$, which implies $|a_n| < |L| + 1$. Let $Q := \max_{n < N} a_n$, take $M := \max\{Q, |L| + 1\}$, then $M$ bounds $(a_n)$.
	\end{proof}
	
	\begin{theorem}[Algebraic Limit Theorem]
		Let $(a_n) \to a, (b_n) \to b$ be convergent sequences, and $c \in \R$, then
		\begin{enumerate}[(i)]
			\item $(c a_n) \to c a$;
			\item $(a_n + b_n) \to a + b$;
			\item $(a_n b_n) \to ab$;
			\item $\left(\frac{a_n}{b_n}\right) \to \frac{a}{b}$, provided $b_n, b \neq 0$.
		\end{enumerate}
	\end{theorem}
	\begin{proof}[Proof (i).]
		Let $\varepsilon > 0$, there exists $N \in \N$ such that $\forall n \geq N$, $|a_n - a| < \frac{\varepsilon}{|c|}$. Then, for every $n \geq N$, $|c a_n - ca| = |c| |a_n - a| < \varepsilon$.
	\end{proof}
	
	\begin{proof}[Proof (ii).]
		Let $\varepsilon > 0$, there exists $N_1, N_2 \in \N$ such that $|a_n - a| < \frac{\varepsilon}{3}\ \forall n \geq N_1$ and $|b_n - b| < \frac{\varepsilon}{3}\ \forall n \geq N_2$. Take $N := \max\{N_1, N_2\}$, let $n \geq N$,
		\begin{align}
			|a_n + b_n - a - b| &\leq |a_n - a| + |b_n - b| < \frac{2\varepsilon}{3} < \varepsilon
		\end{align}
	\end{proof}
	
	\begin{proof}[Proof (iii).]
		Note that
		\begin{align}
			|a_nb_n - ab| &= |a_n b_n + a_nb - a_nb - ab| \\
			&\leq |a_n b_n - a_n b| + |a_n b - ab| \\
			&\leq |a_n| |b_n - b| + |b| |a_n - a|
		\end{align}
		Let $N_1 \in \N$ such that $|a_n - a| < \frac{\varepsilon}{2 |b|}$ for every $n \geq N_1$. Because $(a_n)$ is convergent, let $M$ denote its bound such that $|a_n| < M\ \forall n \in \N$. Let $N_2 \in \N$ such that $|b_n - b| < \frac{\varepsilon}{2 M}$. Then for every $n \geq N_3 := \max\{N_1, N_2\}$, $|a_nb_n - ab| < \varepsilon$.
	\end{proof}
	\begin{proof}[Proof (iv).]
		\textbf{Claim i:} when $n$ is sufficiently larger, $|b_n| > 0$ is bounded away from zero by $M$. Let $\varepsilon = \frac{|b|}{10}$, then there exists $N_1 \in \N$ such that for every $n \geq N_1$, $|b_n - b| < \frac{|b|}{10}$. Note that for every such $n$,
		\begin{align}
			|b_n| &= |b_n - b - (-b)| \\
			&\geq | |b_n - b| - |b| | \\
			&\geq |b| - |b_n - b| \\
			&> \frac{9|b|}{10}
		\end{align}
		\textbf{Claim ii:} $\left(\frac{1}{b_n}\right) \to \frac{1}{b}$. Let $\varepsilon > 0$, note that
		\begin{align}
			\left | \frac{1}{b_n} - \frac{1}{b} \right | &= \left | \frac{b}{b_nb} - \frac{b_n}{b_nb} \right | \\
			&= \frac{1}{|b_n||b|} |b_n - b|
		\end{align}
		from the first claim, $\frac{1}{|b_n|} < \frac{10}{9 |b|}$ for every $n \geq N_1$. Since $(b_n) \to b$, there exists $N_2 \in \N$ such that for every $n \geq N_2$, $|b_n - b| < \frac{10 \varepsilon}{9|b|^2}$. Consequently, for every $n \geq N_3 := \max\{N_1, N_2\}$, $\left | \frac{1}{b_n} - \frac{1}{b} \right | < \varepsilon$. Then the result is immediate from property (iii) in the algebraic limit theorem. 
	\end{proof}
	
	\begin{theorem}[Order Limit Theorem]
		Let $(a_n) \to a$ and $(b_n) \to b$, then
		\begin{enumerate}[(i)]
			\item $a_n \geq 0\ \forall n \in \N \implies a \geq 0$;
			\item $a_n \leq b_n\ \forall n \in \N \implies a \leq b$;
			\item $\exists c \in \R\ s.t.\ c \leq b_n\ \forall n \in \N \implies c \leq b$;
			\item $\exists c \in \R\ s.t.\ a_n \leq c\ \forall n \in \N \implies a \leq c$.
		\end{enumerate}
	\end{theorem}
	
	\begin{proof}
		(i) Assume, for contradiction, $a < 0$. Take $\varepsilon = \frac{|a|}{2}$, then for some $N \in \N$, for every $n \geq N$ $a_n \in V_\varepsilon(a)$. However, this contradicts the fact that $a_n \geq 0$. \\
		(ii) Consider sequence $(b_n - a_n)$ in which $b_n - a_n \geq 0$ for every $n \in \N$. $(b_n - a_n) \to (b - a)$ by the algebraic limit theorem. By property (i), $b - a \geq 0$. \\
		(iii) and (iv) Consider constant sequence defined as $(c_n)$ such that $c_n = c$ for every $n \in \N$, the results are immediate by applying (ii).
	\end{proof}
	
	\begin{theorem}[Squeeze Theorem]
		Let $(x_n) \to L$ and $(z_n) \to \ell$. If for every $n \in \N$, $x_n \leq y_n \leq z_n$, then $(y_n) \to \ell$.
	\end{theorem}
	
	\begin{proof}
%		\textbf{Claim i:} $(y_n)$ converges to some limit $y$. Let $\varepsilon > 0$, then
%		\begin{align}
%			|y_n - \ell| &= |y_n + \ell - x_n - z_n + x_n - \ell + z_n - \ell| \\
%			&\leq |y_n - x_n| + |z_n - \ell| + |x_n - \ell| + |z_n - \ell|
%		\end{align}
%		
%		Note that $0 \leq y_n - x_n \leq z_n - x_n$ for every $n \in \N$. \\
%		\textbf{Claim ii:} $(y_n)$ converges to $\ell$. By the order limit theorem, $a_n \leq y_n \implies \ell \leq y$, and $y_n \leq z_n \implies \ell \geq y$. Therefore, $y = \ell$.
	Suppose, for contradiction, $(y_n) \centernot \to \ell$, then there exists $\varepsilon > 0$ such that for every $N \in \N$, there exists  a $n \geq N$ satisfying $y_n \notin V_\varepsilon (\ell)$. Take the same $\varepsilon > 0$, there exists $N_1 \in \N$ such that for every $n \geq N_1$, $x_n, z_n \in V_\varepsilon(\ell)$. Note that every $y_n \in [x_n, z_n]$ can be written as a convex combination of $x_n, z_n$, and since $V_\varepsilon(\ell)$ is convex, $y_n \in V_\varepsilon(\ell)$. Taking $N := N_1$, this clearly contradicts our previous conclusion.
	\end{proof}
	
\end{document}























d