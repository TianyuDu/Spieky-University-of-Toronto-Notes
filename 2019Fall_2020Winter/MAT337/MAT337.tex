\documentclass[11pt]{article}
\usepackage{spikey}
\usepackage{amsmath}
\usepackage{amssymb}
\usepackage{soul}
\usepackage{float}
\usepackage{graphicx}
\usepackage{hyperref}
\usepackage{xcolor}
\usepackage{chngcntr}
\usepackage{centernot}
\usepackage{datetime}
\usepackage[shortlabels]{enumitem}

\usepackage[margin=1truein]{geometry}
\usepackage{setspace}
\linespread{1.15}

\title{Introduction to Real Analysis}
\date{\today}
\author{Tianyu Du}

\usepackage[
    type={CC},
    modifier={by-nc},
    version={4.0},
]{doclicense}

\counterwithin{equation}{section}
\counterwithin{theorem}{section}
\counterwithin{lemma}{section}
\counterwithin{corollary}{section}
\counterwithin{proposition}{section}
\counterwithin{remark}{section}
\counterwithin{example}{section}
\counterwithin{definition}{section}

\begin{document}
	\maketitle
	\doclicenseThis
	\tableofcontents
	\newpage
	\section{The Axiom of Completeness}
	\subsection{Preliminaries}
	\begin{definition}
		A set $A \subset \R$ is \textbf{bounded above} if
		\begin{align}
			\exists u \in \R\ s.t.\ \forall a \in A,\ u \geq a
		\end{align} 
		It is said to be \textbf{bounded below} if
		\begin{align}
			\exists l \in \R\ s.t.\ \forall a \in A,\ l \leq a
		\end{align}
	\end{definition}
	
	\begin{example}
		The set of integers, $\Z$, is neither bounded from above nor below. Sets $\{1, 2, 3\}$ and $\{\frac{1}{n}: n \in \N \}$ are bounded from both above and below.
	\end{example}
	
	\begin{notation}
		Let $A \subset \R$, we use $A^\uparrow$ and $A^\downarrow$ to denote collections of upper bounds of $A$ and lower bounds of $A$. Note that $A^\uparrow$ and $A^\downarrow$ are potentially empty.
	\end{notation}
	
	\begin{definition}
		A real number $s \in \R$ is the \textbf{least upper bound(supremum)} for a set $A \subset \R$ if $s \in A^\uparrow$ and $\forall u \in A^\uparrow,\ s \leq u$. Such $s$ is denoted as $s := \sup A$.
	\end{definition}
	
	\begin{definition}
		A real number $f \in \R$ is the \textbf{greatest lower bound (infimum)} for $A$ if $f \in A^\downarrow$ and $\forall l \in A^\downarrow,\ l \leq f$. Such $f$ is often written as $f := \inf A$.
	\end{definition}
	
	\begin{axiom}[The Axiom of Completeness/Least Upper Bounded Property]
		$\forall \es \neq A \subset \R$ such that $A^\uparrow \neq \es$, $\exists \sup A$.
	\end{axiom}
	
	\begin{definition}
		Let $\es \neq A \subset \R$, $a_0 \in A$ is the \textbf{maximum} of $A$ if $\forall a \in A, a_0 \geq a $; $a_1 \in A$ is the \textbf{minimum} of $A$ if 
		$\forall a \in A, a_1 \leq a$.
	\end{definition}
	
	\begin{example}
		$\Q \subset \R$ does not satisfy the axiom of completeness.
	\end{example}
	
	\begin{proposition}
		Let $\es \neq A \subset \R$ bounded above, and $c \in \R$. Define $c + A := \{a + c: a \in A\}$. Then
		\begin{align}
			\sup (c + A) = c + \sup A
		\end{align}
	\end{proposition}
	
	\begin{proof}
		\emph{Step 1: Show $c + \sup A \in (c + A)^\uparrow$:} \\
		Let $x \in c + A$, $\exists\ a \in A\ s.t.\ x = c + a$. Then, $x = c + a \leq c + \sup A$. Therefore, $x \leq c + \sup A\ \forall x \in A$, which implies what desired. \\
		\emph{Step 2: Show $\forall u \in (c + A)^\uparrow,\ c + \sup A \leq u$: }\\
		Let $u \in (c + A)^\uparrow$, then $u \geq c + a\ \forall a \in A \implies u - c \geq a\ \forall a \in A \implies u - c \in A\uparrow \implies u - c \geq \sup A \implies u \geq c + \sup A$. \\
		Hence, $\sup (c + A) = c + \sup A$.
	\end{proof}
	
	\begin{lemma}[Alternative Definition of Supremum]
		Let $s \in A^\uparrow$ for some nonempty $A \subset \R$. The following statements are equivalent:
		\begin{enumerate}[(i)]
			\item $s = \sup A$;
			\item $\forall \varepsilon, \exists a \in A,\ s.t.\ a > s - \varepsilon$ (i.e. $s - \varepsilon \notin A^\uparrow$).
		\end{enumerate}
	\end{lemma}
	
	\begin{proof}
		Immediately.
	\end{proof}
	
	\begin{theorem}[Nested Interval Property]
		Let $(I_n)_n$ be a sequence of closed intervals $I_n := [a_n, b_n]$ such that these intervals are \emph{nested} in a sense that
		\begin{align}
			I_{n+1} \subset I_n\ \forall n \in \N
		\end{align}
		Then,
		\begin{align}
			\bigcap_{n \in \N} I_n \neq \es
		\end{align}
	\end{theorem}
	
	\begin{proof}
		Note that the sequence $(a_n)_{n \in \N}$ is bounded above by any $b_k$, by the completeness axiom, there exists $a^* := \sup_{n \in \N} a_n$. Since $a^* \in (a_n)^\uparrow$, $a^* \geq a_n\ \forall n \in \N$. Further, because $a^*$ is the \emph{least} upper bound, then for every upper bound $b_n$, it must be $a^* \leq b_n\ \forall n \in \N$. Therefore, $x^* \in [a_n, b_n]\ \forall n \in \N$. That is, $x^* \in \bigcap_{n \in \N} I_n$.
	\end{proof}
	
	Note that NIP requires all intervals to be closed. One instance when this fails to hold: $\bigcap_{n \in \N} \left(0, \frac{1}{n} \right) = \es$.
	
	\begin{theorem}[Archimedean Property] \quad
		\begin{enumerate}[(i)]
			\item $\forall x \in \R,\ \exists n \in \N\ s.t.\ n > x$;
			\item $\forall y \in \R_{++},\ \exists n \in \N\ s.t. \frac{1}{n} < y$.
		\end{enumerate}
	\end{theorem}
	
	Archimedean property \emph{of natural numbers} can be interpreted as \emph{there is no real number that bounds $\N$}. This interpretation can be seen by considering the negations of above statements:
	\begin{enumerate}[(i)]
		\item $\exists x \in \R\ s.t.\ \forall n \in \N,\ n \leq x$;
		\item $\exists y \in \R_{++}\ s.t.\ \forall n \in \N,\ y \leq \frac{1}{n}$.
	\end{enumerate}
	
	\begin{proof}[Proof of (i) by Contradiction.]
		Suppose the negated statement (i) is true, $\N$ is bounded above. By the completeness axiom, there exists $a^* := \sup \N$. $\exists n \in \N\ s.t.\ a^* - 1< n$. In this case, $a^* < n + 1 \in \N$, which means $a^* \notin \N^\uparrow$ and leads to a contradiction.
	\end{proof}
	
	\begin{proof}[Proof of (ii).]
		Let $y^* \in \R_{++}$, take $x = \frac{1}{y}$. By statement (i), there exists $n^* \in \N$ such that $n >	\frac{1}{y}$. Because $y > 0$, $\frac{1}{n} < y$.
	\end{proof}
	
	\subsection{Density of Rational Numbers}
	\begin{theorem}
		For every $a, b \in \R$ such that $a < b$, there exists $r \in \Q$ such that $a < r < b$.
	\end{theorem}
	
	The above theorem says $\Q$ is in fact \textbf{dense} in $\R$. More generally, one says a set $A \subset X$ is dense whenever the closure of $A$, $\overline{A} = X$.
	
	\begin{proof}
		\emph{Step 1:} Since $b - a > 0$, by the first Archimedean property, there exists $n \in \N$ such that $n > \frac{1}{b - a}$. Such natural number satisfies $\frac{1}{n} < b - a$.\\
		\emph{Step 2:} Let $m$ be smallest integer such that $m > an$. That is, $m - 1 \leq an < m$. Obviously, $a < \frac{m}{n}$ since $n > 0$. Further, since $m \leq an + 1$, with results from step (i), $m < bn - 1 + 1 = bn$, and $\frac{m}{n} < b$. Therefore $\frac{m}{n} \in (a, b)$.
	\end{proof}
	
	\begin{theorem}
		$\exists \alpha \in \R\ s.t.\ \alpha^2 = 2$.
	\end{theorem}
	
	\begin{proof}
		Let $\Omega := \{t \in \R: t^2 < 2\}$, which is obviously a set in $\R$ bounded from above. By the completeness axiom, $\Omega$ possesses a supremum, and we claim $\alpha := \sup \Omega$ satisfies $\alpha^2 = 2$. Suppose $\alpha^2 > 2$, then there exists $\varepsilon > 0$ such that $\alpha^2 - 2 \alpha \varepsilon + \varepsilon^2 > 2$. Therefore, $\alpha > \alpha - \varepsilon \in \Omega^\uparrow$, which contradicts the fact that $\alpha$ is the least upper bound. Suppose $\alpha^2 < 2$, then there exists some $\varepsilon > 0$ such that $\alpha + \varepsilon \in \Omega$, which contradicts the assumption that $\alpha$ is an upper bound. Hence, it must be the case that $\alpha^2 = 2$.
	\end{proof}
	
	\section{Sequences}
	\begin{theorem}[Triangle Inequality]
		Let $a, b \in \R$, then $|a + b| \leq |a| + |b|$.
	\end{theorem}
	
	\begin{corollary}
		Let $a, b \in \R$, then
		\begin{align}
			\large| |a| - |b| \large| \leq |a - b|
		\end{align}
	\end{corollary}
	
	\begin{proof}
		Note that $|a| = |a - b + b| \leq |a - b| + |b|$, which implies $|a| - |b| \leq |a - b|$. And $|b| = |b - a + a| \leq |b - a| + |a| = |a - b| + |a|$, which implies $|b| - |a| \leq |a - b|$. Therefore, by taking the absolute value, $||a| - |b|| \leq |a - b|$.
	\end{proof}
	
	\begin{definition}
		A sequence $(a_n) \subset \R$ \textbf{converges} to $a \in \R$ if 
		\begin{align}
			\forall \varepsilon > 0,\ \exists N \in \N,\ n \geq N \implies |a_n - a| < \varepsilon
		\end{align}
	\end{definition}
	
	Let $a \in \R$ and $\varepsilon > 0$, the open ball centred at $a$ with radius $\varepsilon$ is denoted as 
	\begin{align}
		V_\varepsilon(a) := \left\{x \in \R : |x - a| < \varepsilon \right\}
	\end{align}
	
	\begin{theorem}
		The limit of any convergent sequence is unique.
	\end{theorem}
	
	\begin{proof}
		Let $(a_n)$ be a convergent sequence, assume, for contradiction, that $(a_n) \to L_1$ and $(a_n) \to L_2$ such that $L_1 \neq L_2$. Let $\varepsilon = \frac{|L_1 - L_2|}{3}$, because $(a_n) \to L_1$, there exists $N\in\N$ such that $n \geq N \implies |a_n - L_1| < \frac{|L_1 - L_2|}{3}$. Therefore, for every $n \geq N$,
		\begin{align}
			|a_n - L_2| &= |a_n - L_1 - (L_2 - L_1)| \\ 
			&\geq ||a_n - L_1| - |L_2 - L_1|| \\
			&= ||L_1 - L_2| - |a_n - L_1|| \\
			&= 3\varepsilon - |a_n - L_1| \\
			&> 2 \varepsilon
		\end{align}
		Therefore, there does not exist any $N' \in \N$ such that $|a_n - L_2| < \varepsilon$ for every $n \geq \N$.
	\end{proof}
	
	\begin{definition}
		A sequence $(a_n)$ is \textbf{divergent} if it does not converge.
	\end{definition}
	
	\begin{example}
		The sequence $(a_n) := (1, -1/2, 1/3, 1/4, -1/5, 1/5, -1/5, 1/5, \cdots)$ is divergent.
	\end{example}
	
	\begin{proof}
		Let $\varepsilon := \frac{2}{5 \times 3}$, assume, for contradiction, that $(a_n) \to L$ for some $L \in \R$. Then there exists $N \in \N$ such that for every $n \geq N$, $|a_n - L| < \frac{2}{15}$. Since the sequence is alternating, it must be the case that $\left|L - \frac{1}{5} \right| < \frac{2}{15}$. Similarly, 
		\begin{align}
			\left| - \frac{1}{5} - L\right| &= \left|\frac{1}{5} + L \right|\\
			&= \left|\frac{1}{5} + L - \frac{1}{5} + \frac{1}{5} \right| \\
			&= \left|(L - \frac{1}{5}) - (- \frac{2}{5})\right| \\
			&\geq \left | \left | L - \frac{1}{5} \right | - \frac{6}{15} \right|\\
			&= \frac{6}{15} - \left|L - \frac{1}{5}\right| \\
			&> \frac{4}{15} \\
			&> \varepsilon
		\end{align}
		the strict inequality suggests there cannot be a $M \in \N$ such that $|a_n - L| < \varepsilon$ for every $n \geq M$.
	\end{proof}
	
	\begin{proof}[Alternative Proof.]
		If $(a_n)$ is convergent, then all of its subsequences must converge to the same limit. Obviously, there are subsequences of $(a_n)$ converging to $\frac{1}{5}$ and $-\frac{1}{5}$ respectively, this leads to a contradiction.
	\end{proof}
	
	\begin{definition}
		A sequence is \textbf{bounded} if $\exists M \in \R$ such that $\forall n \in \N,\ |a_n| < M$.
	\end{definition}
	
	\begin{theorem}
		Every convergent sequence is bounded.
	\end{theorem}
	
	\begin{proof}
		Let $(a_n) \to L$, take $\varepsilon = 1$, then there exists $N \in \N$ such that $|a_n - L| < 1$ for every $n > N$. Note that $|a_n| - |L| \leq ||a_n| - |L|| \leq |a_n - L| < \varepsilon$, which implies $|a_n| < |L| + 1$. Let $Q := \max_{n < N} a_n$, take $M := \max\{Q, |L| + 1\}$, then $M$ bounds $(a_n)$.
	\end{proof}
	
	\begin{theorem}[Algebraic Limit Theorem]
		Let $(a_n) \to a, (b_n) \to b$ be convergent sequences, and $c \in \R$, then
		\begin{enumerate}[(i)]
			\item $(c a_n) \to c a$;
			\item $(a_n + b_n) \to a + b$;
			\item $(a_n b_n) \to ab$;
			\item $\left(\frac{a_n}{b_n}\right) \to \frac{a}{b}$, provided $b_n, b \neq 0$.
		\end{enumerate}
	\end{theorem}
	\begin{proof}[Proof (i).]
		Let $\varepsilon > 0$, there exists $N \in \N$ such that $\forall n \geq N$, $|a_n - a| < \frac{\varepsilon}{|c|}$. Then, for every $n \geq N$, $|c a_n - ca| = |c| |a_n - a| < \varepsilon$.
	\end{proof}
	
	\begin{proof}[Proof (ii).]
		Let $\varepsilon > 0$, there exists $N_1, N_2 \in \N$ such that $|a_n - a| < \frac{\varepsilon}{3}\ \forall n \geq N_1$ and $|b_n - b| < \frac{\varepsilon}{3}\ \forall n \geq N_2$. Take $N := \max\{N_1, N_2\}$, let $n \geq N$,
		\begin{align}
			|a_n + b_n - a - b| &\leq |a_n - a| + |b_n - b| < \frac{2\varepsilon}{3} < \varepsilon
		\end{align}
	\end{proof}
	
	\begin{proof}[Proof (iii).]
		Note that
		\begin{align}
			|a_nb_n - ab| &= |a_n b_n + a_nb - a_nb - ab| \\
			&\leq |a_n b_n - a_n b| + |a_n b - ab| \\
			&\leq |a_n| |b_n - b| + |b| |a_n - a|
		\end{align}
		Let $N_1 \in \N$ such that $|a_n - a| < \frac{\varepsilon}{2 |b|}$ for every $n \geq N_1$. Because $(a_n)$ is convergent, let $M$ denote its bound such that $|a_n| < M\ \forall n \in \N$. Let $N_2 \in \N$ such that $|b_n - b| < \frac{\varepsilon}{2 M}$. Then for every $n \geq N_3 := \max\{N_1, N_2\}$, $|a_nb_n - ab| < \varepsilon$.
	\end{proof}
	\begin{proof}[Proof (iv).]
		\textbf{Claim i:} when $n$ is sufficiently larger, $|b_n| > 0$ is bounded away from zero by $M$. Let $\varepsilon = \frac{|b|}{10}$, then there exists $N_1 \in \N$ such that for every $n \geq N_1$, $|b_n - b| < \frac{|b|}{10}$. Note that for every such $n$,
		\begin{align}
			|b_n| &= |b_n - b - (-b)| \\
			&\geq | |b_n - b| - |b| | \\
			&\geq |b| - |b_n - b| \\
			&> \frac{9|b|}{10}
		\end{align}
		\textbf{Claim ii:} $\left(\frac{1}{b_n}\right) \to \frac{1}{b}$. Let $\varepsilon > 0$, note that
		\begin{align}
			\left | \frac{1}{b_n} - \frac{1}{b} \right | &= \left | \frac{b}{b_nb} - \frac{b_n}{b_nb} \right | \\
			&= \frac{1}{|b_n||b|} |b_n - b|
		\end{align}
		from the first claim, $\frac{1}{|b_n|} < \frac{10}{9 |b|}$ for every $n \geq N_1$. Since $(b_n) \to b$, there exists $N_2 \in \N$ such that for every $n \geq N_2$, $|b_n - b| < \frac{10 \varepsilon}{9|b|^2}$. Consequently, for every $n \geq N_3 := \max\{N_1, N_2\}$, $\left | \frac{1}{b_n} - \frac{1}{b} \right | < \varepsilon$. Then the result is immediate from property (iii) in the algebraic limit theorem. 
	\end{proof}
	
	\begin{theorem}[Order Limit Theorem]
		Let $(a_n) \to a$ and $(b_n) \to b$, then
		\begin{enumerate}[(i)]
			\item $a_n \geq 0\ \forall n \in \N \implies a \geq 0$;
			\item $a_n \leq b_n\ \forall n \in \N \implies a \leq b$;
			\item $\exists c \in \R\ s.t.\ c \leq b_n\ \forall n \in \N \implies c \leq b$;
			\item $\exists c \in \R\ s.t.\ a_n \leq c\ \forall n \in \N \implies a \leq c$.
		\end{enumerate}
	\end{theorem}
	
	\begin{proof}
		(i) Assume, for contradiction, $a < 0$. Take $\varepsilon = \frac{|a|}{2}$, then for some $N \in \N$, for every $n \geq N$ $a_n \in V_\varepsilon(a)$. However, this contradicts the fact that $a_n \geq 0$. \\
		(ii) Consider sequence $(b_n - a_n)$ in which $b_n - a_n \geq 0$ for every $n \in \N$. $(b_n - a_n) \to (b - a)$ by the algebraic limit theorem. By property (i), $b - a \geq 0$. \\
		(iii) and (iv) Consider constant sequence defined as $(c_n)$ such that $c_n = c$ for every $n \in \N$, the results are immediate by applying (ii).
	\end{proof}
	
	\begin{theorem}[Squeeze Theorem]
		Let $(x_n) \to L$ and $(z_n) \to \ell$. If for every $n \in \N$, $x_n \leq y_n \leq z_n$, then $(y_n) \to \ell$.
	\end{theorem}
	
	\begin{proof}
%		\textbf{Claim i:} $(y_n)$ converges to some limit $y$. Let $\varepsilon > 0$, then
%		\begin{align}
%			|y_n - \ell| &= |y_n + \ell - x_n - z_n + x_n - \ell + z_n - \ell| \\
%			&\leq |y_n - x_n| + |z_n - \ell| + |x_n - \ell| + |z_n - \ell|
%		\end{align}
%		
%		Note that $0 \leq y_n - x_n \leq z_n - x_n$ for every $n \in \N$. \\
%		\textbf{Claim ii:} $(y_n)$ converges to $\ell$. By the order limit theorem, $a_n \leq y_n \implies \ell \leq y$, and $y_n \leq z_n \implies \ell \geq y$. Therefore, $y = \ell$.
%	Suppose, for contradiction, $(y_n) \centernot \to \ell$, then there exists $\varepsilon > 0$ such that for every $N \in \N$, there exists  a $n \geq N$ satisfying $y_n \notin V_\varepsilon (\ell)$. Take the same $\varepsilon > 0$, there exists $N_1 \in \N$ such that for every $n \geq N_1$, $x_n, z_n \in V_\varepsilon(\ell)$. Note that every $y_n \in [x_n, z_n]$ can be written as a convex combination of $x_n, z_n$, and since $V_\varepsilon(\ell)$ is convex, $y_n \in V_\varepsilon(\ell)$. Taking $N := N_1$, this clearly contradicts our previous conclusion.
%	\end{proof}
	Let $\varepsilon > 0$, because both $(x_n) \to \ell$ and $(y_n) \to \ell$,
	\begin{align}
		\exists N_1\ s.t.\ n \geq N_1 \implies |x_n - \ell| < \varepsilon \implies x_n > \ell - \varepsilon \\
		\exists N_2\ s.t.\ n \geq N_2 \implies |z_n - \ell| < \varepsilon \implies z_n < \ell +  \varepsilon
	\end{align}
	Take $N_3 := \max\{N_1, N_2\}$, then for every $n \geq N_3$,
	\begin{align}
		\ell - \varepsilon < x_n \leq &y_n \leq z_n < \ell + \varepsilon \\
		\implies y_n &\in V_\varepsilon(\ell)
	\end{align}
	therefore $(y_n) \to \ell$ by definition.
	\end{proof}
	
	\subsection{Monotone Convergence Theorem}
	\begin{definition}
		A sequence $(a_n)$ is said to be \textbf{monotone} if it is either increasing ($a_{n+1} \geq a_n\ \forall n \in \N$) or decreasing ($a_{n+1} \leq a_n\ \forall n \in \N$).
	\end{definition}
	
	\begin{theorem}[Monotone Convergence Theorem]
		If a sequence $(a_n)$ is bounded, then it converges.
	\end{theorem}
	
	\begin{proof}
		WLOG, assume $(a_n)$ is increasing, let $\Gamma := \{a_n: n \in \N\} \subset \R$, because $\Gamma$ is bounded, $s := \sup_n \Gamma$ is well-defined by the completeness of real numbers.\\
		\textbf{Claim:} $(a_n) \to s$. Let $\varepsilon > 0$, by the definition of supremum, $\exists N \in \N$ such that $a_N > s - \varepsilon$. Because the sequence is increasing and $s + \varepsilon \in \Gamma^\uparrow$, $n \geq N \implies s - \varepsilon < a_n  < s + \varepsilon$. $(a_n) \to s$ by definition.
	\end{proof}
	
	\subsection{Series}
	\begin{definition}
		Let $(a_i)$ be a sequence, then the $n$-th \textbf{partial sum} is defined as $s_n := \sum_{i=1}^n a_i$. And the \textbf{infinite sum/series} of $(a_n)$ is defined as 
		\begin{align}
			\sum_{i=1}^\infty a_i
			= \begin{cases}
				s &\tx{ if } (s_n) \to s \\
				\tx{undefined/diverges} &\tx{ otherwise}
			\end{cases}
		\end{align}
	\end{definition}
	
	\begin{example}
		$\sum_{i=1}^\infty \frac{1}{i^2}$ converges.
	\end{example}
	
	\begin{proof}
		Obviously the corresponding partial sums are increasing because the sequence $(\frac{1}{i^2})$ is positive. \\
		\textbf{Claim:} $(s_n)$ is bounded from above. Let $n \in \N$, observe
		\begin{align}
			\sum_{i=1}^n \frac{1}{i^2} &= 1 + \frac{1}{2 \times 2} + \frac{1}{3 \times 3} + \cdots + \frac{1}{n \times n} \\
			&\leq 1 + \frac{1}{1 \times 2} + \frac{1}{2 \times 3} + \cdots + \frac{1}{(n-1) \times n} \\
			&= 2 - \frac{1}{n} \leq 2
		\end{align}
		The result is immediate by the monotone convergence theorem.
	\end{proof}
	
	\begin{example}[Harmonic Series]
		$\sum_{n=1}^\infty \frac{1}{n}$ diverges.
	\end{example}
	
	\begin{proof}
		\textbf{Claim:} there exists a subsequence of $(s_n)$ diverges, so $(s_n)$ cannot be convergent. Consider the subsequence $(s_k)$ constructed by defining $s_k := s_{2^k}$. Note that
		\begin{align}
			s_{2^k} &= 1 + \frac{1}{2} + 
			\left(
			\frac{1}{3} + \frac{1}{4}
			\right) +
			\left(
			\frac{1}{5} + \frac{1}{6} + \frac{1}{7} + \frac{1}{8}
			\right) +
			\cdots +
			\left(
			\frac{1}{2^{k-1} + 1} + \cdots + \frac{1}{2^k}
			\right) \\
			&> 1 + \frac{1}{2} k
		\end{align}
		Clearly, the subsequence is unbounded, and therefore cannot be convergent.
	\end{proof}
	
	\begin{definition}
		Let $(a_n)$ be a sequence, then for every \ul{strictly} increasing sequence $(n_i)_i$ in $\N$, $(a_{n_i})$ is a \textbf{subsequence} of $(a_n)$.
	\end{definition}
	
	\begin{theorem}
		All subsequences of a convergent sequence converge to the same limit as the original sequence.
	\end{theorem}
	
	\begin{proof}
		Let $(a_n) \to \ell$, let $(a_{n_k})$ be a subsequence of $(a_n)$. Let $\varepsilon > 0$, there exists $N \in \N$ such that $n \geq N \implies a_n \in V_\varepsilon(\ell)$. By the definition of subsequences, there exists some $K \in \N$ such that $n_K = N$. Take such $K$, then for every $k \geq K$, it must be $n_k \geq N$. Therefore $a_{n_k} \in V_\varepsilon(\ell)$ for every $k \geq K$, and $(a_{n_k}) \to \ell$ by definition.
	\end{proof}
	
	\begin{corollary}
		A sequence $(a_n)$ must be divergent if there exists two subsequences of it converge to two different limits.
	\end{corollary}
	
	\begin{proof}
		Immediate by taking the contrapositive form of above theorem.
	\end{proof}
	
	\begin{theorem}[Bolzano–Weierstrass]
		Every bounded sequence contains a convergent subsequence.
	\end{theorem}
	
	\begin{proof}
		Suppose $(a_n)$ is bounded by certain $M > 0$, that's, for every $n \in \N$, $-M < a_n < M$. Consider the split $I_1^\ell := [-M, 0]$ and $I_1^u := [0, M]$. At least one of above closed intervals contain an infinitely many elements of $(a_n)$. Define the interval as $I_2$. At each $I_n$, one can split it evenly into two closed intervals such that at least one of these sub-intervals contain infinitely many element in the sequence, and $I_{n+1}$ is defined to be such sequence. Note that the sequence of closed intervals constructed from above recursive procedure is in fact nested. Obviously $\lim_{n \to \infty} |I_n| = 0$. Further, by the nested interval property, one can show that $\cap_{n \in \N} I_n \neq \varnothing$. Then $\cap_{n \in \N} I_n$ must be a singleton with $a$ in it. One can construct such that $a_{n_k} \in I_k$. Note that $|I_n| = \frac{1}{2^{n-1}}$, therefore, for every $\varepsilon > 0$, one can take $N \geq \log_2 \left({\frac{1}{\varepsilon}}\right) + 1$, so that for every $k \geq N$, by definition of subsequences, $n_k \geq n$, so that $a_{n_k}, a \in I_N$. This implies $a_{n_k} \in V_{\varepsilon}(a)$ and $(a_{n_k}) \to a$.
	\end{proof}
	
	\subsection{Cauchy Criterion}
	\begin{definition}
		A sequence $(a_n)$ is a \textbf{Cauchy} sequence if
		\begin{align}
			\forall \varepsilon > 0\ \exists N \in \N\ s.t.\ m, n \geq N \implies |a_n - a_m| < \varepsilon
		\end{align}
	\end{definition}
	
	\begin{proposition}
		Every convergent sequence is Cauchy.
	\end{proposition}
	
	\begin{proof}
		Let $(a_n) \to \ell$, let $\varepsilon > 0$. By the convergence of sequence, $\exists N \in \N$ such that for every $ n \geq N$, $|a_n - \ell| < \frac{\varepsilon}{2}$, which turns out to imply $a_n, a_m \in V_\varepsilon(\ell)$.
	\end{proof}
	
	\begin{lemma}
		Every Cauchy sequence is bounded.
	\end{lemma}
	
	\begin{proof}
		Let $(a_n)$ be a Cauchy sequence, take $\varepsilon = 1$, then there exists $N \in \N$ such that for every $m, n \geq N$, $|a_n - a_m| < 1$. In particular, take $m = N$, for every $n \geq N$, $|a_n - a_N| < 1$, and $|a_n| \leq |a_N| + 1$. Then $(a_n)$ is clearly bounded by:
		\begin{align}
			M := \max \{|a_n|: n \leq N\} \cup \{|a_N| + 1\}
		\end{align}
	\end{proof}
	
	\begin{theorem}[Cauchy Criterion]
		A sequence \emph{of real numbers} is convergent if and only if it's Cauchy.
	\end{theorem}
	
	\begin{proof}
		($\impliedby$) Suppose $(a_n)$ is Cauchy, by the lemma established above, $(a_n)$ is bounded. By the Bolzano–Weierstrass theorem, there exists a subsequence $(a_{n_k}) \to \ell$. \\
		\textbf{Claim:} $(a_n) \to \ell$. Let $\varepsilon > 0$, there exists $N_1 \in \N$ such that for every $n_k, n \geq N_1$, $|a_{n_k} - a_n| < \frac{\varepsilon}{2}$. 
		And there exists another $N_2 \in \N$ such that for every $n_k \geq N_2$, $|a_{n_k} - \ell| < \frac{\varepsilon}{2}$. Take $N_3 := \max\{N_1, N_2\}$.
		Note that for every $n \geq N_3$, one can choose some $n_k \geq n$ and derive
		\begin{align}
			|a_n - \ell| &= |a_n - a_{n_k} + a_{n_k} - \ell| \\
			&\leq |a_n - a_{n_k}| + |a_{n_k} - \ell| \\
			&< \varepsilon
		\end{align}
		($\implies$) Already shown in previous proposition.
	\end{proof}
	
	\subsection{Convergence Test for Series}
	\begin{theorem}[$n$-th term test; necessary condition for convergent series]
		Series $\sum_{i=1}^\infty a_n$ converges $\implies \lim_{n \to \infty} a_n = 0$.
	\end{theorem}
	
	\begin{proof}
		Suppose the partial sums converges to $\ell$, by the definition of partial sums, $a_n = s_{n+1} - s_{n}$. Further, the convergence of partial sums guarantees the convergence of $(a_n)$. By taking limit on both sides of above identity, it can be shown $\lim_{n \to \infty} a_n = 0$.
	\end{proof}
	
	\begin{theorem}[Cauchy Criterion for Series]
		For series $\sum_{n=1}^\infty a_n$, TFAE:
		\begin{enumerate}[(i)]
			\item Series converges;
			\item $\forall \varepsilon > 0,\ \exists N \in \N\ s.t.\ \forall n \geq N,\ \left |\sum_{k=\red{n+1}}^{\red{\infty}} a_k \right | < \varepsilon$ (i.e. \emph{tail} sum sequence converges);
			\item $\forall \varepsilon > 0,\ \exists N \in \N\ s.t.\ \forall m > n \geq N,\ \left | \sum_{k=n+1}^m a_k \right | < \varepsilon$. (i.e. partial sum is Cauchy)
		\end{enumerate}
	\end{theorem}
	\begin{proof}
		$(i) \implies (ii)$: Suppose $(S_n)$ converges, let $\varepsilon > 0$, $\exists N\ s.t.\ \forall n \geq N, |S_n - L| < \varepsilon$. Note that 
		\begin{align}
			L - S_n &= \lim_{m \to \infty} \sum_{k=1}^m a_k  - S_n \\
			&= \lim_{m \to \infty} \left [ \sum_{k=1}^m a_k  - S_n \right ]\\
			&= \lim_{m \to \infty} \sum_{k=n+1}^m a_k
		\end{align}
		which implies the convergence of tail sums. \\
		$(ii) \implies (iii)$: Suppose the tail sum converges, let $\varepsilon > 0$, note that
		\begin{align}
			\left | 
			\sum_{k=n+1}^{m} a_k
			\right| &= \left |
			\sum_{k=m+1}^\infty a_k - \sum_{k=n+1}^\infty a_k
			\right| \\
			&\leq \left |
			\sum_{k=m+1}^\infty a_k \right | + \left | \sum_{k=n+1}^\infty a_k
			\right|
		\end{align}
		Both terms can be made arbitrarily small by $(ii)$, specifically, one can choose $N_1$ and $N_2$ such that both terms are strictly bounded by $\frac{\varepsilon}{2}$, and $N_3 := \max\{N_1, N_2\}$ is the desired value. \\
		$(iii) \implies (i)$: Since the partial sum is a Cauchy sequence in a complete space, it must converges, so the series is well-defined.
	\end{proof}
	
	\subsubsection{The Comparison Test}
	\begin{definition}
		A sequence $(a_n)$ is a \textbf{geometric sequence} with coefficient $r$ if $a_{n+1} = r a_n$.
	\end{definition}
	
	\begin{proposition}
		Geometric sequences whenever $r \in (-1, 1)$. Note that when $r = -1$, the sequence becomes an alternating sequence, and the convergence property is indefinite.
	\end{proposition}
	
	\begin{proposition}
		Let $(a_n)$ be a geometric sequence with coefficient $r$, then for every $m \in \N$,
		\begin{align}
			r S^a_m &= ra_0 + r^2 a_0 + \cdots + r^{n+1} a_0 \\
			\implies (r - 1) S^a_m &= r^{n+1} a_0 - a_0 \\
			\implies S^a_m &= a_0 \frac{1 - r^{m+1}}{1 - r}
		\end{align}
	\end{proposition}
	
	\begin{theorem}[The Comparison Test]
		Let $(a_n)$ and $(b_n)$ be two sequences satisfy $|a_n| \leq b_n$ for every $n \in \N$. Then
		\begin{enumerate}[(i)]
			\item $\sum_{i=1}^\infty b_n$ converges $\implies$ $\sum_{i=1}^\infty a_n$ converges;
			\item $\sum_{i=1}^\infty a_n$ diverges $\implies$ $\sum_{i=1}^\infty b_n$.
		\end{enumerate}
	\end{theorem}
	
	\begin{proof}
		\emph{Part 1}: Suppose $(b_n)$ converges, it is therefore Cauchy. Let $\varepsilon > 0$.
		Note that for every $m > n$:
		\begin{align}
			\abs{S^a_m - S^a_n} &= \abs{\sum_{k=n+1}^m a_k} \\
			&\leq \sum_{k=n+1}^m \abs{a_k} \\
			&\leq \sum_{k=n+1}^m b_k
		\end{align}
		Therefore exists $N \in \N$ such that $\sum_{k=n+1}^m b_k \leq \abs{\sum_{k=n+1}^m b_k} < \varepsilon$ for every $m, n \geq N$. Taking such $N$ provides the cutoff needed for $(S_n^a)$ to be Cauchy. Because $(S_n^a) \subset \R$, it converges. \\
		\emph{Part 2}: The result is immediate by taking the contrapositive form of the previous statement.
	\end{proof}
	
	\subsubsection{The Root Test}
	
	\begin{definition}
		Let $(a_n)$ be a bounded sequence, then
		\begin{align}
			\limsup (a_n) &:= \sup_{\red{n} \to \infty} \{a_k : k \geq \red{n}\} \\
			\liminf (a_n) &:= \inf_{\red{n} \to \infty} \{a_k : k \geq \red{n}\} \\
		\end{align}
	\end{definition}
	
	\begin{theorem}[The Root Test]
		Let $(a_n)$ be a sequence in which $a_n \geq 0$ for every $n \in \N$, let $\ell = \limsup a_n^{1/n}$, then
		\begin{enumerate}[(i)]
			\item If $\ell < 1$, then $(S^a_n)$ converges;
			\item If $\ell > 1$, then $(S^a_n)$ diverges;
			\item If $\ell = 0$, inconclusive.
		\end{enumerate}
	\end{theorem}
	
	\begin{proof}
		\emph{Part 1:(Idea: compare with geometric series with $r < 1$)} Suppose $\ell < 1$, pick $r \in (\ell, 1)$, and let $\varepsilon = r - \ell$. By the convergence of supremum, there exists $N \in \N$ such that for every $n \geq N$,
		\begin{align}
			\abs{
			\sup_{k \geq n} a_k^{1/k} - \ell
			} &< \varepsilon \\
			\implies a_n^{1/n} \leq \sup_{k \geq n} a_k^{1/k} &< \ell + \varepsilon =: r
		\end{align}
		Therefore, for every $n \geq N$, $a_n < r^{n}$. Because $(a_n)$ is assumed to be a non-negative sequence, then $|a_n| < r^{n}$. Construct new sequences:
		\begin{align}
			b_k = 
			\begin{cases}
				a_k\ \forall k < N \\
				r^k\ \forall k \geq N \\
			\end{cases}
		\end{align}
		Then, clearly $|a_n| \leq b_k$ for every $k \in \N$. And $(b_n)$ is a sequence with geometric tails (which has coefficient less than one). So $\sum^\infty b_k$ converges, which implies $\sum^\infty a_k$ converges by the comparison test.
		\\
		\emph{Part 2:} Suppose $\ell > 1$. \\
		Note that the necessary condition for $\sum a_n^{1/n}$ to converge is $\lim_{n \to \infty} a_n^{1/n} = 0$, which implies every subsequence of $(a_n^{1/n})$ converges to zero. We are going to prove the divergence of series by constructing a subsequence of $(a_n^{1/n})$ does not converge to zero.\\
		Take $\varepsilon = \ell - 1 > 0$, there exists $N$ such that for every $n \geq N$:
		\begin{align}
			\ell - \varepsilon &< \sup_{k \geq n} a_k^{1/k} \\
			\implies 1 &< \sup_{k \geq n} a_k^{1/k}
		\end{align}
		By definition of supremum, there exists $n_1 \geq n$ such that
		\begin{align}
			a_{n_1}^{1/n_1} > 1
		\end{align}
		For every $n \geq \N$, we can construct a subsequence of $(a_n^{1/n})$ such that every term in it is strictly greater than 1, which means it cannot converge to 0. Therefore, series diverges.
		\end{proof}
	
	\subsubsection{Other Tests}
	\begin{theorem}[Limit Comparison Test]
		Let $\ser{a}{n}$ and $\ser{b}{n}$ satisfy:
		\begin{enumerate}[(i)]
			\item $b_n \geq 0$;
			\item $\limsup_n \frac{|a_n|}{b_n} < \infty$;
			\item $\ser{b}{n}$ converges.
		\end{enumerate}
		Then $\ser{a}{n}$ converges as well.
	\end{theorem}
	
	\begin{theorem}[Ratio Test]
		Given sequence $\seq{a}{n}$ such that $a_n \geq 0$, then 
		\begin{enumerate}
			\item If $\limsup \frac{a_{n+1}}{a_n} < 1$, $\ser{a}{n}$ converges;
			\item If $\limsup \frac{a_{n+1}}{a_n} > 1$, $\ser{a}{n}$ diverges.
		\end{enumerate}
	\end{theorem}
	
	\begin{theorem}[Integral Test]
		Let $f(x)$ be a \emph{positive} and \emph{monotone decreasing} function on $[1, \infty)$. Consider $(f(x_n))$, then
		\begin{align}
			\sum_{n=1}^\infty f(n) \tx{ conv} \iff \int_1^\infty f(x)\ dx < \infty
		\end{align}
	\end{theorem}
	
	\begin{theorem}[Alternating Series Test]
		For an alternating sequence $\sum_{n=1}^\infty (-1)^n a_n$, if $(a_n) \searrow 0$, then the series converges.
	\end{theorem}
	
	\begin{proof}
		\hl{TODO}
	\end{proof}


	\subsection{Absolute and Conditional Convergence}
	
	\begin{corollary}[Corollary of Comparison Test]
		If $\sum_{i=1}^\infty |a_n|$ converges, then $\ser{a}{n}$ converges.
	\end{corollary}
	
	\begin{definition}
		For any series $\ser{a}{n}$, if 
		\begin{enumerate}
			\item $\sum_{i=1}^\infty |a_n|$ converges, $\ser{a}{n}$ \textbf{converges absolutely};
			\item $\sum_{i=1}^\infty |a_n|$ does not converge, then $\ser{a}{n}$ \textbf{converges conditionally}.
		\end{enumerate}
	\end{definition}
	
	\begin{example}
		Alternating harmonic series converges conditionally. \\However, $\sum_{n=1}^\infty \frac{(-1)^n}{n^2}$ converges absolutely.
	\end{example}
	
	\begin{definition}
		$\ser{b}{n}$ is called a \textbf{rearrangement} of series $\ser{a}{n}$ if there exists $f: \N \to \N$ such that $f$ is a bijection and $b_{f(k)} = a_k$ for every $k \in \N$.
	\end{definition}
	
		\begin{theorem}[Riemann Series Theorem]
		If series $\ser{a}{n}$ converges \ul{conditionally}, for every $\alpha \in \R$, there exists a rearrangement $\ser{b}{n}$ converges to $\alpha$.
	\end{theorem}
	
	\begin{proof}
		The proof is non-trivial and omitted.
	\end{proof}
	
	\begin{theorem}
		If series $\ser{a}{n}$ converges \ul{absolutely} to some value $A \in \R$, then every rearrangement $\ser{b}{n}$ converges to $A$.
	\end{theorem}
	
	\begin{proof}
		Define partial sum sequences
		\begin{align}
			S_n := \sum_{k=1}^n a_k \quad
			T_m := \sum_{k=1}^m b_k
		\end{align}
		Suppose $(S_n) \to A$, want to show: $(T_n) \to A$.\\
		Let $\varepsilon > 0$ fixed. \\
		By convergence of $(S_n)$, there exists $N_1 \in \N$ such that
		\begin{align}
			n \geq N_1 \implies \abs{S_n - A} < \frac{\varepsilon}{2}
		\end{align}
		Because $\ser{a}{n}$ converges absolutely, by the Cauchy criterion for convergent series (i.e. partial sum sequence is Cauchy), there exists $N_2 \in \N$ such that
		\begin{align}
			n > m \geq N_2 \implies \sum_{k=n+1}^m \abs{a_k} < \frac{\varepsilon}{2}
		\end{align}
%		Note that, for every $m, n \in \N$:
%		\begin{align}
%			\abs{T_M - S_N} = \abs{b_1 + \cdots + b_M - a_1 - \cdots - a_N}
%		\end{align}
		Define $N := \max\{N_1, N_2\}$, $M := \max\{f(k): 1 \leq k \leq N\}$.
		\begin{align}
			\abs{T_m - S_N} &= \abs{b_1 + \cdots + b_m - a_1 - \cdots - a_N} \\
			&= \abs{b_1 + \cdots + b_m	 - b_{f(1)} - \cdots - b_{f(N)}}
		\end{align}
		Note that for every $m \geq M$,
		by construction, $\{b_{f(1)}, \cdots b_{f(N)}\} \subset \{b_1 \cdots, b_m\}$. \\
		Note that for each $b_{f(k)} \in \{b_1 \cdots b_m\}$, either $k > N$ or $k \leq N$. But all $b_{f(k)}$ with $k \leq N$ were subtracted, so $b_{f(k)}$ elements left are all from $\{a_k: k \geq N+1\}$.
		\begin{align}
			... &= \abs{\sum_{k \in \mc{I} \geq N + 1} a_k} \\
			&\leq \sum_{k=N+1}^\infty |a_k| < \frac{\varepsilon}{2}
		\end{align}
		Therefore, for all $m \geq M$, 
		\begin{align}
			\abs{T_m - A} &= \abs{T_M - S_n + S_n - A} \\
			&\leq |T_M - S_n| + |S_n - A| \\
			&< \varepsilon
		\end{align}
		The desired result is immediate.
	\end{proof}
	
	\section{Topology in $\R$}
	
	\begin{definition}
		\begin{align}
			V_\varepsilon(x_0) := \{x \in \R: \norm{x, x_0} < \varepsilon\}
		\end{align}
	\end{definition}
	
	\begin{definition}
		A subset of $\mc{O} \subset \R$ is \textbf{open} if
		\begin{align}
			\forall x \in \mc{O}\ \exists\ \varepsilon > 0\ s.t.\ V_\varepsilon(x)\ s.t.\ V_\varepsilon(x) \subset \mc{O}
		\end{align}
	\end{definition}
	
	\begin{theorem}
		Arbitrary union of open sets is open; Any finite intersection of open sets is open.
	\end{theorem}
	
	\begin{proof}
		Let $\mc{O}_\alpha$ open for all $\alpha \in \mc{A}$. Let $\mc{O} := \bigcup_{\alpha \in \mc{A}} \mc{O}_\alpha$. If $x \in \mc{O}$, there exists some $\alpha \in \mc{A}$ such that $x \in \mc{O}_\alpha$. There exists $V_\varepsilon(x) \subset \mc{O}_{\alpha} \subset \mc{O}$. Hence $\mc{O}$ is open.
		\\
		Let $\{\mc{O}_i: 1 \leq i \leq n\}$ be a collection of open sets, let $\mc{O} := \bigcap_{i=1}^\infty \mc{O}_i$. If $x \in \mc{O}$, there exists $\varepsilon_i > 0$ such that $V_{\varepsilon_i}(x) \subset \mc{O}_i$ for every $i$. Take $\varepsilon := \max\{\varepsilon_i\}$, which exists and is strictly positive by finiteness of index set. Therefore $V_\varepsilon(x) \subset \mc{O}_i$ for every $i$, and therefore in $\mc{O}$.
	\end{proof}
	
	\begin{definition}
		$x$ is a \textbf{limit point} of $A$ if $\forall \varepsilon > 0$,
		\begin{align}
			V_\varepsilon(x) \cap A - \{x\} \neq \varnothing
		\end{align}
		\emph{Remark: this definition does not require $x$ to be an element of $A$.}
	\end{definition}
	
	\begin{theorem}
		$x$ is a limit point $A$ if and only if there exists a sequence $\seq{a}{n} \subset A$ such that \ul{$a_n \neq x\ \forall n \in \N$} and $\seq{a}{n} \to x$.
	\end{theorem}
	\begin{proof}
		$(\implies)$ Let $x$ be a limit point, take $\varepsilon = \frac{1}{n}$, immediate by the definition of limit point. \\
		$(\impliedby)$ Trivially by definition of sequential convergence.
	\end{proof}
	
	\begin{definition}
		$X \subset \R$ is \textbf{closed} if it contains all its limit points.
	\end{definition}
	
	\begin{definition}
		$x \in A$ is an \textbf{isolated point} is it is not a limit point of $A$.
	\end{definition}
	
	\begin{definition}
		$A \subset X$ is \textbf{dense} in $X$ if $\overline{A} = X$.
	\end{definition}
	
	\begin{theorem}
		Let $x \in \R$, there exists a sequence $\seq{q}{n} \subset \Q$ such that $\seq{q}{n} \to x$.
	\end{theorem}
	
	\begin{proof}
		Let $x \in \R$. Note that $\forall u < v \in \R$, there exists $q \in (u, v) \cap \Q$.
		Hence, for every $n \in \N$, $\exists q_n \in \Q$ such that $x - \frac{1}{n} < q_n < x + \frac{1}{n}$. It is evident that $\seq{q}{n} \to x$.
	\end{proof}
	
	\begin{definition}
		The \textbf{closure} of $A$, denoted as $\overline{A}$, is defined to be the union of $A$ and all limit points of $A$.
	\end{definition}
	
	\begin{lemma}
		$\overline{A}$ is the smallest closed set containing $A$.
	\end{lemma}
	\begin{proof}
		It is evident that $\overline{A}$ is a closed set containing $A$. \\
		Now show the closure is in fact the smallest closed set. Let $B \subsetneq \overline{A}$ be a proper subset of the closure, we are going to show that $B$ is not closed. Let $x \in \overline{A} - B \neq \varnothing$.\\
		Note that $\overline{A} \equiv A \cup A'$, then either $x \in A$ or $x \in A'$. If $x \in A$, then $B$ does not contain $A$. If $x \in A'$, then $B$ does not contain all limit points of $A$, so it is not closed.
	\end{proof}
	
	\begin{theorem} Equivalent definitions of openness and closedness:
		\begin{enumerate}[(i)]
			\item $\mc{O}$ is open if and only if $\mc{O}^c$ is closed;
			\item $\mc{O}$ is closed if and only if $\mc{O}^c$ is open.
		\end{enumerate}
	\end{theorem}
	
	\begin{proof}
		($\implies$) Let $\mc{O}$ be an open set, let $(x_n) \to x$ be a convergent sequence in $\mc{O}^c$. It is evident that if $x \in \mc{O}$, infinitely many elements in the tail of $(x_n)$ would be in $V_\varepsilon(x) \subset \mc{O}$, which leads to a contradiction. Therefore $\mc{O}^c$ contains all of its limit poitns, and $\mc{O}^c$ is therefore closed.
		\\($\impliedby$) Let $\mc{O}^c$ be a closed set, suppose $\mc{O}$ is not open, there exists $x \in \mc{O}$ such that for all $\varepsilon > 0$, $V_\varepsilon(x) \cap \mc{O}^c \neq \varnothing$. Then we can construct a sequence in $\mc{O}^c$ converge to $x$, which leads to a contradiction that there is a limit point of a sequence in $\mc{O}^c$ not contained by $\mc{O}^c$. \\
		The second part is immediate.
	\end{proof}
	
	\begin{theorem}
		Any intersection of closed sets is closed; any finite union of closed  sets is closed.
	\end{theorem}
	
	\begin{proof}
		Direct result from De Morgan's law and the previous theorem.
	\end{proof}
	
	\paragraph{} \emph{Remark: Limit points and boundary points are completely different.} Example: let $\Omega = [1, 2] \cup {3}$, then $3$ is a boundary point but not a limit point (i.e. it is isolated). And $0.5$ is a limit point but not a boundary point.
	
	\begin{definition}
		A set $K \subset \R$ is \textbf{compact} if every sequence in $K$ has a convergent subsequence converges to some limit $x \in K$.
	\end{definition}
	
	\begin{theorem}
		A set $K \subset \R$ is compact \ul{if and only if} it is closed and bounded.
	\end{theorem}
	
	\begin{proof}
		($\implies$) Suppose $K \subset \R$ is compact. \\
		\emph{Show $K$ is bounded}: suppose, for contradiction, $K$ is unbounded, then for every $N \in \N$, one can construct a sequence as following: $a_1 \in K$ and $a_{n+1} > \max\{a_n, n\}$. Such sequence diverges to positive infinity, and every subsequence of it converges to infinity as well (easy to verify). This leads to a contradiction to the compactness of $K$. \\
		\emph{Show $K$ is closed}: Suppose, for contradiction, $K$ is not closed, then there exists some limit point of $K$ say $x \notin K$. Consider the sequence $(x_n) \to x$ in $K$, because every subsequence of such convergent sequence converges to the same limit $x \notin K$, which leads to a contradiction of compactness. \\
		($\impliedby$) Let $(x_n) \subset K$, then $(x_n)$ is bounded and therefore possesses a convergent subsequence by Bolzano-Weierstrass Theorem. Further, because $K$ is closed, then the limit point must be in $K$.
	\end{proof}
	
	\begin{theorem}[Nested Compact Set Property]
		Let $\R^n \supset K_1 \supset K_2 \supset \cdots \supset K_n \supset \cdots$, where $K_n \neq \varnothing$ are all compact sets, then
		\begin{align}
			\bigcap_{n \in \N} K_n \neq \varnothing
		\end{align}
	\end{theorem}
	
	\begin{proof}
		Construct a sequence such that $x_n \in K_n$ for every $n \in \N$. In particular, $(x_n) \subset K_1$. Because $K_1$ is compact, it has  a convergent subsequence $(x_{n_k}) \to x \in K_1$. Then every subsequence of $(x_{n_k})$ converges to the same limit $x$.
		\\Note that by dropping out the first element of the subsequence, the resulted sequence starts with $x_{n_2}$. By the definition of subsequences, $n_2 \geq 2$, therefore, the truncated subsequence is contained in $K_2$ because of the compactness of $K_2$. As a result, $x \in K_2$. Applying the same argument on all natural numbers, it is immediate that $x \in K_n\ \forall n \in \N$. So $x \in \bigcap_{n \in \N} K_n$.
	\end{proof}
	
	\begin{proof}[Proof. (Cantor's Argument)]
		Suppose, for contradiction, the intersection is empty. Define $U_n := K_1 \backslash K_n$. Note that $U_n = K_1 \cap K_n^c = K_n^c$, which is open. Further, $\bigcup_{n \in \N} U_n = \bigcup_{n \in \N} K_1 \cap K_n^c = K_1 \cap (\bigcup_{n \in \N}K_n^c) = K_1 \cap (\bigcap_{n \in \N}K_n)^c = K_1 \backslash \bigcap_{n \in \N}K_n = K_1$. Therefore, $\mc{C} = \{U_n: n \in \N\}$ is an open cover of $K_1$. Because $K_1$ is compact, there exists a finite subcover of $\mc{C}$. Take $n^*$ to be the greatest index in this finite subcover, then for every $x' \in K_{n^*+1} \subset K_1$, $x'$ is not in the union of the constructed subcover, which leads to a contradiction.
	\end{proof}
	
	\begin{example}
		Note that the closedness itself is not sufficient for the nest compact set property to hold. For instance, the following sequence of closed sets are nested: $F_n := [n, \infty)$, but indeed, for every $x \in \R$, there exists a natural number $n > x$, so that $x \notin \bigcap_{n \in \N} F_n$. Therefore, $\bigcap_{n \in \N} F_n = \varnothing$.
	\end{example}
	
	\begin{definition}
		Let $A \subset \R$, an \textbf{open cover} for $A$ is a collection of open sets $\{\mc{O}_\lambda: \lambda \in \Lambda\}$ such that $A \subseteq \bigcup_{\lambda \in \Lambda} \mc{O}_\lambda$.
	\end{definition}
	
	\begin{theorem}[Heine-Borel]
		Let $K \subset \R$, then the following are equivalent:
		\begin{enumerate}[(i)]
			\item $K$ is (sequentially) compact;
			\item $K$ is closed and bounded;
			\item Every open cover of $K$ has a finite subcover.
		\end{enumerate}
	\end{theorem}
	
	\begin{proof}
		The equivalence of $(i)$ and $(ii)$ has been proven previously. \\
		\emph{Show} $(iii) \implies (ii)$: suppose every open cover of $K$ has a finite subcover, consider the following cover of $K$: $\mc{C} = \{[-n, n]: n \in \N\}$. Let $M$ be the greatest index in the finite subcover $\mc{C}$, and obviously $K$ is bounded by $M$. \\
		Suppose, for contradiction, that $K$ is not closed. Let $y$ be a limit point of $K$ but $y \notin K$. Then, for every $\varepsilon > 0$, $V_\varepsilon^o(y) \cap K \neq \varnothing$. We've shown that $K$ is bounded, take $M \in \R$ such that $(-M, M) \supset K$. Define the following cover:
		\begin{align}
			\mc{C} := \left\{
			(-M, M)\ \backslash\ \overline{V_\varepsilon(y)}
			:\varepsilon \in \R_{++}
			\right\}
		\end{align}
		Because $K$ is compact, there exists a finite subcover of $\mc{C}$, which is clearly a contradiction.
	\end{proof}
\end{document}
























