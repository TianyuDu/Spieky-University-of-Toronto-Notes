\documentclass[11pt]{article}

\usepackage{amsmath}
\usepackage{amssymb}
\usepackage{enumitem}

\title{ECO102 Lecture Notes}
\author{Tianyu Du}
\date{\today}

\begin{document}
\maketitle	
	
	\section{Jan. 24 2018}
	\subsection{Factors of production over time.}
	\paragraph{Focus on Capital} Excluding changes in inventory from this definition of investment.
	\begin{align*}
		Capital_{t+1} = Capital_t - Depreciation_t + Investment_t
	\end{align*}
	\paragraph{Simplest Model}
	\begin{align*}
		GDP = Y = C + I \implies S = I = Y - C \emph{(gross saving)}
	\end{align*}
	
	\paragraph{Still Simple Model}
	\begin{align*}
		Y = C + I + G \implies I = Y - C - G \\
		\text{Government Spending}
		\begin{cases}
			\text{Consumption} \\
			\text{Investment}
		\end{cases}
	\end{align*}
	\paragraph{Assets/work increase slowly in a mature economy} \quad
	\newline Consider:
	\begin{itemize}
		\item Stock of capital Assets.
		\item Stock per capita and per worker.
		\item Flow. Investment goes into (i)\underbar{increase in per labor capital} and (ii)\underbar{depreciation} as well.
		\item Investment Increase.
		\item Capital Assets Increase.
		\item Capital Assets per worker.
	\end{itemize}
	\paragraph{Motivating the importance of productivity}
	\begin{align*}
		GDP = A\ f(L,K,H) \\
		\frac{GDP}{L} = A\ f(\frac{K}{L}, \frac{H}{L})
	\end{align*}
	With A and T fixed and \underbar{constant} returns to scale in L and K, we have increasing GDP only if the growth rate of K is greater than the growth rate of L.
	\newline \textbf{Promoting productivity:}
	\begin{itemize}
		\item Increased access to education.
		\item Encourage saving and investment.
		\item Promote research and development.
		\item Public goods and infrastructure.
	\end{itemize}
	\[
		\text{Higher wage} \implies \text{more productivity}
	\]
	\emph{Sometimes, higher concentration leads to higher productivity.}
	\subsection{Increasing Returns: it's all about knowledge}
	\subsubsection{Knowledge}
	\paragraph{knowledge leaks} Cannot prevent knowledge from being known by others.t
	\paragraph{Complementarity} The private returns are increasing in the level of knowledge.
	\subsubsection{Implications of increasing returns}
	\paragraph{Matching} Concentration of industry due to complementarity: e.g. HollyWood, Finance, Agricultural: Denmark
	\paragraph{Social v.s. private returns} Effects:
		\begin{enumerate}
			\item Private return $<$ social returns.
			\item Private return increasing in knowledge base.
		\end{enumerate}
	\paragraph{Virtuous cycle}
	\paragraph{Vicious cycle} Starting with low knowledge base (Effect 1 $>$ Effect 2)
	\newline
	$\implies$ Little incentive to invest in knowledge.
	\newline
	$\implies$ No new knowledge.
	\newline
	$\implies$ No increasing in incentive in invest.
	
	\subsubsection{Implications for poor countries}
	\paragraph{Problem} How do we transition from the vicious-cycle equilibrium to the virtuous-cycle equilibrium.
	\paragraph{Public policy solution} South Korea
		\begin{itemize}
			\item Promoted electronic industry.
			\item 
		\end{itemize}
		
	\subsubsection{Institutions}
	\paragraph{Property of goods institutions.}
	\begin{itemize}
		\item Political stability.
		\item Role of law. (Transparent and non arbitiray laws)
		\item Effective public service.
		\item Enforced property rights.
		\item No corruption.
	\end{itemize}
\end{document}

























