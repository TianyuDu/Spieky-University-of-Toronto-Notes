\documentclass{article}
\usepackage{amsmath}
\author{Tianyu Du}
\date{Aug 2017}
\title{Self-study Notes on Linear Algebra and its Applications 5e, UT MAT223}
\begin{document}
	\maketitle
	\tableofcontents
	\section{Linear Equations in Linear Algebra}
	\subsection{Systems of Linear Equations}
	\paragraph{Linear Equation} A \textbf{linear equation} in the variables $x_1,...,x_n$ is an equation that can be written in the form
	\[
	\Sigma_{i=1}^n a_i*x_i = b
	\]
	where $b$ is the \textbf{coefficients} $a_1,...,a_n$ are real or complex numbers, usually known in advance.
	\paragraph{System of Linear Equations} A \textbf{system of linear equations} (or a \textbf{linear system}) is a \emph{collection} of one or more linear equations involving the same variables.
	\paragraph{Solution} A \textbf{solution} of the system is a list $(s_1,s_2,...,s_n)$ of numbers that makes each equation a true statement when values $s_1,...,s_n$ are substituted for $x_1,...,x_n$,respectively. The set of all possible solutions is called the \textbf{solution set} of the linear system. Two systems are called \textbf{equivalent} if they have the same solution set.
	\paragraph{} A system of linear equations has
	\begin{enumerate}
		\item no solution\emph{(inconsistent)}, or
		\item exactly one solution, or
		\item infinitely many solutions.
	\end{enumerate}
	\paragraph{Matrix Notation} \textbf{Coefficient Matrix} and \textbf{Augment Matrix}.
	\paragraph{Solving a Linear System: basic strategy} \emph{To replace one system with an equivalent system that is easier to solve.}
	\paragraph{Elementary Row Operations}
	\begin{enumerate}
		\item \emph{\textbf{Replacement}} Replace one row by the sum of itself and a multiple of another row.
		\item \emph{\textbf{Interchange}} Interchange two rows.
		\item \emph{\textbf{Scaling}} Multiply all entries in a row by nonzero constant.
	\end{enumerate}
	Two matrices are called \textbf{row equivalent} if there is a sequence of elementary row operations that transforms one matrix into the other. If the augmented matrices of two linear systems are row equivalent, then the two systems have the same solution set.
	\subsection{Row Reduction and Echelon Forms}
	\paragraph{Leading Entry} A \textbf{leading entry} of a row refers to the leftmost nonzero entry (in a nonzero row).
	\paragraph{Echelon Form / Row Echelon Form} has the following properties:
	\begin{enumerate}
		\item All nonzero rows are above any rows of all zeros.
		\item Each leading entry of a row is in a column to the right of the leading entry of the row above it.
		\item (Consequence of (2)) All entries in a column below leading entry are zeros.
	\end{enumerate}
	\underbar{Additional} properties for \textbf{reduced echelon form} (or \textbf{reduced row echelon form})
	\begin{enumerate}
		\item The leading entry in each nonzero row is 1.
		\item Each leading 1 is the only nonzero entry in its column.
	\end{enumerate}
	\paragraph{Theorem 1: Uniqueness of the Reduced Echelon Form} Each matrix is row equivalent to one and only one reduced echelon matrix. Consequently, \emph{the leading entries are always in the same positions in any echelon form obtained from a given matrix}.
	\paragraph{Pivot Position} A \textbf{pivot position} in a matrix $A$ is a location in $A$ that corresponds to a leading 1 in the reduced echelon form of $A$. A \textbf{pivot column} is a column of $A$ that contains a pivot position.
	\paragraph{} Whenever a system is consistent, the solution set can be described explicitly by solving the \emph{reduced} system of equations for the \textbf{basic variables} in terms of the \textbf{free variables}. When a consistent system has free variables, the solution set has many \emph{parametric descriptions}.
	\paragraph{Theorem 2: Existence and Uniqueness Theorem} A linear system is \emph{consistent} if and only if the rightmost column of the augmented matrix is \emph{not} a pivot column. That is, a consistent augmented matrix does \emph{not} has a row like
	\[
	\begin{bmatrix}
		0 & ... & 0 & b\\
	\end{bmatrix}
	 \quad \text{with $b$ nonzero}
	 \]
	 For a consistent system, the solution set contains
	 \begin{enumerate}
	 	\item \emph{\textbf{A unique solution}}, if there are no free variables.
	 	\item \emph{\textbf{Infinitely many solutions}}, if there is at least one free variables.
	 \end{enumerate}
	 \paragraph{Using Row Reduction To Solve Linear System}
	 \begin{enumerate}
	 	\item Write augmented matrix of the system.
	 	\item Use \emph{row reduction algorithm} to obtain the equivalent echelon form. If the system is inconsistent, stop.
	 	\item Obtain reduced echelon form
	 	\item Write system of equations corresponding to the matrix obtained in (3).
	 	\item Rewrite each nonzero equation from (4) so that its one basic variable is expressed in terms of any free variables appearing in the equation.
	 \end{enumerate}
\end{document}





