\documentclass[]{article}
\title{ECO325: Lecture Notes \\ Macroeconomic Theory}
\author{Tianyu Du}
\date{\today}

\usepackage{spikey}
\usepackage{amsmath}
\usepackage{amssymb}
\usepackage{soul}
\usepackage{float}


\begin{document}
    \maketitle 
    \tableofcontents
    \newpage
    
    \section{Lecture 1. September 6. 2018}
        \begin{defn}
            A \textbf{growth miracle} are episodes where thee growth in a country far exceeds the world average over a extended period of time. \ul{Result} the country experiencing the miracle moves up the wold income distribution.
        \end{defn}
        
        \begin{defn}
            A \textbf{growth disaster} is an episodes where the growth in a country falls short at the world average for an extended period of time. \ul{Result} the country moves down in the world income distribution.
        \end{defn}
    
        \paragraph{Facts} Data from the $20^{th}$ century suggest that
        \begin{enumerate}
            \item Real output grows at a (more or less) constant rate.
            \item Stock of real capital grows at a (more or less) constant rate (but it grows faster than labor input).
            \item Growth rates of real output and the stock of capital are about the same.
            \item The rate of growth of output per capita varies greatly across countries.
        \end{enumerate}
        
        \subsection{Solow Growth Model (continuous time version)}
            \par Solow growth model decomposes the growth in output per capita into portions accounted for by increase in inputs and the portion contributed to increases in productivity.
            \par In the baseline model we denote $K$ as capital, $L$ as labor and $A$ as technology.
            
        \subsubsection{Production Function}
            \begin{remark}
                Harrod-neutral technology here, refer to Uzawa's theorem.
            \end{remark}
            \begin{defn}
                The \textbf{effective labor input} is defined as $A(t)L(t)$
            \end{defn}
            
            \begin{defn}
                The production function is defined as
                \[
                    Y(t) = F(K(t), A(t)L(t))
                \]
                Typically Cobb-Douglas form is taken
                \[
                    Y(t) = K(t)^\alpha (A(t)L(t))^{1 - \alpha},\ \alpha \in (0, 1)
                \]
            \end{defn}
            
            \begin{prop} Properties on production function.
                \begin{enumerate}
                    \item \ul{CRS in $K$ and $AL$}: $Y(cK, cAL) = cY(K, AL),\ \forall c \geq 0$ implies
                        \begin{itemize}
                            \item All gains from specialization have been exhausted.
                            \item Inputs other than $K$ and $AL$ are unimportant.
                        \end{itemize}
                \end{enumerate}
            \end{prop}
            
            \begin{defn}
                Define $c := \frac{1}{AL}$, the \textbf{intensive form} of production function is
                \[
                    y = \frac{Y}{AL} = f(k)
                \]
                where $y := \frac{Y}{AL}$ denotes the \textbf{output per unit of effective labor} and $k := \frac{K}{AL}$ denote the capital stock per unit of effective labor.
            \end{defn}
\end{document}

