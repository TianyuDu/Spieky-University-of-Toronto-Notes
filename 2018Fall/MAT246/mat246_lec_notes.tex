\documentclass[11pt]{article}

\author{Tianyu Du}
\date{\today}
\title{MAT246: Concepts in Abstract Mathematics: \\ \small Lecture 0101 Notes}

\usepackage{amsmath}
\usepackage{amssymb}
\usepackage{float}
\usepackage{soul}
\usepackage{spikey}
\usepackage{xcolor}
\usepackage{centernot}

\usepackage[
    type={CC},
    modifier={by-nc},
    version={4.0},
]{doclicense}

\begin{document}
	\maketitle
	\doclicenseThis
	
	\tableofcontents
	\newpage
	\section{Lecture 1 Sep. 7 2018}
	
	\begin{definition}
		Let $\mathbb{N} := \{1, 2, 3, \dots\}$ be the set of \textbf{natural numbers}.
	\end{definition}
	
	\begin{theorem}[Principle of Mathematical Induction]
		Suppose $S$ is a set of natural numbers, $S \subseteq \mathbb{N}$. If
		\begin{enumerate}
			\item $1 \in S$
			\item $k \in S \implies k + 1 \in S,\ \forall k \in \mathbb{N}$
		\end{enumerate}
		then, $S = \mathbb{N}$
	\end{theorem}
	
	\begin{example}
		Show that 
		\[
			1^2 + 2^2 + \dots + n^2 = \frac{n(n+1)(2n+1)}{6}\ \forall n \in \mathbb{N}
		\]
		\begin{proof}
			
		\end{proof}
	\end{example}
	
	\section{Lecture 2 Sep. 10 2018}
	
	\begin{theorem}[Extended Principle of Mathematical Induction]
		Suppose set $S \subseteq \mathbb{N}$ and let $n_0 \in \mathbb{N}$ fixed, if
		\begin{enumerate}
			\item $n_0 \in S$
			\item $\forall k \geq n_0, k \in S \implies k + 1 \in S$
		\end{enumerate}
		then $\{n_0, n_0 + 1, n_0 + 2, \dots \}\ \textcolor{red}{\subseteq}\ S$
	\end{theorem}
	
	\begin{example}
		Show that 
		\[
			n! \geq 3^n\ \forall n \geq 7
		\]
		\begin{proof}
			
		\end{proof}
	\end{example}
	
	\begin{theorem}[Well-Ordering Principle]
		Every non-empty subset of natural number has a smallest element.
	\end{theorem}

	\begin{proof} (Principle of Mathematical Induction)\\
		Let $S \subseteq \mathbb{N}$ \\
		Suppose $1 \in S \land (k \in S \implies k+1 \in S, \forall k \in \mathbb{N})$ \\
		Show: $S = \mathbb{N}$ \\
		Let $T = \mathbb{N} \backslash S$ \\
		Suppose $T \neq \emptyset$ \\
		By Well-Ordering Principle, there exists a smallest element of T, denoted as $t_0 \in \mathbb{N}$. \\
		Since $1 \in S$, therefore $t_0 \neq 1$. \\
		Therefore $t_0 > 2$. \\
		Thus $t_0 - 1 \in \mathbb{N}$ and since $t_0 = \min{T}$, $t_0 - 1 \notin T$ \\
		Therefore $t_0 - 1 \in S$, then, $t_0 - 1 + 1 = t_0 \in S$, \\
		Contradict the assumption that $t_0 \in T$. \\
		Thus $T = \emptyset$ and $S = \mathbb{N}$. \\
	\end{proof}
	
	\begin{remark}
		We can use principle of Mathematical Induction to prove Well-Ordering Principle as well.
	\end{remark}
	
	\section{Lecture 3 Sep. 12 2018}
	\begin{definition}
		Let $a, b \in \mathbb{N}$ and $a$ \textbf{divides} $b$, written as $a | b$ if 
		\[
			\exists c \in \mathbb{N}\ s.t.\ b = ac
		\]
		And $a$ is a \textbf{divisor} of $b$.
	\end{definition}
	
	\begin{definition}
		A natural number $p$ (except $1$) is called \textbf{prime} if the only divisors of $p$ are $1$ and $p$.
	\end{definition}
	
	\begin{lemma}[Prime numbers are building blocks of natural numbers]
		Every natural number other than $1$ is a \emph{product}\footnote{Product could mean the product of a single number.} of prime numbers.
	\end{lemma}
	
	\begin{theorem}[Principle of Complete Induction]
		Suppose $S \subseteq \mathbb{N}$ and if 
		\begin{enumerate}
			\item $n_0 \in S$
			\item $n_0, n_0 + 1, \dots, k \in S \implies k+1 \in S,\ \forall k \geq n_0$
		\end{enumerate}
		then 
		\[
			\{n_0, n_0 + 1, \dots\} \subseteq S
		\]
	\end{theorem}
	\begin{proof} [Proof of Lemma]
		Let $S \subseteq \mathbb{N}$ for which the lemma is true, \\
		Want to show: $S = \mathbb{N} \backslash \{1\}$ \\
		(Base Case) For $2$ it's a product of prime. Thus $2 \in S$ \\
		(Inductive Step) Suppose $\{2,3, \dots k\} \subseteq S$ \\
		Consider $k+1$, if $k+1$ is a prime then $k+1$ can be written as a product of itself, as a product of one single prime.\\
		Else, if $k+1$ is not a prime, then $\exists 1<m,n<k+1\ s.t.\ k+1 = mn$.\\
		By induction hypothesis of strong induction, $m,n$ can both be written as product of primes. \\
		$m = \prod_{i=1}^\ell {p_i},\ n = \prod_{i=1}^t {q_i}$ where $p_i, q_i$ are all primes.\\
		and $k+1 = \prod_{i=1}^t{q_i}\prod_{i=1}^\ell {p_i}$ \\
		thus $k+1 \in S$ \\ 
		by principle of strong induction, $\{2,3,\dots,\} \subseteq S$.
	\end{proof}
	
	\begin{theorem}
		There is no largest prime number.
	\end{theorem}
	
	\begin{proof}
		(By contradiction) \\
		Assume there is a largest prime $p$, \\
		then $\{2,3,5,\cdots,p\}$ is the set of all primes \\
		Let $M := (2*3*5*\cdots*p)+1 \in \mathbb{N}$ \\
		$M$ is either prime or not. \\
		Suppose $M$ is not a prime, then by Lemma 3.1, $\exists p' $ dividing $M$. \\
		Obviously $\forall i \in \{2*3*5*\cdots*p\},\ i \centernot | M$. \\
		There is no prime dividing $M$, which contradict Lemma 3.1 \\
		Thus $M$ is a prime, and $M > p$, which contradicts assumption \\
		Therefore there is no largest prime.
	\end{proof}
	
	\section{Lecture 4 Sep. 14 2018}
	\begin{theorem}[the Fundamental Theorem of Arithmetic]
		Every natural (except 1) is a product of prime(s), and the prime(s) in the product are unique including multiplicity except for the order.
	\end{theorem}
	\begin{proof}
		We have already proven that the existential parts of this theorem in Lemma 3.1. \\
		(Proof for the uniqueness part) Suppose there exists natural number (not 1) has 2 different prime factorizations. \\
		By well ordering principle, there is a smallest $n$, which has two distinct prime factorizations. \\
		Say $n = p_1 p_2 \dots p_k = q_1 q_2 \dots q_\ell $ where $p_i, q_i$ are all primes. \\
		Notice that $p_i \neq q_j$ for any combination of $(i,j)$ since if so $\frac{n}{p_i}=\frac{n}{q_j}$ is a natural number smaller than $n$ having 2 distinct prime factorization, which contradicts our assumption above. \\
		Specifically, $p_1 \neq q_1$. \\
		(Case 1: $p_1 < q_1$) \\
		Let $m := n - p_1 q_2 \dots q_\ell \in \mathbb{N}$ \\
		Notice $m = p_1 (p_2 p_3 \dots p_k - q_2 q_3 \dots q_\ell)$ \\
		Also $m = (q_1 - p_1) (q_2 q_3 \dots q_\ell)$ \\
		$\implies m = p_1 \dots p_k = q_2 q_3 \dots q_\ell (q_1 - p_1)$ \\
		$\implies p_1 | m $ also notices that $p_1 \centernot | q_2 q_3 \dots q_\ell$ \\
		$\implies p_1 |  (q_1 - p_1) \implies p_1 | q_1 \implies p_1 = q_1$ \\
		Contradicts the assumption that $p_q < q_1$ \\
		The other case goes a similar proof.
	\end{proof}
	
	\begin{definition}
		A natural number $n$ is called \textbf{composite} if it's not 1 or a prime number.
	\end{definition}
	
	\begin{remark}
		Natural numbers are partitioned into 3 categories, 1, prime and composite numbers.
	\end{remark}
	
	\begin{example}
		Find 20 consecutive composite numbers.
		\[
		(21!)+2, (21!)+3, \dots, (21!)+21
		\]
	\end{example}
	
	\begin{example}
		Find $k$ consecutive composite numbers.
		\[
			(k+1!)+2, (k+1)!+3, \dots ,(k+1!)+k+1
		\]
	\end{example}
	
	\section{Lecture 5 Sep. 17 2018}
	
	\begin{definition}
		Let $a, b \in \mathbb{Z}$, and let $m \in \mathbb{N}$. If $m | a - b$ then we say "$a$ and $b$ are congruent modulo $m$"
	\end{definition}
	
	\begin{remark}
		Regular Induction $\iff$ Complete Induction $\iff$ Well-Ordering Principle
	\end{remark}
	
	\begin{proof}
		(WTS: Complete Induction $\implies$ Well-Ordering Principle) \\
		Let $S \subseteq \mathbb{N}$ and $S \neq \emptyset$ \\
		(WTS, $S$ has the smallest element) \\
		Assume $S$ does not have the smallest element. \\
		Let $T := S^c$ \\
		Clearly $1 \in T$ (prop 1) \\
		Since other wise 1 could be the smallest element of $S$. \\
		Let $k \in \mathbb{N}$. \\
		Suppose $1, 2, 3, \dots, k \in T$, if $k+1 \notin T$, then $k+1 \in S$ and $k+1$ becomes the smallest element of $S$ and contradicts our assumption above. \\
		Therefore $1, 2, 3, \dots k \in T \implies k+1 \in T$.\\ 
		By principle of strong induction, $T = \mathbb{N}$. \\
		Thus, $S = \emptyset$, and contradicts our definition of $S$. \\
		Therefore $\forall S \subseteq \mathbb{N}\ s.t.\ S \neq \emptyset$, $S$ has the smallest element (Well-Ordering Principle). 
	\end{proof}
	
	\begin{example}[Application 2]
		Is $2^{29} + 3$ divisible by 7? \\ 
		\begin{proof}[Solution] Notice $2^2 \equiv 4 \mod 7$ and $2^3 \equiv 1 \mod 7$. \\
		$\implies (2^3)^9 \equiv 1^9 \mod 7$ \\
		$\implies 2^{27} \equiv 1 \mod 7 $ \\
		$\implies 2^{29} \equiv 4 \mod 7 $ \\
		Also $3 \equiv 3 \mod 7$ \\
		$\implies 2^{29} + 3 \equiv 4 + 3 \mod 7$ \\
		$\implies 2^{29} + 3 \equiv 7 \mod 7$ \\
		$\implies 7 | 2^{29}+3$.
		\end{proof}
	\end{example}
	
	\begin{theorem}[Rules on computing congruence ]
		Let $a, b, c, d \in \mathbb{Z}$ and $m \in \mathbb{N}$.
		\begin{enumerate}
			\item $a \equiv b \mod m \land c \equiv d \mod m \implies a + c \equiv b + d \mod m$
			\item $a \equiv b \mod m \land c \equiv d \mod m \implies ac \equiv bd \mod m$
		\end{enumerate}
	\end{theorem}
	
	\begin{proof}
		Let $a, b, c, d \in \mathbb{Z}$ and $m \in \mathbb{N}$, \\
		suppose $a \equiv b \mod m \land c \equiv d \mod m$ \\
		by definition of congruence, $\exists p, q\in \mathbb{Z}\ s.t.\ (a-b) = pm \land (c-d) = qm$\\
		$\implies (a+c-b-d) = (p+q)m,\ (p+q) \in \mathbb{Z}$ \\
		$\implies a + c \equiv b + d \mod m$ \\
		And $a = b + pm \land c = d + qm$ \\
		$ac - bd = (b + pm)(d+qm) - bd$ \\
		$=bd +dpm + qbm + pqm^2 - bd$ \\
		$=(dp+qb+pqm)m$ \\
		$\implies m | ac-bd$ \\
		$\implies ac \equiv bd \mod m$ 
	\end{proof}
	
	\begin{proposition}[Corollary from theorem 5.1]
		\[
		a \equiv b \mod m \implies a + c \equiv b + c \mod m 
		\] and
		\[
			a \equiv b \mod m \implies a^k \equiv b^k \mod m,\ \forall k \in \mathbb{Z}_{\geq 0}
		\]
	\end{proposition}
	
	\section{Lecture 6 Sep. 19 2018}
	
	\begin{theorem}
		Let $a,b \in \mathbb{Z}$,
		\[
			a = b \implies a \equiv b \mod m\ \forall m \in \mathbb{N}
		\]
	\end{theorem}
	
	\begin{example}
		What is the reminder when $3^{202}+5^9$ is divided by 8		
		\begin{proof}[Solution]
			Notice $3^2 \equiv 1 \mod 8$ \\
			Therefore, $(3^2)^{101} \equiv 1^{101} \mod 8$ \\
			That's, $3^{202} \equiv 1 \mod 8$ \\
			Also $5^2 \equiv 1 \mod 8$ \\
			$\implies (5^2)^4 \equiv 1^4 \mod 8$ \\
			$\implies 5^9 \equiv 5 \mod 8$ \\
			$\implies 3^{202} + 5^9 \equiv 5 + 1 \mod 8$ \\
			$\implies$ the reminder is $6$. \\
			(Notice that $3^{202} + 5^9 \equiv 6 \equiv 14 \equiv 22 \equiv \dots \mod 8 $, and the reminder is the smallest integer satisfying above relation.)
		\end{proof}
	\end{example}
	
	\begin{theorem}
		Let $M \in \mathbb{Z}$ and $M = d_N \dots d_2 d_1 d_0,\ d_i \in \{0,1,\dots,9\}$ \footnote{This means the integer $M$ is constructed from digits $d_i$. For example, $M=256$, then $d_0=6, d_1=5,d_2=2$}, then
		\[
			3 | M \iff 3\ |\ \sum_{i=0}^N{d_i}
		\]
	\end{theorem}
	
	\begin{proof}
		Notice $10 \equiv 1 \mod 3$, $100 \equiv 1 \mod 3$ and so on, \\
		(Fact) $10^k \equiv 1 \mod 3,\ \forall k \in \mathbb{Z}_{\geq 0}$ \\
		Then $d_i 10^i \equiv d_i \mod 3,\ \forall i$ \\
		Therefore, $\sum_{i=0}^N{10^i d_i} \equiv \sum_{i=0}^N{d_i} \mod 3$ \\
		Therefore $\sum_{i=0}^N{10^i d_i} \equiv 0 \mod 3 \iff \sum_{i=0}^N{d_i} \equiv 0 \mod 3$ \\
	\end{proof}
		
	\begin{theorem}
		Let $M \in \mathbb{Z}$ and $M = d_N \dots d_2 d_1 d_0,\ d_i \in \{0,1,\dots,9\}$, then 
		\[
			11 | M \iff 11\ |\ \sum_{i=0}^N{(-1)^i d_i}
		\]
	\end{theorem}
	
	\begin{proof}
		Notice $10^i \equiv (-1)^i \mod 11$ \\
		Therefore $10^i d_i \equiv (-1)^i d_i$ \\
		Thus, $\sum_{i=0}^N{10^i d_i} \equiv \sum_{i=0}^N{(-1)^i d_i} \mod 11$ \\
		Then, $\sum_{i=0}^N{10^i d_i} \equiv 0 \mod 11 \iff \sum_{i=0}^N{(-1)^i d_i} \equiv 0 \mod 11$ \\
	\end{proof}
	
	
\end{document}



























