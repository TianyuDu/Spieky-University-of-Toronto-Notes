\documentclass[11pt]{article}

\author{Tianyu Du}
\date{\today}
\title{MAT246: Concepts in Abstract Mathematics: \\ \small Lecture 0101 Notes}

\usepackage{amsmath}
\usepackage{amssymb}
\usepackage{float}
\usepackage{soul}
\usepackage{spikey}
\usepackage{xcolor}
\usepackage{centernot}
\usepackage{txfonts}

\usepackage[
    type={CC},
    modifier={by-nc},
    version={4.0},
]{doclicense}

\begin{document}
	\maketitle
	\doclicenseThis
	
	\tableofcontents
	\newpage
	\section{Lecture 1 Sep. 7 2018}
	
	\begin{definition}
		Let $\mathbb{N} := \{1, 2, 3, \dots\}$ be the set of \textbf{natural numbers}.
	\end{definition}
	
	\begin{theorem}[Principle of Mathematical Induction]
		Suppose $S$ is a set of natural numbers, $S \subseteq \mathbb{N}$. If
		\begin{enumerate}
			\item $1 \in S$
			\item $k \in S \implies k + 1 \in S,\ \forall k \in \mathbb{N}$
		\end{enumerate}
		then, $S = \mathbb{N}$
	\end{theorem}
	
	\begin{example}
		Show that 
		\[
			1^2 + 2^2 + \dots + n^2 = \frac{n(n+1)(2n+1)}{6}\ \forall n \in \mathbb{N}
		\]
		\begin{proof}
			
		\end{proof}
	\end{example}
	
	\section{Lecture 2 Sep. 10 2018}
	
	\begin{theorem}[Extended Principle of Mathematical Induction]
		Suppose set $S \subseteq \mathbb{N}$ and let $n_0 \in \mathbb{N}$ fixed, if
		\begin{enumerate}
			\item $n_0 \in S$
			\item $\forall k \geq n_0, k \in S \implies k + 1 \in S$
		\end{enumerate}
		then $\{n_0, n_0 + 1, n_0 + 2, \dots \}\ \textcolor{red}{\subseteq}\ S$
	\end{theorem}
	
	\begin{example}
		Show that 
		\[
			n! \geq 3^n\ \forall n \geq 7
		\]
		\begin{proof}
			
		\end{proof}
	\end{example}
	
	\begin{theorem}[Well-Ordering Principle]
		Every non-empty subset of natural number has a smallest element.
	\end{theorem}

	\begin{proof} (Principle of Mathematical Induction)\\
		Let $S \subseteq \mathbb{N}$ \\
		Suppose $1 \in S \land (k \in S \implies k+1 \in S, \forall k \in \mathbb{N})$ \\
		Show: $S = \mathbb{N}$ \\
		Let $T = \mathbb{N} \backslash S$ \\
		Suppose $T \neq \emptyset$ \\
		By Well-Ordering Principle, there exists a smallest element of T, denoted as $t_0 \in \mathbb{N}$. \\
		Since $1 \in S$, therefore $t_0 \neq 1$. \\
		Therefore $t_0 > 2$. \\
		Thus $t_0 - 1 \in \mathbb{N}$ and since $t_0 = \min{T}$, $t_0 - 1 \notin T$ \\
		Therefore $t_0 - 1 \in S$, then, $t_0 - 1 + 1 = t_0 \in S$, \\
		Contradict the assumption that $t_0 \in T$. \\
		Thus $T = \emptyset$ and $S = \mathbb{N}$. \\
	\end{proof}
	
	\begin{remark}
		We can use principle of Mathematical Induction to prove Well-Ordering Principle as well.
	\end{remark}
	
	\section{Lecture 3 Sep. 12 2018}
	\begin{definition}
		Let $a, b \in \mathbb{N}$ and $a$ \textbf{divides} $b$, written as $a | b$ if 
		\[
			\exists c \in \mathbb{N}\ s.t.\ b = ac
		\]
		And $a$ is a \textbf{divisor} of $b$.
	\end{definition}
	
	\begin{definition}
		A natural number $p$ (except $1$) is called \textbf{prime} if the only divisors of $p$ are $1$ and $p$.
	\end{definition}
	
	\begin{lemma}[Prime numbers are building blocks of natural numbers]
		Every natural number other than $1$ is a \emph{product}\footnote{Product could mean the product of a single number.} of prime numbers.
	\end{lemma}
	
	\begin{theorem}[Principle of Complete Induction]
		Suppose $S \subseteq \mathbb{N}$ and if 
		\begin{enumerate}
			\item $n_0 \in S$
			\item $n_0, n_0 + 1, \dots, k \in S \implies k+1 \in S,\ \forall k \geq n_0$
		\end{enumerate}
		then 
		\[
			\{n_0, n_0 + 1, \dots\} \subseteq S
		\]
	\end{theorem}
	\begin{proof} [Proof of Lemma]
		Let $S \subseteq \mathbb{N}$ for which the lemma is true, \\
		Want to show: $S = \mathbb{N} \backslash \{1\}$ \\
		(Base Case) For $2$ it's a product of prime. Thus $2 \in S$ \\
		(Inductive Step) Suppose $\{2,3, \dots k\} \subseteq S$ \\
		Consider $k+1$, if $k+1$ is a prime then $k+1$ can be written as a product of itself, as a product of one single prime.\\
		Else, if $k+1$ is not a prime, then $\exists 1<m,n<k+1\ s.t.\ k+1 = mn$.\\
		By induction hypothesis of strong induction, $m,n$ can both be written as product of primes. \\
		$m = \prod_{i=1}^\ell {p_i},\ n = \prod_{i=1}^t {q_i}$ where $p_i, q_i$ are all primes.\\
		and $k+1 = \prod_{i=1}^t{q_i}\prod_{i=1}^\ell {p_i}$ \\
		thus $k+1 \in S$ \\ 
		by principle of strong induction, $\{2,3,\dots,\} \subseteq S$.
	\end{proof}
	
	\begin{theorem}
		There is no largest prime number.
	\end{theorem}
	
	\begin{proof}
		(By contradiction) \\
		Assume there is a largest prime $p$, \\
		then $\{2,3,5,\cdots,p\}$ is the set of all primes \\
		Let $M := (2*3*5*\cdots*p)+1 \in \mathbb{N}$ \\
		$M$ is either prime or not. \\
		Suppose $M$ is not a prime, then by Lemma 3.1, $\exists p' $ dividing $M$. \\
		Obviously $\forall i \in \{2*3*5*\cdots*p\},\ i \centernot | M$. \\
		There is no prime dividing $M$, which contradict Lemma 3.1 \\
		Thus $M$ is a prime, and $M > p$, which contradicts assumption \\
		Therefore there is no largest prime.
	\end{proof}
	
	\section{Lecture 4 Sep. 14 2018}
	\begin{theorem}[the Fundamental Theorem of Arithmetic]
		Every natural (except 1) is a product of prime(s), and the prime(s) in the product are unique including multiplicity except for the order.
	\end{theorem}
	\begin{proof}
		We have already proven that the existential parts of this theorem in Lemma 3.1. \\
		(Proof for the uniqueness part) Suppose there exists natural number (not 1) has 2 different prime factorizations. \\
		By well ordering principle, there is a smallest $n$, which has two distinct prime factorizations. \\
		Say $n = p_1 p_2 \dots p_k = q_1 q_2 \dots q_\ell $ where $p_i, q_i$ are all primes. \\
		Notice that $p_i \neq q_j$ for any combination of $(i,j)$ since if so $\frac{n}{p_i}=\frac{n}{q_j}$ is a natural number smaller than $n$ having 2 distinct prime factorization, which contradicts our assumption above. \\
		Specifically, $p_1 \neq q_1$. \\
		(Case 1: $p_1 < q_1$) \\
		Let $m := n - p_1 q_2 \dots q_\ell \in \mathbb{N}$ \\
		Notice $m = p_1 (p_2 p_3 \dots p_k - q_2 q_3 \dots q_\ell)$ \\
		Also $m = (q_1 - p_1) (q_2 q_3 \dots q_\ell)$ \\
		$\implies m = p_1 \dots p_k = q_2 q_3 \dots q_\ell (q_1 - p_1)$ \\
		$\implies p_1 | m $ also notices that $p_1 \centernot | q_2 q_3 \dots q_\ell$ \\
		$\implies p_1 |  (q_1 - p_1) \implies p_1 | q_1 \implies p_1 = q_1$ \\
		Contradicts the assumption that $p_q < q_1$ \\
		The other case goes a similar proof.
	\end{proof}
	
	\begin{definition}
		A natural number $n$ is called \textbf{composite} if it's not 1 or a prime number.
	\end{definition}
	
	\begin{remark}
		Natural numbers are partitioned into 3 categories, 1, prime and composite numbers.
	\end{remark}
	
	\begin{example}
		Find 20 consecutive composite numbers.
		\[
		(21!)+2, (21!)+3, \dots, (21!)+21
		\]
	\end{example}
	
	\begin{example}
		Find $k$ consecutive composite numbers.
		\[
			(k+1!)+2, (k+1)!+3, \dots ,(k+1!)+k+1
		\]
	\end{example}
	
	\section{Lecture 5 Sep. 17 2018}
	
	\begin{definition}
		Let $a, b \in \mathbb{Z}$, and let $m \in \mathbb{N}$. If $m | a - b$ then we say "$a$ and $b$ are congruent modulo $m$"
	\end{definition}
	
	\begin{remark}
		Regular Induction $\iff$ Complete Induction $\iff$ Well-Ordering Principle
	\end{remark}
	
	\begin{proof}
		(WTS: Complete Induction $\implies$ Well-Ordering Principle) \\
		Let $S \subseteq \mathbb{N}$ and $S \neq \emptyset$ \\
		(WTS, $S$ has the smallest element) \\
		Assume $S$ does not have the smallest element. \\
		Let $T := S^c$ \\
		Clearly $1 \in T$ (prop 1) \\
		Since other wise 1 could be the smallest element of $S$. \\
		Let $k \in \mathbb{N}$. \\
		Suppose $1, 2, 3, \dots, k \in T$, if $k+1 \notin T$, then $k+1 \in S$ and $k+1$ becomes the smallest element of $S$ and contradicts our assumption above. \\
		Therefore $1, 2, 3, \dots k \in T \implies k+1 \in T$.\\ 
		By principle of strong induction, $T = \mathbb{N}$. \\
		Thus, $S = \emptyset$, and contradicts our definition of $S$. \\
		Therefore $\forall S \subseteq \mathbb{N}\ s.t.\ S \neq \emptyset$, $S$ has the smallest element (Well-Ordering Principle). 
	\end{proof}
	
	\begin{example}[Application 2]
		Is $2^{29} + 3$ divisible by 7? \\ 
		\begin{proof}[Solution] Notice $2^2 \equiv 4 \mod 7$ and $2^3 \equiv 1 \mod 7$. \\
		$\implies (2^3)^9 \equiv 1^9 \mod 7$ \\
		$\implies 2^{27} \equiv 1 \mod 7 $ \\
		$\implies 2^{29} \equiv 4 \mod 7 $ \\
		Also $3 \equiv 3 \mod 7$ \\
		$\implies 2^{29} + 3 \equiv 4 + 3 \mod 7$ \\
		$\implies 2^{29} + 3 \equiv 7 \mod 7$ \\
		$\implies 7 | 2^{29}+3$.
		\end{proof}
	\end{example}
	
	\begin{theorem}[Rules on computing congruence ]
		Let $a, b, c, d \in \mathbb{Z}$ and $m \in \mathbb{N}$.
		\begin{enumerate}
			\item $a \equiv b \mod m \land c \equiv d \mod m \implies a + c \equiv b + d \mod m$
			\item $a \equiv b \mod m \land c \equiv d \mod m \implies ac \equiv bd \mod m$
		\end{enumerate}
	\end{theorem}
	
	\begin{proof}
		Let $a, b, c, d \in \mathbb{Z}$ and $m \in \mathbb{N}$, \\
		suppose $a \equiv b \mod m \land c \equiv d \mod m$ \\
		by definition of congruence, $\exists p, q\in \mathbb{Z}\ s.t.\ (a-b) = pm \land (c-d) = qm$\\
		$\implies (a+c-b-d) = (p+q)m,\ (p+q) \in \mathbb{Z}$ \\
		$\implies a + c \equiv b + d \mod m$ \\
		And $a = b + pm \land c = d + qm$ \\
		$ac - bd = (b + pm)(d+qm) - bd$ \\
		$=bd +dpm + qbm + pqm^2 - bd$ \\
		$=(dp+qb+pqm)m$ \\
		$\implies m | ac-bd$ \\
		$\implies ac \equiv bd \mod m$ 
	\end{proof}
	
	\begin{proposition}[Corollary from theorem 5.1]
		\[
		a \equiv b \mod m \implies a + c \equiv b + c \mod m 
		\] and
		\[
			a \equiv b \mod m \implies a^k \equiv b^k \mod m,\ \forall k \in \mathbb{Z}_{\geq 0}
		\]
	\end{proposition}
	
	\section{Lecture 6 Sep. 19 2018}
	
	\begin{theorem}
		Let $a,b \in \mathbb{Z}$,
		\[
			a = b \implies a \equiv b \mod m\ \forall m \in \mathbb{N}
		\]
	\end{theorem}
	
	\begin{example}
		What is the reminder when $3^{202}+5^9$ is divided by 8		
		\begin{proof}[Solution]
			Notice $3^2 \equiv 1 \mod 8$ \\
			Therefore, $(3^2)^{101} \equiv 1^{101} \mod 8$ \\
			That's, $3^{202} \equiv 1 \mod 8$ \\
			Also $5^2 \equiv 1 \mod 8$ \\
			$\implies (5^2)^4 \equiv 1^4 \mod 8$ \\
			$\implies 5^9 \equiv 5 \mod 8$ \\
			$\implies 3^{202} + 5^9 \equiv 5 + 1 \mod 8$ \\
			$\implies$ the reminder is $6$. \\
			(Notice that $3^{202} + 5^9 \equiv 6 \equiv 14 \equiv 22 \equiv \dots \mod 8 $, and the reminder is the smallest integer satisfying above relation.)
		\end{proof}
	\end{example}
	
	\begin{theorem}
		Let $M \in \mathbb{Z}$ and $M = d_N \dots d_2 d_1 d_0,\ d_i \in \{0,1,\dots,9\}$ \footnote{This means the integer $M$ is constructed from digits $d_i$. For example, $M=256$, then $d_0=6, d_1=5,d_2=2$}, then
		\[
			3 | M \iff 3\ |\ \sum_{i=0}^N{d_i}
		\]
	\end{theorem}
	
	\begin{proof}
		Notice $10 \equiv 1 \mod 3$, $100 \equiv 1 \mod 3$ and so on, \\
		(Fact) $10^k \equiv 1 \mod 3,\ \forall k \in \mathbb{Z}_{\geq 0}$ \\
		Then $d_i 10^i \equiv d_i \mod 3,\ \forall i$ \\
		Therefore, $\sum_{i=0}^N{10^i d_i} \equiv \sum_{i=0}^N{d_i} \mod 3$ \\
		Therefore $\sum_{i=0}^N{10^i d_i} \equiv 0 \mod 3 \iff \sum_{i=0}^N{d_i} \equiv 0 \mod 3$ \\
	\end{proof}
		
	\begin{theorem}
		Let $M \in \mathbb{Z}$ and $M = d_N \dots d_2 d_1 d_0,\ d_i \in \{0,1,\dots,9\}$, then 
		\[
			11 | M \iff 11\ |\ \sum_{i=0}^N{(-1)^i d_i}
		\]
	\end{theorem}
	
	\begin{proof}
		Notice $10^i \equiv (-1)^i \mod 11$ \\
		Therefore $10^i d_i \equiv (-1)^i d_i$ \\
		Thus, $\sum_{i=0}^N{10^i d_i} \equiv \sum_{i=0}^N{(-1)^i d_i} \mod 11$ \\
		Then, $\sum_{i=0}^N{10^i d_i} \equiv 0 \mod 11 \iff \sum_{i=0}^N{(-1)^i d_i} \equiv 0 \mod 11$ \\
	\end{proof}
	
	\section{Lecture 7 Sep. 21 2018}
	\begin{theorem}
		Suppose $p$ is a prime and $a,b \in \mathbb{N}$, if $p|ab$ then $p | a \lor p | b$.
	\end{theorem}
	\begin{proof}
		If $a = 1 \lor b = 1$, then done. And for the case $a = b = 1$, the proposition is vacuously true.\\
		Let $a, b > 1$, \\
		By the fundamental theorem of arithmetic, we can write $a, b $ as their unique prime factorization \\
		$a = p_1^{\alpha_1} \dots p_k^{\alpha_k},\ \alpha_j \geq 1$ and $b = q_1^{\beta_1}\dots q_\ell^{\beta_\ell},\ \beta_j \geq 1$\\
		then $ab = p_1^{\alpha_1} \dots p_k^{\alpha_k}q_1^{\beta_1}\dots q_\ell^{\beta_\ell}$ is the unique prime factorization of $ab$. \\
		Since $p \in \mathbb{P}$, therefore, $p=p_j \lor p = q_j$ 
		$\implies p | a \lor p | b$ \\
	\end{proof}
	
	\begin{remark}
		We have shown that $a \equiv b \mod m \implies ca \equiv cb \mod m$. But notice that 
		\[
			ca \equiv cb \mod m \centernot \implies a \equiv b \mod m
		\]
	\end{remark}
	
	\begin{definition}
		Let $a,b \in \mathbb{Z}$, then we say $a$ and $b$ are \textbf{relatively prime} if they have no prime factor in common.
	\end{definition}
	
	\begin{theorem}
		Suppose \ul{$p$ is a prime} and $a \in \mathbb{Z}$ and $p \centernot | a$, then $ax \equiv ay \mod p \implies x \equiv y \mod p$.
	\end{theorem}
	
	\begin{proof}
		Let $x,y,a \in \mathbb{N}$ and $p \in \mathbb{P}$. \\
		Suppose $ax \equiv ay \mod p$ \\
		Then $p | a(x-y)$ \\
		By theorem 7.1, $p | a \lor p | (x-y)$\\
		But by our assumption, $p \centernot | a$, therefore $p | (x-y)$ \\
		Thus $x \equiv y \mod p$ \\
	\end{proof}
	
	\begin{theorem}[Generalization of Theorem 7.2]
		Let $m \in \mathbb{N}$ and $a \in \mathbb{Z}$ and $a$ and $m$ are relatively prime. Then 
		\[
			ax \equiv ay \mod m \implies x \equiv y \mod m
		\]
	\end{theorem}
	
	\begin{proof}
		Suppose $ax \equiv ay \mod m$ \\
		Then $m | a(x-y)$ \\
		Therefore $m | a \lor m | (x - y)$ \\
		For $m$ to divide $a$, all of $m's$ prime factors have to be in the prime factorization of $|a|$. \\
		But $m$ and $a$ are relatively prime, therefore $m \centernot | a$. \\
		Therefore $m | (x-y)$ and that's $x \equiv y \mod m$\\
	\end{proof}
	
	\begin{theorem}
		Any integer $a$ is congruent to mod $m$ to exactly one of $\{0, 1, \dots, m-1\}$.
	\end{theorem}
	
	\begin{theorem}[Fermat's Little Theorem]
		If $p$ is a prime and $p \centernot | a$ (i.e. $a$ and $p$ are relatively prime), then
		\[
			a^{p-1} \equiv 1 \mod p
		\]
	\end{theorem}
	
	\begin{proof}
		Let $S := \{a1, a2, \dots a(p-1)\}$ \\
		Notice that if $ax_i \equiv ax_j \mod p$, since $p \centernot | a$, $x_1 \equiv x_2 \mod p$. \\
		Since $1 \leq x_i, x_j \leq p-1$, then $x_i = x_j$. \\
		Therefore all elements in $S$ are distinct with mod $p$ \\
		i.e. $x_i \centernot \equiv x_j \mod p,\ \forall (i,j) \in \mathbb{Z}^2$. \\
		Since $p \centernot | a \land p \centernot | m,\ \forall m \in \{1, 2, \dots, (p-1)\}$ \\
		So no element in $S$ is congruent to $0 \mod p$. \\
		Thus, $S$ contains $p-1$ numbers and no two of them are congruent mod $p$. \\
		Also none of them are congruent to $0 \mod p$. \\
		By theorem 7.4, each element in $S$ is congruent to one corresponding element in set $\{1,2,\dots, p-1\}$. \\
		Therefore $(a1)(a2)\dots(a(p-1)) \equiv 1*2*\dots*(p-1) \mod p$ \\
		That's $a^{p-1}(1*2* \dots *(p-1)) \equiv 1*2*\dots*(p-1) \mod p$ \\
		Clearly $p \centernot | (1*2*\dots (p-1))$, since if a prime divides a product of natural numbers, the prime must divide at least one of elements in the product. \\
		Therefore $a^{p-1} \equiv 1 \mod p$ \\
	\end{proof}
	
	\section{Lecture 8 Sep. 24 2018}
	\begin{definition}
		Let $p \in \mathbb{N}$ and $a \in \mathbb{Z}$. The \textbf{multiplicative inverse} $\mod p$ of $a$ is an integer $b$ such that
		\[
			ab \equiv 1 \mod p
		\]
	\end{definition}
	\begin{remark}
		Notice that the multiplicative inverse is generally not unique but unique up to $\mod p$.		
	\end{remark}
	
	\begin{corollary}
		Let $p \in \mathbb{P}$, $a \in \mathbb{N}$ and $p \centernot | a$. Then 
		\[
			\exists b \in \mathbb{Z},\ s.t.\ ba \equiv 1 \mod p
		\]
	\end{corollary}
	
	\begin{proof}
		Let $p \in \mathbb{Z}$ and $a \in \mathbb{Z}$ \\
		Suppose $p \centernot | a$, then by Fermat's little theorem, \\
		$a^{p-1} \equiv 1 \mod p \implies a^{p-2}a \equiv 1 \mod p$ \\
		Take $b = a^{p-2} \in \mathbb{Z}$ and $ab \equiv 1 \mod p$
	\end{proof}
	
	\begin{example}
		Let $a=8$ and $p=5$. Obviously $p \centernot | a$. By corollary above, 
		\[
			\exists b \in \mathbb{Z},\ s.t.\ 8b \equiv 1 \mod 5
		\]
		Notice $b=2$ satisfies above equation.
	\end{example}
	
	\begin{remark}
		Corollary 8.1 requires $p$ to be a prime.
	\end{remark}
	
	\begin{corollary}[Generalization]
		Let $a$ and $m \in \mathbb{N}$ and $a$ and $m$ are \ul{relatively prime}, then 
		\[
			\exists b \in \mathbb{Z},\ s.t.\ ab \equiv 1 \mod m
		\]
	\end{corollary}
	
	\begin{theorem}[Wilsons' Theorem]
		Let $p \in \mathbb{P}$ then
		\[
			(p-1)! \equiv -1 \mod p 
		\]
	\end{theorem}
	
	\begin{proof}
		Let $p \in \mathbb{P}$ \\
		if $p = 2 \lor p = 3$, then $1! \equiv -1 \mod 2$ and $2! \equiv -1 \mod 3$. \\
		Otherwise, suppose $p > 3$, \\
		Consider, let $S := \{2, 3, 4, \dots, p-2\}$ \\
		Notice that none of $S$ is divisible by $p$. \\
		Therefore $p$ is relatively prime to all elements in $S$. \\
		Then by Corollary 8.1, $\exists b_i \in \mathbb{Z}\ s.t.\ b_i s_i \equiv 1 \mod p,\ \forall s_i \in S$. \\
		Notice that $0$ has no multiplicative inverse and 
		\[
			(p-1)(p-1)=p^2 - 2p + 1 \equiv 1 \mod p
		\]
		That's, $1$ and $(p-1)$ have themselves as their multiplicative inverse. \\
		Also notice that for any $s_i \in S$, $s_i$ does not have itself as its multiplicative inverse. \\
		If $a \in S$ has itself as it's multiplicative inverse, then
		\begin{gather*}
			a^2 \equiv 1 \mod p\\
			\implies a^2 - 1 \equiv 0 \mod p \\
			\implies (a+1)(a-1) \equiv 0 \mod p \\ 
			\implies p | (a+1)(a-1)
		\end{gather*}
		Notice that at last one of $(a+1)$ and $(a-1)$ is in set $S$ since $p > 3 \implies S \neq \emptyset$. This contradicts what we argued above, \emph{none of $S$ is divisible by $p$}.\\
		That's \[s_i s_i \centernot \equiv 1 \mod p,\ \forall s_i \in S\]
		\emph{Note that if $y$ is a multiplicative inverse of $x$, then $x$ is a multiplicative inverse of $y$.} \\
		Notice that for any $s_i \in S$, by Corollary 8.1, \\
		there exists an integer $b_i$ s.t. $s_i b_i \equiv 1 \mod p$\\
		And the multiplicative inverse is unique up to $\mod p$, \\
		Thus $s_i(b_i \mod p) \equiv 1 \mod p$ and $(b_i \mod p) \in S$.\\
		And for all elements in $S$ has one of their multiplicative inverse in $S$,
		
		That's
		\[
			s_i s_j \equiv 1 \mod p,\ i \neq j
		\]
		Notice $p > 3$ implies $p$ is odd, so $|S|$ is even. \\
		Match every pair of multiplicative inverses in $S$ and they collapse to 1 $\mod p$ \\
		Therefore 
		\begin{gather*}
			2\cdot 3 \cdot 4 \cdots (p-2) \equiv 1 \mod p \\
			\implies 2\cdot 3 \cdot 4 \cdots (p-2) \cdot (p-1) \equiv (p - 1) \mod p \\
			\implies (p-1)! \equiv -1 \mod p
		\end{gather*}
	\end{proof}
	
	\section{Lecture 9 Sep. 26 2018}
	\begin{remark}
		Recall that an integer $n$ is even iff $n \equiv 0 \mod 2$ and is odd iff $n \equiv 1 \mod 2$.
	\end{remark}
	
	\begin{theorem}
		There are infinitely many primes of the form $4k+3$, where $k \in \mathbb{Z}$.
	\end{theorem}
	\begin{proof}
		Note that odd numbers $n$ can be classified as $n \equiv 1 \mod 4$ and $n \equiv 3 \equiv -1\mod 4$ \\
		(Suppose 1) there are only finitely many primes in the form $4k+3$. \\
		Let finite set $S := \{p_1, p_2, \dots p_m\}$ denotes the collection of them. \\
		And notice that $p_i \equiv -1 \mod 4,\ \forall p_i \in S$. \\
		Let 
		\[
			M := (p_1 \cdot p_2 \cdots p_m)^2 + 2
		\]
		and $M \equiv 1 + 2 \equiv 3 \equiv -1 \mod 4$. \\
		Therefore $M$ is an odd natural number. \\
		By the Fundamental Theorem of Arithmetic, $M$ can be factorized into product of primes.\\
		\[
			M = \prod_{i=1}^{\ell} q_i
		\]
		and since $M$ is odd, $q_i \neq 2\ \forall \ i$. Thus all $q_i$ are odd.\\
		(Suppose 2) All $q_i \equiv 1 \mod 4$. \\
		Then $M \equiv 1 \mod 4$. \\
		Contradict the fact that $M \equiv -1 \mod 4$. Thus (Suppose 2) is false. \\
		Therefore $\exists i,\ s.t.\ q_i \equiv -1 \mod 4$. \\
		From (Suppose 1), $S$ is the collection of all primes that $\equiv -1 \mod 4$. \\
		Therefore $q_i = p_j$ for some $j$. \\
		Therefore $p_j | M$.\\
		Also note that $p_j | (p_1 \cdot p_2 \cdots p_m) \implies p_j | (p_1 \cdot p_2 \cdots p_m)^2$ \\
		$\implies p_j | 2 \implies p_j = 2$ contradicts the fact that $p_j$ is odd. \\
		Therefore (Suppose 1) is false, there are infinitely many primes taking the form $4k+3$. \\
	\end{proof}
	
	\begin{example}
		Find $7^{20^{30}} \mod 5$.
		\begin{proof}[Solution]
			Let $n := 20^{30}$. \\
			Notice that $7^4 \equiv 1 \mod 5$. \\
			And if $n \equiv r \mod 4$ where $r \in \mathbb{Z}$, \\
			$n = 4k+r$ and $7^n \equiv 7^{4k+r} \equiv (7^4)^k \times 7^r \equiv 1^k \times 7^r \equiv 7^r \mod 5$. \\
			Notice that $20 \equiv 0 \mod 4 \implies 20^{30} \equiv 0 \mod 4$. \\
			Thus $r = 0$. \\
			Therefore $7^n \equiv 7^0 \equiv 1 \mod 5$. \\
			Thus $7^{20^{30}} \mod 5 = 1$.
		\end{proof}
	\end{example}
	
	\begin{example}
		Find $10^{3^{30}} \mod 7$.
		\begin{proof}[Solution]
			Notice that $10^6 \equiv 1 \mod 7$. \\
			And $3 \equiv 3 \mod 6, 3^2 \equiv 3 \mod 6, 3^3 \equiv 3 \mod 6 \dots $ \\
			Using induction, we can show that 
			\[
				3^k \equiv 3 \mod 6,\ \forall k \in \mathbb{Z}_{\geq 0}
			\]
			Therefore $3^{30} \equiv 3 \mod 6$. \\
			That's $3^{30} = 6k + 3$ for some $k$. \\
			Thus $10^{3^{30}} \equiv (10^6)^k \times 10^3 \equiv (1)^k \times 10^3 \equiv -1 \equiv 6 \mod 7$. \\
			So $10^{3^{30}} \mod 7 = 6$. \\
		\end{proof}
	\end{example}
	
	\section{Lecture 10 Sep. 28 2018}
	\begin{example}
		Find $8^{9^{10^{11}}} \mod 5$.
	\end{example}
	\begin{proof}[Solution]
		Let $n := 9^{10^{11}}$ \\
		And notices that $8^4 \equiv 1 \mod 5$. \\
		Then find $n \mod 4$ \\
		Note that $9 \equiv 1 \mod 4 \implies 9^{10^{11}} \equiv 1 \mod 4$. \\
		Thus $n = 4k + 1$. \\
		Therefore $8^{9^{{10}^{11}}} \equiv (8^4)^k \cdot 8 \equiv 1 \cdot 3 \mod 5$. \\
		That's $8^{9^{10^{11}}} \mod 5 = 3$. \\
	\end{proof}
	
	\begin{definition}[Euler $\phi$-function] Let $m \in \mathbb{N}$ and $\phi(m): \mathbb{N} \to \mathbb{N}$ is defined as \emph{the number of elements in $\{1, 2, \dots, m-1\}$ that are relatively prime to $m$}.	
	\end{definition}
	
	\begin{example}
		For $m = 8$, note that $\{1, 3, 5, 7\} \subset \{1,2,\dots,7\}$ are relatively prime with 8, therefore $\phi(8) = 4$. \\
		And for $m=11$, since $m$ is a prime, then every integer between $1$ and $m-1$ are relatively prime with $11$. Therefore $\phi(11)=10$. \\
		And notice that $\phi(p) = p-1$ if $p \in \mathbb{P}$. (Fermat's Little Theorem)\\
	\end{example}
	
	\begin{proposition}
		Let $p, q$ be two distinct primes, then \[ \phi(pq) = (p-1)(q-1) \]
	\end{proposition}
	\begin{proof}
		Let $S:= \{1, 2, \dots, pq - 1\}$. \\
		WLOG, assume $p < q$. \\
		We need find all elements in $S$ that with either $p$ or $q$ in their prime factorization to find elements in $S$ that are \ul{not} relatively prime to $pq$. \\
		And those elements are multiples of $p$ and multiples of $q$. \\
		And since $pq \notin S$, the largest multiple of $p$ in $S$ is $(q-1)p$ and the largest multiple of $q$ in $S$ is $q(p-1)$. \\
		And since there is no multiple of \ul{both} $p$ and $q$ in set $S$, therefore there's no overlapping between multiples of $p$ and multiples of $q$.\\
		Therefore exists $(p-1)+(q-1)$ elements that are \ul{not} relatively. prime to $pq$. \\
		Therefore $\phi(pq) = (pq-1)-(p-1)-(q-1)$ \\
		$=pq-p-q+1$ \\
		$=(p-1)(q-1)$
	\end{proof}
	
	\begin{proposition}
		For any natural number $m \in \mathbb{N}$. Therefore $m$ can be expressed as 
		\[
			m = p_1^{\alpha_1} p_2^{\alpha_2} \cdots p_k^{\alpha_k}
		\]
		Then 
		\[
			\phi(m) = \phi(p_1^{\alpha_1}) \phi(p_2^{\alpha_2}) \cdots \phi(p_k^{\alpha_k})
		\]
		And 
		\[
			\phi(p^\alpha) = p^\alpha - p^{\alpha - 1}= p^{\alpha - 1} (p - 1)
		\]
		Therefore
		\[
			\phi(m) = (p_1^{\alpha_1} - p_1^{\alpha_1 - 1})(p_2^{\alpha_2} - p_2^{\alpha_2 - 1}) \cdots (p_k^{\alpha_k} - p_k^{\alpha_k - 1})
		\]
	\end{proposition}
	
	\begin{example}
		\begin{gather*}
			\phi(6) = \phi(2^1 3^1) \\
			= \phi(2^1) \phi(3^1) \\
			= (2^1 - 2^0) (3^1 - 3^0) \\
			= (2 - 1)(3 - 1)
			= 2 \\
		\end{gather*}
	\end{example}
	\begin{example}
		\begin{gather*}
			\phi(8) = \phi(2^3)\\
			= (2^3 - 2^2) = 4
		\end{gather*}
	\end{example}
	
	\begin{theorem}[Euler's Theorem]
		Suppose $m \in \mathbb{N} \backslash \{1\}$. And $a \in \mathbb{N}$ \footnote{Also true for $a \in \mathbb{Z}$}Assume $a$ and $m$ are relatively prime, then 
		\[
			a^{\phi(m)} \equiv 1 \mod m
		\]
	\end{theorem}
	
	\begin{remark}
		This theorem is a generalization of Fermat's Little Theorem. When $m \in \mathbb{P}$, it becomes Fermat's Little Theorem.
	\end{remark}
	
	\begin{proof}\renewcommand{\qedsymbol}{\textcolor{red}{$ \ensuremath\varheartsuit$}}
		Let $S := \{r_1, r_2, \dots r_{\phi(m)}\}$ be the set of all elements in $\{1, 2, \dots, m-1\}$ that are relatively prime to $m$. \\
		Let $T := \{a r_1, a r_2, \dots a r_{\phi(m)}\}$. \\
		(Observation 1) that no two elements in $S$ are congruent to each other $\mod m$. Since all elements are in the range $[1, m-1]$ and they are the reminder while $r_i$ is divided by $m$. \\
		Also notice that elements in $T$ are not congruent to each other $\mod m$.\\
		Since, suppose \[a r_i \equiv a r_j \mod m\] for some $(i, j)$. \\
		Since $a$ and $m$ are relatively prime, therefore we could use cancellation law.\\
		\[
			r_i =\equiv r_j \mod m
		\]
		This would contradict our observation 1 \\
		(Observation 2) elements in $T$ are not congruent to each other $\mod m$. \\
		Therefore elements in $S$ are congruent to elements in $T \mod m$ \ul{in some order}. \\
		Therefore \[ r_1 r_2 r_3 \cdots r_{\phi(m)} \equiv a^{\phi(m)} r_1 r_2 \cdots r_{\phi(m)} \mod m \]\\
		 And notice $r_1 r_2 r_3 \cdots r_{\phi(m)}$ is a product of natural numbers relatively prime to $m$. \\
		 Therefore $r_1 r_2 r_3 \cdots r_{\phi(m)}$ is relatively prime to m. \\
		 And by cancellation law, we have 
		 \[
		 	a^{\phi(m)} \equiv 1 \mod m
		 \]
	\end{proof}
	
	\section{Lecture 11 Oct. 2 2018}
	\subsection{Rational and Irrational Numbers}
	\begin{definition}
		A \textbf{rational number} is an expression in form 
		\[
			\frac{m}{n},\ m,n \in \mathbb{Z},\ n \neq 0
		\]
	\end{definition}
	
	\begin{definition}
		Two rational numbers $\frac{m_1}{n_1}, \frac{m_2}{n_2} \in \mathbb{Q}$ are \textbf{equal} if and only if  $m_1 n_2 = m_2 n_1$.
	\end{definition}
	
	\begin{definition}
		Arithmetic on $\mathbb{Q}$ are defined as 
		\begin{itemize}
			\item \textbf{Addition} $+: \frac{m_1}{n_1} + \frac{m_2}{n_2} := \frac{m_1 n_2 + m_2 n_1}{n_1 n_2}$
			\item \textbf{Multiplication} $\times: \frac{m_1}{n_1} \times \frac{m_2}{n_2} := \frac{m_1 m_2}{n_1 n_2}$
			\item \textbf{Subtraction} $-: \frac{m_1}{n_1} - \frac{m_2}{n_2} := \frac{m_1 n_2 - m_2 n_1}{n_1 n_2}$
			\item \textbf{Division} $\div: \frac{\frac{m_1}{n_1}}{\frac{m_2}{n_2}} := \frac{m_1 n_2}{n_1 m_2}$, defined only if $m_2 \neq 0$.
		\end{itemize}
	\end{definition}
	
	\begin{definition}
		The \textbf{multiplicative inverse} of a \ul{non-zero} rational number $x \neq 0$ is a rational number $y$ such that $xy=1$.
	\end{definition}
	\begin{remark}
		Let $x = \frac{m}{n} \neq 0$, then the multiplicative inverse $y = \frac{n}{m}$.
	\end{remark}
	
	\begin{example}
		Claim: $\sqrt{2}$ is not rational.
	\end{example}
	\begin{proof}
		Assume $\sqrt{2}$ is rational, \\
		by definition of rational numbers, $\sqrt{2} = \frac{m}{n}$ where $m,n \in \mathbb{Z}, n \neq 0$. \\
		Divide numerator and denominator by their common prime factors (if any). \\
		Assume $m$ and $n$ have been reduced so that they are relatively prime. \\
		\begin{gather*}
			\implies 2 = \frac{m^2}{n^2}\\
			\iff 2n^2 = m^2 \\
			\implies 2 | m^2 
		\end{gather*}
		Consider if $2 \centernot | m$, then $m$ is odd, then $2 \centernot | m^2$. \\
		Take the contraposition, $2 | m^2 \implies 2 | m$. \\
		\begin{gather*}
			\implies 2 | m \\
			\implies m = 2q,\ q \in \mathbb{Z} \\
			\implies 2n^2 = 4 q ^2 \\
			\implies n^2 = 2q^2 \\
			\implies 2 | n^2 \\
			\implies 2 | n
		\end{gather*}
		That's $2 | m \land 2 | n$, which contradicts our assumption that $m$ and $n$ are relatively prime. \\
		Therefore $\sqrt{2}$ cannot be rational.
	\end{proof}
	
	\begin{definition}[non-rigorous definition]
		\textbf{Real numbers}, denoted as $\R$, are numbers representing distance of points on a line from $0$.
	\end{definition}
	
	\begin{definition}
		\textbf{Irrational numbers} are real numbers which are not rational. ($\mathbb{R}\backslash \mathbb{Q}$)
	\end{definition}
	
	\begin{proposition}
		Let $p \in \mathbb{P}$ and $m \in \mathbb{Z}$, then 
		\[
		p | m^2 \implies p | m
		\]
	\end{proposition}
	\begin{proof}
		Let $m = q_1 q_2 \dots q_\ell$ be the unique prime factorization. \\
		Suppose $p \centernot | m$, then $p \notin \{q_1, q_2, \dots, q_\ell\}$. \\
		Obviously, $m^2 = q_1^2 q_2^2 \dots q_\ell^2$ as it's prime factorization. \\
		Then $p \centernot | m^2$.
	\end{proof}
	
	\begin{example}
		$\sqrt{p} \notin \mathbb{Q},\ \forall p \in \mathbb{P}$.
	\end{example}
	
	\begin{proof}
		Let $p \in \mathbb{P}$,
		Suppose $\sqrt{p} \in \mathbb{Q}$. \\
		Therefore $\sqrt{p} = \frac{m}{n}$ where $m, n \in \mathbb{Z}$ and $n \neq 0$. \\
		Assume $\frac{m}{n}$ has been reduced such that $m$ and $n$ are relatively prime. \\
		\begin{gather*}
			\implies p n^2 = m^2 \\
			\implies p | m^2 \\
			\implies p | m \\
			\implies m = pr,\ r \in \mathbb{Z}. \\
			\implies pn^2 = p^2 r^2 \\
			\implies n^2 = p r^2 \\
			\implies p | n^2 \\
			\implies p | n
		\end{gather*}
		Contradicts the assumption that $m$ and $n$ are relatively prime.
	\end{proof}
	
\end{document}




























